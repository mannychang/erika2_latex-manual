\chapter{Task API Example}
\label{cha:taskapi}

The Task API Example, also called the {\em Christmas Tree} example, is
a simple demo that shows the usage of the following primitives:
\begin{itemize}
\item \fn{DeclareTask};
\item \fn{ActivateTask};
\item \fn{TerminateTask};
\item \fn{Schedule}.
\end{itemize}



\section{Demo structure}

The demo consists of two tasks, \fn{Task1} and \fn{Task2}.
\fn{Task1} repeatedly puts on and off a sequence of LEDs, like a
Christmas Tree. \fn{Task2} simply turns on and off a LED, and is
activated by the press of a button. Task2 is de facto a {\em
  disturbing} task that, depending on the configuration parameters,
may preempt \fn{Task1}.

The demo can be compiled in four different configurations, showing in
each case a different scheduling behavior. To select a configuration,
please look at the end of the OIL file containing the configuration of
the example, and uncomment {\em only one} configuration at a time.

\section{Configuration 1: Full preemptive}

This configuration is characterized by the following properties:
\begin{itemize}
\item An Altera HAL alarm \fn{Task1_alarm_callback} is linked to the
  System Clock (typically an Altera Interval Timer). The alarm simply
  activates \fn{Task1}.
\item Each instance of \fn{Task1} simply shows a {\em Christmas Tree},
  that is, \fn{Task1} controls a sequence of LEDs blinking from LED 0
  to LED 5.
\item each time the first button in the board is pressed, \fn{Task2}
  is activated.
\item \fn{Task2} always preempts \fn{Task1} (it has higher
  priority). \fn{Task2} does the following actions:
  \begin{itemize}
  \item LED 6 or LED 7 blink alternatively;
  \item a string is printed to the JTAG UART displaying the number of
    alarms fired, the number of button interrupt fired, the number of
    times Task2 has run.
  \end{itemize}
\end{itemize}

\subsection{Running configuration 1}

The behavior of the two tasks in configuration 1 can be evaluated
looking at the evaluation board LEDs together with the console output.

When the demo starts, the first string that appears on the console is
the banner
\begin{lstlisting}
Welcome to the ERIKA Enterprise Christmas Tree!
\end{lstlisting}
meaning the application has started correctly.

After that, the main() function calls \fn{StartOS()}. \fn{StartOS} is
an \ee\ primitive that initializes the \ee\ environment. The call to
\fn{StartOS()} must precede any call to the \ee\ API functions.

\fn{StartOS} automatically activates \fn{Task2}. This can be
configured by modifying the following property of \fn{Task2} in the
OIL configuration file:

\begin{lstlisting}
AUTOSTART = TRUE;
\end{lstlisting}

As a result, \fn{Task2} runs, blinking LED 6 and printing the following string:
\begin{lstlisting}
Task2 - Timer:   0 Button:   0 Task2:   1  
\end{lstlisting}

\begin{itemize}
\item The first number is the number of times the HAL alarm has run so
  far.  Every execution of the alarm callback activates \fn{Task1}, that in
  turn blinks LED 0 to 5.

\item The second number displays the number of executions of the
  button interrupt.  Every time the button is pressed or released, an
  interrupt is generated. The interrupt activates \fn{Task2}.
  
\item The third number displays the number of times \fn{Task2} has
  run.  As we will see later, the number may differ from the number of
  button activations.
\end{itemize}

Now press the first button in the evaluation board {\em for MORE THAN
ONE SECOND}, and release it.  As a result, two lines are printed on
the console.  The first line corresponds to the \fn{Task2} activation
issued by the interrupt generated when you pressed the button.  The
second line correspond to the \fn{Task2} activation issued by the
interrupt generated when you released the button.

Moreover, looking at the LEDs on the evaluation board, every time you
press or release the button, LEDs 6 or 7 blink regardless of the
current status of the Christmas Tree (that is, the current code being
run by \fn{Task1}).

The behavior happens because \fn{Task2} has an {\em higher priority}
than \fn{Task1}: when \fn{Task2} is activated, it has a higher
priority than \fn{Task1}, and as a result \fn{Task2} "preempts"
\fn{Task1}\footnote{Please note that this behavior will change in
Configuration 2, 3, and 4.}.

Now please press the button rapidly a few times. As a result, a few
lines are printed on the console. As it can be seen looking at the
number printed for button presses and \fn{Task2} executions, the
number of \fn{Task2} Activations are less than the number of Button
IRQs. The behavior happens because the OIL file for configuration 1
contains the line:
\begin{lstlisting}
KERNEL_TYPE = BCC1;
\end{lstlisting}

\noindent that is, \ee\ is configured with a \const{BCC1} conformance class,
which forces every task to maintain only one activation at a
time. \fn{Task2} activations issued when \fn{Task2} is in the ready
queue or is running are {\em lost} (in that case the \fn{ActivateTask}
primitive returns \const{E_OS_LIMIT}). We will address this issue in
Configuration 4.

You are now ready to proceed to Configuration 2.

To change configuration, go at the end of the OIL file, and leave
uncommented only the Configuration you want to try.  After that,
please rebuild the application.

\section{Configuration 2: Non preemptive}

This configuration differs from Configuration 1 only for the fact
\fn{Task1} is configured as {\em NON preemptive} in the OIL file by
using
\begin{lstlisting}
SCHEDULE = NON;
\end{lstlisting}
in the task attributes of the OIL file.

\subsection{Running configuration 2}
The effect of \fn{Task1} being non-preemptive is that \fn{Task2} only
runs when \fn{Task1} (the Christmas Tree) is not running. The fact can
be perceived by looking at the board LEDs: LEDs 6 or 7 only blinks
when the Christmas Tree is off, that is when \fn{Task1} ended its
current activation.

If compared with configuration 1, the lines printed on the console by
\fn{Task2} show the correct number of alarm executions, with the
correct number of alarm and button interrupts, meaning that interrupts
are handled also when a non-preemptive task is running.

On the other side, the number of times \fn{Task2} has run could be
less than in configuration 1 because some more activations could have
been lost due to the fact \fn{Task2} cannot preempt \fn{Task1}.



\section{Configuration 3: Preemption points.}

This configuration differs from configuration 2 for the fact
\fn{Task1} calls \fn{Schedule()} in the middle of the display of the
Christmas Tree.  To implement that, the OIL file specifies a global 

\begin{lstlisting}
EE_OPT = "MYSCHEDULE";
\end{lstlisting}
option, that translates in a 
\begin{lstlisting}
#define MYSCHEDULE
\end{lstlisting}
statement in the generated files when compiling the application source
code (see the automatically generated file
\file{Debug/default_cpu/eecfg.h} inside your project directory). Then,
\fn{Task1} source code includes a conditional compiler directive to
add the call to \fn{Schedule()}.


\subsection{Running configuration 3}
This configuration shows the usage of the
\fn{Schedule()} primitive, which can be used to implement a preemption
point inside a non-preemptive task.

In particular, \fn{Task2} can now preempt \fn{Task1} in the middle of
the Christmas tree (between the blinking of LED 2 and LED 3). The fact
is visible looking at the board LEDs: if the button is pressed while
LED 0, 1, or 2 are on, then in any case LED 6 (or 7) will blink
between the blink of LED 2 and 3.


\section{Configuration 4: Multiple Activations.}
This configuration differs from configuration 3 because \fn{Task2} has
now the possibility of storing a few pending activations. To do that,
the number of pending activations for \fn{Task2} is set to 6 by having the line
\begin{lstlisting}
ACTIVATION = 6;
\end{lstlisting}
to control the pending activations of \fn{Task2} and the line
\begin{lstlisting}
KERNEL_TYPE = BCC2;
\end{lstlisting}
to set the kernel conformance class to \const{BCC2}.

\subsection{Running configuration 4}
When running configuration 4, \fn{Task2} now stores up to 6 pending
activations, which means that if the task is activated more than once
(by pressing and releasing the button rapidly a few times) while Task1
is running , then \fn{Task2} is executed consecutively for a
corresponding number of times up to the maximum number of pending
activations specified in the OIL File.

\section{Configuration 5: Minimal configuration.}
This configuration differs from configuration 1 because the system is running with the \const{FP} conformance class. As a result, pending activations are stored inside an integer, and for that reason no activations are lost when clicking the button.

\subsection{Running configuration 5}
When running configuration 5, \fn{Task2} now stores all your pending
activations, which means that if the task is activated more than once
(by pressing and releasing the button rapidly a few times) while Task1
is running , then \fn{Task2} is executed consecutively for a
corresponding number of times. Pleae note that the order of
activations of tasks with the same priority is not maintained as it is with conformance class \const{BCC2} and \const{ECC2}.
