\chapter{Resource and Application Mode API Example}



The \emph{Resource and Application Mode API Example} is a simple demo
that shows the usage of the following primitives:
\begin{itemize}
\item \fn{GetActiveApplicationMode};
\item \fn{GetResource};
\item \fn{ReleaseResource}; 
\end{itemize}
The demo also uses \fn{ActivateTask} and
other task related primitives. Please refer to Section
\ref{cha:taskapi} for a better explanation of these primitives.

\section{Demo structure}

The demo consists of two tasks, \fn{LowTask} and \fn{HighTask} that
share a \emph{resource}.  \fn{LowTask} is a periodic low priority
task, activated by a timer, with a \emph{long} execution time. Almost
all its execution time is spent inside a critical section on the
resource (that is, between a call to \fn{GetResource} and
\fn{ReleaseResource}). To show this behavious, LED 0 is turned on when
\fn{LowTask} is inside the critical section.

\fn{HighTask} is a high priority task that increments (decrements) a
counter depending on the application mode being \const{ModeIncrement}
(\const{ModeDecrement}). The task is \emph{aperiodic}, and is
activated by the ISR linked to the button. Most of the execution time
of \fn{HighTask} is spent inside a critical section on the shared
resource. To show this behaviour, LED 1 is turned on when
\fn{HighTask} is inside the critical section. 

The application uses \emph{Application Mode}s to implement a task
behavior dependent on the Application Mode. When the application
starts, it checks the first pin of the Button PIO to select one out of
the two different application modes (\const{ModeIncrement} or
\const{ModeDecrement}).

An {\em Application Mode} is basically a system configuration that is
selected when the application starts. Application modes are set at
application startup, and are not changed after the call to
\fn{StartOS()}.  Application modes can be used to model particular
application configurations such as "Debug Mode", "Normal Mode", and so
on. The application tasks can read the application mode using
\fn{GetActiveApplicationMode}, and adjust their behavior depending on
the particular mode. Please also note that the \const{AUTOSTART}
property in the OIL file for tasks and alarms may have different
values for each application mode.

After the call to \fn{StartOS()}, all the calls to \fn{printf} are
always done in mutual exclusion inside the \const{mutex} critical
section, to ensure that there is no concurrency issues between the
various \fn{printf} calls.

\fn{HighTask} and \fn{LowTask} are configured to share the same stack
by setting the following line inside the OIL task properties:
\begin{lstlisting}
STACK = SHARED;
\end{lstlisting}
As a result, only a single stack will be used, with a dimension that
(in this case) is the sum of the stack space required by both
tasks. In general, by appropriately using priorities and preemption
thresholds the stack space can be reduced significantly.

Access to shared resources is managed by using the implementation of
the Immediate Priority Ceiling protocol in \ee. 
%
% TBD Forse una citazione non guasterebbe.
% da mettere quando sara' pronto il capitolo sullo scheduling nel
% manuale di ERIKA 
%
In practice, whenever \fn{LowTask} locks the resources with primitive
\fn{GetResource()}, its priority is raised to the one of
\fn{HighTask}. This fact prevents \fn{HighTask} from preempting
\fn{LowTask} while in the critical section, ensuring data consistency.
Of course, \fn{LowTask} will return to its priority when calling
\fn{ReleaseResource()} and the scheduler will be called to check
whether a preemption is necessary.
 
\section{Running the example}
Compile and run the application as usual. When the application starts,
the Button PIO is read, and the appropriate mode is passed to
\fn{StartOS}. let's suppose that the Button is not pressed: the
application will start with the application mode set to
\const{ModeIncrement}.

Each click of the first button in the evaluation board activates
\fn{HighTask}: if \fn{LowTask} is already executing in critical
section, \fn{HighTask} will have to wait for \fn{LowTask} to complete
its critical section before being able to execute.

Pressing the button too fast will activate \fn{HighTask} too fast. In
that case, it is likely that some activation will be lost (as it
happened in the Task API Example).
  
Starting the application with the first Button pressed, causes
\fn{StartOS} to be called with the \const{ModeDecrement} application
mode. In this example, the application mode simply influences the
increment or decrement of a counter, as it can be seen looking at the
console outputs.
