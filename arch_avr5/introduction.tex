\chapter[Introduction]{Introduction}
\label{cha:introduction}

Embedded microcontroller units are spreading in thousands of
applications, ranging from single to distributed systems, control
applications, multimedia, communication, medical applications and many
others. Modern microcontrollers, which are growing in computational
power, speed and interfacing capabilities, are more and more feeling
the need of tools to make the development of complex scalable
applications easier.

This manual describes the porting details of the \ee\ kernel
for the \avr\ family of microcontrollers. The \avr\ family produced by
Atmel represents a widely used 8-bit RISC microcontroller, with a full
range of interfaces available.

\subsection{\ee\ and \rtd\ for \avr}

Embedded applications often require tight control on the temporal
behavior of each single activity in the system. The research in the
field of real-time systems brought the team of Evidence Srl to design
a small, efficient, modular real-time kernel that can be used to
easily guarantee real-time constraints in every embedded applications.

\ee\ and \rtd\ represent the answer of Evidence Srl for the
development of scalable real-time applications for the \avr\
family.

\ee\ provides \avr\ developers the following features:

\begin{description}
\item[Traditional RTOS features] ~
  \begin{itemize}
  \item Support for four conformance classes to match different
    application requirements;
  \item Support for preemptive and non-preemptive multitasking;
  \item Support for fixed priority scheduling;
  \item Support for stack sharing techniques, and one-shot task model
    to reduce the overall stack usage;
  \item Support for shared resources;
  \item Support for periodic activations using Alarms;
  \item Support for centralized Error Handling;
  \item Support for hook functions before and after each context
    switch.
  \end{itemize}

\item[\rtd\ development environment] ~
  \begin{itemize}
  \item Development environment based on the Eclipse IDE;
  \item Support for the OIL language for the specification of the RTOS
    configuration;
  \item Graphical configuration plugin to easily generate the OIL
    configuration file and to easily configure the RTOS parameters;
  \item Full integration with the Cygwin development environment to
    provide a Unix-style scripting environment;
  \item Apache ANT scripting support for code generation;
  \end{itemize}

\item[\avr\ integration features] ~
  \begin{itemize}
  \item Installation setup which integrates the AVR gcc compiler and
    AVRStudio together with Evidence \ee\ and \rtd;
  \item Support for the Atmel JTAG debugger;
  \item Full support for \avr\ microcontroller series avr5;
  \item Full support for the development boards Atmel STK50X boards;
  \item Full support of the RF230 RF solution from Atmel, with usage
    of the IEEE 802.15.4 MAC.
  \item Full support for Crossbow MIB5X0 boards used to program Mica
    motes;
  \end{itemize}
\end{description}


\subsection{Integration with other tools for \avr}

\ee\ and \rtd\ aims to the best integration with the existing tools
for development available from the \avr\ microcontrollers.

\rtd\ will be used to quickly configure the application, setting
temporal parameters of real-time tasks, memory requirements, stack
allocation and many other parameters. \rtd\ generates the application
template, and leaves the developer the task to implement the logic of
each single task.

While programming the application, the developer can exploit the power
and flexibility offered by the primitives of the \ee\ real-time
kernel.

\ee\ also supports the compiling environments provided by Atmel,
providing also direct support for the programming and JTAG solutions
of Atmel.

\section{Content of this document}

The purpose of this document is to describe all the information needed
to create, develop and modify an \ee\ application for the \avr\ family
of microcontrollers. In particular, the document describes:
\begin{itemize}
\item The design flow which should be used to generate an \ee\ application;
\item The configuration of the development environment;
\item The options which are available to configure the system.
\end{itemize}

As a final note, all the settings which are explained in this document
apply both to \ee\, if not otherwise stated.

\begin{note}
If you are looking for a step-by-step / quick guide tutorial on how to
use \ee\ and \rtd\ with \avr, please read the ``\ee\ Tutorial for the
\avr\ microcontrollers'', available for download on the Evidence Web
site.
\end{note}
