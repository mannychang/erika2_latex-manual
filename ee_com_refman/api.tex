
\chapter{API reference}
\label{sec:api_reference}

\section{Introduction}
% provide an introduction to the OSEK COM stack
\ee\ provides a communication environment according to
\oc\ specification, version 3.0.3. \oc\ endows \ee\ with a mechanism
to transfer data between tasks and/or interrupt service routines
(ISRs). The communication is based on message objects. A message
containing application data is sent to sending message objects and
received by receiving message objects.  Message objects are identified
by message identifiers. For internal communication, sending messages
can store messages in zero or more receiving message objects. Zero or
more senders can send messages to the same sending message
object. This allows a n:m communication. All message objects are
statically defined using the OIL language.

The current implementation supports internal communication for queue
and unqueued messages. If a receiving message object is defined as
"queued", any message received by this object can only be read once
and the read operation removes the oldest message from the queue. For
a receiving message object defined as "unqueued", each message can be
read more times. The message object returns the last received message
every time it is read.

%aggiungere due parole sui meccanismi di notifica.
%aggiungere due parole gestione errori.
%aggiungere gli header da includere



\subsection{Communication Conformance Classes}
% provide a description of the available conformance classes and their features
Depending on the specific application and the hardware platform,
different levels of communication service can be
required. \oc\ defines these service levels through Communication
Conformance Classes (CCCs). The current version of the
\ee\ \oc\ implementation provides two conformance classes:

\begin{itemize}
  \item \const{CCCA} provides internal communication only. It supports
    only unqued messages. Message status is not supported. It provides
    the notification mechanism defined by Notification Class 1, except
    for the Flag notification mechanism.

  \item \const{CCCB} provides internal communication only. It supports
    queued and unqued message objects. It provides all the
    notification mechanisms provided by Notification Class
    1. Moreover, it supports message status information services.
\end{itemize}




\subsection{Notfication Classes}
Only Notification Class 1 is supported for internal communication. 

The following notification mechanisms are provided:
\begin{itemize}
\item Callback routines provided by the application;
\item Flags set by the \oc\ layer, read and reset by the application through the specific API service;
\item Tasks activated by the \oc\ layer;
\item Events set for application tasks by the \oc\ layer.
\end{itemize}


\pagebreak

\section{Constants}
\label{sec:constants}

This is a list of the \ee\ constants that can be used by the developer
for writing applications.

\begin{constant}{Error List}
  \begin{constantdescription}
    This is the list of the error values returned by the kernel primitives:
    \begin {lstlisting}
#define E_OK          0
    \end{lstlisting}
  \end{constantdescription}
\end{constant}

\begin{constant2}{INVALID\_TASK}{INVALID_TASK}
  \begin{constantdescription}
    This constant represent an invalid task ID, and is returned by
    \reffun{GetTaskID} when the function is called and no task is
    running.
  \end{constantdescription}
\end{constant2}

\begin{constant}{OSService IDs}
  \begin{constantdescription}
    This is the list of Service IDs values that can be returned by
    \reffun{OSErrorGetServiceId}:
    \begin {lstlisting}
#define OSServiceId_ActivateTask      1U
    \end{lstlisting}
    Please note that the primitives:
    \begin{itemize}                              \vspace{-2mm}
      \item \reffun{DisableAllInterrupts}        \vspace{-2mm}
    \end{itemize}
    never return an error, and for that reason they are not listed
    here.
  \end{constantdescription}
\end{constant}

\begin{constant}{OSDEFAULTAPPMODE}
  \begin{constantdescription}
    This is the default Application Mode. This value is always a valid
    Application Mode that can be passed to \reffun{StartOS}.
  \end{constantdescription}
\end{constant}


\pagebreak

%%%%%%%%%%%%%%%%%%%%%%%%%%%%%%%%%%%%%%%%%%%%%%%%%%%%%%%%%%%%%%%%%%%%%%%%%%%%


%PJ: most of these sections are here as a placeholder and can be removed if not used!

\section{Types}
\label{sec:types}

This Section contains a description of the data types used by the OS
interface of \ee. When the size of a type is specified to be of the
size of a machine register, it is intended that the type has the same
size of the CPU general purpose register.

\begin{type}{AlarmBaseType}
  \begin{typedescription}
    This structure is used to store the basic information about
    Counters. It has the following fields:
    \begin{description}
    \item[TickType maxallowedvalue] Is the maximum allowed count value
      in ticks for a counter.
    \item[TickType ticksperbase] It is the number of ticks required to
      reach a counter-specific significant unit.
    \item[TickType mincycle] It is the smallest allowed value for the
      \vr{cycle} parameter of the primitives 
      \reffun{SetRelAlarm}/\reffun{SetAbsAlarm}. This field is only
      present when Extended status is selected.
    \end{description}
  \end{typedescription}
\end{type}

\begin{type}{AlarmBaseRefType}
  \begin{typedescription}
    This is a pointer to \reftype{AlarmBaseType}.
  \end{typedescription}
\end{type}


\pagebreak






%%%%%%%%%%%%%%%%%%%%%%%%%%%%%%%%%%%%%%%%%%%%%%%%%%%%%%%%%%%%%%%%%%%%%%



\section{Object Declarations}
The following declarations have to be used to declare Tasks, Resources,
Alarms, and Events within the application code.

\begin{function_nopb}{DeclareAlarm}
  \synopsis{DeclareAlarm (AlarmIdentifier)}
  \begin{fundescription}
    Declares an alarm.

    This declaration is currently not mandatory because alarm
    identifiers are all declared within the code generated by \rtd.
  \end{fundescription}
  \begin{funconformance}
    BCC1, BCC2, ECC1, ECC2
  \end{funconformance}
\end{function_nopb}

%%%%%%%%%%%%%%%%%%%%%%%%%%%%%%%%%%%%%%%%%%%%%%%%%%%%%%%%%%%%%%%%%%%%%%

\pagebreak










%%%%%%%%%%%%%%%%%%%%%%%%%%%%%%%%%%%%%%%%%%%%%%%%%%%%%%%%%%%%%%%%%%%%%%

\section{Object Definitions}
The following macro have to be used when defining Tasks, ISRs and
Alarm Callbacks.

\begin{function_nopb}{ALARMCALLBACK}
  \synopsis{ALARMCALLBACK(t)}
  \begin{fundescription}
    This macro is used to declare and to define an alarm callback.
  \end{fundescription}
  \begin{funparameters}
    \fpar{t}{Name of the alarm callback.}
  \end{funparameters}
  %  \begin{funreturn}
  %    \fret{E_OK}{No error.}
  %  \end{funreturn}
  \begin{funconformance}
    BCC1, BCC2, ECC1, ECC2
%PJ: should this be CCCA and CCCB?
  \end{funconformance}
\end{function_nopb}

%%%%%%%%%%%%%%%%%%%%%%%%%%%%%%%%%%%%%%%%%%%%%%%%%%%%%%%%%%%%%%%%%%%%%%





\pagebreak

%%%%%%%%%%%%%%%%%%%%%%%%%%%%%%%%%%%%%%%%%%%%%%%%%%%%%%%%%%%%%%%%%%%%%%

\section{XXX Primitives}

% short general description

\pagebreak

%% 
%% Basic template 
%%
%% \begin{function}{FunctionName}
%% \synopsis{}
%%   \begin{fundescription}
%%   \end{fundescription}
%%   \begin{funparameters}
%%     \fpar{}{}
%%   \end{funparameters}
%%   \begin{funreturn}
%%   \fret{}{}
%%   \fret{}{(Extended)}.
%%   \end{funreturn}
%%   \begin{funconformance}
%%     BCC1, BCC2, ECC1, ECC2
%%   \end{funconformance}
%% \end{function}

% provide a list of the functions and of the contexts where they can be called
\begin{function}{StartCOM}
  \synopsis{StatusType StartCOM (COMApplicationModeType Mode)}
  \begin{fundescription}
    The user shall call this primitive to initialise and start the
    communication system in a specific application mode, defined by
    \vr{Mode}. This routine shall be called from a task. When
    \const{COMSTARTCOMEXTENSION} is set to \const{TRUE} within the OIL
    configuration file, before returning \reffun{StartCOM} calls
    \reffun{StartCOMExtension}. \reffun{StartCOMExtension} is provided
    by the application.
  \end{fundescription}
  \begin{funparameters}
    \fpar{Mode}{COM application mode}
  \end{funparameters}
  \begin{funreturn}
    \fret{E_OK }{ No errors. }
    \fret{A value returned from \reffun{StartCOMExtension} if different from E_OK.}.
    \fret{E_COM_ID}{(Extended) if \vr{Mode} is greater than  \const{EE_COM_N_MODE}.}
  \end{funreturn}
  \begin{funconformance}
    CCCA, CCCB
  \end{funconformance}
\end{function}

\begin{function}{StopCOM}
  \synopsis{StatusType StopCOM (COMShutdownModeType Mode)}
  \begin{fundescription}
    This primitive is called to terminate all communication activities
    immidiatly. The only possibile value for \vr{Mode} is
    \vr{COM_SHUTDOWN_IMMEDIATE}.
  \end{fundescription}
  \begin{funparameters}
    \fpar{Mode}{Shutdown mode}
  \end{funparameters}
  \begin{funreturn}
    \fret{E_OK }{ No errors. }
    \fret{E_COM_ID}{(Extended) if \vr{Mode} is different from \vr{COM_SHUTDOWN_IMMEDIATE}.}
  \end{funreturn}
  \begin{funconformance}
    CCCA, CCCB
  \end{funconformance}
\end{function}


\begin{function}{GetCOMApplicationMode}
  \synopsis{COMApplicationModeType GetCOMApplicationMode()}
  \begin{fundescription}
    This primitive returns the current COM application mode. It shall
    not be called before \reffun{StartCOM}.
  \end{fundescription}
  \begin{funparameters}
    \fpar{None}{}
  \end{funparameters}
  \begin{funreturn}
    \fret{Current COM application mode}
  \end{funreturn}
  \begin{funconformance}
    CCCA, CCCB
  \end{funconformance}
\end{function}

\begin{function}{InitMessage}
  \synopsis{StatusType InitMessage (MessageIdentifier Message, ApplicationDataRef DataRef)}
  \begin{fundescription}
    This primitive 
  \end{fundescription}
  \begin{funparameters}
    \fpar{None}{}
  \end{funparameters}
  \begin{funreturn}
    \fret{StatusType}{StatusType}
  \end{funreturn}
  \begin{funconformance}
    CCCA, CCCB
  \end{funconformance}
\end{function}

\begin{function}{SendMessage}
  \synopsis{StatusType SendMessage (MessageIdentifier Message, ApplicationDataRef DataRef)}
  \begin{fundescription}
    This primitive 
  \end{fundescription}
  \begin{funparameters}
    \fpar{None}{}
  \end{funparameters}
  \begin{funreturn}
    \fret{StatusType}{StatusType}
  \end{funreturn}
  \begin{funconformance}
    CCCA, CCCB
  \end{funconformance}
\end{function}

\begin{function}{SendZeroMessage}
  \synopsis{StatusType SendZeroMessage (MessageIdentifier Message)}
  \begin{fundescription}
    This primitive 
  \end{fundescription}
  \begin{funparameters}
    \fpar{None}{}
  \end{funparameters}
  \begin{funreturn}
    \fret{StatusType}{StatusType}
  \end{funreturn}
  \begin{funconformance}
    CCCA, CCCB
  \end{funconformance}
\end{function}

\begin{function}{ReceiveMessage}
  \synopsis{StatusType ReceiveMessage (MessageIdentifier Message, ApplicationDataRef DataRef)}
  \begin{fundescription}
    This primitive 
  \end{fundescription}
  \begin{funparameters}
    \fpar{None}{}
  \end{funparameters}
  \begin{funreturn}
    \fret{StatusType}{StatusType}
  \end{funreturn}
  \begin{funconformance}
    CCCA, CCCB
  \end{funconformance}
\end{function}

\begin{function}{GetMessageStatus}
  \synopsis{StatusType GetMessageStatus()}
  \begin{fundescription}
    This primitive 
  \end{fundescription}
  \begin{funparameters}
    \fpar{None}{}
  \end{funparameters}
  \begin{funreturn}
    \fret{}
  \end{funreturn}
  \begin{funconformance}
    CCCB
  \end{funconformance}
\end{function}

\begin{function}{COMErrorGetServiceId}
  \synopsis{COMServiceIdType COMErrorGetServiceId (void)}
  \begin{fundescription}
    This primitive 
  \end{fundescription}
  \begin{funparameters}
    \fpar{None}{}
  \end{funparameters}
  \begin{funreturn}
    \fret{}
  \end{funreturn}
  \begin{funconformance}
    CCCA, CCCB
  \end{funconformance}
\end{function}

\begin{function}{ReadFlag}
	\synopsis{FlagValue ReadFlag_<Flag>()}
	\begin{fundescription}
	This routine reads the status of the flag identified by the name $<$Flag$>$.
   	\end{fundescription}
%   	\begin{funparameters}
 %    	\fpar{None}{None}
 %  	\end{funparameters}
  	\begin{funreturn}
  	\fret{COM_TRUE }{$<$Flag$>$ is set}
   	\fret{COM_TRUE}{$<$Flag$>$ is not set}.
   	\end{funreturn}
	\begin{funconformance}
     	 CCCB
     	\end{funconformance}
\end{function}

\begin{function}{ResetFlag}
	\synopsis{void ResetFlag_<Flag>()}
	\begin{fundescription}
	This routine reset the status of the flag identified by the name $<$Flag$>$.
   	\end{fundescription}

	\begin{funconformance}
     	CCCB
     	\end{funconformance}
\end{function}

%%%%%%%%%%%%%%%%%%%%%%%%%%%%%%%%%%%%%%%%%%%%%%%%%%%%%%%%%%%%%%%%%%%%%%

%PJ: to be removed if not used
\begin{function}{ActivateTask}
  \synopsis{StatusType ActivateTask(TaskType TaskID);}
  
  \begin{fundescription}
    This primitive activates a task \vr{TaskID}, putting it in the
    \const{READY} state, or in the \const{RUNNING} state if the
    scheduler finds that the activated task should become the running
    task.

    Once activated, the task will run for an instance, starting from
    its first instruction. For the BCC2 and ECC2 Conformance classes,
    pending activations can be stored if the task has been configured
    with a number of activations greater than 1 within the OIL
    configuration file.

    The function can be called from the Background task (typically,
    the \fn{main()} function).

  \end{fundescription}
  
  \begin{funparameters}
    \fpar{TaskID}{Task reference.}
  \end{funparameters}
  
  \begin{funreturn}
    \fret{E_OK}{No error.}
    \fret{E_OS_LIMIT}{Too many pending activations of \vr{TaskID}.}
    \fret{E_OS_ID}{(Extended) \vr{TaskID} is invalid.}
  \end{funreturn}
  
  \begin{funconformance}
    BCC1, BCC2, ECC1, ECC2
  \end{funconformance}
\end{function}

%%%%%%%%%%%%%%%%%%%%%%%%%%%%%%%%%%%%%%%%%%%%%%%%%%%%%%%%%%%%%%%%%%%%%%

\pagebreak





%%%%%%%%%%%%%%%%%%%%%%%%%%%%%%%%%%%%%%%%%%%%%%%%%%%%%%%%%%%%%%%%%%%%%%

\section{Primitives provided by the application}

%%%%%%%%%%%%%%%%%%%%%%%%%%%%%%%%%%%%%%%%%%%%%%%%%%%%%%%%%%%%%%%%%%%%%%

\pagebreak





%%%%%%%%%%%%%%%%%%%%%%%%%%%%%%%%%%%%%%%%%%%%%%%%%%%%%%%%%%%%%%%%%%%%%%

\section{ErrorHook Macros}
\label{sec:errorhook-macros}
These macros are meaningful inside the \reffun{ErrorHook} Hook
function, and are used to better understand the source of the
error. In particular, \reffun{ErrorHook} receives as parameter the
error that is raised by the primitive. Then, a call to
\reffun{OSErrorGetServiceId} returns informations about which
primitive caused the error. Finally, calls to the macros
\fn{OSError_XXX_YYY} returns the values of the \const{YYY} parameter
of the primitive \const{XXX}.

%%%%%%%%%%%%%%%%%%%%%%%%%%%%%%%%%%%%%%%%%%%%%%%%%%%%%%%%%%%%%%%%%%%%%%

\begin{function_nopb2}{OSErrorGetServiceId}{OSErrorGetServiceId}
  \synopsis{OSServiceIdType OSErrorGetServiceId(void)}
  \begin{fundescription}
    The function may be used inside \reffun{ErrorHook} to return the
    Service ID that generated the error that caused the call to
    \reffun{ErrorHook}.
  \end{fundescription}
  \begin{funreturn}
    \fret{Service ID}{The service ID causing the error.}
  \end{funreturn}
  \begin{funconformance}
    BCC1, BCC2, ECC1, ECC2
  \end{funconformance}
\end{function_nopb2}


%%%%%%%%%%%%%%%%%%%%%%%%%%%%%%%%%%%%%%%%%%%%%%%%%%%%%%%%%%%%%%%
