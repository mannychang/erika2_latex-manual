\chapter{API reference}
\label{sec:api_reference}

\section{Introduction}
% provide an introduction to the OSEK COM stack
\ee\ provides a communication environment according to \oc\ specification, 
version 3.0.3. \oc\ endows \ee\ with a mechanism to transfer data between tasks 
and/or interrupt service routines (ISRs). The communication is based on message 
objects. A message containing application data is sent by sending message objects 
and received by receiving message objects.  Message objects are identified by 
message identifiers (IDs). For internal communication, sending message objects 
can store messages in zero or more receiving message objects. Zero or more senders 
can send messages to the same sending message object. This allows a n:m communication. 
All message objects are statically defined using the OIL language.    

The current implementation supports internal communication for queue and 
unqueued messages. If a receiving message object is defined as "queued", 
any message received by this object can only be read once and the read 
operation removes the oldest message from the queue. For a receiving message 
object defined as "unqueued", each message can be read more times. The message 
object returns the last received message every time it is read.

In order to used the COM support, the application should add the following includes
in its source files:
\begin {lstlisting}
  #include "com/ee_com.h"
  #include "com/com/inc/ee_cominit.h"
  #include "com/com/inc/ee_api.h"
\end{lstlisting}

The COM layer and the message objects are configurated in the OIL configuration 
file.


\subsection{Communication Conformance Classes}
% provide a description of the available conformance classes and their features
Depending on the specific application and the hardware platform, different 
levels of communication service can be required. \oc\ defines these service 
levels through  Communication Conformance Classes (CCCs). The current version of 
the \ee\ \oc\ implementation provides two conformance classes:

\begin{itemize}
\item CCCA (internal communication only). It supports only unqued
  messages.  Message status is not supported. It provides the
  notification mechanism defined by Notification Class 1, except for
  the Flag notification mechanism.
\item CCCB (internal communication only). It supports queued and
  unqued message objects. It provides all the notification mechanisms
  provided by Notification Class 1. Moreover, it supports message
  status information services.
\end{itemize}

\subsection{Available primitives}
\ee\ provides a set of primitives to exchange messages between tasks according to 
\oc\ specification. The primitives can be called in different situations. 
The complete list of primitives is listed in Table \ref{tab:api-restrictions}, 
together with the locations where it is legal to call these functions.

\begin{table}
\begin{centering}
\begin{tabular}{|c|c|c|c|c|}
\hline 
Service & \begin{sideways}
Background Task
\end{sideways} & \begin{sideways}
Task
\end{sideways} & \begin{sideways}
ISR2
\end{sideways} & COMErrorHook\tabularnewline
\hline 
\hline 
\reffun{StartCOM} & $\surd$ & $\surd$ & $\surd$ & \tabularnewline
\hline 
\reffun{StopCOM} & $\surd$ & $\surd$ & $\surd$ & \tabularnewline
\hline 
\reffun{GetCOMApplicationMode} & $\surd$ & $\surd$ & $\surd$ & \tabularnewline
\hline 
\reffun{InitMessage} & $\surd$ & $\surd$ & $\surd$ & \tabularnewline
\hline 
\reffun{SendMessage} & $\surd$ & $\surd$ & $\surd$ & \tabularnewline
\hline 
\reffun{ReceiveMessage} & $\surd$ & $\surd$ & $\surd$ & \tabularnewline
\hline 
\reffun{GetMessageStatus} & $\surd$ & $\surd$ & $\surd$ & \tabularnewline
\hline 
\reffun{COMErrorGetServiceId} &  &  &  & $\surd$\tabularnewline
\hline 
\reffun{ReadFlag} & $\surd$ & $\surd$ & $\surd$ & \tabularnewline
\hline 
\reffun{ResetFlag} & $\surd$ & $\surd$ & $\surd$ & \tabularnewline
\hline 
\end{tabular}
\par\end{centering}

\caption{\label{tab:api-restrictions}This table lists the environments where
primitives can be called.}
\end{table}


\subsection{Notfication Classes}
To determine the final status of a send or receive operation, the application can
exploit the notification services provided by the COM layer. The application does 
not need to check directly the occurrence of an event by calling a COM API primitive, 
but it is notified as soon as the specific event has occurred. A notification is
configured for each message object in the OIL configuration file.


Only notification class 1 is supported for conformance classes CCCA and CCCB. 
This means that the notification mechanism is only used to notify the receiver 
of a message. The configured notification mechanism is called whenever a message 
has been stored in the receiving message object.


The following notification mechanisms are provided:
\begin{itemize}
\item Callback routines provided by the application;
\item Flags set by the COM layer, read and reset by the application through 
the specific API service;
\item Tasks activated by the COM layer;
\item Events set for application tasks by the COM layer.
\end{itemize}


\pagebreak

\section{Constants}
\label{sec:constants}

This section provides a list of the constants that can be used by the developer
for writing applications that use the COM support.

\begin{constant}{Error List}
  \begin{constantdescription}
    This is the list of the error values returned by the COM primitives:
    \begin {lstlisting}
#define E_OK                                    0
#define E_COM_ID                                1
#define E_COM_LENGTH                            2
#define E_COM_LIMIT                             3
#define E_COM_NOMSG                             4
#define E_COM_SYS_DISCONNECTED                  5
    \end{lstlisting}
  \end{constantdescription}
\end{constant}

\begin{constant}{Boolean constants}
\label{subsec:const_bool}
  \begin{constantdescription}
    These constants are used to represent boolean values FALSE and TRUE.
    \begin {lstlisting}
#define COM_FALSE                               0
#define COM_TRUE                                1
    \end{lstlisting}
  \end{constantdescription}
\end{constant}

\begin{constant}{COMService IDs}
  \begin{constantdescription}
    This is the list of Service IDs values that can be returned by
    \reffun{COMErrorGetServiceId}:
    \begin {lstlisting}
#define COMServiceId_NoError                    0
#define COMServiceId_StartCOM                   1
#define COMServiceId_StopCOM                    2
#define COMServiceId_GetCOMApplicationMode      3
#define COMServiceId_InitMessage                4
#define COMServiceId_ReadFlag                   7
#define COMServiceId_ResetFlag                  8
#define COMServiceId_SendMessage                9
#define COMServiceId_ReceiveMessage            10
#define COMServiceId_SendZeroMessage           13
#define COMServiceId_GetMessageStatus          14
#define COMServiceId_COMErrorGetServiceId      15

    \end{lstlisting}
  \end{constantdescription}
\end{constant}

\begin{constant2}{COM\_SHUTDOWN\_IMMEDIATE}{COM_SHUTDOWN_IMMEDIATE}
  \begin{constantdescription}
    This is the default and only possible value for COM Suthdown Mode that can 
    be passed to \reffun{StopCOM}.
  \end{constantdescription}
\end{constant2}

\pagebreak


%%%%%%%%%%%%%%%%%%%%%%%%%%%%%%%%%%%%%%%%%%%%%%%%%%%%%%%%%%%%%%%%%%%%%%%%%%%%


\section{Types}
\label{sec:types}

This section contains a description of the data types used by the COM
interface of \ee. When the size of a type is specified to be of the
size of a machine register, it is intended that the type has the same
size of the CPU general purpose register.

\begin{type}{ApplicationDataRef}
  \begin{typedescription}
    This is a pointer to application data. This pointer can be passed to 
    \reffun{InitMessage}, \reffun{ReceiveMessage} and \reffun{SendMessage}.  
  \end{typedescription}
\end{type}

\begin{type}{COMApplicationModeType}
  \begin{typedescription}
   This type is an unsigned 8-bit integer type used to store COM Application Mode 
   IDs.
  \end{typedescription}
\end{type}

\begin{type}{COMServiceIdType}
  \begin{typedescription}
    This is an enumeration type used to store COM Service IDs, and it is used 
    within the \reffun{COMErrorGetServiceId}.
  \end{typedescription}
\end{type}

\begin{type}{COMShutdownModeType}
  \begin{typedescription}
    This type is an unsigned 8-bit integer type used to store COM Shutdown Mode 
    IDs. The default and only possibile value is \vr{COM_SHUTDOWN_IMMEDIATE}. 
  \end{typedescription}
\end{type}

\begin{type}{FlagValue}
  \begin{typedescription}
    This is an enumeration type used to store the current state of a message flag.
    Possible values are \vr{COM_FALSE} and \vr{COM_TRUE}, see 
    Section \ref{subsec:const_bool}
  \end{typedescription}
\end{type}

\begin{type}{MessageIdentifier}
  \begin{typedescription}
    This is an unsigned char used as ID for message objects.
  \end{typedescription}
\end{type}

\begin{type}{StatusType}
  \begin{typedescription}
    This type is an unsigned char used to store function error return values.
  \end{typedescription}
\end{type}


\pagebreak

%%%%%%%%%%%%%%%%%%%%%%%%%%%%%%%%%%%%%%%%%%%%%%%%%%%%%%%%%%%%%%%%%%%%%%

\section{Object Definitions}
The following macro has to be used when defining Message Callbacks. A message
callback is a notification mechanism that can be configured for each message 
object in the OIL configuration file.

\begin{function_nopb}{COMCallback}
  \synopsis{COMCallback(c)}
  \begin{fundescription}
    This macro is used to declare and to define a message callback.
  \end{fundescription}
  \begin{funparameters}
    \fpar{c}{Name of the message callback.}
  \end{funparameters}
  \begin{funconformance}
    CCCA, CCCB
  \end{funconformance}
\end{function_nopb}

%%%%%%%%%%%%%%%%%%%%%%%%%%%%%%%%%%%%%%%%%%%%%%%%%%%%%%%%%%%%%%%%%%%%%%





\pagebreak

%%%%%%%%%%%%%%%%%%%%%%%%%%%%%%%%%%%%%%%%%%%%%%%%%%%%%%%%%%%%%%%%%%%%%%

\section{COM Initialization Primitives}
The \oc\ implementation of \ee\ supports the specification of a set of {\it COM 
Application Modes}. Application modes are startup configurations that are used to
configure the communication system support for a certain mode of operation. 
COM Application modes are passed to \reffun{StartCOM} in order to start
the system in a specific mode. Application modes are defined by the application in 
the OIL configuration file. Examples of COM Application Modes are "Debug Mode", 
"Releas Mode", "Test Mode" and so forth. 

The COM support is initialized by \reffun{StartCOM} which initializes the system 
resources utilized by the COM module. \reffun{StartCOM} sets the initial value of 
an unqueued message to 0 if an initial value is not specified in the OIL
configuration file. If a message object has an initial value specified in the OIL
file, \reffun{StartCOM} initializes the message object with that specific value.
\reffun{StartCOM} initilaizes queued message objects with the number of receieved 
messages equal to 0. For internal transmit messages there is not any initialization 
procedure. An application can directly initialize a message by \reffun{InitMessage}. If 
configured in the OIL file, the application can provides a \reffun{StartCOMExtension} 
primitive used to add specific initialization functions, e.g. it can use 
\reffun{InitMessage} to initialize the application messages.

Finally, \reffun{StopCOM} shall be used to stop the communication system releasing 
its resources. Note that, after calling \reffun{StopCOM} the system can be re-initialized 
be calling \reffun{StartCOM} which resets all communication resources to their 
initial status. 

The start-up primitives are listed in the following Sections.

\pagebreak

\begin{function}{StartCOM}
  \synopsis{StatusType StartCOM (COMApplicationModeType Mode)}
  \begin{fundescription}
    The user shall call this primitive to initialize and start the  communication 
    system in a specific application mode, defined by \vr{Mode}. This routine 
    shall be called from a task. When \const{COMSTARTCOMEXTENSION}  is set to \const{TRUE}
    within the OIL configuration file, before returning, this primitive calls 
    \reffun{StartCOMExtension}. \reffun{StartCOMExtension} is provided by the 
    application.  
  \end{fundescription}
  \begin{funparameters}
    \fpar{Mode}{COM Application Mode}
  \end{funparameters}
  \begin{funreturn}
    \fret{E_OK }{ No errors. }
    \fret{value}{A value returned from StartCOMExtension if different from \const{E_OK}.}
    \fret{E_COM_ID}{(Extended) If \vr{Mode} is not valid.}
  \end{funreturn}
  \begin{funconformance}
    CCCA, CCCB
  \end{funconformance}
\end{function}

\begin{function}{StopCOM}
  \synopsis{StatusType StopCOM (COMShutdownModeType Mode)}
  \begin{fundescription}
    This primitive is called to terminate all communication activities and release
    their resources. 
    The only possible value for \vr{Mode} is \vr{COM_SHUTDOWN_IMMEDIATE}. 
  \end{fundescription}
  \begin{funparameters}
    \fpar{Mode}{Shutdown Mode}
  \end{funparameters}
  \begin{funreturn}
    \fret{E_OK }{ No errors. }
    \fret{E_COM_ID}{(Extended) If \vr{Mode} is different from 
      \vr{COM_SHUTDOWN_IMMEDIATE}.}
  \end{funreturn}
  \begin{funconformance}
    CCCA, CCCB
  \end{funconformance}
\end{function}


\begin{function}{GetCOMApplicationMode}
  \synopsis{COMApplicationModeType GetCOMApplicationMode(void)}
  \begin{fundescription}
    This primitive returns the current COM Application Mode. It shall not be 
    called before StartCOM.
  \end{fundescription}
  \begin{funreturn}
    \fret{value}{Current COM Application Mode}
  \end{funreturn}
  \begin{funconformance}
    CCCA, CCCB
  \end{funconformance}
\end{function}

\begin{function}{InitMessage}
  \synopsis{StatusType InitMessage (MessageIdentifier Message, ApplicationDataRef DataRef)}
  \begin{fundescription}
    This primitive can be used to initializes a message object with application data
    passed by the pointer \vr{DataRef}. To change the default initialization, it is possible 
    to call this primitive from \reffun{StartCOMExtension}.
  \end{fundescription}
  \begin{funparameters}
    \fpar{Message}{Message object ID}
    \fpar{DataRef}{Pointer to application data for the message object being initialized.}
  \end{funparameters}
  \begin{funreturn}
    \fret{E_OK }{ No errors.}
    \fret{E_COM_ID}{(Extended) If the message identifier \vr{Message} is not valid.}
  \end{funreturn}
  \begin{funconformance}
    CCCA, CCCB
  \end{funconformance}
\end{function}

\pagebreak

%%%%%%%%%%%%%%%%%%%%%%%%%%%%%%%%%%%%%%%%%%%%%%%%%%%%%%%%%%%%%%%%%%%%%%

\section{Message Communication Primitives}
This section provides the COM primiteves used to transmit and receive messages.

\begin{function}{SendMessage}
  \synopsis{StatusType SendMessage (MessageIdentifier Message, ApplicationDataRef DataRef)}
  \begin{fundescription}
    This primitive stores in the message object identified by \vr{Message} the 
    application datapointed by \vr{DataRef}.
  \end{fundescription}
  \begin{funparameters}
    \fpar{Message}{Message object ID}
    \fpar{DataRef}{Pointer to the application data.}
  \end{funparameters}
  \begin{funreturn}
    \fret{E_OK }{ No errors.}
    \fret{E_COM_ID}{(Extended) If message identifier \vr{Message} is not valid.}
  \end{funreturn}
  \begin{funconformance}
    CCCA, CCCB
  \end{funconformance}
\end{function}


\begin{function}{ReceiveMessage}
  \synopsis{StatusType ReceiveMessage (MessageIdentifier Message, ApplicationDataRef DataRef)}
  \begin{fundescription}
    This primitive stores the data from the message object \vr{Message} in the 
    application message memory pointed by \vr{DataRef}. All flags associated with 
    \vr{Message} are reset. If a message queue overflow happened since the last 
    call of this primitive, an \vr{E_COM_LIMIT} is returned, nevertheless a message is 
    store in \vr{DataRef} and the overflow is cleared.
  \end{fundescription}
  \begin{funparameters}
    \fpar{Message}{Message object ID}
    \fpar{DataRef}{Pointer to the application message data.}
  \end{funparameters}
  \begin{funreturn}
    \fret{E_OK }{ No errors.}
    \fret{E_COM_NOMSG }{If \vr{Message} is a queued message object and it is empty.}
    \fret{E_COM_LIMIT}{If \vr{Message} is a queued message object and a queue overflow 
          occurred since the last call of this primitive for \vr{Message}. 
         \vr{E_COM_LIMIT} indicates that at least a message was discarded.}
    \fret{E_COM_ID}{(Extended) If message identifier \vr{Message} is not valid.}
  \end{funreturn}
  \begin{funconformance}
    CCCA, CCCB
  \end{funconformance}
\end{function}


\begin{function}{GetMessageStatus}
  \synopsis{StatusType GetMessageStatus(MessageIdentifier Message)}
  \begin{fundescription}
    This primitive returns the current status of the message object identified by 
    \vr{Message}.  
  \end{fundescription}
  \begin{funparameters}
    \fpar{Message}{Message object ID}
  \end{funparameters}
  \begin{funreturn}
    \fret{E_COM_NOMSG }{If \vr{Message} is a queued message object and it is empty.}
    \fret{E_COM_LIMIT}{If \vr{Message} is a queued message object and a queue overflow 
          occurred since the last call of the primitive for \vr{Message}}. 
    \fret{E_OK }{ If there are no error indications, if \vr{Message} is not empty 
                 and an overflow condition has not occurred since the last call 
                 to \reffun{ReceiveMessage} }
    \fret{E_COM_ID}{(Extended) If the message identifier \vr{Message} is not valid 
     (e.g. \vr{Message} is invalid if it is not a queued message object).}
  \end{funreturn}
  \begin{funconformance}
    CCCB
  \end{funconformance}
\end{function}


\pagebreak

%%%%%%%%%%%%%%%%%%%%%%%%%%%%%%%%%%%%%%%%%%%%%%%%%%%%%%%%%%%%%%%%%%%%%%

\section{Notification Primitives}
The following primitives can be used to read and reset the status of 
the flags used as notification mechanism.

\begin{function}{ReadFlag}
 \synopsis{FlagValue ReadFlag_<Flag>(void)}
 \begin{fundescription}
   This primitive reads the status of the flag identified by the name \vr{<Flag>}.
 \end{fundescription}
 \begin{funreturn}
   \fret{COM_TRUE}{\vr{<Flag>} is set}
   \fret{COM_FALSE}{\vr{<Flag>} is not set}.
 \end{funreturn}
 \begin{funconformance}
  CCCB
  \end{funconformance}
\end{function}

\begin{function}{ResetFlag}
  \synopsis{void ResetFlag_<Flag>(void)}
  \begin{fundescription}
    This routine reset the status of the flag identified by \vr{<Flag>}.
   \end{fundescription}
   \begin{funconformance}
    CCCB
   \end{funconformance}
\end{function}


%%%%%%%%%%%%%%%%%%%%%%%%%%%%%%%%%%%%%%%%%%%%%%%%%%%%%%%%%%%%%%%%%%%%%%

\pagebreak





%%%%%%%%%%%%%%%%%%%%%%%%%%%%%%%%%%%%%%%%%%%%%%%%%%%%%%%%%%%%%%%%%%%%%%

\section{Primitives Provided by the Application}

The following routines are optional and must be provided by the application. 
These routines are enabled through the OIL configuration file 
of the specific application.

%%%%%%%%%%%%%%%%%%%%%%%%%%%%%%%%%%%%%%%%%%%%%%%%%%%%%%%%%%%%%%%%%%%%%%

\begin{function_nopb}{StartCOMExtension}
  \synopsis{StatusType StartCOMExtension(void)}
  \begin{fundescription}
    This primitive is executed at the end of the \reffun{StartCOM} routine. It can
    be used to add start-up configurations provided by the user. This 
    primitive is called by \reffun{StartCOM} if \const{COMSTARTCOMEXTENSION}  is set to 
    \const{TRUE} in the OIL configuration file. 
  \end{fundescription}
  \begin{funreturn}
   \fret{E_OK }{No errors}
   \fret{} {An implementation specific status code if an error occurred.}
   \end{funreturn}
  \begin{funconformance}
    CCCA, CCCB
  \end{funconformance}
\end{function_nopb}

\begin{function_nopb}{COMErrorHook}
  \synopsis{void COMErrorHook (StatusType err)}
  \begin{fundescription}
    This primitive is called by the COM services, such as \reffun{StartCOM}, 
    \reffun{ReceiveMessage} etc., when a status code different from \vr{E_OK} is 
     returned. This primitive is enabled by setting \const{COMERRORHOOK} to \const{TRUE} in 
     the OIL configuration file. 
  \end{fundescription}
  \begin{funparameters}
    \fpar{err}{ID of the error.}
  \end{funparameters}
  \begin{funconformance}
    CCCA, CCCB
  \end{funconformance}
\end{function_nopb}


\pagebreak

%%%%%%%%%%%%%%%%%%%%%%%%%%%%%%%%%%%%%%%%%%%%%%%%%%%%%%%%%%%%%%%%%%%%%%

\section{Error Handling Primitives and COMErrorHook Macros}
\label{sec:errorhook-macros}
These macros are meaningful inside the \reffun{COMErrorHook} Hook
function, and are used to better understand the source of the
error. In particular, \reffun{COMErrorHook} receives as parameter the
error that is raised by the primitive. Then, a call to
\reffun{COMErrorGetServiceId} returns informations about which
primitive caused the error. Finally, calls to the macros
\fn{COMError_XXX_YYY} returns the values of the \const{YYY} parameter
of the primitive \const{XXX}. 

%%%%%%%%%%%%%%%%%%%%%%%%%%%%%%%%%%%%%%%%%%%%%%%%%%%%%%%%%%%%%%%%%%%%%%

\begin{function_nopb2}{COMErrorGetServiceId}{COMErrorGetServiceId}
  \synopsis{COMServiceIdType COMErrorGetServiceId(void)}
  \begin{fundescription}
    This function may be used inside \reffun{COMErrorHook} to return the
    Service ID that generated the error that caused the call to
    \reffun{COMErrorHook}. 
  \end{fundescription}
  \begin{funreturn}
    \fret{Service ID}{The service ID causing the error.}
  \end{funreturn}
  \begin{funconformance}
    CCCA, CCCB
  \end{funconformance}
\end{function_nopb2}


\begin{function_nopb2}{COMError\_GetMessageStatus\_Message}{COMError_GetMessageStatus_Message}
  \synopsis{MessageIdentifier COMError_GetMessageStatus_Message(void)}
  \begin{fundescription}
    The function returns the \const{Message} parameter passed to
    \reffun{GetMessageStatus}. The function must be used inside \reffun{COMErrorHook} 
     after having discovered using \reffun{COMErrorGetServiceId} that the error 
     was caused by that function.
  \end{fundescription}
  \begin{funreturn}
     \fret{Message}{The value of the \const{Message} ID passed to \reffun{GetMessageStatus}.}
  \end{funreturn}
  \begin{funconformance}
   CCCA, CCCB
  \end{funconformance}
\end{function_nopb2}

\begin{function_nopb2}{COMError\_InitMessage\_Message}{COMError_InitMessage_Message}
  \synopsis{MessageIdentifier COMError_InitMessage_Message(void)}
  \begin{fundescription}
    The function returns the \const{Message} parameter passed to
    \reffun{InitMessage}. The function must be used inside \reffun{COMErrorHook} 
     after having discovered using \reffun{COMErrorGetServiceId} that the error 
     was caused by that function.
  \end{fundescription}
  \begin{funreturn}
     \fret{Message}{The value of the \const{Message} ID passed to \reffun{InitMessage}.}
  \end{funreturn}
  \begin{funconformance}
   CCCA, CCCB
  \end{funconformance}
\end{function_nopb2}

\begin{function_nopb2}{COMError\_InitMessage\_DataRef}{COMError_InitMessage_DataRef}
  \synopsis{ApplicationDataRef COMError_InitMessage_DataRef(void)}
  \begin{fundescription}
    The function returns the \const{DataRef} parameter passed to
    \reffun{InitMessage}. The function must be used inside \reffun{COMErrorHook} 
     after having discovered using \reffun{COMErrorGetServiceId} that the error 
     was caused by that function.
  \end{fundescription}
  \begin{funreturn}
     \fret{DataRef}{The value of the pointer \const{DataRef} passed to \reffun{InitMessage}.}
  \end{funreturn}
  \begin{funconformance}
   CCCA, CCCB
  \end{funconformance}
\end{function_nopb2}

\begin{function_nopb2}{COMError\_ReceiveMessage\_Message}{COMError_ReceiveMessage_Message}
  \synopsis{MessageIdentifier COMError_ReceiveMessage_Message(void)}
  \begin{fundescription}
    The function returns the \const{Message} parameter passed to
    \reffun{ReceiveMessage}. The function must be used inside \reffun{COMErrorHook} 
     after having discovered using \reffun{COMErrorGetServiceId} that the error 
     was caused by that function.
  \end{fundescription}
  \begin{funreturn}
     \fret{Message}{The value of the \const{Message} ID passed to \reffun{ReceiveMessage}.}
  \end{funreturn}
  \begin{funconformance}
   CCCA, CCCB
  \end{funconformance}
\end{function_nopb2}

\begin{function_nopb2}{COMError\_ReceiveMessage\_DataRef}{COMError_ReceiveMessage_DataRef}
  \synopsis{ApplicationDataRef COMError_SendMessage_DataRef(void)}
  \begin{fundescription}
    The function returns the \const{DataRef} parameter passed to
    \reffun{ReceiveMessage}. The function must be used inside \reffun{COMErrorHook} 
     after having discovered using \reffun{COMErrorGetServiceId} that the error 
     was caused by that function.
  \end{fundescription}
  \begin{funreturn}
     \fret{DataRef}{The value of the pointer \const{DataRef} passed to \reffun{ReceiveMessage}.}
  \end{funreturn}
  \begin{funconformance}
   CCCA, CCCB
  \end{funconformance}
\end{function_nopb2}

\begin{function_nopb2}{COMError\_SendMessage\_Message}{COMError_SendMessage_Message}
  \synopsis{MessageIdentifier COMError_SendMessage_Message(void)}
  \begin{fundescription}
    The function returns the \const{Message} parameter passed to
    \reffun{SendMessage}. The function must be used inside \reffun{COMErrorHook} 
     after having discovered using \reffun{COMErrorGetServiceId} that the error 
     was caused by that function.
  \end{fundescription}
  \begin{funreturn}
     \fret{Message}{The value of the \const{Message} ID passed to \reffun{SendMessage}.}
  \end{funreturn}
  \begin{funconformance}
   CCCA, CCCB
  \end{funconformance}
\end{function_nopb2}

\begin{function_nopb2}{COMError\_SendMessage\_DataRef}{COMError_SendMessage_DataRef}
  \synopsis{ApplicationDataRef COMError_SendMessage_DataRef(void)}
  \begin{fundescription}
    The function returns the \const{DataRef} parameter passed to
    \reffun{SendMessage}. The function must be used inside \reffun{COMErrorHook} 
     after having discovered using \reffun{COMErrorGetServiceId} that the error 
     was caused by that function.
  \end{fundescription}
  \begin{funreturn}
     \fret{DataRef}{The value of the pointer \const{DataRef} passed to \reffun{SendMessage}.}
  \end{funreturn}
  \begin{funconformance}
   CCCA, CCCB
  \end{funconformance}
\end{function_nopb2}

\begin{function_nopb2}{COMError\_StartCOM\_Mode}{COMError_StartCOM_Mode}
  \synopsis{COMApplicationModeType COMError_StartCOM_Mode(void)}
  \begin{fundescription}
    The function returns the \const{Mode} parameter passed to
    \reffun{StartCOM}. The function must be used inside
    \reffun{COMErrorHook} after having discovered using
    \reffun{COMErrorGetServiceId} that the error was caused by that
    function.
  \end{fundescription}
  \begin{funreturn}
    \fret{Mode}{The value of the \const{Mode} parameter passed to \reffun{StartCOM}.}
  \end{funreturn}
  \begin{funconformance}
   CCCA, CCCB
  \end{funconformance}
\end{function_nopb2}

\begin{function_nopb2}{COMError\_StopCOM\_Mode}{COMError_StopCOM_Mode}
  \synopsis{COMShutdownModeType COMError_StopCOM_Mode(void)}
  \begin{fundescription}
    The function returns the \const{Mode} parameter passed to
    \reffun{StopCOM}. The function must be used inside
    \reffun{COMErrorHook} after having discovered using
    \reffun{COMErrorGetServiceId} that the error was caused by that
    function.
  \end{fundescription}
  \begin{funreturn}
     \fret{Mode}{The value of the \const{Mode} parameter passed to \reffun{StopCOM}.}
  \end{funreturn}
  \begin{funconformance}
   CCCA, CCCB
  \end{funconformance}
\end{function_nopb2}




%%%%%%%%%%%%%%%%%%%%%%%%%%%%%%%%%%%%%%%%%%%%%%%%%%%%%%%%%%%%%%%
