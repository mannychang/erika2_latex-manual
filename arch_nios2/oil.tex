%=====================================================================
\chapter{OIL Extensions for the Nios II target}
%=====================================================================

This chapter contains OIL extensions specific for the Nios II target. It
addresses only the extensions for Nios II.

For a detailed description of the OIL features supported by \ee\ and \rtd,
which are coomon to all the supported architectures, please refer to the
\rtd\ Reference Manual. 

%---------------------------------------------------------------------
\section{Task Extensions}
%---------------------------------------------------------------------

%---------------------------------------------------------------------
\subsection{Stack size}
%---------------------------------------------------------------------

Task stacks can be shared or private. If the system is using shared
stack spaces, this information is ignored. In case the system is
configured for private stack spaces, each task needs to declare the
dimension of its stack space, expressed in bytes, with the following 
notation.

Definition:

\begin{lstlisting}
  ENUM [
    SHARED,
    PRIVATE_NIOSII {
      UINT32 SYS_SIZE;
    }
  ] STACK = SHARED;
\end{lstlisting}

Example of declaration:

\begin{lstlisting}
  STACK = PRIVATE_NIOSII {
    SYS_SIZE = 0x1000;
    /* or, alternatively
    SYS_SIZE = 4096;
    */
  }
\end{lstlisting}

%---------------------------------------------------------------------
\section{Nios II specific extensions}
%---------------------------------------------------------------------

%---------------------------------------------------------------------
\subsection{Configuration parameters}
%---------------------------------------------------------------------

The \const{NIOS2_DO_MAKE_OBJDUMP} option has the same effect of the
Nios II preference that produces an objdump of the ELF file on each
compilation.

The \const{NIOS2\_APP\_CONFIG} and \const{NIOS2\_SYS\_CONFIG} options
specify the configuration options for the application and for the
system libraries. Typical values for these options are \const{Debug}
and \const{Release}. Basically, these options specifies a set of
configuration parameters used to compile the source code.

The \const{NIOS2\_JAM\_FILE} and \const{NIOS2\_PTF\_FILE} options are
mandatory and are used to specify the JAM file (in case of debugging
using the Lauterbach Trace32), and the PTF file used for system
generation.

Definition:

\begin{lstlisting}
  STRING NIOS2_SYS_CONFIG;
  STRING NIOS2_APP_CONFIG;
  BOOLEAN NIOS2_DO_MAKE_OBJDUMP = FALSE;
  STRING NIOS2_JAM_FILE;
  STRING NIOS2_PTF_FILE;
\end{lstlisting}

Example:

\begin{lstlisting}
  NIOS2_SYS_CONFIG = "Debug";
  NIOS2_APP_CONFIG = "Debug";
  NIOS2_DO_MAKE_OBJDUMP = TRUE;
  NIOS2_JAM_FILE = "C:/mydirectory/myjamfile.jam";
  NIOS2_PTF_FILE = "C:/mydirectory/myptffile.ptf";
\end{lstlisting}


%---------------------------------------------------------------------
\subsection{CPU data}
%---------------------------------------------------------------------

Extensions to the CPU\_DATA include the following options:

\begin{description}
\item [{\tt SYSTEM\_LIBRARY\_NAME}] The name of the Altera system
  library to which the cpu is linked against.

\item [{\tt SYSTEM\_LIBRARY\_PATH}] The path where the Altera system
  library for the cpu is located.

\item [{\tt IPIC\_LOCAL\_NAME}] The input PIO that is used to raise
  interprocessor interrupts on the CPU.
\end{description}

Example:

\begin{lstlisting}
  CPU_DATA = NIOSII {
    ID = "cpu2";
    MULTI_STACK = TRUE {
      IRQ_STACK = FALSE;
      DUMMY_STACK = SHARED;
    };

    APP_SRC = "cpu2_startup.c";
    STACK_TOP = 0x20004000;
    SHARED_SYS_SIZE = 1800;
    SYS_SIZE = 0x1000; 

    SYSTEM_LIBRARY_NAME = "standard_2cpu_cpu2";
    SYSTEM_LIBRARY_PATH = "C:/altera/kits/.../mylibrary";
    IPIC_LOCAL_NAME = "IPIC_IN_2";
  };
\end{lstlisting}

%---------------------------------------------------------------------
\subsection{Mutex options}
%---------------------------------------------------------------------

In the HW configuration of a multiprocessor system the definition of
an Altera Avalon Mutex Peripheral is required. 

For Nios II versions 6.0 and above, the OIL parameter
\const{NIOS2\_MUTEX\_BASE} is used to specify the base address of the
Altera HAL mutex device as it can be found in the \file{system.h}
file.

Definition:

\begin{lstlisting}
  STRING NIOS2_MUTEX_BASE;
\end{lstlisting}

Example:

\begin{lstlisting}
   NIOS2_MUTEX_BASE = "MUTEX_BASE";
\end{lstlisting}

For Nios II versions 5.0 and 5.1, the OIL parameter
\const{MUTEX\_DEVICE\_NAME} is used to specify the name of the Altera
HAL mutex device as it can be found in the \file{system.h} file.

Definition:

\begin{lstlisting}
  STRING MUTEX_DEVICE_NAME;
\end{lstlisting}

Example:

\begin{lstlisting}
  MUTEX_DEVICE_NAME = "/dev/mutex";
\end{lstlisting}

%---------------------------------------------------------------------
\subsection{Startup}
\label{sec:startup}
%---------------------------------------------------------------------

Erika-specific parameters for Nios II multiprocessors include a few
startup options:

\begin{description}
\item [{\tt STARTUPSYNC}] This item may be declared as \const{TRUE} or
  \const{FALSE}. If \const{TRUE}, the startup barrier is enabled at
  system startup, all CPUs start executing the first line of code at
  the same time instant. The default value is \const{TRUE}.

\item [{\tt USEREMOTETASK}] If set at \const{IFREQUIRED}, a task is
  declared inside the CPU declaration of the processor where it is
  going to be executed. If a task handles signals coming from
  tasks/alarms executing on a remote CPU, the task needs to be
  declared also in those CPUs as a remote task. If
  \const{USEREMOTETASK} is set to \const{ALWAYS}, then all tasks are
  declared in the hosting CPUs and in all the other CPUs, regardless
  of the fact that they are referred by local code or not. The default
  value is \const{IFREQUIRED}. When using the value
  \const{IFREQUIRED}, the user must specify intertask activations
  using the \const{LINKED} attributes (see section
  \ref{sec:intertask-notifications}). 
\end{description}

Definition:

\begin{lstlisting}
  BOOLEAN STARTUPSYNC = TRUE;
  ENUM [ALWAYS, IFREQUIRED] USEREMOTETASK = IFREQUIRED; 
\end{lstlisting}

Example:

\begin{lstlisting}
  STARTUPSYNC = TRUE;
  USEREMOTETASK = ALWAYS;
\end{lstlisting}


\subsection{Intertask notifications}
\label{sec:intertask-notifications}
If a task needs to activate other tasks or set Events to other
extended tasks, then a corresponding declaration needs to be added to
the OIL declaration part. If the tasks are mapped on remote cpus,
\rtd\ provides the \ee\ kernel with the necessary information so that
remote signalling mechanisms (instead of local) are used. The OIL
declaration requires the names/identifiers of the tasks to which the
signals are directed. This option is typically used to reduce the
footprint of the kernel configuration data, because the generation of
the kernel data structure that supports remote activations are
generated only when needed, saving space in the kernel data structures
generated by \rtd.

\begin{warning}
  The specification of these options can be avoided if the option
  \const{USEREMOTETASK} is used with the \const{ALWAYS} value (see
  Section \ref{sec:startup}).
\end{warning}

Definition:

\begin{lstlisting}
  TASK_TYPE LINKED[];
\end{lstlisting}

Example:
\begin{lstlisting}
CPU mySystem {
  TASK myTask {
    LINKED = "task1";
    LINKED = "task2";
    ...
  };
}
\end{lstlisting}


%---------------------------------------------------------------------
\subsection{Compiler options}
%---------------------------------------------------------------------

Nios II supports an additional parameter \const{LDDEPS} that has the
same meaning of the \const{LDDEPS} parameter in the Altera Makefiles.

Definition:

\begin{lstlisting}
  STRING LDDEPS[];
\end{lstlisting}

Example:
\begin{lstlisting}
  /* this example includes allthe compilation parameters 
     for Nios II */
  CFLAGS = "-DALT_DEBUG -G0";
  CFLAGS = "-O0 -g";
  /* The previous two lines are equivalent to 
   * CFLAGS = "-DALT_DEBUG -G0 -O0 -g";
   */
  CFLAGS = "-Wall -Wl,-Map -Wl,project.map";
  ASFLAGS = "-g";
  LDDEPS = "\";
  LIBS = "-lm";
\end{lstlisting}

%---------------------------------------------------------------------
\section{OIL Extensions for Multiprocessing}
%---------------------------------------------------------------------

%---------------------------------------------------------------------
\subsection{Interprocessor Interrupts Extensions}
%---------------------------------------------------------------------

When using multiprocessor systems based on Altera Nios II, the user
must specify the Altera HAL names used to implement 
interprocessor interrupts.

The user must specify inside the OIL file the name of the global
Output PIO, by defining the attribute \const{IPIC_GLOBAL_NAME}, and
the name of the Input PIO local to each CPU with the definition of the
\const{IPIC_LOCAL_NAME} as specified in the following example:

\begin{lstlisting}
CPU test_application {
  OS EE {
    ...
    IPIC_GLOBAL_NAME = "IPIC_OUTPUT_PIO";
    CPU_DATA = NIOSII {
      ...
      IPIC_LOCAL_NAME = "IPIC_INPUT_PIO_CPU1";
    };
    ...
  };
  ...
};
\end{lstlisting}
