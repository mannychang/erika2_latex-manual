\chapter[Introduction]{Introduction}
\label{cha:introduction}


\section{\ee\ on multicores}

Today's embedded systems are continuously being extended to support
additional and more complex functionality. In many domains
(automotive, telecommunications, domotics), availability of powerful
hardware at low prices and cost/time market pressure are pushing for
integration of functionality on system-on-chip devices capable of
processing large amount of data in a short time.

Multicore systems are being considered as an economically viable
alternative to support this increasing computational demand. In this
context, Evidence proposes OS-level solutions and tools for embedded
multicore-on-a-chip.

The Evidence \rtd/\ee\ solution, originally developed for small-scale
OSEK/VDX compatible embedded systems for the automotive powertrain
market has been ported to the Altera Nios II environment, easing the
adoption of reconfigurable FPGA capable of supporting multiple
softcores on the same FPGA.

The \rtd/\ee\ pair has been designed to handle multicore
development and programming by hiding the use of multicore
synchronization primitives. With multicore hiding, it is
possible to seamlessly migrate application code from a single
processor to multicore without changing a single line of the
source code.

Hiding helps customers preserving their code base. Retargetting an
application from single to multicore architectures only requires
different OIL configurations, but allows retaining the same source
code in all the configurations.




\subsection{Systems-On-Chip and FPGAs: the future?}

\paragraph{Trends and challenges from the automotive market.}
Electronic components in today's cars are mandatory elements to
satisfy tight safety, efficiency and regulation constraints. Current
vehicle electronics systems consist of up to 100 embedded controllers
connected in a real-time distributed electronic system with several
network clusters and electronic control units (ECU). Setup and
Management of such complex distributed configurations is now the
rapidly becoming the cost, performance and reliability bottleneck of
the system. To satisfy tighter cost and packaging constraints the
number of ECU should be reduced compared to today cars resulting in an
integration of several functions.

To cope with this issue and the related design efficiency, the AUTOSAR
\cite{AUTOSAR} industrial initiative is standardizing a common set of
hardware and software components and an accommodating platform that
allows the integration of applications provided by different
sub-system and component makers. This trend and the ever increasing
application complexity (e.g. x-by-wire) demand architectures with
multiple computational units.

\paragraph{FPGAs: the opportunity.}
On the silicon technology side, the continuous scaling provides
hardware fabrics that allows unprecedented integration of functions in
a system-on-chip with very high parallelism. This trend promises that,
in the very near future, even the use of reconfigurable hardware and
softcores (CPU implemented with FPGAs) will be cost effective for
selected applications.

Moreover, several CPU providers (Intel,Motorola, IBM, ARM, etc) have
already announced multi-processor platforms with a winning
cost/performance trade-off for the consumer and multimedia markets.

The Janus system, developed by ST Microelectronics/Magneti-Marelli
Powertrain and Parades, is an example of a single-chip dual-processor
platform for power train applications (featuring two 32-bit ARM
processors and four memory banks connected by a crossbar switch)
offering better cost/performance trade-offs than traditional
architectures.

Reconfigurable platforms, such as Altera FPGAs, are providing soft-core
solutions like Nios II, offering the opportunity for implementing 4 or
more CPUs into even the cheapest Altera Cyclone II chips.


\section{Solving problems when moving code to multicores}
The development trends for next generation embedded devices clearly
states that the future lies to multicore devices on the same chip. In
particular, power management, leakage power issues, frequency limits,
cost reduction and other factors are driving many companies toward the
implementation of multicore system-on-a-chip solutions
integrating together CPUs, memories, and peripherals.

Altera Nios II is an example of these multicore
platforms. Thanks to the configurability of FPGAs, developers can
design multicore in minutes. Although on one side hardware
design is simplified by tools like Altera SOPCBuilder, on the other
side application development is made more complex by the fact that the
application code have to be spread out among different processors.

In particular, partitioning the application among the different CPUs,
and implementing an efficient communication mechanism between the CPUs
are one of the first issues to be addressed at the first stages of the
development process. Unfortunately these choices are very critical for
later development phases, since changes in the partitioning scheme can
heavily change application design.

\rtd\ and \ee\ give support to developers to solve the partitioning
issues in multicore applications based on Altera Nios II,
enabling the developer to perform code partitioning and then to easily
change it at later stages in the design.

\subsection{Design and programming.} 
Design and Programming paradigms must exploit the parallelism of these
architectures, but programmers are usually not trained for writing
code executing in parallel on multiple processors, and designers need
to find the best trade-offs for exploiting computing capabilities
without incurring in excessive blocking over shared resources or
because of synchronization.

With \ee, each task can be thought to run on a single processor
multi threaded environment. Multicore issues like data cache
disabling and mutual exclusion between different CPUs accessing
concurrently the same data structures are handled automatically by
\rtd\ and \ee, simplifying the application design and
verification.

\subsection{Code placement and resource sharing.}
The choice of which software has to be placed on which processor is
usually called code placement or software binding or partitioning.

The task of the final user is to figure out the right communication
pattern(s) / architecture(s) and to implement it. Today's approach
used by many development tools is non conclusive, leaving out to the
designer the job of choosing the best partitioning and communication
scheme for its application, with the risk of making early decisions
that may impact heavily on the application code with bad application
performance.

In reality, changing the code placement must not impact on the way
people design and program applications. The multicore structure
must be hided to the user whenever possible, and source code
compatibility between mono and multicore, with automatic mapping
of application facilities to the new multicore features. This is
exactly what the \rtd\ code generator offers, exploiting the
multicore support provided by \ee.

Future versions of \rtd\ will also enable users to perform
well-reasoned partitioning choices integrating the results of timing
analysis of the application with schedulability analysis to guide
design choices.

\subsection{Porting of legacy code to new architectures and to
  multicores.}

One of the problems that typically arises when adopting
multicores is that application rarely start from scratch with a
multicore approach. More often, new applications are upgrades to
existing (working) systems that are adapted to particular new
environment.

\rtd\ and \ee\ helps porting existing legacy code to multicores
because:
\begin{itemize}
\item The kernel API is the same for both single and multicore
  systems.
\item A multicore system can be view as an "extension" of a single
  core. You can add functionality to a system on a separate core,
  without perturbating too much the application running on the
  first CPU.
\item Application code does not have to be changed when changing the
  partitioning scheme.
\end{itemize}

\subsection{Resource sharing and OS-level mechanisms.}
When moving from single processor systems to multicore systems,
standard programming paradigms used in real-time systems to access
shared resources does not work anymore.

In particular, all the solutions used to avoid the Priority Inversion
problem, such as the Immediate Priority Ceiling protocol implemented
in the OSEK/VDX standard and in other APIs does not scale to
multicores like Altera Nios II.

\rtd\ and \ee\ offer an innovative way to automatically provide support
for resource sharing among different processors, automatically
integrating resource consistency protocols for multicores when a
resource is shared among different CPUs.



\section{Multicore API implementation}

While hardware technology is offering the opportunity for greatly
increased performance, conventional RTOS implementations and standards
are not yet ready to support multicore architectures. The
OSEK/VDX standard for Real-Time Operating Systems has gone a long way
in introducing mechanisms for ensuring predictable (schedulable)
real-time behavior and has provided the developers of automotive
applications with a stable and sound API.

Unfortunately, OSEK/VDX was not meant to be and it is not ready to
cope with the challenges posed by a multicore system.  In
particular, OSEK/VDX fits best on systems where different processors
are physically separated and connected through a network (e.g.,
Flexray, CAN, LIN, etc.), but it does not best fit in systems that
implement multicores on the same ECU.

In detail, the OIL configuration language supports the definition of
single processor systems possibly communicating using OSEK COM, but
provides no support for fundamental multicore issues such as:
distribution of the computational activities (tasks) among processors;
multicore communication and synchronization; and multicore
details hiding for applications compilation and execution.  

\subsection{\ee\ and multicore hiding}

With multicore hiding - that is the possibility to seamlessly
migrate application code from a single processor to multicore
without changing the source code - the programmer's view of the
application is not compromised, meaning that the RTOS API is supported
without changes and only minimal OIL extensions. Furthermore, \ee\ 
extends the real-time and memory saving features of automotive RTOS to
multicore systems.

In detail, \ee\ provides:
\begin{itemize}
\item Innovative mechanisms that allow exploiting the main features of
  Immediate Priority Ceiling in multicore systems.
\item Predictable real-time behavior.
\item Kernel features and OIL extensions that allow seamless execution
  of single processor code in multiple processor architectures.
\item Tools that support task placement on processors; easy
  reconfiguration of kernel calls for resource sharing and
  communication mechanisms from intraprocessor to interprocessor
  usage.
\item Minimal footprint in terms of ROM and RAM usage.
\item Real-time performance (such as interrupt and scheduler
  latencies, and context switch times) in line with the best market
  options and the possibility of configuring the kernel for constant
  time O(1) scheduling.
\end{itemize}

\ee\ hides the multiple processor structure by providing an automatic
mapping of the RTOS API calls to the single processor or the multiple
processor implementation, according to the target of the call. This
mapping is obtained by exploiting an enhanced remote notification
mechanism based on multicore interrupts. Remote notification is
used to provide an implementation to kernel primitives acting on tasks
(resources) allocated to remote processors. Moreover, the notification
mechanism is hidden inside the implementation of the \ee primitives,
maintaining the same interface to the developer both for single and
for multicore systems.

The additional overhead caused by the remote execution of calls is
mainly due to the "send" of the remote notification from one side, and
to the interprocessor interrupt that is raised on the receiving
side. Most of the overhead of the two calls is due to the internal
usage of the Altera Avalon Mutex calls. However, this overhead does
not have a great impact on typical applications where remote
activations are not as frequent as local activations.  

\subsection{Contention for multicore shared data structures.}

In addition to multicore hiding, the mechanisms typically
provided by common RTOS for sharing resources with a predictable and
bounded blocking time, such as the Immediate Priority Ceiling protocol
and the often related mechanisms for sharing stack space among
application tasks lose some of their hard real-time properties (such
as predictable blocking times).

\ee\ implements an advanced protocol used when sharing
resources among tasks allocated to different processors. The protocol,
known as Multicore Stack Resource Policy (MSRP), makes internal
use of spin-locks among different processors to ensure data
consistency. 

We expect that contention for multicore shared resources among
different systems will be the main source of overhead in the
system. To reduce them, critical section for multicore shared
data structures should be reduced as much as possible, for example
making local copies of the data structures when possible.


\section{Erika Enterprise for the Altera NIOS II platform}

This document describes the details of the \ee\ porting for the Altera
Nios II platform. Altera Nios II is a powerful soft-core processor,
which is basically a 32-bit RISC microcontroller especially designed
for the Altera FPGAs which can be customized to meet high performance
or low hardware (in terms of Logic Elements) requirements. The Altera
Nios II core can also be highly customized adding custom peripherals,
custom assembler instructions, debug features, caches, and other
features.

As you will note by reading the following pages, \ee\ fully supports
the various features of the Nios II platform, maintaining the Altera
workflow based on Quartus II, SOPCBuilder, and Nios II IDE.

\ee\ runs on every typical single core Nios II design without
requiring any modification to the hardware. All the Nios II HAL
functions, drivers, and features are retained on a \ee\ system.

On the other hand, \ee\ shows its performance and its ease of use when
the designer moves from simple single CPU Nios II designs to more
complex multicore Nios Ii designs. In particular, \ee\ has been
designed to address in a simple way all the problems that can arise
when dealing with multicore systems.

Designing multicore System-on-a-chip (SOC) has always been a difficult
task, because it involves a deep knowledge of hardware details and
application details that are needed to perform a correct co-design
between the application parts that have to be implemented in hardware,
and those that have to be implemented in software. Moreover, in a
multicore system, the user have to understand how the various software
functionality maps to the available computational resources, in a way
to optimize the system performance. Choosing the right partitioning
scheme for a multicore system is one of the most difficult tasks,
because a wrong application partition heavily impact on application
performance and response times. In fact, tuning a multicore design is
usually made in various design steps. Each design step in fact refers
to different stages in the application design, and often the result is
a tradeoff with the design choices that were made earlier in the
application definition.

The design tools provided by Altera allow designers to easily create
and test multicore hardware platforms based on the Altera Nios II
Processor. Basically the developer creates a multicore platform by
simply inserting CPUs and connecting peripherals using the Altera
SOPCBuilder Tool. Using this tool the designer can modify the hardware
architecture easily, adding and removing CPUs, modifying the bus
connections, and so on. Once the hardware platform has been created
and generated, the developer can develop the software separately for
each CPU using the Nios II IDE based on Eclipse. Altera provides a
nice environment where for each CPU the user can instantiate the
device drivers of the hardware components connected to the particular
CPU (using the provided Altera HAL and the System Libraries
Projects). After that, the user can create a C/C++ application that is
linked to a single System Library (that is, to a single CPU).

It is our believe that the common workflow during the design of
complex multicore applications requires a frequent number of change in
the partitioning of the software among the different CPUs. That is
typically needed because application requirements often change during
the development of an embedded system, and the choices that has been
made earlier in the design may not be the right ones at the end of the
development. In this sense, the developer must have the power to
change the software partitioning among processors with little effort,
as it is currently done for the hardware part in SOPCBuilder, where
the user can change the hardware configurations with a few mouse
clicks.

That is exactly what Evidence Srl aims to provide to Altera Nios II
customers: a way to simply deal with the complexity of multicore
software development, easing the partitioning of applications on
different processors, giving the possibility to developers to change
their partitioning without changing their source code.

Briefly, Evidence Srl provides a powerful design and configuration
tool named \rtd, and an RTOS, named \ee. The \rtd\ Toolset and \ee\
RTOS exploit the multicore capabilities of the Altera FPGA boards and
allow to fully use the power of the multicore designs.

On the host side, \rtd\ offers an integrated view of a multicore
application under the Nios II IDE, allowing the user to partition the
application jobs on the different available CPU. To do that, \rtd\
offers a concept of ``application'' project that spans different CPUs,
allowing an easy move of features across the execution units. \rtd\ is
a set of plugins for the Eclipse framework that are integrated with
the Altera Nios II IDE, giving the possibility to the user to:
\begin{itemize}
\item develop a multicore application using an \rtd\ project, that is
      an Eclipse Project that is able to handle the software for all
      the CPUs in the multicore system\footnote{A single \rtd\ Project
      can be viewed as a replacement for the set of Altera Application
      Projects for each CPU.};
\item specify one or more partitioning schemes to be used for a
      multicore system;
\item generate the configuration code for the \ee\ RTOS;
\item analyze the results of the partitioning using schedulability
      analysis \footnote{The schedulability analysis plugin will be
      available together with the next version of \rtd};
\end{itemize}


On the target side, \ee\ offers a multiprogramming environment that
can be used to deploy the application architecture defined in \rtd\ on
the real hardware. \ee\ helps the partitioning job supporting code
deployment that is independent from the particular CPU, allowing the
use to move code between CPUs without changing it. \ee\ is a Real-Time
Operating System (RTOS) that offers a multicore partitioned
multithread environment that internally takes care of multicore issues
like activation and run of remote tasks, periodic alarms, and shared
resources among tasks allocated to different processors. Of course, as
said before, the restriction to single processor designs is
supported. \ee\ currently offers support for fixed priority
scheduling, with preemptive, non preemptive and mixed preemption
support.

\ee\ is ready for formal compliancy with the OSEK/VDX
operating system standard.

The main feature in the Evidence approach for Nios II is the fact that
the \rtd\ Toolset and \ee\ do not change the current Altera
Workflow. For that reason, users accustomed with the Nios II IDE
editing and debugging features can easily reuse their knowledge.

Some of the features supported by \rtd\ and \ee\ are:

\begin{itemize}
\item Support for the definition, and deployment of software for
      Nios II multicore systems with shared memory.
\item Support for partitioning of the application functionalities on
      different processors with negligible modification to the source
      code base.
\item Support for easy change of the partitioning scheme allowing the
      developer to easily try and test different partitioning schemes.
\item Support for the migration of single processor source code
      to multicore systems.
\item Support for the Altera development toolchain: \rtd\ and \ee\ are
      perfectly integrated with Altera Nios II, reducing the learning
      curve of the tools.
\end{itemize}

Specific Altera integration features include:
\begin{itemize}
\item Integrated as a component in the Altera HAL;
\item Compatible with the Altera HAL peripheral drivers and system
libraries;
\item Support for nested interrupts;
\item Support for a RTOS configuration code generator compliant
with the OSEK OIL specifications;
\end{itemize}

Moreover, the multicore support of \ee\ includes:
\begin{itemize}
\item Advanced software partitioning support:
\begin{itemize}
\item developers can decide which task goes on which processor;
\item developers can move tasks between processors without
  changing the source code;
\item Transparent handling of shared resource locks behavior
  depending on the partitioning of the application;
\end{itemize}
\item Interprocessor interrupt support;
\item Shared resource support using queuing spin locks on top of the
Altera Mutex Peripheral;
\item Automatic cache disabling technique without changes to the
user source code (only changes to data definitions);
\item Support for multicore scheduling algorithms with bounded
blocking times on multicore;
\end{itemize}

The following chapters describe in detail the single and multicore
support for the Altera Nios II platform that is offered by \ee\ and by
\rtd, highlighting the various features available and the main
architecture design requirements of the system.

%
%
%
%\nb{e' il caso di parlare qui di una roadmap dei prodotti Evidence?
% probabilmente conviene prendere la cosa di petto e mettere 
% fuori un apposito documento!!!}
%
%
%Future versions of the product will include:
%\begin{itemize}
%\item schedulability analysis
%\item graphical configurator
%\item ORTI language for kernel awareness
%\item Official OSEK/VDX compliancy
%\item and much much more!
%\end{itemize}
