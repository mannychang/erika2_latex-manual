\chapter{Introduzione all'API minimale}

\section{\ee\ }

Evidence presenta il sistema operativo real-time \ee\, un sistema operativo
minimale per microcontrollori che fornisce un semplice e compatto ambiente
di esecuzione multithreading, con il supporto di algoritmi di scheduling
real-time avanzati e il supporto allo stack sharing.

Il kernel \ee\ e' stato sviluppato con l'intenzione di fornire un insieme
minimale di primitive che possono essere utilizzate per implementare
un sistema di ambiente multithreading. Le API minimali di \ee\ costituiscono un
insieme ridotto delle API OSEK/VDX, che forniscono il supporto per
funzionalita' quali l'attivazione di thread, la mutua esclusione, gli
allarmi e i counting semaphores.

Il consorzio OSEK/VDX ha messo a punto il linguaggio di specifica OIL 
(OSEK Implementation Language) come standard per la configurazione delle
applicazioni. Esso viene utilizzato per la definizione statica di una 
serie di componenti e funzionalita' da istanziare in una applicazione.
\ee\ supporta appieno il linguaggio OIL per la configurazione di 
applicazioni real-time.

Per far fronte alla complessita' che deriva dalla manipolazione di un file 
di configurazione scritto in linguaggio OIL, Evidence ha sviluppato un tool
di configurazione e profiling apposito: \rtd\. \rtd\ permette di 
configurare una applicazione in ogni sua componente, impostando
i parametri in modo semplice attraverso una interfaccia visuale 
che automaticamente genera il codice di configurazione in linguaggio 
OIL.

Il tipico flusso di progetto e design di una applicazione include la definizione
di un file di configurazione in linguaggio OIL, che definisce gli oggetti, i 
componenti che sono utilizzati all'interno dell'applicazione real-time.
In questa fase, \rtd\ aiuta lo sviluppatore ad impostare i parametri dei singoli
oggetti e provvede alla generazione del relativo file di configurazione in
linguaggio OIL, nonche' alla generazione del codice sorgente e dei makefiles
richiesti per compilare l'applicazione. L'ultimo step prevede l'effettiva 
compilazione del sorgente e la generazione del codice eseguibile sulla
macchina target.

Alcune delle features fornite da \ee\ allo sviluppatore, tipiche di un sistema real-time,
sono le seguenti:
\begin{itemize}
  \item Supporto per il multitasking sia di tipo preemptive che non-preemptive;
  \item Supporto per algoritmi di schedulazione a priorita' fisse;
  \item Supporto per tecniche di stack sharing e di modelli di task one-shot 
  al fine di ridurre l'utilizzo di stack da parte del programma;
  \item Supporto per risorse condivise;
  \item Supporto per l'attivazione periodica di task per mezzo degli Alarms;
\end{itemize}

Lo scopo di questo documento e' quello di descrivere in dettaglio le funzionalita' e la
API di \ee\. Presso il sito web di Evidence e' disponibile ulteriore documentazione
riguardo al porting di \ee\ per varie architetture e microcontrollori.
