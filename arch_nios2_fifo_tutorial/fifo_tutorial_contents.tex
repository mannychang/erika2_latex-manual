\chapter{Introduction}

This short tutorial shows how to build a simple multiprocessor
application on the Altera Nios II platform composed of a producer
\const{WriteTask} task that periodically sends information (an
\const{alt_u32} 32-bit unsigned integer) to a consumer
\const{ReadTask} task. The information is sent using a shared data
structure that implements a FIFO queue.

The demo is organised as follows: the writer task \const{WriteTask}
writes one item every second, while the reader task \const{ReadTask}
reads one item every two seconds. A third task called
\const{ButtonReadTask} is also present. Its behavior is to flush all
the content of the FIFO queue.

The demo focuses on the following features:
\begin{itemize}
\item The allocation of the shared memory used to implement the FIFO
  queue is completely handled by the \ee\ makefile scripts. There is
  not any need for explicitly allocating or reserving memory
  areas. This marks a remarkable difference, for instance, from Altera
  Mailboxes.
\item Task allocation to a particular CPU can be changed by changing
  only the \const{CPU_ID} field of a task inside the OIL File. The
  code of the tasks remains unchanged when changing the used
  partitioning.
\end{itemize}

This tutorial assumes the reader's familiarity with design of
multiprocess systems, as introduced in Altera document named
{}``Creating Multiprocessor Nios II systems tutorial'' available on
the Altera web site \cite{Altera-multicpu-tutorial}, and with the 2
CPU \ee\ tutorial included inside the \ee\ manual
\cite{Evidence-multicpu-tutorial}.

\chapter{Running the tutorial}

The following basic steps will guide you in running the application
included in this tutorial:

\begin{enumerate}
\item First of all, the hardware design. The hardware design that is
  needed to run this demo is the same 2-CPU Nios II hardware design
  used for the 2 CPU \ee\ tutorial. Please refer to the 2 CPU \ee\
  tutorial inside the \ee\ multicore tutorial
  \cite{Evidence-multicpu-tutorial} for information on the creation of
  multiprocessor designs compatible with \ee. In the following points,
  we suppose that the location of the hardware design is the same as
  specified in the 2 CPU \ee\ tutorial. If not, remember to change the
  OIL directories accordingly as explained in the section ``Updating
  the OIL File'' of that tutorial.

\item Then, you need to create two Altera System Libraries. Again,
  these libraries are the same as in the 2 CPU \ee\ tutorial
  \cite{Evidence-multicpu-tutorial}. Please refer to it for more
  information.

\item From the File menu of the Nios II IDE select ``New
  Project...''. Choose ``RT-Druid Nios Project'' from the Evidence tab
  of the New Project Dialog box. A dialog box appears. Choose the FIFO
  Template. Name the project \file{demo_fifo} and press the Finish
  button.

\item Right click on the project name, and select ``Build Project''.
  The demo application will be compiled, and two ELF binaries will be
  produced.

\item To run the demo, please follow the instruction described in the
  \ee\ 2 CPU tutorial \cite{Evidence-multicpu-tutorial}: two debug
  configuration (one for each CPU) will be created along with a
  multiprocessor collection. After that, you have to program the FPGA,
  and run the just created multiprocessor collection.
\end{enumerate}

The result of the application run is the following:

\begin{enumerate}
\item The \fn{WriteTask} is activate once every second. The ReadTask
  is activated once every two seconds. Each activation prints a
  message to the JTAG UART. As a result, the FIFO read frequency is half
  the FIFO write frequency.
\item After a few seconds, the FIFO fills up (by default it can store
  only 8 items), the \fn{WriteTask} starts failing the write
  operation, printing the message \const{FIFO buffer full!}.
\item At that point, pressing one of the buttons causes the activation
  of the \fn{ButtonReadTask}, that reads all the buffer. One click of
  the button typically activates the task twice (at the press and at
  the release of the button).
\end{enumerate}

Here is a typical execution trace copied from the Nios II consoles for
CPU 0:

\begin{lstlisting}
...
Hello from CPU 0!
WriteTask: written 1
WriteTask: written 2
WriteTask: written 3
WriteTask: written 4
WriteTask: written 5
WriteTask: written 6
WriteTask: written 7
WriteTask: written 8
WriteTask: written 9
WriteTask: written 10
WriteTask: written 11
WriteTask: written 12
WriteTask: written 13
WriteTask: written 14
WriteTask: written 15
WriteTask: FIFO buffer full!
WriteTask: written 17
WriteTask: FIFO buffer full!
WriteTask: written 19
WriteTask: FIFO buffer full!
ButtonReadTask: read 11
ButtonReadTask: read 12
ButtonReadTask: read 13
ButtonReadTask: read 14
ButtonReadTask: read 15
ButtonReadTask: read 17
ButtonReadTask: read 19
ButtonReadTask: FIFO buffer empty, ending task!
ButtonReadTask: FIFO buffer empty, ending task!
WriteTask: written 21
WriteTask: written 22
WriteTask: written 23
WriteTask: written 24
...
\end{lstlisting}

...and for CPU 1:

\begin{lstlisting}
...
Hello from CPU 1
ReadTask: read 1
ReadTask: read 2
ReadTask: read 3
ReadTask: read 4
ReadTask: read 5
ReadTask: read 6
ReadTask: read 7
ReadTask: read 8
ReadTask: read 9
ReadTask: read 10
ReadTask: read 21
ReadTask: read 22
...
\end{lstlisting}


\chapter{Final comments}

This short chapter contains some details about the design choices used
when developing the example, that we think useful for people that want
to design \ee\ applications.

\paragraph{Partitioning the software}

\rtd\ and \ee\ allow an easy partitioning of the application
software. Changing the partitioning of the tasks into the CPUs is very
simple: you just need to change the \const{CPU_ID} parameter in the
task section of the OIL file.

For example, you can put all the three tasks to CPU 0, or the tasks to
CPU 1, or choose an intermediate partitioning. In all these cases,
multiprocessor resource handling and task activations will be hided by
\ee, without changing the application software.  Please note that a
different partitioning scheme does not require a change in the
application source code, but only in the OIL configuration file.

To test a different partitioning, just change the OIL File, recompile
and rerun the demo.

\paragraph{Resource usage.}
The simple implementation of the FIFO queue presented in this tutorial
shows a typical usage of the \fn{GetResource} and \fn{ReleaseResource}
primitives for the implementation of a simple non-blocking shared data
structure. The demo defines a \const{fifo_t} abstract data type, and
uses two resources for each instance FIFO queue: a first resource is
passed to the FIFO write function to handle concurrent accesses from
different writers, whereas the second resource is passed to the FIFO
read function to handle concurrent accesses from different writers.

Using two resources means that read and write operations can proceed
in parallel, in the same way as it happens for the Altera Mailbox.

The main difference with the Altera Mailbox is that if the tasks using
the resources are allocated to the same CPU, then the Altera
Mutex peripheral is not used at all.

\paragraph{Using a single resource.}
To limit the resource usage, the FIFO can work using a single resource
for both readers and writers. The behavior in this case will be that
only one task (writer or reader) will get access to the FIFO at the
same time.

\paragraph{Implementing a Blocking (spinning) FIFO.}
The Altera Mailbox exports two functions called
\fn{altera_avalon_mailbox_post} and
\fn{altera_avalon_mailbox_pend}. The behavior of these function is to
actively spin until the FIFO becomes empty.  This approach leads to
satisfactory results on multiprocessor platforms where some CPU write
and the other read; however, it does not allow a flexible partitioning
of the code, that needs to be modified when the partitioning
changes. To extend the example to have a spinning read primitive like
in the Altera Mailbox, you need to put a cycle outside the read or
write function, in a way similar to what happens to the
\fn{ButtonReadTask} task.

The implementation of a blocking (not spinning) primitive requires the
usage of the ECC1 or ECC2 \ee\ configuration, and the usage of the
\fn{WaitEvent} primitive. Consider that implementing such kind of
primitive requires a disabling of the stack sharing mechanism for the
tasks using the FIFO.

\paragraph{Activating the tasks upon reads and writes.}
Instead of actively spinning waiting for the arrival of a data on the
FIFO, another nice technique that can be used in this case is to map
the read of the FIFO to a particular task, and to activate it every
time there is a new data written on the FIFO. When considering this
approach, you need to take into account the overhead introduced by
interprocessor interrupts that may possible occurr upon remote task
activations".

\paragraph{Passing Resource IDs as parameters.}
The FIFO functions provided in the example receives as a parameter the
Resource that have to be locked when accessing the data structure for
read or write operations.

The Resource IDs are not passed to the initialization function called
in the Master CPU because some Resource IDs may be in general not in
the scope of the Master CPU. That situation happens when the resource
becomes local to a CPU other than the Master CPU\footnote{That is, all
the tasks using the Resource are allocated to a CPU other than the
Master CPU.}.

In general, Resource IDs of Global Resources are in the scope of {\em
all} the CPUs in a multiprocessor Nios II design. Resource IDs of
Local Resources are in the scope of the particular CPU only.

