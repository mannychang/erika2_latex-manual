\chapter{Standalone version of \rtd}
\label{cha:Commandline}

\section{Introduction}

This chapter describes the standalone version of \rtd. The idea behind
of the standalone version is to provide the \rtd\ code generator
plugin packed {\em without} the Eclipse Framework, to have a simple and
fast way to generate code and templates from the command line.

The standalone version of \rtd\ is stored inside the \file{bin}
directory inside the \rtd\ installation directory.

\section{Code generation}
To generate code using the standalone version of \rtd, please run the
following command:

\begin{lstlisting}
rtdruid_launcher.bat --oil filename --output dir
\end{lstlisting}

\noindent Where:
\begin{itemize}
\item \const{filename} is the name of the OIL file.
\item \const{dir} is the directory where the generator should put the
  generated files. The \const{--output} option is optional. If not
  specified, the default directory name used is \const{Debug}.
\end{itemize}

\section{Template code instantiation}
This section explains how to obtain the automatic generation of a
template example, as it is done from the ``New Project'' menu item
from the Eclipse Framework.

To obtain the list of available templates, please run the following
command:

\begin{lstlisting}
template.bat --list
\end{lstlisting}

\noindent as a result, the command displays a list of the available
templates, with their IDs.

Then, to instantiate a template, please run the following command:

\begin{lstlisting}
template.bat --template ID --output dir
\end{lstlisting}

\noindent Where:
\begin{itemize}
\item \const{ID} is one of the IDs returned using the \const{--list} command..
\item \const{dir} is the directory where the generator should put the
  generated files. The \const{--output} option is optional. If not
  specified, the default directory name used is the current directory.
\end{itemize}
