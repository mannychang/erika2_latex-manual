\chapter{Script File}
\label{cha:Script-File}

\section{Introduction}

The \rtd\ Toolset provides an uniform way of running a batch
computation, that supports both non-GUI usage (useful for automatic
code generation) and an usage that takes advantage of the Eclipse GUI
%PJ: questo riferimento mi sa non esiste + 
%described in Chapter \ref{cha:RTDruid-GUI}. 
. To do that, the \rtd\ Toolset enhanced the support for the Apache Ant
build tool. Apache Ant has been chosen because:

\begin{itemize}
\item it allows useful scripting with an expressive power similar to
  conventional makefiles;
\item it is expandible with customized features;
\item it supports seamlessy integration with the Eclipse platform,
  which has been used as base for the graphical environment of the
  \rtd\ Toolset.
\end{itemize}
It is out of the scope of this section to describe Ant in detail.  For
more informations about Apache Ant, please refer to
\url{http://ant.apache.org}; a detailed manual of Ant can be found at
\url{http://ant.apache.org/manual/intro.html}.

%
%\begin{lyxgreyedout}
%Il tool deve capire da solo se e' il caso di usare schedulabilita'
%con offset o senza?
%\end{lyxgreyedout}



\section{Quick introduction to ANT}


Ant is a Java-based build tool produced by the Apache
Foundation\footnote{Small parts of this section are derived from the
manual available at
\protect\url{http://ant.apache.org/manual/intro.html}. Please refer to
it for a complete explaination. }.  Ant build files are somehow
similar to makefiles, except that they are written as XML files.


\paragraph{Targets.}

As in makefiles, Ant can handle ``targets''. Each buildfile contains
one project and at least one (default) target. Targets contains
(provoke the execution of) tasks. Each task is run by a Java object
that implements a particular Task interface. A target can depend on
other targets.  You might have a target for compiling, for example,
and a target for creating a distributable. In that sense, ``target''s
are similar to makefile ``rules''. You can only build a distributable
when you have compiled first, so the ``distribute'' target depends on
the ``compile'' target. Ant resolves these dependencies.


\paragraph{Tasks.}

A task is a piece of code that can be executed. A task can have
multiple attributes (or arguments, if you prefer). The value of an
attribute might contain references to a property. These references
will be resolved before the task is executed. A task is more or less a
method of a Java class inside Ant that run whenever it has to be
executed. We can think as a shell command inside a makefile rule is
like the invocation of an Ant Task whose behavior is the same as the
command. The \rtd\ Toolset expanded the Ant tasks providing a set of
new tasks for doing, for example, schedulability analysis and code
generation.


\paragraph{Types.}

A Type is used whenever the user needs to provide informations to a
task, or have to represent list of items, files, and so on. For
example, the execution of the \const{delete} task (for example inside a
\const{clean} target) needs the list of files to be deleted, that is
passed using a type.


\paragraph{Properties.}

A project can have a set of properties. These might be set in the
buildfile by the property task, or might be set outside Ant. A
property has a name and a value; the name is
case-sensitive. Properties may be used in the value of task
attributes. This is done by placing the property name between \const{\$\{}
and \const{\}} in the attribute value. For example, if there is a
\const{builddir} property with the value \const{build}, then this could be
used in an attribute like this: \const{\$\{builddir\}/classes}. This is
resolved at run-time as build/classes. Properties are very similar to
makefile variables.


\paragraph{Case sensitivity.}

In general, the following rules apply for case sensitivity:

%\begin{lyxgreyedout}
%Penso che tutto il file sia case sensitive, fatta eccezione dei file
%sotto windows
%\end{lyxgreyedout}


\begin{itemize}
\item XML element tags in the script files are case sensitive;
\item attribute values surrounded by brackets are case sensitive;
\item if attribute values refer to file names, the rules of the host
  OS applies (e.g. Windows is case insensitive, whereas Linux is case
  sensitive);
\item case sensitive words are written in the manual using a
  \texttt{typewriter} font;
\end{itemize}




\section{Launching Ant}

Ant can be launched from the command line.
%, using the \file{rtdruid.sh}
%script provided inside the \file{bin} directory of the documentation
%package. 
For example, to execute the script \file{build.xml} with a
parameter \const{all} as target (the same as doing \const{make all}
with makefiles), you should execute the following command (this command works on Windows; we used ``\textbackslash" to split the lines which are too long):

%\begin{lstlisting}
%rtdruid.sh build.xml all
%\end{lstlisting}

%For completeness, the rtdruid.sh script simply executes a command
%like the following:

\begin{lstlisting}
set CLASSPATH="C:\Programmi\j2sdk1.4.2_05\lib\tools.jar; \
    C:\Programmi\Evidence\eclipse\startup.jar"
java org.eclipse.core.launcher.Main                      \
     -application org.eclipse.ant.core.antRunner all     \
     -buildfilebuild.xml
\end{lstlisting}

%java~-cp~\$ECLIPSE\_HOME/startup.jar~org.eclipse.core.launcher.Main~
%~~~~~-application~org.eclipse.ant.core.antRunner~
%~~~~~-buildfile~build.xml~mytarget~

%\begin{lyxgreyedout}
%l'opzione buildfile e' necesaria soltanto se lo script non e' build.xml,
%build.xml e' il file di default
%\end{lyxgreyedout}

that is, the \const{antRunner} plugin of Eclipse is started, asking to
execute the \file{build.xml} script file running the \const{all}
target. If the \const{-buildfile} option is not specified, the default
build file is \file{build.xml}.
% Please refer to section
%\ref{sec:launching_ant_from_Eclipse} for more informations about how
%to run the Ant from inside the Eclipse platform GUI.


\section{An Ant build script file format example}

An Ant build script is currently composed by an XML file that (more or
less) sequentially lists the commands run by the Ant. This section
does not describe in detail the Ant syntax. For details about the
syntax of an Ant build script, please refer to
\url{http://ant.apache.org/manual/intro.html}.

The following lines show an Ant script that uses some of the \rtd
Toolset features.

\begin{lstlisting}
<?xml version="1.0" encoding="UTF-8"?>
<project name="rtdruid" default="all" basedir=".">

  <target name="all">
    <rtdruid.Oil.Configurator 
      inputfile="conf.oil" 
      outputdir="Debug"
    />
  </target>
</project>
\end{lstlisting}

A project named \const{rtdruid} is declared, with one target named \const{all}.

Target \const{all} calls a \const{rtdruid.Oil.Configurator} task,
which performs the code generation for a given OIL file. In
particular, the target \const{all} reads the file \file{conf.oil}
stored in the current directory, and write the outputs inside the
\file{Debug} directory.




\section{Task ``rtdruid.Oil.Configurator''}
\label{sec:rtdruid.oil.configurator}

This task can be used to perform the code generation from an OIL
file. The task accepts two parameters, which are \const{inputfile} (an
input OIL file) and \const{outputdir} (the output directory where the
generated files should be stored). If the output directory is created
if it does not alreday exist.

The typical usage of this ANT task is to exexute it inside the
application working directory, generating the configuration files in a
subdirectory. Then, the user can run the makefile which has been
generated to compile the application.

An example of an \const{rtdruid.Oil.Configurator} task is the following:

\begin{lstlisting}
...
    <rtdruid.Oil.Configurator 
      inputfile="conf.oil" 
      outputdir="Debug"
    />
...
\end{lstlisting}







%% questo e' il testo della versione presente nel manuale di \rtd originale
%% \section{An Ant build script file format example}

%% questo era l'esempio alla fine della sezione

%% \begin{lstlisting}
%% <?xml version="1.0" encoding="UTF-8"?>
%% <project name="rtdruid" default="all" basedir=".">
%%   <target name="all">
%%     <antcall target="t001"/>
%%     <antcall target="t002"/>
%%   </target>

%%   <target name="t001">
%%     <rtdruid.schedulability store="ris1.rtd">
%%        <filelist dir="./001" files="data1.rtd data2.rtd"/>
%%        <filelist dir="./003" files="dataA.rtd dataB.rtd"/>
%%     </rtdruid.schedulability>
%%   </target>

%%   <target name="t002">
%%     <rtdruid.schedulability store="ris2.ertd">
%%        <filelist dir="./002" files="data.rtd"/>
%%     </rtdruid.schedulability>
%%   </target>
%% </project> 
%% \end{lstlisting}

%% A project named \const{rtdruid} is declared, with three targets:
%% \texttt{all}, \texttt{t001}, and \texttt{t002}. 

%% The \texttt{all} target simply executes two \texttt{antcall}s. In
%% practice, the executon of \texttt{all} simply executes target \texttt{t001},
%% and then \texttt{t002}. Target \texttt{t001} (\texttt{t002} does it
%% in a similar way) calls a \texttt{rtdruid.schedulability} task, that,
%% as you can see in Section \ref{sec:rtdruid.schedulability}, performs
%% the schedulability analysis without considering offsets. In particular,
%% the target \texttt{t001} reads the files \texttt{data1.rtd} and \texttt{dataA.rtd},
%% stored in the respective directories, and write the outputs inside
%% \texttt{ris1.rtd}.


%% \section{Common task structure\label{sec:common_task_structure}}

%% Currently, tasks require the specification of a set of input files
%% and of an output file. Input files are loaded all together to form
%% the system representation (each file can contain a different part
%% of the system). The output file is used to save the result of the
%% analysis.

%% Moreover, each schedulability analysis task allow the possibility
%% to select the \emph{application mode} which the test is done with.

%% The template DTD structure for a Task is the following:

%% \begin{lyxcode}
%% <!ELEMENT~rtdruid.xxx~(filelist){*}>

%% <!ATTLIST~rtdruid.xxx

%% ~~mode~~~~~~~~CDATA~\#IMPLIED

%% ~~priorities~~CDATA~\#IMPLIED

%% ~~store~~~~~~~CDATA~\#IMPLIED

%% >
%% \end{lyxcode}
%% In particular:

%% \begin{description}
%% \item [{mode}] represents the application mode that have to be considered
%% during the analysis. The application mode, for example, influences
%% the mapping phase and the execution times. If not present, the default
%% mode is \emph{considered}.
%% \item [{priorities}] the priority attribute can have the following values:

%% \begin{description}
%% \item [{user}] proprities are specified by the user (that is the default
%% value);
%% \item [{byDeadline}] priorities are assigned to tasks inversely proportional
%% to their deadlines;
%% \item [{byPeriod}] priorities are assigned to tasks inversely proportional
%% to their periods;
%% \end{description}
%% \item [{store}] specifies the output file, where all the results need to
%% be saved. Currently all the data structure obtained by the union of
%% the input files plus the schedulability results are saved in the same
%% output file.
%% \item [{filelist}] allows the specification of the list of input files.
%% That is one of the predefined Ant type, and it is mainly composed
%% by the dir and file attributes (both mandatory). The \texttt{dir}
%% attribute represents the directory that have to be considered as prefix
%% for the file list listed in the \texttt{file} attribute. Files in
%% the \texttt{file} attribute are separated using {}``,'' and spaces%
%% \footnote{File names that contain spaces should not be specified.%
%% }.
%% \end{description}
%% The \RTDruid Toolset does not support the Ant feature that allows
%% task to be executed concurrently. In general, Tasks can execute concurrently
%% only if they belong to different Java Virtual Machines.

%% %
%% \begin{lyxgreyedout}
%% Il prossimo futuro

%% blocchi di comandi, nel quale � presente una sola struttura dati condivisa
%% tra tutti i comandi. Ovviamente nell'uscire da un blocco, tutte le
%% informazioni in esso presenti e/o calcolate vengono perse. I comandi
%% disponibili (quelli che pensavo di implementare) sono:

%% \begin{enumerate}
%% \item indicare uno o + file da caricare, eventualmente taggandoli con un
%% id in modo da caricarli una volta ed utilizzarli eventualmente + volte,
%% visto che il caricamento con trasformazione richiede abbastanza tempo.
%% Dopo il primo caricamento, i dati vengono anche inseriti nella struttura
%% corrente, faccendo un merge con i dati preesistenti (si potrebbe mettere
%% un flag che abilita o meno la sovvrascrittura di valori settati in
%% modo diverso).

%% \begin{enumerate}
%% \item pulire l'albero corrente.
%% \item ri-inserire nell'albero corrente file gia' caricati (una sorta di
%% {}``re-load ID'').
%% \item effettuare dei test di schedulabilita' coi dati presenti nell'albero
%% corrente (comporta una successiva modifica dell'albero).
%% \item effettuare dei salvataggi dell'albero corrente, eventualmente indicando
%% quali sezioni salvare. Nel caso si renda disponibile la scelta di
%% sotto rami, si potrebbe accettare che l'utente specifichi tutti i
%% sottorami che vuole salvare (i caratteri jolly potrebbero essere comodi
%% ma hanno un impatto da non sottovalutare ).
%% \item Stampare valori dell'albero o fare dei test sui loro valori .... magari
%% + avanti nel tempo, anche perche' altrimenti diventa un secondo linguaggio
%% interno al linguaggio di ant.
%% \end{enumerate}
%% \end{enumerate}
%% NOTA 2:

%% In ogni caso, almeno per adesso non e' possibile eseguire piu' test/blocchi/script
%% in parallelo, in quanto l'albero corrente e' memorizzato in una variabile
%% statica, quindi comune a tutto l'ambiente. Quest'ulteriore residuo
%% del vecchio modello andra' eliminato ma non in questa versione.

%% Volendo esaminare alcuni casi :

%% \begin{enumerate}
%% \item l'uso del costrutto di ant che permette la parallizzazione delle esecuzioni,
%% che comprende task di rtdruid, e' ovviamente da evitare
%% \item in assenza del costrutto indicato sopra, le esecuzioni da riga di
%% comando sono sempre {}``isolate'', in quanto per ciascuna di queste
%% viene caricato un ambiente distinto, dunque possono essere eseguiti
%% in parallelo
%% \item poiche' per mandare in esecuzione uno script ant dall'interfaccia
%% grafica e' necessario utilizzare lo stesso ambiente di eclipse, non
%% e' possibile mandare eseguire piu' test in parallelo (compresi i test
%% lanciati direttamente da menu)
%% \end{enumerate}

%% \end{lyxgreyedout}



%% \section{Task {}``rtdruid.Schedulability''\label{sec:rtdruid.schedulability}}

%% This task can be used to perform the schedulability test without considering
%% task offsets. The system is composed by the composition of different
%% input files; the result can then be saved on an output file, following
%% the specification described in Section \ref{sec:common_task_structure}.

%% An example of an \texttt{rtdruid.Schedulability} task is the following:

%% \begin{lyxcode}
%% ...

%% ~~~<rtdruid.schedulability~store=\char`\"{}results.rtd\char`\"{}>

%% ~~~~~~<!-{}-~NB:~input1.rtd~and~input2.ertd~are~two~different~files!~-{}->

%% ~~~~~~<filelist~dir=\char`\"{}./first\char`\"{}~~files=\char`\"{}input1.rtd~input2.ertd\char`\"{}/>~

%% ~~~~~~<filelist~dir=\char`\"{}./second\char`\"{}~files=\char`\"{}other\_data.rtd\char`\"{}/>

%% ~~~</rtdruid.schedulability>

%% ...


%% \end{lyxcode}

%% \section{Task {}``rtdruid.OffsetSchedulability.Exact''}

%% This task can be used to perform the schedulability test considering
%% task offsets (and an exaustive search). The system is composed by
%% the composition of different input files; the result can then be saved
%% on an output file, following the specification described in Section
%% \ref{sec:common_task_structure}.

%% In this case, an \emph{exaustive} \emph{search} is performed to find
%% the best offset that makes the system schedulable (that is different
%% to what happens with Task rtdruid.OffsetSchedulability.Sufficient,
%% see Section \ref{sec:offset-sufficient-analysis}). 

%% Warning: That search can take a long time to be computed!

%% An example of an \texttt{rtdruid.OffsetSchedulability.Exact} task
%% is the following:

%% \begin{lyxcode}
%% ...

%% ~~~<rtdruid.OffsetSchedulability.Exact~priority=\char`\"{}byDeadline\char`\"{}

%% ~~~~store=\char`\"{}../results.rtd\char`\"{}>

%% ~~~~~~<filelist~dir=\char`\"{}.\char`\"{}~files=\char`\"{}some\_data.rtd~myDir/other\_data.rtd\char`\"{}/>

%% ~~~</rtdruid.OffsetSchedulability.Exact>

%% ...


%% \end{lyxcode}

%% \section{Task {}``rtdruid.OffsetSchedulability.Sufficient''\label{sec:offset-sufficient-analysis}}

%% This task can be used to perform the schedulability test considering
%% task offsets (and a partial search). The system is composed by the
%% composition of different input files; the result can then be saved
%% on an output file, following the specification described in Section
%% \ref{sec:common_task_structure}.

%% In this case, only a partial search is done to find a (hopefully)
%% good offset assignment. Only a partial search is performed, and only
%% a limited number of tasks are examined together. This task works only
%% with FP priority assignments.

%% %
%% \begin{lyxgreyedout}
%% Per ora solo FP, in futuro anche EDF
%% \end{lyxgreyedout}


%% For this reason, this task adds to the template showed in Section
%% \ref{sec:common_task_structure} the following two attributes:

%% \begin{description}
%% \item [{F}] number of tasks that have to be considered together (F stands
%% for Fixed tasks). Values less than 2 and greater than the number of
%% tasks in the system are non-sense. This parameter must have a value
%% between 2 and the number of tasks that are mapped on the RTOS%
%% \footnote{If more than one RTOS is present, the F value is the same for everyone.
%% That means that F should be less then the minimum number of tasks
%% present on all the RTOS (excluding of course RTOSes with no tasks).%
%% } . The default value is F equal to 2.
%% \item [{%
%% \begin{lyxgreyedout}
%% Per ora il {}``tipo e' disabilitato''\\


%% type This is the kind of analysis that have to be used. This attribute
%% is case insensitive and it can have only the following two values
%% \texttt{EDF} and \texttt{RT}; \texttt{EDF} is the default value..
%% \end{lyxgreyedout}
%% }]~
%% \end{description}
%% An example of an \texttt{rtdruid.OffsetSchedulability.Sufficient}
%% task is the following:

%% \begin{lyxcode}
%% ...

%% ~~~<rtdruid.OffsetSchedulability.Sufficient~priority=\char`\"{}byDeadline\char`\"{}~

%% ~store=\char`\"{}results.rtd\char`\"{}~F=\char`\"{}3\char`\"{}>

%% ~~~~~~<filelist~dir=\char`\"{}.\char`\"{}~files=\char`\"{}data.rtd\char`\"{}/>

%% ~~~</rtdruid.OffsetSchedulability.Sufficient>

%% ...


%% \end{lyxcode}

%% \section{%
%% \begin{lyxgreyedout}
%% Non e' presente in questa versione\protect \\


%% Task {}``rtdruid.OffsetSchedulability.Assignment''

%% This task can be used to assign offset to tasks, without performing
%% a schedulability analysis. The system is composed by the composition
%% of different input files; the result can then be saved on an output
%% file, following the specification described in Section \ref{sec:common_task_structure}.

%% An example of an \texttt{rtdruid.OffsetSchedulability.Assignment}
%% task is the following:

%% \begin{lyxcode}
%% ...

%% ~~~<rtdruid.OffsetSchedulability.Assignment~store=\char`\"{}withOffset.rtd\char`\"{}>

%% ~~~~~~<filelist~dir=\char`\"{}.\char`\"{}~files=\char`\"{}withoutOffset.rtd\char`\"{}/>

%% ~~~</rtdruid.OffsetSchedulability.Assignment>

%% ...
%% \end{lyxcode}

%% \end{lyxgreyedout}
%% }


%% \section{Task {}``rtdruid.Convert''}

%% This task can be used to convert a file between the two formats (RTD
%% and ERTD) supported by the \RTDruid Toolset. Moreover, it can be
%% used to perform a composition of a set of files already existing,
%% since its input file, as in the task template in Section \ref{sec:common_task_structure},
%% is a list of files. The mode and priorities attributes are not present.

%% An example of an \texttt{rtdruid.Convert} task is the following:

%% \begin{lyxcode}
%% ...

%% ~~~<rtdruid.Convert~store=\char`\"{}withOffset.rtd\char`\"{}>

%% ~~~~~~<filelist~dir=\char`\"{}.\char`\"{}~files=\char`\"{}data.rtd\char`\"{}/>

%% ~~~~~~<filelist~dir=\char`\"{}./001\char`\"{}~files=\char`\"{}data1.ertd\char`\"{}/>

%% ~~~~~~<filelist~dir=\char`\"{}./002\char`\"{}~files=\char`\"{}data1.rtd\char`\"{}/>

%% ~~~</rtdruid.Convert>

%% ...
%% \end{lyxcode}
