
\chapter{Introduction}
\label{cha:intro}

This document provides the user with a basic understanding of the
architecture, the features and the operations of the \rtd\ Code
generator tool and the associated
%OSEK
\ee\ Kernel.

The Code generator tool is part of the \rtd\ design framework for
architecture level modeling. The \rtd\ toolset consists of a {\em
core} component, required for all operations, and a number of plugins
providing time verification and automatic generation of the
implementation of real-time embedded software.

The modular structure of RT-Druid is now fully integrated with the
open Eclipse framework. The Eclipse environment can easily be extended
with third party components and plug-ins, further improving the
configurability and extensibility of \rtd{}.


%The Code generator component is currently available for OSEK/VDX real-time
%OS implementations only.

This user guide document covers the code generation plugin. It
consists of an overview, explaining the architectural concepts, the
standards and the inputs and outputs of the code generator tool. In
the first section: Overview of \rtd\ Code Generator and \ee, the tool
and the
%OSEK
real-time \ee\ OS are introduced, the code generation process is
outlined and the relationships among the products and the Eclipse
development environment are explained. 
%The second section Overview of
%OSEK provides a basic understanding of the OSEK standard for
%automotive applications and its real-time features.

The second part contains the basic information for operating with the
tool and providing the right configuration input for the following
code generation phase. Chapter \ref{cha:creating-rtdruid-project}
explains the basic steps that are necessary to start an Rt-Druid
project and how to define the basic configuration info that is
required by the tool. A fundamental part of the configuration tool is
contained in the OIL input file. Syntax and methods for
generating the OIL description of the system are the subject of
Chapter \ref{cha:oil-syntax}.

The \ee\ specific extensions to the OIL language that are necessary to
define task placement and other features of multiprocessor systems are
described in Section \ref{sec:oil-multicore}.

The operations that are required for the code generation phase,
together with a detailed description of the input and output data at
each step is the subject of the Chapter \ref{cha:code-generation}.
The kernel configuration and the explanation of the programming model
that needs to be used for \rtd/\ee\ applications are also described in
Chapter \ref{cha:code-generation}.

