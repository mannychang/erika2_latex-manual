\chapter{OIL syntax and OIL file generation}
\label{cha:oil-syntax}

OIL (OSEK Implementation Language) is a part of the OSEK/VDX standard,
that is used for OS and application configuration. The specification
of the OIL file structure and syntax is provided in the OSEK/VDX web
site at \url{http://www.osek-vdx.org} \cite{OSEKOIL}.

In the \rtd\ and in the \ee\ RTOS the configuration of the system is
defined inside an OIL file. In this chapter we only provide a quick
introduction of the OIL Language (see \cite{OSEKOIL} for a complete
description), together with a specification of the specific OIL
attributes implemented by \rtd.

Standard OIL has no knowledge of multiprocessor systems, nor of 
distribution of threads and resources. \ee\ provides mechanisms for 
resource sharing with predictable blocking time in a distributed 
environment. We defined a set of OIL extensions (see Section 
\ref{sec:oil-multicore}) which explicitly deals with the additional 
syntax features that are needed for the definition of a multiprocessor 
system, including placement of threads and resources. 

\section{OIL Basics}

In \ee\ all the RTOS objects like tasks, alarms, resources, are static
and predefined at application compile time. To specify which objects
exist in a particular application, \ee\ uses the OIL Language, which
is a simple text description language.

Here is an example of the OIL File for the \dspic\ device:

\begin{lstlisting}
CPU mySystem {
        OS myOs {
                EE_OPT = "DEBUG";
                CPU_DATA = PIC30 {
                        APP_SRC = "code1.c";
                        APP_SRC = "code2.c";
                        MULTI_STACK = FALSE;
                        ICD2 = TRUE;
                };

                MCU_DATA = PIC30 {
                        MODEL = 33FFJ256GP710;
                };

                BOARD_DATA = EE_FLEX {
                        USELEDS = TRUE;
                }

                KERNEL_TYPE = FP;
        };

        TASK myTask {
                PRIORITY = 1;
                STACK = SHARED;
                SCHEDULE = FULL;
        };

        TASK myTask {
                PRIORITY = 1;
                STACK = SHARED;
                SCHEDULE = FULL;
        };
};
\end{lstlisting}

The example contains a single object called \const{CPU}, which
contains all the specifications and the values used by the system. Inside
the \const{CPU}, are described the objects which are present in
the application: an \const{OS}, which specifies the global attributes
for the system, and, in the example, two \const{TASK}s.

The OIL File is parsed by the \rtd\ code generator and, as a result,
part of the RTOS source code is generated and compiled together with
the application.

An OIL file consists of two parts: a set of definitions and a set of
declarations. Definitions are used to define data types, constants and
kernel objects that need to be provided in the declaration part for
configuring a specific kernel. In other words, the definition part
tells the configurator that there exists different objects like
tasks, resources, and so on, describing their attributes and types,
like in a C struct declaration. Then, the declaration part is used to
specify which objects are really present in a particular application.

In \rtd, the definition part of the OIL file is fixed and is
contained inside the \rtd\ Eclipse Plugins. The definition part
including all the attributes which can be specified by users is
included in Appendix \ref{cha:oil-definition}. The user has only to
provide the declaration part, specifying for a particular application
the objects to be created.

The OIL file basically contains the description of a set of objects. A
\const{CPU} is a container of these objects. Other objects include the following:
\begin{itemize}
\item \const{OS} is the Operating System which runs on the CPU. This
  object contains all the global settings which influence the
  compilation process and the customization of the \ee\ RTOS.
\item \const{APPMODE} defines the different application mode.
  These modes are then used to control the autostart
  feature for tasks and alarms in the OIL file.
\item \const{TASK} is an application task handled by the OS.
\item \const{RESOURCE} is a resource (basically a binary mutex) used
  for mutual exclusion.
\item \const{EVENT} is a synchronization flag used by extended tasks.
\item \const{COUNTER} is a software source for periodic / one shot
  alarms.
\item \const{ALARM} is a notification mechanism attached to a counter
  which can be used to activate a task, set an event, or call a
  function.
\end{itemize}

All the attributes in the OIL file can be:

\begin{itemize}
\item numbers, i.e. the \const{PRIORITY} attribute;
\item strings, i.e. the \const{APP_SRC} attribute;
\item enumerations, i.e. the \const{KERNEL_TYPE} attribute.
\end{itemize}

Attributes can have a default value, as well as an {\em automatic}
value specified with the keyword \const{AUTO}. Some of the attributes
can be specified more than once in the OIL file, such as the
\const{APP_SRC}, and the configurator treats them as a {\em set} of
values; i.e., in the case of \const{APP_SRC}, the set of application
files to be compiled.

Finally, some items can in reality contain a set of sub-attributes,
like in a C-language struct definition. For example, \const{CPU_DATA}
contains a \const{PIC30} object, which is detailed by a set of
attributes.

\section{The CPU Object}

The CPU object is only used as a container of all the other objects,
and does not have any specific attribute.

\section{The OS Object}

The \oil{OS} Object is used to define the \ee\ global configuration as
well as the compilation parameters.

The attributes which can be specified for the \const{OS} object are 
specified in the following subsections.

\subsection{Compilation attributes}

The OIL file includes a set of fields for controlling the command line
parameters which are passed to the compiler tools. The meaning of those
elements is the following:

\begin{itemize}
\item \oil{EE_OPT} contains a list of additional compilation flags
  passed to the \ee\ makefile. In practice, the \const{EE_OPT}
  makefile variable controls which files has to be compiled and with
  which options. The \const{EE_OPT} attributes are translated in
  \const{#define}s in the C code.
\item \oil{CFLAGS} contains the list of additional C compiler options.
\item \oil{ASFLAGS} contains the list of additional assembly options.
\item \oil{LDFLAGS} Contains the list of additional linker parameters.
\item \oil{LDDEPS} Contains the list of additional library
  dependencies which have to be added to the makefile rules.
\item \oil{LIBS} Contains the list of additional libraries that needs
  to be linked.
\end{itemize}

Example of declaration:

\begin{lstlisting}
CPU mySystem {
  OS myOs {
    EEOPT = "MYFLAG1";
    EEOPT = "MYFLAG2";
    CFLAGS = "-G0";
    CFLAGS = "-O0 -g";
    CFLAGS = "-Wall -Wl,-Map -Wl,project.map";
    ASFLAGS = "-g";
    LIBS = "-lm";
    ...
  };
  ...
}
\end{lstlisting}

\subsection{OSEK attributes}
The OIL file includes a set of attributes which are part of the
OSEK/VDX specification. The meaning of those attributes is the
following:

\begin{itemize}
\item \oil{STATUS} specifies if the kernel should be compiled with
  \const{STANDARD} status or \const{EXTENDED} status. With the
  \const{STANDARD} status, only a subset of the error codes are 
  reported by the kernel primitives to reduce the system footprint. 
  This setting only applies to the OSEK/VDX conformance classes.
\item The settings \oil{STARTUPHOOK}, \oil{ERRORHOOK},
  \oil{SHUTDOWNHOOK}, \oil{PRETASKHOOK}, \oil{POSTTASKHOOK} specifies
  which particular hook routine should be included in the kernel.
\item \oil{USEGETSERVICEID} specifies if the Service ID debugging
  functionality of the \fn{ErrorHook()} routine should be included in
  the kernel.
\item \oil{USEPARAMETERACCESS} specifies if the \fn{ErrorHook()}
  should have access to the parameters passed to the primitives.
\item \oil{USERESSCHEDULER} specifies if the kernel includes the
  \const{RES_SCHEDULER} resource.
\end{itemize}

Example of declaration:

\begin{lstlisting}
CPU mySystem {
  OS myOs {
    STATUS = STANDARD;
    STARTUPHOOK = TRUE;
    ERRORHOOK = TRUE;
    SHUTDOWNHOOK = TRUE;
    PRETASKHOOK = FALSE;
    POSTTASKHOOK = FALSE;
    USEGETSERVICEID = FALSE;
    USEPARAMETERACCESS = FALSE;
    USERESSCHEDULER = TRUE;
    ...
  };
  ...
}
\end{lstlisting}

\subsection{Multi-core attributes}
The attributes \oil{STARTUPSYNC}, and \oil{USEREMOTETASK} are
described in the \ee\ Manual for the Altera Nios II target, since they
are specific for that architecture.

\subsection{Nios II target attributes}
The attributes \oil{NIOS2_MUTEX_BASE}, \oil{NIOS2_SYS_CONFIG},
\oil{NIOS2_APP_CONFIG}, \oil{IPIC_GLOBAL_NAME}, \oil{IPIC_LOCAL_NAME},
\oil{MP_SHARED_RAM}, \oil{MP_SHARED_ROM}, \oil{NIOS2_DO_MAKE_OBJDUMP},
\oil{SYSTEM_LIBRARY_NAME}, \oil{SYSTEM_LIBRARY_PATH}, \oil{NIOS2_PTF_FILE}, 
are described in the \ee\ Manual for the Altera Nios II target.

\subsection{CPU\_DATA sections}

The \oil{CPU_DATA} section of the \const{OS} object is used to specify
the configuration of a core in a single or in a multiple core device.

In general, the OIL file will contain a \const{CPU_DATA} section for
each core in the system. There is a specific \const{CPU_DATA} section
for each architecture supported by \ee. 

In particular, the \const{CPU_DATA} sections currently supported are
\oil{NIOSII}, \oil{PIC30} and \const{AVR_5}, which contain the following
attributes:

\begin{itemize}
\item \oil{ID} is a symbolic name uniquely identifying the
  CPU. The name used for the \const{CPU_ID} attribute must be the same
  name that is used when allocating objects to a particular CPU.

  CPUs with no name automatically get a default name
  \const{default_cpu}\index{default\_cpu}. If more than one CPU gets
  \const{default_cpu}, an error is raised, because different CPUs
  cannot have the same name.

  \const{default_cpu} is subsumed also when allocating Tasks (see
  Section \ref{sub:task-mapping-to-cpu}), and counters (see Section
  \ref{sec:counter-extensions}) to a CPU, and when the Master CPU is
  assigned (see Section \ref{sub:master-cpu}).

  For single processor systems, it is safe to avoid any declaration of
  the \const{CPU_ID} field in the entire OIL file. In this way, all
  the objects will be mapped to the only CPU in the system, named
  \const{default_cpu}.

  Example of declaration:

\begin{lstlisting}
CPU mySystem {
  OS myOs {
    CPU_DATA = NIOSII {
      ID = "mycpu";
      ...
    }
    ...
  };
  ...
}
\end{lstlisting}

\item \oil{APP_SRC} declares a list of all files containing code to
  be executed on the CPU.

  Example of declaration:

\begin{lstlisting}
CPU mySystem {
  OS myOs {
    CPU_DATA = NIOSII {
      APP_SRC = "file1.c";
      APP_SRC = "file2.c";
      ...
    }
    ...
  };
  ...
}
\end{lstlisting}


\item \oil{MULTI_STACK} defines if the system supports multiple
  stacks for the application tasks (\const{TRUE}) or not
  (\const{FALSE}). The default value is \const{FALSE}.

  If set to \const{TRUE}, it is possible to specify if IRQs are
  executed using a dedicated stack space. The attribute
  \oil{IRQ_STACK} is used for this purpose.

  Some architectures also allow the specification of a
  \oil{DUMMY_STACK}, which specifies if the background task is using
  a shared stack (\const{SHARED} value) or a dedicated stack segment
  (\const{PRIVATE} value). \ee\ schedules the \fn{main()}
  function as a background task, also called ``dummy'' task. For
  example, the Altera Nios II architecture provides support for the
  above described mechanism, while the \dspic\ family does not support
  it.

\item \oil{STACK_TOP} contains the highest address from which the
  stack space starts. The address can be provided as a symbol,
  assuming that the symbol is associated to a value in some other part
  of the OIL declaration or in some application file. For example, in
  the Altera Nios II HW version of \ee, the typical value for this
  attribute is \const{__alt_stack_pointer}, that is the symbol used
  inside the Altera Nios II System libraries as the initial stack
  pointer.

\item \oil{SYS_SIZE} is used to declare the total size of the memory
  that is allocated to the task stacks.

\item \oil{SHARED_MIN_SYS_SIZE} used to declare the minimum size of
  the shared stack space. The dimension of the shared stack space is
  computed as the difference between the available space
  (\const{SYS_SIZE}) and the space required for implementing the private
  stack spaces. \rtd\ guarantees that the remaining size is higher
  than or equal to the value defined with the
  \const{SHARED_MIN_SYS_SIZE} directive (an error is raised
  otherwise). The default value for this attribute is zero.

\item The attributes \oil{ICD2} and \oil{ENABLE_SPLIM} are described
  in the \ee\ Manual for the Microchip PIC24, dsPIC30 (R) DSC and
  dsPIC33 (R) DSC targets.

\item The specific attributes about the \oil{AVR_5} architecture are
  described in the \ee\ Manual for the Atmel AVR5 targets.
\end{itemize}

Here is an example of a declaration of a Nios II \const{CPU_DATA}:

\begin{lstlisting}
CPU mySystem {
  OS myOs {
    CPU_DATA = NIOSII {
      ID = "cpu2";
      MULTI_STACK = TRUE {
	IRQ_STACK = FALSE;
	DUMMY_STACK = SHARED;
      };
      APP_SRC = "cpu2_startup.c";
      STACK_TOP = 0x20004000;
      SHARED_MIN_SYS_SIZE = 1800;
      SYS_SIZE = 0x1000; 
      IPIC_LOCAL_NAME = "IPIC_INPUT_CPU0";
    };
    ...
  };
  ...
}
\end{lstlisting}

The same example can be written in two stages by splitting the
declaration of the structure. The only requirement is that the
separate declarations do not contain any conflicting assignment to the
same field name. The previous example can be rewritten as follows:

\begin{lstlisting}
CPU mySystem {
  OS myOs {
    CPU_DATA = NIOSII {
      ID = "cpu2";
      MULTI_STACK = TRUE {
	IRQ_STACK = FALSE;
	DUMMY_STACK = SHARED;
      };
      APP_SRC = "cpu2_startup.c";
    };
    
    CPU_DATA = NIOSII {
      ID = "cpu2";
      STACK_TOP = 0x20004000;
      SHARED_SYS_SIZE = 1800;
      SYS_SIZE = 0x1000; 
      IPIC_LOCAL_NAME = "IPIC_INPUT_CPU0";
    };
    
    CPU_DATA = NIOSII {
      /* The ID is not defined, this section refers
         to the "default_cpu" */
      STACK_TOP = "alt_data_end";
    };
    ...
  };
  ...
}
\end{lstlisting}

%\item [ARM7]
%
%  In case of an ARM7 target, the declaration of a stack space requires
%  both \const{SYS\_SIZE} e \const{IRQ\_SIZE}, for the system and IRQ
%  stack, respectively.

\subsection{MCU\_DATA sections}

The \oil{MCU_DATA} section of the \const{OS} object is used to specify
the configuration of peripherals which are present in a specific
microcontroller.

The following microcontrollers are supported:
\begin{itemize}
\item Microchip PIC24 microcontrollers and dsPIC DSCs. Please refer to
  the \ee\ Manual for the Microchip PIC24, dsPIC30 (R) DSC and dsPIC33
  (R) DSC targets.
\end{itemize}

\subsection{BOARD\_DATA sections}

The \oil{BOARD_DATA} section of the \const{OS} object is used to
specify the configuration of the board where the microcontroller is
placed. For example, the board configuration includes the
configuration of the external devices like leds, buttons, displays,
and other peripherals.

The following boards are supported:
\begin{itemize}
\item \oil{NO_BOARD} is a placeholder to specify that no board
  configuration is required.
\item \oil{EE_FLEX} is the Evidence / Embedded Solutions \flex\ Board
  based on the Microchip \dspic. For details, please refer to the
  \ee\ Manual for the Microchip PIC24, dsPIC30 (R) DSC and dsPIC33 (R)
  DSC targets.
\item \oil{MICROCHIP_EXPLORER16} is the Microchip Explorer 16
  evaluation board. For details, please refer to the \ee\ Manual for the
  Microchip PIC24, dsPIC30 (R) DSC and dsPIC33 (R) DSC targets.
\item \oil{MICROCHIP_DSPICDEM11PLUS} is the Microchip dsPIC Demo Plus
  1.1 evaluation board. For details, please refer to the \ee\ Manual
  for the Microchip PIC24, dsPIC30 (R) DSC and dsPIC33 (R) DSC
  targets.
\item \oil{ATMEGA_STK50X} is the Atmel STK 500 evaluation board for
  the AVR5 architecture. For details, please refer to the \ee\ Manual
  for the Atmel AVR5 targets.
\item \oil{XBOW_MIB5X0} is the Crossbow MIB 5x0 board used to program
  wireless sensor network hardware. For details, please refer to the
  \ee\ Manual for the Atmel AVR5 targets.
\end{itemize}

\subsection{Library configuration}
Typical microcontroller applications needs to link external libraries
to the application code. \ee\ supports both the linking of external
binary libraries as well as the development of library code that can
be automatically built by the \ee\ build scripts.

\subsubsection{Linking an external third-party binary library}
A third-party binary library is typically provided as a binary archive
with a ``.a'' extension.

If you need to link this kind of library to your executable, just add
the following lines to the OIL file:
\begin{lstlisting}
  CPU mySystem {
    OS myOs {
      LDFLAGS = "-Llibrarypath";
      LIBS = "-llibraryname";
    };
  };
\end{lstlisting}

These lines have the effect to add the proper option to tell the
linker to load the library you specified.

\subsubsection{Building and using libraries which are integrated in the \ee\ build system}

In this case, the target is to use the library code which is provided
in the \ee\ build tree. Basically, the OIL file can specify a set of
libraries which are distributed with or are supported by \ee\ and that
have to be linked together with the application. The specification is
done by using the \oil{LIB} attribute.

An example of this kind of library are the Scicos library, and other
communication libraries which can be found under the \file{ee/contrib}
directory of the \ee\ source tree.

The list of supported libraries depends on the target and
can be found in the \ee\ Manual for the specific target. 

The \const{LIB} attribute can be used in one of the following ways:

\begin{itemize}

\item
  This first option helps to build the library files -only-. In
  particular, \const{LIB} can be used to specify an OIL file which
  only compiles the supported libraries (that is, the OIL file is used
  to configure the libraries but {\em not} the application). The
  following example is an OIL file which only compiles the library
  \file{mylib.a}:

\begin{lstlisting}
  CPU mySystem {
    OS myOs {
      EE_OPT = "__BUILD_LIBS__";
      LIB = ENABLE { NAME = "mylib"; };
      CPU_DATA = PIC30;
    };
  };
\end{lstlisting}

\begin{note}
To compile {\em all} the libraries that are supported by a particular
architecture with a single OIL file, the following OIL file
configuration can be used:

\begin{lstlisting}
  CPU mySystem {
    OS myOs {
      EE_OPT = "__BUILD_ALL_LIBS__";
      CPU_DATA = PIC30;
    };
  };
\end{lstlisting}

\end{note}


\item
  \const{LIB} can be used for the {\em on-the-fly} creation of the
  library during the application compilation process. That is, the
  build process will create the library as well as the \ee\ library
  \file{libee.a}. After that, the application code will be compiled
  and linked with all the libraries just created. The following
  example can be used to compile and link the library \file{mylib.a}.

\begin{lstlisting}
  CPU mySystem {
    OS myOs {
      EE_OPT = "__ADD_LIBS__";
      LIB = ENABLE { NAME = "mylib"; };
      ...
    };
    ...
  };
\end{lstlisting}

\item
  In this case, an application will be linked with a library which has
  been generated using a separate OIL file. The following example
  shows the OIL file which can be used to link an already existing
  library which is located in the directory
  \file{librarypath}. \file{librarypath} typically is the \file{Debug}
  directory of a project used to build a library, as explained in the
  first bulled of this list.

\begin{lstlisting}
  CPU mySystem {
    OS myOs {
      LDFLAGS = "-Llibrarypath";
      LIB = ENABLE { NAME = "mylib"; };
      ...
    };
    ...
  };
\end{lstlisting}

\item
  Finally, please note that more than one library can be specified in
  a OIL file in one of the following two ways:
\begin{lstlisting}
  CPU mySystem {
    OS myOs {
      LIB = ENABLE { NAME = "mylib1"; };
      LIB = ENABLE { NAME = "mylib2"; };
      LIB = ENABLE { NAME = "mylib3"; };
      ...
    };
    ...
  };

  CPU mySystem {
    OS myOs {
      LIB = ENABLE {
	NAME = "mylib1"; 
	NAME = "mylib2"; 
	NAME = "mylib3"; 
      };
      ...
    };
    ...
  };
\end{lstlisting}

\end{itemize}


\subsection{Kernel Conformance class}

An explicit declaration of the kernel conformance class is required in
the \oil{KERNEL_TYPE} definition. The definition is shown below:

\begin{lstlisting}
  ENUM [
    FP {
      BOOLEAN NESTED_IRQ;
    },
    EDF {
      BOOLEAN NESTED_IRQ;
      STRING TICK_TIME; 
      BOOLEAN REL_DEADLINES_IN_RAM = FALSE;
    },
    FRSH {
      ENUM [
	CONTRACT {
	  STRING NAME;
          UINT32 BUDGET;
	  UINT32 PERIOD;
	  STRING CPU_ID;
        }
      ] CONTRACTS[];
      BOOLEAN USE_SYNC_OBJ;
      STRING TICK_TIME;
    },
    BCC1,
    BCC2,
    ECC1,
    ECC2
  ] KERNEL_TYPE;
\end{lstlisting}

For the \const{EDF} and \const{FRSH} kernels, it is possible to
specify the tick length. Given the tick length for the circular timer,
which is then used to compute the values to put in the relative
deadline task parameter. The tick time can be specified in various
unit measures, such as seconds (``s''), milliseconds (``ms''),
microseconds (``us''), and nanoseconds (``ns''). Please check the
manual for the CPU architecture you are currently using for the proper
configuration of the tick parameter. The EDF kernel has an additional
option which allows to specify that relative deadlines should be
stored in RAM and not in Flash, to allow them to be changed at
runtime.

For the \const{FRSH} kernel\footnote{The FRSH kernel is the result of
the IST FP6 FRESCOR Project, \url{http://www.frescor.org}}, a set of
contracts have to be specified. For each contract, the user has to
specify the contract name, the budget and period (either in clock
ticks or using unit measures as for the tick time described above),
and the \const{CPU_ID} (which must only be specified on multicore
applications).

For more details on the \const{FRSH} kernel, please see
% *** broken sentence

Some examples of use within the declaration part are the following:

\begin{enumerate}

\item
  To configure the \const{BCC1} conformance class:
  \begin{lstlisting}
    KERNEL_TYPE = BCC1;
  \end{lstlisting}
  
\item
  To configure the \const{FP} conformance class: 
  
  \begin{lstlisting}
    KERNEL_TYPE = FP;
  \end{lstlisting}
  
  By default, nested IRQs are set to \const{FALSE}.
  
\item
  To configure the \const{EDF} conformance class:
  
  \begin{lstlisting}
    KERNEL_TYPE = EDF { 
      NESTED_IRQ = TRUE; 
      TICK_TIME = "10.5ns"; 
      REL_DEADLINES_IN_RAM = TRUE;
    };
  \end{lstlisting}
  
  Nested IRQs are set to \const{TRUE}.

\item
  To configure the \const{FRSH} conformance class:
  
  \begin{lstlisting}
    KERNEL_TYPE = FRSH {
      TICK_TIME = "20ns";
      USE_SYNC_OBJ = TRUE;
      CONTRACTS = CONTRACT {
	CPU_ID = "cpu1";
        NAME = "c1";
        BUDGET = 20000;
        PERIOD = 100000;
      };
      CONTRACTS = CONTRACT {
	CPU_ID = "cpu3";
        NAME = "c7";
        BUDGET = "10ms";
        PERIOD = "50ms";
      };
    };
  \end{lstlisting}

  In this case, the \const{FRSH} kernel is configured with two
  contracts on different CPUs. budget and period are specified either
  with a unit measure or with clock cycles.
  
\end{enumerate}



\subsection{ORTI file generation and kernel awareness with Lauterbach Trace32}
\label{sub:orti}
% ripreso in buona parte dall'event demo...

This section describes the steps to use the Lauterbach Trace32 ORTI
support in \ee. To generate the ORTI information, the
\oil{ORTI_SECTIONS} attribute has to be specified inside the
\const{OS} object. The definition of \const{ORTI_SECTIONS} is the following:

\begin{lstlisting}
  ENUM [
    NONE,
    ALL,
    OS_SECTION,
    TASK_SECTION,
    RESOURCE_SECTION,
    STACK_SECTION,
    ALARM_SECTION
  ] ORTI_SECTIONS[];
\end{lstlisting}

Basically, each ORTI section can be selected separately. If
\const{ALL} is specified, then all the ORTI sections are generated.

Notice that the ORTI support currently applies only to the Altera Nios
II target for the \const{BCC1}, \const{BCC2}, \const{ECC1},
\const{ECC2}, \const{FRSH} conformance classes.

\rtd\ provides the possibility to automatically generate an
ORTI\footnote{ORTI is a standard file format specified by the OSEK/VDX
consortium.} file.  An ORTI file is basically a text file that
specifies which data structures the kernel information are stored
in. The file is parsed by an ORTI--enabled debugger, providing useful
feedback to the application developer during debug sessions.

Also, \rtd\ automatically produces a set of scripts that can be used
to automatically launch a Lauterbach Trace32 debugger
\cite{Lauterbach}. The provided scripts automatically load the FPGA
hardware, and start a debug session for each CPU in the system.

To enable all these features, you need to specify a JAM file
name\footnote{JAM is one of the file formats containing the FPGA
configuration that is accepted by Lauterbach Trace32} inside the OS
section of the OIL file, as well as the specification of the ORTI
sections that should be generated, as follows:

\begin{lstlisting}
CPU test_application {
  OS EE {
    ...
    NIOS2_JAM_FILE = "JAM_filename.jam";
    ORTI_SECTIONS = ALL;
  }
  ...
}
\end{lstlisting}

In the above example, \const{ALL} causes the generation of all the
ORTI information, and \file{JAM_filename.jam} is the path name of 
the JAM file specified in the \oil{NIOS2_JAM_FILE} attribute. If not
specified, \file{../../fpga.jam} is used.

As a result of the compilation process, a set of files are produced
inside the \file{Debug} directory (see Table \ref{tab:t32files} for a
detailed list).

Please refer to Section \ref{sec:trace32} for information on how to
use the ORTI Files and launch a Lauterbach Trace32 session.

%
\begin{table}
\begin{center}
\begin{tabular}{|c|p{8cm}|}
\hline 
File name&
Description\tabularnewline
\hline
\hline 
\file{debug.bat}&
This batch script loads the FPGA hardware and starts a T32 instance
for each CPU. You can double click it on the Nios II IDE to directly
launch the debug session.\tabularnewline
\hline 
\file{debug\_nojam.bat}&
This batch script starts a T32 instance for each CPU. You can double
click it on the Nios II IDE to directly launch the debug session.
You can use it if the FPGA has been already programmed with the hardware
contents.\tabularnewline
\hline 
\file{t32.cmm}&
Main PRACTICE script, responsible for loading the JAM file and starting
all the T32 instances on every CPU.\tabularnewline
\hline 
\file{testcase\_data.cmm}&
Internal file used for automatic testcase generation.\tabularnewline
\hline 
\file{t32/{*}}&
Internal PRACTICE scripts. They are a copy of the files inside 
\file{components/evidence\_ee/ee/pkg/cpu/nios2} 
\file{/debug/lauterbach/t32}.\tabularnewline
\hline 
\file{cpuname/config.t32}&
Configuration file for T32. Contains the Multicore configuration 
information.\tabularnewline
\hline 
\file{cpuname/orti.men}&
Trace32 menu automatically generated using the Lauterbach ORTI menu
generator.\tabularnewline
\hline
\file{cpuname/system.orti}&
The ORTI file, for each CPU.\tabularnewline
\hline
\file{cpuname/t32.cmm}&
The main script file executed by each CPU.\tabularnewline
\hline
\end{tabular}
\end{center}

\caption{\label{tab:t32files} Files generated for the Lauterbach
Trace32 support. ({\em cpuname} represents the name of the CPU as specified 
in the OIL file).}
\end{table}


\section{The Application mode Object}
The \oil{APPMODE} object is contained by the \const{CPU} object and is
used to define an application mode.

Example:

\begin{lstlisting}
CPU test_application {
  ...
  APPMODE myAppMode1;
  APPMODE myAppMode2;
  APPMODE myAppMode3;
  ...
}
\end{lstlisting}




\section{The Task Object}

The \oil{TASK} object is contained inside the \const{CPU} object and it is used to specify the properties of a task.

\subsection{Autostart attribute}
The \oil{AUTOSTART} attribute specifies if the task should be
automatically activated at system startup by the \fn{StartOS()}
primitive.

If the task must be activated at startup, the \const{AUTOSTART}
attribute has a value \const{TRUE}. When \const{TRUE}, the
\const{APPMODE} sub-attribute lists the application modes for which
the task is autostarted.

Example:

\begin{lstlisting}
CPU test_application {
  ...
  TASK myTask1 {
    AUTOSTART = TRUE { APPMODE = myAppMode1; };
    ...
  };
  TASK myTask2 {
    AUTOSTART = FALSE;
    ...
  };
  ...
}
\end{lstlisting}

\subsection{Priority attribute}
In the FP kernel, the \oil{PRIORITY} attribute specifies the task
priority. In the EDF kernel, the value specifies the task preemption
level. The value is used by \rtd\ as a relative ordering of priorities
and not as an absolute priority value. Higher values correspond to
higher priorities.

Example:

\begin{lstlisting}
CPU test_application {
  ...
  TASK myTask1 {
    PRIORITY = 1;
    ...
  };
  ...
}
\end{lstlisting}

\subsection{Relative Deadline attribute}
The \oil{RELDLINE} attribute specifies the task relative deadline. The
value is used by \rtd\ to compute the numerical value of the timing
attribute. The value can be expressed as a timing quantity such as
seconds (``s''), milliseconds (``ms''), microseconds (``us''), or
nanoseconds (``ns''). The value is interpreted as a time, and it is
divided by the \const{TICK_TIME} attribute specified inside the OS
attribute \const{KERNEL_TYPE} to obtain the final tick value which is
then programmed inside the microcontroller.

If a number is specified without any time unit (e.g., ``1234'' and not
``1234ms''), then the number is taken ``as is'' and programmed to
the target device.

Please remember that, to be complete, the OIL
file should also include a specification of the preemption level of
the task by using the \const{PRIORITY} field.

Example:

The following example specifies the preemption level and the relative
deadline of an EDF task.

\begin{lstlisting}
CPU test_application {
  OS myOS {
    ...
    KERNEL = EDF;
  };
  ...
  TASK myTask1 {
    PRIORITY = 3;
    REL_DEADLINE = "10ms";
    ...
  };
  ...
  TASK myTask2 {
    PRIORITY = 4;
    REL_DEADLINE = "1234";
    ...
  };
  ...
}
\end{lstlisting}


\subsection{Activation attribute}
The \oil{ACTIVATION} attribute specifies the number of pending
activations which can be stored by a task. It is only used in the
\const{BCC1}, \const{BCC2}, \const{ECC1}, and \const{ECC2} conformance
classes.

Example:

\begin{lstlisting}
CPU test_application {
  ...
  TASK myTask1 {
    ACTIVATION = 3;
    ...
  };
  ...
}
\end{lstlisting}

\subsection{Schedule attribute}
The \oil{SCHEDULE} attribute specifies if a task is full preemptive
(\const{FULL}) or non preemptive (\const{NON}).  The default is \const{NON}.

Example:

\begin{lstlisting}
CPU test_application {
  ...
  TASK myTask1 {
    SCHEDULE = FULL;
    ...
  };
  TASK myTask2 {
    SCHEDULE = NON;
    ...
  };
  ...
}
\end{lstlisting}

\subsection{Event attribute}
The \oil{EVENT} attribute is used to list the Events which belong to
a task. It is used in conformance classes \const{ECC1} and
\const{ECC2}.

Example:

\begin{lstlisting}
CPU test_application {
  ...
  TASK myTask1 {
    EVENT = "TimerEvent";
    EVENT = "ButtonEvent";
    ...
  };
  ...
}
\end{lstlisting}

\subsection{Resource attribute}
The \oiltwo{RESOURCE}{RESOURCEtask} attribute is used to list the
Resources used by a task.

Example:

\begin{lstlisting}
CPU test_application {
  ...
  TASK myTask1 {
    RESOURCE = "Resource1";
    RESOURCE = "Resource2";
    ...
  };
  ...
}
\end{lstlisting}


\subsection{Contract attribute}
The \oiltwo{CONTRACT}{CONTRACTtask} attribute is used to specify the \const{CONTRACT} statically linked to the task. The \const{CONTRACT} name must be one of the contracts listed in the \const{FRSH} \refoillst{KERNEL_TYPE}! section.

Example:

\begin{lstlisting}
CPU test_application {
  ...
  TASK myTask1 {
    CONTRACT = "Contract1";
    ...
  };
  ...
}
\end{lstlisting}

\subsection{Stack attribute}
The \oil{STACK} attribute is used to specify if the task stack is shared or of the task should have a separate private stack.

Example:

\begin{lstlisting}
CPU test_application {
  ...
  TASK myTask1 {
    STACK = SHARED;
    ...
  };
  TASK myTask2 {
    STACK = PRIVATE {
      SYS_SIZE = 128;
    };
  };
  ...
}
\end{lstlisting}

\subsection{Mapping tasks to CPUs using the CPU ID attribute}
\label{sub:task-mapping-to-cpu}

The \oiltwo{CPU_ID}{CPU_IDtask} attribute is used to specify the CPU
to which the task is allocated. The placement of the tasks on the CPUs
is defined before the compile time and can not be changed during the
system execution time. If the CPU identifier is missing, then \rtd\
assumes the default value ``default\_cpu''. An error is generated if
the CPU identifier specified for the task does not exist in the
system.

Example:

\begin{lstlisting}
CPU test_application {
  ...
  TASK myTask1 {
    CPU_ID = "cpu1";
    ...
  };
  ...
}
\end{lstlisting}



\subsection{Source files for each CPU}
The source files implementing the tasks can be declared inside the OIL
file in a dedicated section named \oiltwo{APP_SRC}{APP_SRCtask}. This allows identification of the
required files when producing the executable code for each CPU. The
\file{makefile} is automatically generated based on this declaration,
so that only the files implementing the task executing on the CPU need
to be compiled. If the task is moved to another CPU, the makefile is
automatically updated.

Example of declaration:

\begin{lstlisting}
CPU mySystem {
  TASK myTask {
    APP_SRC = "file1.c";
    APP_SRC = "file2.c";
    ...
  }
  ...
}
\end{lstlisting}

\subsection{Intertask notifications}
\label{sec:intertask-notifications}
The attribute \oil{LINKED} is related to intertask notifications on
multicore architectures and is described in the \ee\ Manual for the
Altera Nios II target.





\section{The Resource Object}

The \oil{RESOURCE} object is contained inside the \const{CPU} object
and it is used to specify the properties of a resource.

The Resource object contains an attribute named \oil{RESOURCEPROPERTY}
which can take the following values: 
\begin{itemize}
\item \const{STANDARD} is the default used for a normal resource. In
  that case, a set of source files can be specified using the
  \const{APP_SRC} sub-attribute. These files typically contain the
  resource data definition.
\item \const{LINKED} resources are only links/alias for other
  resources.
\item \const{INTERNAL} resources are currently not implemented.
%used for stack sharing techniques
%, only \const{STANDARD} type resources require the
%declaration of the source files implementing them. Please note that on
%a multiprocessor system \const{INTERNAL} resources can only be
%local. Global Internal resources are forbidden.
\end{itemize}

Example:

\begin{lstlisting}
CPU mySystem {
  RESOURCE mutex {
    RESOURCEPROPERTY = STANDARD {
      APP_SRC = "shareddata.c";
    };
  };
  ...
};
\end{lstlisting}

\section{The Event Object}
The \oiltwo{EVENT}{EVENTobj} object is used to define a bit mask which
then can be used by extended tasks. Events with the same name are
identical, and have the same mask. Events with the same mask are not
identical. If the value \const{AUTO} is specified for a mask, then
\rtd\ automatically computes the mask value.

Example:

\begin{lstlisting}
CPU mySystem {
  EVENT myEvent1 {
    MASK = 0x01;
  };
  EVENT mtEvent2 {
    MASK = AUTO;
  };
  ...
};
\end{lstlisting}



\section{The Counter object}
\label{sec:counter-extensions}

The \oil{COUNTER} object is the timing reference that is used by alarms.

The attributes of a counter are the following:
\begin{itemize}
\item \const{CPU_ID} is an indication
  of the CPU on which the counter is mapped. The default value is
  \const{default\_cpu}. If the identifier of the CPU does not exist in
  the system, an error is generated.
\item \const{MINCYCLE} is currently ignored by \ee.
\item \const{MAXALLOWEDVALUE} is currently ignored by \ee.
\item \const{TICKSPERBASE} is currently ignored by \ee.
%
% toOl: che significa che vengono ignorati?!?
%
\end{itemize}

Example:

\begin{lstlisting}
CPU mySystem {
  COUNTER myTimer {
    MINCYCLE = 32;
    MAXALLOWEDVALUE = 127;
    TICKSPERBASE = 23;
  };
  ...
};
\end{lstlisting}

\section{The Alarm Object}

The \oil{ALARM} Object is used to implement an asynchronous
notification which can activate a task, set an event or call a
callback function. Alarms can be autostarted at boot time depending on
the application mode.

The attributes of an alarm are the following:
\begin{itemize}
\item \const{COUNTER} specifies the counter to which the alarm is
  statically linked.
\item \const{ACTION} specifies the kind of action which has to be
  implemented when the alarm fires. The action is specified using one
  of the following sub-attributes:
  \begin{itemize}
  \item \const{ACTIVATETASK} specifies that a task has to be
    activated. The name of the task must be specified inside the
    \const{TASK} sub-attribute.
  \item \const{SETEVENT} specifies that an event has to be set on a
    task. The task name and event must be specified inside the
    \const{TASK} and \const{EVENT} sub-attributes.
  \item \const{ALARMCALLBACK} specifies that an alarm callback has to
    be called. The name of the callback is specified inside the
    attribute \const{ALARMCALLBACKNAME}.
  \end{itemize}
\item \const{AUTOSTART} specifies if the alarm has to be autostarted
  at system startup. If \const{TRUE}, the alarm properties and the
  application modes for which the alarm should be autostarted have to be
  specified in the sub-attributes \const{ALARMTIME}, \const{CYCLETIME},
  and \const{APPMODE}.
\end{itemize}

\section{Notes on source files}

If a system object (CPU, TASK or RESOURCE) is implemented by more than
one source file, all the corresponding file names need to appear in
one or more corresponding OIL declarations (not necessarily in order,
as shown by the following examples for a CPU, a task and a resource):

\begin{lstlisting}
  CPU_DATA = NIOSII {
    ID = "cpu0";
    APP_SRC = "cpu0_src0.c";
  };
  CPU_DATA = NIOSII {
    ID = "cpu0";
    APP_SRC = "cpu0_src1.c";
    APP_SRC = "cpu0_src2.c";
    APP_SRC = "cpu0_src3.c";
    APP_SRC = "cpu0_src4.c";
  };
\end{lstlisting}

\begin{lstlisting}
  TASK thread1 {
    CPU_ID = "cpu1";
    APP_SRC = "thread1_a.c";
    APP_SRC = "thread1_b.c";
    APP_SRC = "thread1_c.c";
  };
\end{lstlisting}

\begin{lstlisting}
  RESOURCE mutex {
    RESOURCEPROPERTY = STANDARD {
      APP_SRC = "res_a.c";
      APP_SRC = "res_b.c";
    };
  };
\end{lstlisting}

It is possible, even if we discourage its use, to list several file
names in the same declaration, separated by spaces:

\begin{lstlisting}
  APP_SRC = "cpu0_src1.c cpu0_src2.c";
\end{lstlisting}

If a file name appears more than once, all declarations following the
first one are ignored.

\section{OIL extensions for multiprocessing}
\label{sec:oil-multicore}

\subsection{Partitioning support}
When developing a multiprocessor application, the developer faces the
job of mapping a multitask application on the CPUs that are available
in the system.  That mapping procedure involves the partitioning of
the application tasks into the CPUs, meeting all application
constraints.

As the starting point, each CPU runs a copy of \ee, and all the copies 
of \ee\ on the CPUs have the same configuration. Depending on the 
application needs, for example, the kernel can be configured to have 
monostack or multistack support, which is useful when dealing with 
blocking primitives, and to support debugging features such as hooks, 
and extended error status report. All these features are set at the same 
time for all CPUs. For example, the case where a CPU runs with monostack 
support while other CPUs run with multistack support is not possible. 

From the OIL configuration point of view, the developer defines a set
of CPUs in the OIL configuration file using the \const{CPU_DATA}
sections inside the \const{OS} object, and the job of partitioning
consists in placing the OIL objects into the existing CPUs.

The OIL Objects that must be explicitly mapped to a processor are
Tasks and Counters. As explained in Sections
\ref{sub:task-mapping-to-cpu} and \ref{sec:counter-extensions} the OIL
extensions implemented by \rtd\ allow the specification of the CPU to
which a \const{TASK} or \const{COUNTER} is allocated by using the
attribute \const{CPU_ID}. The link between the particular object (Task
or Counter) and the CPU is {\em static} and specified at compile time,
and cannot be changed at runtime.

Other OIL Objects are automatically mapped by the system. In
particular, Resources will be {\em local} if the tasks using them are
allocated to the same CPU or {\em global} otherwise. Alarms are linked
to a Counter (that in turn is mapped to a CPU). However, please note
that Alarm notifications can result in activation of remote tasks, and
setting of events on remote tasks.

Finally, other objects are local and are replicated on each CPU. Hooks
and Application Modes fall in this category. This means that each CPU
will have its own copy of the hook routines, and each CPU will be
initialized passing an appropriate Application Model. It is
responsibility of the application developer that all CPUs are
initialized with the same Application mode.


\subsection{Master CPU}
\label{sub:master-cpu}
%
%\nb{PJ: questo lo ho spiegato bene anche nella parte multiprocessore,
%forse conviene riprenderlo da li o citare quello!!!}
%

When designing multiprocessor systems, the user must specify which CPU
acts as ``Master CPU''\footnote{For details about the role of the Master CPU
please look at the Multiprocessor Sections of the \ee\ manuals}. Here
is the definition of the \oil{MASTER_CPU} attribute:

\begin{lstlisting}
  STRING MASTER_CPU = "default_cpu";
\end{lstlisting}

The default value for the \const{MASTER_CPU} attribute is
\const{default_cpu}\index{default\_cpu}. 

Example:
\begin{lstlisting}
  MASTER_CPU = "cpu0";
\end{lstlisting}




\subsection{Specification of the source files.}

In order to make easier the change among different application 
partitionings, each
system object should be implemented in a separate file. Then, each
file is specified into the OIL file as the implementation of the
corresponding object, such as \const{CPU_DATA}, \const{TASK},
\const{RESOURCE}. \rtd\ uses the information to create the
\file{subdir.mk} files that are used by the makefile scripts to
compile the source code.

\rtd\ generates a \file{subdir.mk} file for each CPU, including all
the files that refer to the objects allocated to the CPU.

When a source file is specified inside a \const{CPU_DATA} element, the
file is inserted in the \file{subdir.mk} of that CPU.

When a source file is specified inside a \const{TASK} element, the
file is inserted in the CPU where the task is allocated.

When a source file is specified inside a \const{RESOURCE} element, the
file is inserted in the CPU where all the tasks using it are allocated
in case it is a local resource, or on the Master CPU if it is a global
resource.

A source file name can be specified more than once inside the OIL
file. However, it will be inserted at most once for each CPU.

To better understand this situation, consider the following example:

\begin{lstlisting}
CPU test_application {

  OS EE {
    MASTER_CPU = "cpu0";			

    CPU_DATA = NIOSII {
      ID = "cpu0";
      APP_SRC = "cpu0.c";
    };

    CPU_DATA = NIOSII {
      ID = "cpu1";
      APP_SRC = "cpu1.c";
    };
  };

  TASK task0 {
    CPU_ID = "cpu0";
    APP_SRC = "task0.c";
    RESOURCE = globalmutex;
  };

  TASK task1 {
    CPU_ID = "cpu1";
    APP_SRC = "task1.c";
    RESOURCE = globalmutex;
    RESOURCE = localmutex;
  };

  TASK task2 {
    CPU_ID = "cpu1";
    APP_SRC = "task2.c";
    RESOURCE = localmutex;
  };
  
  RESOURCE globalmutex {
    RESOURCEPROPERTY = STANDARD { APP_SRC = "globalmutex.c"; };
  };

  RESOURCE localmutex {
    RESOURCEPROPERTY = STANDARD { APP_SRC = "localmutex.c"; };
  };
};
\end{lstlisting}

\const{cpu0}'s \file{subdir.mk} will include the files
\file{cpu0.c}, \file{task0.c}, and \file{globalmutex.c};
\const{cpu1}'s \file{subdir.mk} will include the files \file{cpu1.c},
\file{task1.c}, \file{task2.c}, and \file{localmutex.c}.



