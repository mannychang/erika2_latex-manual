\chapter{An overview of \rtd\ Code Generator and \ee}
\label{cha:overview}

The \rtd\ Code Generator is an open and extensible environment, based
on XML and open standards allowing the generation of configuration
code for the \ee\ real-time kernel. The code configuration may start
from an OSEK OIL or an AUTOSAR XML definition, to create the
configuration code for applications running in a variety of
environments.

The code generator plugin is designed aiming at the following general
goals:

\begin{description}
\item [Modularity:] once the kernel module is installed, each design
  activity in the development flow is in charge of a module that can
  be separately used as standalone component.
\item [Portability~across~different~execution~environments:] the tool
  is designed and implemented in Java for maximum portability to
  different environments and operating systems (MS Windows, Linux,
  Macintosh).
\item [Extensibility:] The Code Generator tool includes an XSLT
  transformation engine to easily specify custom modifications to the
  standard OIL code generator.
\end{description}

Similarly, the operation for creating a new project, as shown in the
following Chapter \ref{cha:creating-rtdruid-project} and the editing
of the configuration files follow the standard Eclipse pattern. The
result of the generation of the configuration file by the \rtd\ wizard
and the result of most operations performed by \rtd\ are shown in a
dedicated Eclipse console (as happens for most Eclipse plug-ins, for
details, please see Section \ref{cha:code-generation}).

Integration with Eclipse is not only at GUI interface level, but it
also allows performing operations in batch (command line) mode
according to the ANT standard \cite{ANT}. \rtd\ extends the ANT
commands (``TASK'' in ANT terminology) adding the capability for code
generation and the execution of the compilation scripts starting from
an OIL file.


\section{Code generation}

The \rtd\ Code Generator is a plugin that is used to automatically
generate configuration code at compile time. The steps performed by
the Code Generator upon a compilation request are described in this 
section.

\paragraph{Creation of the build directory and its content}
Starting from an OIL configuration file, the tool creates a directory
that will contain all the generated files. The directory will be the
default directory for all the operations of the C/C++ compiler. In the
following, we assume that the name selected for this directory in the
configuration file is \file{Debug}.

The first file that is created is the \file{makefile}, created inside
the \file{Debug} directory itself. The \file{makefile} is used to
compile the application source code. The makefile structure may depend
on the final target architecture.

Then, typically the files \file{eecfg.h} and \file{eecfg.c} are
created, containing the data structures needed to run an
\ee\ application. Other files may be present depending on the specific
target microcontroller.

\paragraph{Makefile execution}
After the files are created, \rtd\ automatically runs the \file{make}
command in the \file{Debug} directory. As a result, the compilation
process starts and a set of output files are created.

\section{Multicore Version}

\rtd\ provides special support for the development of multicore
applications together with the \ee\ kernel. Currently
supported features include:

\begin{itemize}
\item Multicore systems with shared memory.
\item Support for code placement in a multicore system.
\end{itemize}

Programming-level implementation is independent from code placement on
processors, meaning that the programmer do not need to be aware of the
existence of multiple processors. The code generator provides the
correct implementation of system primitives based on the placement of
the threads and resources as specified in the \rtd\ (OIL)
configuration part. Independence from placement options provides:

\begin{itemize}
\item Easy testing of different placement configurations. 
\item Easy extension to a higher degree of parallelism and seamless
  porting of existing single processor applications
\end{itemize}

On a multicore system, the main makefile is responsible of triggering
the compilation of a separate image for each CPU. In that case,
together with the \file{makefile}, the tool also generates a file
\file{common.mk} (inside the \file{Debug} directory), containing
common makefile settings for the CPUs in the project. Then, one
directory is created for each CPU included into the project. The
content of that directory is similar to the content of the single core
system.
