\chapter{Format of the Input File}
\label{cha:DTD-Input-file}


\section{Introduction}

This chapter describes the structure and the syntax of the RTD XML
format, which is used to describe the main component of a system. The
RTD format has been used since the first versions of \rtd\ as the
source of the information for the schedulability analysis plugin.

The DTD of the RTD specification is contained in the file
\file{templates/evidence.DTD} of the \rtd\ documentation.
% FIXME: check if the file position is still correct.

The structure of this chapter is the following: first, it describes
the development process of an embedded real-time system with \rtd, and
how the different steps of this process are reflected in the sections
of the \rtd\ input file. Then, it details the structure of the DTD,
describing each of its sections with examples. Finally, it describes
other miscellaneous, like common elements and conventions.


\section{Development process with \rtd}

The objective of the \rtd\ schedulability plugins is to help the
system designer in all phases of the design and development of a
complex embedded application. For this reason, the input file is
divided into different sections, each one containing information on a
single phase of the design and development process. In particular, we
envision the following steps:
\begin{itemize}
\item Modeling the functional behavior of the application. In this
  phase, the designer models the behavior of the system and of the
  application without any regard to the implementation of the system
  on a specific platform. This phase is typically done with tools like
  Matlab/Simulink, or similar. The description of the functional model
  of the system is done in the \const{FUNCTIONAL} section of the input
  file (see Section \ref{sec:Functional-section} for details).
\item Modeling the software and hardware architecture of the
  system. In this phase, the designer models the characteristics of
  the platform on which the application will be implemented. This
  involves the definition of the number of computational nodes, the
  number and types of processors for each node, the RTOS used on each
  node, the number of real-time tasks, etc. The architectural model is
  specified in the \const{ARCHITECTURAL} part of the \rtd\ input file
  (see Section \ref{sec:Architectural-section} for more details).
\item Mapping the functional model to the architectural model. In this
  phase, the functional elements defined in the \const{FUNCTIONAL}
  section are ``mapped'' on to the appropriate architectural
  elements. For example, it is important to decide which task will
  perform a specific operation of the applications, and on which
  node/RTOS such task is allocated. Of course, many different mapping
  choices are possible, and the choice of the mapping can influence
  the performance of the system. In the \rtd\ input file, the mapping
  is specified in the \const{MAPPING} section (see Section
  \ref{sec:Mapping-section} for more details).
\item Back-annotation of performance measurement. In this phase, the
  system is run under specific testing conditions, and the performance
  of each element is measured and back-annotated. For example, in this
  phase each task is run under different conditions and the execution
  time is measured and recorded. At the end of the test, the
  average-case and worst case execution time of the task can be
  obtained. These information will be useful for the next phase, the
  schedulability analysis. The back-annotation information is
  contained in the \const{ANNOTATION} section of the \rtd\ input file
  (see Section \ref{sec:Annotation-section} for more details).
\item Schedulability analysis. In this phase, a systematic off-line
  analysis of the system is performed. The analysis will tell the
  designer if the system is schedulable (i.e. if all tasks will
  complete in time) under all conditions. Moreover, the analysis gives
  back additional sensitivity information about the system. For
  example, if the system is not schedulable, the analysis tells us
  which task will miss its deadline and how it is possible to modify
  the system the avoid this.  In case the system is schedulable, it
  tells us how much ``free space'' is left before a deadline is
  missed. The results of the analysis are provided by the tool in the
  \const{SCHEDULABILITY} section (see
  Section\ref{sec:Schedulability-section} for more details).
\end{itemize}

Not all sections are mandatory. In most cases, it is possible to
specify only some section and still be able to analyze the system. For
example, under certain conditions, the \const{FUNCTIONAL} section need
not to be specified. This is useful, for example, when analyzing
systems that are already implemented.


\section{RTD and ERTD files}

The \rtd\ schedulability plugin stores the input files in two
different formats, each one having a different file extension, as
shown in Table \ref{tab:file-formats}.

%
\begin{table}
\begin{centering}
\begin{tabular}{|c|p{7cm}|}
\hline 
File extension & Description
\tabularnewline
\hline
\hline 
.RTD & XML file format that has been created for ease human processing.
\tabularnewline
\hline 
.ERTD & XML file format automatically generated by EMF. That format is not
suited for readability. 
\tabularnewline
\hline
\end{tabular}
\par\end{centering}

\caption{XML file formats used by the \rtd\ Tool set.}
\label{tab:file-formats}
\end{table}


The two formats express the system semantic in a different form: the
.ERTD form is suited for automatic processing by the EMF automatically
generated loader, whereas the .RTD form is more easily readable, and
hence it is more suited for manual editing. This chapter only
describes the .RTD style XML file structure. The .ERTD file structure
is an internal XML representationand and it is not described in this
manual.

\textbf{Note:} \rtd\ uses the file extension to discriminate which
file format should be expected as content. Files in .RTD form are
parsed by an XSLT sheet, converted in .ERTD form, and then loaded.
When saving the file from the GUI it is possible to save in both formats.


\section{.RTD XML file structure}

The purpose of the file format presented in this section is to describe
all the aspects of a complex system. As its root, a system is composed
by a single \const{system} element, that contains a \const{Name}
attribute used to discriminate between different instances of systems.
The following is an example of a system definition:

\begin{lstlisting}
<?xml version="1.0" encoding="UTF-8"?>
<SYSTEM Name="mySystem">
[...]
</SYSTEM>
\end{lstlisting}

Each system is described using sections. Each section is used to
describe a particular phase of the design methodology that
\rtd\ proposes.  Here is a quick list of the various sections that
compose a system description:

\begin{description}
\item [{modes}] The modes section is used to list different working
  modes of a system. See section \ref{sec:Mode-section}.
\item [{functional}] The functional section contains all the elements
  that describe the functional part of the system. The functional part
  of a system is in general composed by different elements, named
  \const{PROC}, \const{VAR}, \const{SUBSYSTEM}, and \const{TRIGGER}
  (external events).  Moreover, the functional part contains the
  \emph{use} relationships between the different elements in the
  system, the \emph{partial ordering} between different events
  (start/end of a method, proc activation, trigger arrival). For each
  element it is possible to specify some temporal characterization,
  such as periodicity, deadline, minimum interarrival time, jitter and
  offset (see sections \ref{sub:Logical-level-design} 
  % FIXME reference not found
  and \ref{sec:Functional-section}).
\item [{architectural}] The architectural section contains the
  specification of different parts of the system that are of interest
  for the purpose of mapping. This section contains, for example,
  elements such as \const{ECU}, \const{CPU}, \const{TASK},
  \const{SEMAPHORE}s, \const{BUS} and related \const{FRAME}s. This
  section does not contain the relation between tasks and CPUs, that
  is instead contained in the mapping section.  If a system includes
  only the architectural but not the functional section, all the
  references to \const{VAR}s are substituted with resources that
  otherwise are ignored (see Sections
  \ref{sub:Software-architecture-design} 
  % FIXME reference not found
  and
  \ref{sec:Architectural-section}).

\item [{mapping}] The mapping section is used to describe the
  assignment of \const{PROC}s to \const{TASK}s, to specify which RTOS
  a task is assigned to, and to specify if a \const{VAR} element
  should be implemented as a mutex or through a bus frame. These
  information depend on the system application modes (see sections
  \ref{sub:Mapping} 
  % FIXME ref not found
  and \ref{sec:Mapping-section}).
\item [{annotation}] The annotation section includes the temporal
  information related to the execution times of each task and of each
  method related to proc, var and resources. This information depends
  on the system application modes (see section
  \ref{sec:Annotation-section}).
\item [{schedulability}] The schedulability section contains the
  result of the schedulability test that is made for each CPU and for
  each task. This information depends on the system application modes
  (see sections \ref{sub:Schedulability-Analysis} and
  \ref{sec:Schedulability-section}).
  % FIXME ref not found
\end{description}

It is not possible to describe more than one system inside a single
input file. Not all the sections just described are mandatory in an
instance of an input file, however it must be clear that some features
may be disabled in case of missing informations. For example,
schedulability tests cannot be performed if computation times are
missing.

All the identifiers in an input file are \emph{case insensitive},
except for the references that must contain the exact name of an
object (that usually has a one-to-one relation with some identifier
used in the source code).


\paragraph{Example}

Here is an example of a typical XML file structure.

\begin{lstlisting}
<?xml version="1.0" encoding="UTF-8"?>
<SYSTEM Name="...">
  <MODES>
    [...]
  </MODES>
  <FUNCTIONAL>
    [...]
  </FUNCTIONAL>
  <ARCHITECTURAL>
    [...]
  </ARCHITECTURAL>
  <MAPPING>
    [...]
  </MAPPING>
  <ANNOTATION>
    [...]
  </ANNOTATION>
  <SCHEDULABILITY>
    [...]
  </SCHEDULABILITY>
</SYSTEM>
\end{lstlisting}

The following sections present in detail the content of each section
in the XML file, detailing all the elements that characterize each
part.


\section{Mode section}
\label{sec:Mode-section}

Modes are abstractions that can be used to specify a configuration of
the system that is used only under particular working conditions.  For
example, an airplane controller probably has different control
algorithms when the airplane is taking off, flying and landing, and
these algorithms have requirements expressed by a different set of
tasks, resources, and so on. For this reason, the specification of
modes in the system is useful for describing the behavior and the
composition of the system in all its aspects and in an unified way.

Modes typically influence the mapping phase, because it is in that
phase that the designer decides:

\begin{itemize}
\item which is the task that contains each PROC;
\item which is the RTOS that contains each Task;
\item how each variable is handled (private, protected using some kind
  of mutual exclusion, or bus variables);
\end{itemize}

As a consequence, the mapping phase influences the computation time
and the blocking time of each task, and the schedulability of each
single task and CPU in the system.

The Mode section is simply composed by a sequence of \const{MODE}
elements, as described below. \rtd\ defines implicitly a \emph{default
  Mode} that is always used for all the elements that do not
explicitly specify one.

Please note that, although the name is similar, \const{MODE}s are not
equivalent to OSEK/VDX Application Modes. OSEK/VDX Application Modes
are used to identify a boot configuration which cannot be changed
after the system starts (for example, normal operation, testing,
...). The \const{MODE} section is instead used to specify runtime
aspects of an application (for example, engine stop, engine running at
minimum RPM, engine running at 1000~rpm, engine running at 2000~rpm,
...). The idea is that each \const{MODE} may define a different
application setup and a different schedulability analysis (for
example, some tasks may have a frequency which is directly linked to
the rotation speed of the engine).

\subsection{\const{MODE}}

\const{MODE} elements simply describe mode names. A \const{MODE}
element contains only one attribute:
\begin{description}
\item [{attribute~Name}] is the symbolic name of a mode.
\end{description}

\paragraph{Example}

In the following example, four modes (\const{Init}, \const{Running,
Fault} and \const{default mode}) are defined:

\begin{lstlisting}
[...]
<MODES>
  <MODE Name="Init"/>
  <MODE Name="Running"/>
  <MODE Name="Fault"/>
</MODES>
[...]
\end{lstlisting}

% FIXME: we should link to an existing example

\section{Functional section}
\label{sec:Functional-section}

The functional section contains a set of elements that describe the
\emph{functional behavior} of the system. These entities will be
mapped onto the architectural elements during the mapping phase. Each
entity is represented by an XML element inside the main
\const{FUNCTIONAL} element. The functional section may contain one or
more instances of the following elements: \const{PROC},
\const{TRIGGER}, \const{VAR}, \const{EVENT}, \const{PARTIALORDER},
\const{TIMECONST}, \const{SUBSYSTEM}.

Element names are grouped in namespaces. Two elements inside the same
namespace must have different names. \rtd\ defines the following
namespaces for the \const{FUNCTIONAL} element:
\begin{itemize}
\item One namespace that contains \const{TRIGGER} names,
  \const{SUBSYSTEM} names, \const{PROC} and \const{VAR} external to
  subsystems.
\item One namespace for each \const{SUBSYSTEM}. This namespace
  contains its internal \const{SUBSYSTEM}s, \const{PROC} and
  \const{VAR};
\item One namespace for each \const{PROC}, \const{VAR},
  \const{REQUIRED\_INTERFACE} and \const{PROVIDED\_INTERFACE.} This
  namespace contains the names of the \const{METHOD} specified inside
  any of the elements listed before.
\item One namespace for each \const{PROC}, \const{VAR},
  \const{TRIGGER}, and \const{REQUIRED\_INTERFACE.} This namespace
  contains the names of the \const{METHODREF} specified inside any of
  the elements listed before.
\item One namespace that contains \const{EVENT} names
  \footnote{In all the functional section, even if they are related to different
    subsystems}.
\end{itemize}

% FIXME: to be changed in the next version. EVENT names should stay in
% the namespaces of the subsystems where they are specified



\subsection{\const{PROC}}
\label{sub:PROC}

A \const{PROC} element allows the description of the minimal
computational entity, that \emph{provides} one or more methods and
\emph{uses} one or more methods exported by other \const{PROC}s.

A \const{PROC} element may be internal or external to a subsystem:

\begin{itemize}
\item if a \const{PROC} element is internal to a subsystem, it can use
  only the \const{METHOD}s provided by the \const{PROC}, and the
  \const{VAR}iables internal to the same subsystem, or by the
  \const{REQUIRED\_INTERFACE} of its own subsystem;
\item if a \const{PROC} element is external to a subsystem, it can use
  the \const{METHOD}s provided by \const{PROC}s and \const{VAR}iables
  external to the subsystems, or the \const{PROVIDED\_INTERFACE} of
  the other subsystems.
\end{itemize}

A \const{PROC} element has only one attribute:

\begin{description}
\item [{attribute~Name}] this attribute specifies the \const{PROC}
  name inside the current namespace.
\end{description}

A \const{PROC} element must contain at least one
\const{METHOD}\footnote{The next versions of \rtd\ will provide a
  default method if no method is specified}. A \const{PROC} can
contain zero or more \texttt{METHODREF} elements.


\subsubsection{\const{METHOD}}

This element represents a method provided by a \const{PROC}. For more
information about methods, see Section \ref{sub:METHOD}.


\subsubsection{\const{METHODREF}}

This element specifies the coordinates that identify a method provided
by some other element that will be called by the \const{PROC}'s
methods (see Section \ref{sub:METHODREF}).


\paragraph{Example}

The following example shows the definition of two \const{PROC} elements
named \const{filter1} and \const{filter2}. \const{filter1} provides
a method \const{method1}, that is required by \const{filter2}.

\begin{lstlisting}
[...]
<FUNCTIONAL>
  <PROC Name="filter1">
    <METHOD Name="method1"/>
    [...]
  </PROC>
  <PROC Name="filter2">
    <METHODREF
      Name="internal_name_for_method1"
      MethodName="filter1/method1"
    />
    [...]
  </PROC>
  [...]
</FUNCTIONAL>
[...]
\end{lstlisting}



\subsection{\const{VAR}}
\label{sub:VAR}

The \const{VAR} element inside the XML representation is used to model
an abstract data type. An abstract data type is typically composed by
some data structures, and by a set of methods that are used to access
and modify the them.

The \const{VAR} element has the following attributes:
\begin{description}
\item [{attribute~Name}] this attribute specifies the \const{VAR} name
  inside the current namespace.
\item [{attribute~Type}] this attribute specifies the type of the VAR
  element.
\end{description}

A \const{VAR} can only contain a set of \const{METHOD} elements.

\subsubsection{\const{METHOD}}

This element represents a method provided by the abstract data type
implemented inside the \const{VAR}. See Section \ref{sub:METHOD} for
details.


\subsubsection{Example}

The following example describes two VAR elements, the first named
\const{image} is implemented as an C integer matrix (\const{int[][]}),
that provides two elements \const{read} and \const{write}; the second
named \const{size} is implemented as an C integer (\const{int}), that
provides two implicit methods \const{read} and \const{write}.

\begin{lstlisting}
[...]
<FUNCTIONAL>
  <VAR Name="image" Type="int[][]">
    <METHOD Name="read"/>
    <METHOD Name="write"/>
  </VAR>
  <VAR Name="size" Type="int"/>
  [...]
</FUNCTIONAL>
[...]
\end{lstlisting}

\subsection{\const{TRIGGER}}

A \const{TRIGGER} element describes an external event that can
activate and use one or more methods provided by other elements in the
functional specification.

A trigger is characterized by a single attribute:

\begin{description}
\item [{attribute~Name}] this attribute specifies the \const{TRIGGER}
  name inside the \const{FUNCTIONAL} specification.
\end{description}

A \const{TRIGGER} can only contain a set of \const{METHODREF}
elements.


\subsubsection{\const{METHODREF}}

This element represent a method invoked by a TRIGGER. For details see
Section \ref{sub:METHODREF}.


\paragraph{Example}

The following example, extracted from the example contained in the
directory \file{examples/001}, describes a trigger named \const{imageArrival}
that requires three methods: \const{write}, \const{decodeX} and
\const{derX}, provided respectively by objects \const{image}, \const{filter1}
and \const{der1}.
%FIXME: check the file name

These reference have been renamed inside the trigger as \const{image.write,}
\const{filter1.decodeX} and \const{der1.derX}.

\begin{lstlisting}
[...]
<FUNCTIONAL>
  <TRIGGER Name="imageArrival" >
    <METHODREF
      Name="image.write"
      MethodName="image/write"
    />
    <METHODREF
      Name="filter1.decodeX"
      MethodName="filter1/decodeX"
    />
    <METHODREF
      Name="der1.derX"
      MethodName="der1/derX"
    />
    [...]
  </TRIGGER>
  [...]
</FUNCTIONAL>
[...]
\end{lstlisting}

Please note the difference between, for example, \const{der1.derX} and
\const{der1/derX}: the former is the name of the \const{METHODREF}
(the ``\const{.}'' is part of the name), whereas the latter is the
path to the \const{METHOD} element (as specified in Section
\ref{sub:Paths-to-elements}, the ``\const{/}'' is used as separator).


\subsection{\const{EVENT}}
\label{sub:EVENT}

An \const{EVENT} element is used to tag the execution of a
\const{METHOD}.  \const{EVENT}s are then used when representing order
between events, or timing constraints between different method
executions. An \const{EVENT} can describe the action of activating a
method (\const{activation}), and the fact that a method has started
(\const{begin}) or ended (\const{end}) its execution on the final
architecture.

Subsystems do not have {}``internal'' \const{EVENT}s. In any case,
\const{EVENT}s can tag internal subsystem methods. An \const{EVENT}
has the following three attributes:
\begin{description}
\item [{attribute~Name}] this attribute specifies the \const{EVENT}
  name inside the functional section. \const{EVENT} names are unique
  in the entire functional section.
\item [{attribute~Type}] specifies the action that the \const{EVENT}
  is tagging:

  \begin{description}
  \item [{activation}] specifies that the \const{EVENT} tags the activation
    of a method. Activating a method means that the method is ready to
    execute, without specifying when the task will be really executed
    on the architecture.
  \item [{begin}] specifies that the \const{EVENT} tags the beginning of
    execution of a method. The \const{begin} of a method always follows
    its \const{activation}.
  \item [{end}] specifies that the \const{EVENT} tags the end of the execution
    of a method. The \const{end} of the execution of a method always
    follows its \const{begin}.
  \end{description}

\item [{attribute~MethodRefName}] specifies the name of the \const{METHODREF}
  that the \const{EVENT} is tagging.
\end{description}

\paragraph{Example}

The following example shows how to represent the activation event
of a method.

\begin{lstlisting}
[...]
<FUNCTIONAL>
  <EVENT 
    Name="my_activation_event"
    Type="activation"
    MethodRefName="imageArrival/der1.derX"
  /> 
  [...]
</FUNCTIONAL>
[...]
\end{lstlisting}

\subsection{\const{PARTIALORDER}}

A \const{PARTIALORDER} element is a container of \const{ORDER} elements.
An \const{ORDER} element can be used to describe a precedence relationship
between two events. A \const{PARTIALORDER} is related to a particular
system mode.

\const{PARTIALORDER} has only one attribute:

\begin{description}
\item [{attribute~ModeRef}] this attribute specifies the Mode to which
  the \const{ORDER} elements contained inside the \const{PARTIALORDER}
  element are related to. If ModeRef is not specified, the default mode
  is used.
\end{description}

The \const{PARTIALORDER} element can contain zero or more \const{ORDER}
elements.


\subsubsection{\const{ORDER}}

An \const{ORDER} element only contains a pair of \const{EVENTREF}
elements. The first \const{EVENTREF} is the predecessor, the second is
the successor. For information about \const{EVENTREF} elements, see
Section \ref{sub:EVENTREF}.


\paragraph{Example}

The following \const{PARTIALORDER} example shows four events with
precedence relationship as shown in Figure
\ref{fig:PARTIALORDER-Example}.

\begin{figure}
  \begin{center}
    \includegraphics[width=6.3cm]{images/partialorder.eps}
  \end{center}
  \caption{The \const{PARTIALORDER} Example.}
  \label{fig:PARTIALORDER-Example}
\end{figure}

\begin{lstlisting}
[...]
<FUNCTIONAL>
  <PARTIALORDER>
    <ORDER>
      <EVENTREF Name="one"/>
      <EVENTREF Name="three"/>
    </ORDER>
    <ORDER>
      <EVENTREF Name="two"/>
      <EVENTREF Name="three"/>
    </ORDER>
    <ORDER>
      <EVENTREF Name="three"/>
      <EVENTREF Name="four"/>
    </ORDER>
  </PARTIALORDER>
  [...]
</FUNCTIONAL>
[...]
\end{lstlisting}

\subsection{\const{TIMECONST}}

A \const{TIMECONST} element is a container of one or two
\const{EVENTREF} followed by a \const{BOUND} element. A
\const{TIMECONST} is used to represent a constraint on events.

When the constraint describes a property of an event (such as an event
periodicity), only one \const{EVENTREF} is needed.

When the constraint describes a property related to two events (such
as an end-to-end deadline), two \const{EVENTREF}s are needed.

A \const{TIMECONST} has only one attribute:
\begin{description}
\item [{attribute~ModeRef}] this attribute specifies the Mode to which
  the \const{TIMECONST} elements is related to. If ModeRef is not
  specified, the default mode is used.
\end{description}


\subsubsection{\const{EVENTREF}}

For information about \const{EVENTREF} elements, see Section
\ref{sub:EVENTREF}.


\subsubsection{\const{BOUND}}

The \const{BOUND} element is used to characterize the type of
constraints represented by a \const{TIMECONST} element.

The available constraints are the following:

\begin{description}
\item [{deadline}] specifies a constraint that the distance between
  two events must not exceed the deadline.
\item [{period}] specifies the period of an event;
\item [{mit}] specifies the Minimum Interarrival Time (MIT) of an
  event.  That is, if an event appeared at time $t$, then the same
  event will not appear again before time $t+mit$.
\item [{jitter}] specifies the maximum jitter between two events;
\item [{offset}] specifies the constant time distance between two
  events;
\end{description}

The \const{BOUND} object has two attributes:

\begin{description}
\item [{attribute~Type}] The type of the constraint; Should be either
  ``deadline'', ``period'', ``mit'', ``jitter'', or ``offset''.
\item [{attribute~Value}] The temporal value related to the
  constraint.
\end{description}

\underbar{The current version of \rtd\ only supports deadline, period
  and offsets.}
% FIXME: is this still true?

\paragraph{Example}

The following example describes how to model the periodicity of an
event and a deadline between two events.

\begin{lstlisting}
[...]
<FUNCTIONAL>
  <TIMECONST>
    <EVENTREF Name="my_periodic_event"/>
    <BOUND Type="period" Value="40ms"/>
  </TIMECONST>
  <TIMECONST> 
    <EVENTREF Name="start_event"/>
    <EVENTREF Name="end_event"/>
    <BOUND Type="deadline" Value="35ms"/>
  </TIMECONST> 
  [..]
</FUNCTIONAL>
[...]
\end{lstlisting}

\subsection{\const{SUBSYSTEM}}
\label{sub:SUBSYSTEM}

This element is a container that can be used to group a set of
\const{SUBSYSTEM}s, \const{PROC}s and \const{VAR}s. It has been
introduced since \rtd\ 0.2 to permit a modular specification of the
functional part of a system. It mimics the concept of subsystem
defined in component-based languages.

A \const{SUBSYSTEM} consists of three sub-elements:

\begin{itemize}
\item an \const{IMPLEMENTATION} element that groups internal
  \const{SUBSYSTEMs, PROC}s and \const{VAR}s; it represents the
  internal implementation of a module or component.
\item a \const{PROVIDED\_INTERFACE} element; it represents the
  interface that this subsystem provides to other subsystems or
  external elements.
\item a \const{REQUIRED\_INTERFACE} element; it represents the
  interface that this subsystem requires from external subsystems or
  elements in order to work correctly.
\end{itemize}

The implementation of a subsystem is hidden to the rest of the system.
In this way, it is possible to change the implementation without
affecting the implementation of the rest of the system. For this
reason, \const{SUBSYSTEM, PROC} and \const{VAR} elements specified in
the \const{IMPLEMENTATION} of a \const{SUBSYSTEM} cannot be referred
directly from external entities. An external entity can only refer to
the \const{METHOD} elements specified in the
\const{PROVIDED\_INTERFACE} section. On the other hand, a \const{PROC}
or a \const{VAR} inside the \const{IMPLEMENTATION} can only refer to
the \const{METHOD} elements specified in its
\const{REQUIRED\_INTERFACE}.

% check the following image before putting it in production
%(see Figure \ref{fig:subsystem}).
%% %
%% \begin{figure}
%% \includegraphics[width=1\columnwidth,keepaspectratio]{subsystem.eps}
%% \caption{\label{fig:subsystem}A subsystem. The figure highlights the purpose
%% of the \emph{provided} and the \emph{required} interface. Please note
%% that \const{EVENT\_PROC\_2} points to the link (the \const{METHODREF})
%% between \const{PROC\_1} and \const{PROC\_2}, not directly to \const{PROC\_2}'s
%% \const{METHOD}.}
%% \end{figure}


A subsystem does not contain \const{EVENT}s. However, \const{EVENTS}
can tag internal subsystem methods (see Section \ref{sub:EVENT}).

Note: subsystems can be nested. That is, a subsystem can contain
another subsystem.

A subsystem has the following attribute:

\begin{description}
\item [{attribute~Name}] This attribute is the name of the subsystem.
\end{description}

A subsystem contains three kinds of elements:

\begin{itemize}
\item \const{IMPLEMENTATION}
\item \const{PROVIDED\_INTERFACE}
\item \const{REQUIRED\_INTERFACE}
\end{itemize}

\subsubsection{\const{IMPLEMENTATION}}

This element contains one or more \const{SUBSYTEM} elements (see
Section \ref{sub:SUBSYSTEM}), one or more \const{PROC} elements (see
Section \ref{sub:PROC}) and one or more \const{VAR} elements (see
Section \ref{sub:VAR}) internal to a subsystem.


\subsubsection{\const{REQUIRED\_INTERFACE}}

This element contains a list of methods that are required by the
subsystem because some \const{PROC} element inside the subsystem needs
that \const{METHOD}. Each one of these methods must be mapped on the
external \const{PROC}, \const{VAR}, \const{SUBSYSTEM} that provides
it, using a \const{METHODREF}. Therefore, the
\const{REQUIRED\_INTERFACE} is composed by a sequence of pairs each
one composed by a \const{METHOD} and its \const{METHODREF}. The
\const{METHODREF} can be missing if the binding of the
\const{SUBSYSTEM} has not been completed (see the example below).


\subsubsection{\const{PROVIDED\_INTERFACE}}

This element contains the list of methods exported by the subsystem.
These exported \const{METHOD}s are mapped to the methods of the
objects inside the subsystem using elements of type
\const{EXPORTED\_METHOD}.  For that reason, the
\const{PROVIDED\_INTERFACE} is composed by a sequence of pairs each
one composed by a \const{METHOD} and its \const{EXPORTED\_METHOD}.
The \const{EXPORTED\_METHOD} can be missing if the binding of the
\const{SUBSYSTEM} has not been completed (see the example below).


\subsubsection{\const{EXPORTED\_METHOD}}

This XML element is used to specify the internal method that is binded
to the provided method exported by the subsystem. This element has the
following attributes:

\begin{description}
\item [{attribute~ObjRef}] is the name of the object that contains the
  method (a \const{PROVIDED\_INTERFACE} of a \const{SUBSYTEM} internal
  to the current subsystem, or a \const{PROC} or a \const{VAR}
  internal to the subsystem);
\item [{attribute~MethodName}] is the method name.
\end{description}

\paragraph{Example}

The following example describes a subsystem named \const{mySub}, which
exports four methods \const{read,} \const{write, start} and
\const{stop} and which requires two methods \const{input} and
\const{output}.  The subsystem contains a \const{PROC} named
\const{subProc} and a var named \const{subData}. The provided methods
are linked to the internal methods. The subsystem requires two
external methods, \const{input} and \const{output}. The first one has
been mapped on method \const{getData} of subsystem \const{yourSum},
whereas the output method has not been mapped yet.\textcolor{red}{}%

%FIXME: for an example with nested subsystems check ``testcase 003''

\begin{lstlisting}
[...]
<FUNCTIONAL>
  <SUBSYSTEM Name="mySub">
    <IMPLEMENTATION>
      <PROC Name="subProc">
        <METHOD Name="start"/>
        <METHOD Name="stop"/>
        <METHODREF
          Name="subProc.read"
          MethodName="mySub/input"
        />
        <METHODREF
          Name="subProc.write"
          MethodName="mySub/output"
        />
        <METHODREF
          Name="subProc.data.read"
          MethodName="subData/read"
        />
        <METHODREF
          Name="subProc.data.write"
          MethodName="subData/write"
        />
      </PROC>

      <!-- implicit read and write methods -->
      <VAR Name="subData" Type="int"/>
    </IMPLEMENTATION>

    <PROVIDED_INTERFACE>
       <METHOD Name="read"/>
       <EXPORTED_METHOD
          ObjRef="subData"
          MethodRef="read"
       />
       <METHOD Name="write"/>
       <EXPORTED_METHOD
          ObjRef="subData"
          MethodRef="write"
       />
      <METHOD Name="start"/>
      <EXPORTED_METHOD
          ObjRef="subProc"
          MethodRef="start"
       />
      <METHOD Name="stop"/>
      <EXPORTED_METHOD
          ObjRef="subProc"
          MethodRef="stop"
       />
     </PROVIDED_INTERFACE>
     <REQUIRED_INTERFACE>
       <METHOD Name="input"/>
       <METHODREF 
          Name="inputMapping"
          MethodName="yourSub/getData"
       />
       <METHOD Name="output"/>
     </REQUIRED_INTERFACE>
  </SUBSYSTEM>
  [...]
</FUNCTIONAL>
[...]
\end{lstlisting}


\section{Architectural section}
\label{sec:Architectural-section}

The architectural section contains a set of elements that describes
the \emph{hw/sw architecture} of the system. These entities will be
mapped together with methods of the functional section during the
mapping phase.

Each architectural entity is represented by an XML element inside the
main \const{ARCHITECTURAL} element. The architectural section may
contain the following elements: \const{ECU}, \const{TASK},
\const{RESOURCE}, \const{BUS}, \const{FRAME}, \const{SIGNAL},
\const{MUTEX}. Element names are grouped in namespaces. Every element
inside a namespace must have a different name. The \rtd\ Toolset
defines the following namespaces for the \const{ARCHITECTURAL}
section:

\begin{itemize}
\item a namespace that contains \const{ECU} names;
\item a namespace that contains \const{RTOS} names;
\item a namespace that contains \const{TASK} names;
\item a namespace that contains \const{RESOURCE} names;
\item a namespace that contains \const{BUS} names;
\item a namespace that contains \const{FRAME} names;
\item a namespace that contains \const{SIGNAL} names;
\item a namespace that contains \const{MUTEX} names.
\end{itemize}


\subsection{\const{ECU}}

The \const{ECU} (Embedded Control Unit) element represents a
computational node composed by one or more \const{CPU} elements;
\const{CPU} elements belonging to the same ECU may be etherogeneous
(e.g., an \const{ECU} can have an ARM7TDMI and an Hitachi H8). The
various \const{CPU} elements mapped to the same \const{ECU} can
communicate using a shared bus (message passing paradigm) and/or
shared variables (shared memory).

Note: the current version of the schedulability analysis of the \rtd\
Toolset does not consider shared resources between tasks allocated
to different \const{CPU} elements. Moreover, there is currently no
check to prevent such situation. A mutex shared between tasks allocated
to different CPUs is treated as a local mutex on both CPUs; no warning
message is printed if this condition appears.

The \const{ECU} element has only one attribute:

\begin{description}
\item [{attribute~Name}] is the name assigned to the \const{ECU}.
\end{description}

The \const{ECU} element contains one or more \const{CPU}s.


\subsubsection{\const{CPU}}

The \const{CPU} element models a computation unit (that in general can
be a generic commercial micro-controller) that is controlled by a
Real-Time Operating System (RTOS).

The current version of the \rtd\ Tool set only allows the existence of
one \const{RTOS} for each \const{CPU}. This constraint will be relaxed
in the following versions.

The \const{CPU} element has the following attributes:

\begin{description}
\item [{attribute~Name}] is the the name of the \const{CPU} inside the
  \const{ECU};
\item [{attribute~Model}] is the CPU model (e.g., ARM7TDMI, PPC z7,
  ...). This information is optional.
\end{description}

A \const{CPU} can contain one or more \const{RTOS}.


\subsubsection{\const{RTOS}}

The \const{RTOS} element is used to model the operating system that
will be installed on the CPU, and it is identified by a name that is
used during the mapping phase to specify the relation between task and
CPU.

The \const{RTOS} element has the following attribute:

\begin{description}
\item [{attribute~Name}] is the the name of the \const{RTOS} element;
\end{description}

The RTOS element contains two elements named \const{OSMODEL} and
\const{OILVAR}, that in future releases of the \rtd\ Toolset will
specify the peculiarities of the Operating System. In the current
version of the \rtd\ Tool set these elements are empty.


\paragraph{Example}

This example shows a single \const{ECU} node, that has two ARM7TDMI
CPUs each one handled by a different operating system.

\begin{lstlisting}
[...]
<ARCHITECTURAL>
[...]
  <ECU Name="ECU0">
    <CPU Name="CPU0" Model="ARM7TDMI">
      <RTOS Name="CPU0.erikaEnterprise"/> </RTOS>
    </CPU>
    <CPU Name="CPU1" Model="ARM7TDMI">
      <RTOS Name="CPU1.erikaEnterprise"/> </RTOS>
    </CPU>
  </ECU>
[...]
</ARCHITECTURAL>
[...]
\end{lstlisting}

\subsection{\const{TASK}}

The \const{TASK} element can be used to represent both an RTOS task or
and RTOS ISR.\footnote{When running the schedulability analysis, the
  \rtd\ Tool set handles task and ISRs in the same way.  When
  generating RTOS configuration files, tasks and ISRs are considered
  in a different way.}

Each \const{TASK} element is characterized by a set of parameters
(e.g., its priority, period or deadline), that in general depends on
the Application Modes.

Moreover, a \const{TASK} contains some information useful for the OIL
standard. These informations are not used in the current version of
the \rtd\ Tool set.

The \const{TASK} element has the following attributes:

\begin{description}
\item [{attribute~Name}] is the name of the \const{TASK} element;
\item [{attribute~Type}] specifies whether the \const{TASK} element
  represents a task or an ISR in the final implementation.
\end{description}

The \const{TASK} element may contain the following elements:


\subsubsection{\const{SCHEDULING}}

The \const{SCHEDULING} element is used to specify the scheduling
parameters of a task.

The \const{SCHEDULING} element has the following attributes:

\begin{description}
\item [{attribute~Priority}] is used to store informations about the
  Task priority. Lower values are considered to be lower priority in
  the system; bigger numbers refer to higher priorities
  \footnote{Please note that this priority assignment extends the OSEK
    Priority assignment used in OIL, where 0 is considered the lowest
    priority and bigger number correspond to higher priorities.};
\item [{attribute~Threshold}] is used to store informations about Task
  Threshold (supposing that the Kernel specifies a Preemption
  Threshold scheduling algorithm).
  %FIXME - bug 146 on the old Evidence bugzilla
\item [{attribute~PreemptionGroupName}] is used to store the
  Non-Preemption Group of a Task. 
\item [{attribute~ModeRef}] is used to specify to which Mode these scheduling
  parameters are related.
\end{description}


\subsubsection{\const{ACTIVATION}}

This element is used to specify the activation parameters of a task.
For example, this element specifies if the task is activated
periodically or not, if it can have pending activations, and moreover
it specifies other activation parameters.

The \const{ACTIVATION} element has the following attributes:
\begin{description}
\item [{attribute~Type}] is the type of the Task/ISR. Can be
  {}``\const{sporadic}'', {}``\const{periodic}'',
  {}``\const{aperiodic}''.
% FIXMEE ISRs should have the same types
\item [{attribute~ActNumber}] is the number of pending activations for
  a task. This value is used for example in RTOS like those compliant
  with OSEK BCC2/ECC2 conformance classes, that can admit more than
  one pending activation for tasks. Typically this value is set to 1.
\item [{attribute~Class}] is the class of the task/ISR. %
% FIXME: is the class attribute used?
\item [{attribute~Period}] is a time value that can be used to specify
  a typical period of a Task/ISR. In the case of periodic tasks, the
  value is the period of the task. In the case of sporadic tasks or
  ISRs, this value represents the Minimum Interarrival Time.%
% FIXME note for an ISR only the mit makes sense, not the period!
\item [{attribute~Offset}] is a time value that specifies the
  activation offset of the task. It is currently used \emph{only} for
  the purpose of scheduling analysis, and does not influence the code
  generated by the \rtd\ Toolset.%
%FIXME - this is not for ISRs
\item [{attribute~Deadline}] is a time value that represents the
  relative deadline of a task. %
\item [{attribute~ModeRef}] is used to specify to which Mode these activation
  parameters are related.%
% FIXME check what happens if nothing is specified. The default mode should be used
\end{description}

\subsubsection{\const{RESOURCEREF}}

A \const{RESOURCEREF} element is used to group the methods of the
resources used by tasks (depending on the mode of
operation). \const{RESOURCEREF} elements are mainly used for
schedulability analysis when a partial system is specified (e.g., a
system with incomplete mapping or incomplete functional
specification). In this sense, \const{RESOURCEREF} elements are not
the main way a user should follow to specify the relationship between
tasks, methods, and mutexes. See Section \ref{sub:RESOURCE} for
details.

A \const{RESOURCEREF} element has the following attribute:
\begin{description}
\item [{attribute~ModeRef}] is used to specify to which Mode these
  \const{RESOURCEREF} element is related to.
\end{description}

A \const{RESOURCEREF} should contain one or more \const{METHODREF}
element (see Section \ref{sub:METHODREF}), that specifies the methods
used by a task that have to be protected by mutexes.


\subsubsection{\const{OILVAR}}

The \const{OILVAR} element is used to store in an RTD file the
corresponding content of an \ee\ OIL File. The idea is the following:
the OIL file contains all the information needed to generate the
configuration code of the \ee\ kernel. Most of these information are
often linked to a specific microcontroller or boards, and it is not
worth to model them in detail into an RTD or ERTD file. For this
reason, when RT-Druid imports an OIL file inside an RTD file, all the
elements which cannot be directly mapped to an RTD element are put
inside the OILVAR container.


\paragraph{Example}

This example describes a periodic TASK, with a 45~ms period, named
Writer. The task has a priority equal to 3, and it uses three
resources, Res8, Res12 and Res6, to do write operations on the first
two resources, and read operations on the latter.

\begin{lstlisting}
[...]
<ARCHITECTURAL>
[...]
  <TASK Name="Writer" Type="task">
     <SCHEDULING Priority="3" />
     <ACTIVATION Type="periodic" Period="45ms" />
     <RESOURCEREF>
        <METHODREF Name="Res8.write"  MethodName="Res8/write" />
        <METHODREF Name="Res12.write" MethodName="Res12/write" />
        <METHODREF Name="Res6.read"   MethodName="Res6/read" />
    </RESOURCEREF>
  </TASK>
[...]
</ARCHITECTURAL>
[...]
\end{lstlisting}


\subsection{\const{RESOURCE}}
\label{sub:RESOURCE}

A \const{RESOURCE} element is used to specify methods of the resources
used by tasks (depending on the mode of operation). \const{RESOURCE}
elements are mainly used for schedulability analysis when a partial
system is specified (e.g., a system with incomplete mapping).

Typically a user specifies (using the elements in the \const{MAPPING}
section) the mapping between \const{TASK} and \const{PROC} elements,
and between \const{VAR} and \const{MUTEX} elements (see Figure
\ref{fig:RESOURCE}).  Then, in the \const{FUNCTIONAL} section the
\const{PROC} elements specifies which \const{VAR} elements they
use. Using these informations the schedulability analyzer can know
which are the resources used by \const{PROC} elements (that
information is useful when doing a schedulability analysis).

\begin{figure}
  \includegraphics[width=1\columnwidth]{images/resource}
  \caption{Use of \const{RESOURCE} elements to enable schedulability
    analysis without complete mapping specified.}
  \label{fig:RESOURCE}
\end{figure}


Suppose now that the user do not have a complete system, because for
example the mapping has not been completed yet. In such a situation,
the user could not perform schedulability analysis, because there is
no way for the system to know which mutexes each task is using.

Unfortunately such a situation happens often, for example when the
user tries to import an OSEK OIL architectural specification (OIL
contains the architectural specification of a system, but not the
functional part). \const{RESOURCEREF} elements is a temporary way the
user can use to link resources to tasks directly inside the
architectural specification making schedulability analysis possible
also with a incomplete mapped system.

If both methods (mapping and \const{RESOURCE} elements) are specified
for an object (e.g. a task) in the a system, the schedulability
analyzer will always use the mapping informations regardless of what
specified in the \const{RESOURCE} elements.

The presence of the mapping relationship between \const{TASK} and
\const{PROC} elements are the discriminant part that is used by the
\rtd\ Toolset for choosing the right path to follow. That is, if the
mapping is present and a \const{VAR} has links with any mutex, it is
supposed that the \const{VAR} is a private data structure that does
not need to be protected with mutexes. That because the Toolset
subsumes that the mapping part can be either not or completely
present.

A \const{RESOURCE} element has the following attribute:

\begin{description}
\item [{attribute~Name}] is the name of the \const{RESOURCE} element.
\end{description}

The \const{RESOURCE} element can contain zero or more \const{METHOD}
elements (same as in Section \ref{sub:METHOD}) and one or more
\const{MUTEXREF} elements (see Section \ref{sub:MUTEXREF}).

A \const{RESOURCE} always has two \emph{predefined} methods named
\const{read} and \const{write}.


\paragraph{Example}

This example shows a resource \const{Res2} that is protected by mutex
\const{Mutex2}.

\begin{lstlisting}
[...]
<ARCHITECTURAL>
[...]
  <RESOURCE Name="Res2">
     <MUTEXREF MutexName="Mutex2"/>
  </RESOURCE> 
[...]
</ARCHITECTURAL>
[...]
\end{lstlisting}


\subsection{\const{BUS}}

The \const{BUS} element is used to represent the bus of the
\const{ECU}.

The \const{BUS} element is currently not used.

A \const{BUS} element has the following attributes:
\begin{description}
\item [{attribute~Name}] is the name of the bus.
\item [{attribute~Type}] is the type of the bus.
\item [{attribute~Speed}] is the speed of the bus.
\end{description}

\paragraph{Example}

No example available for the \const{BUS} element.


\subsection{\const{FRAME}}

The \const{FRAME} element is used to represent a set of messages sent
on the network, such as an OSEK COM IPDU.

The \const{FRAME} element is currently not used.

A \const{FRAME} has the following attributes:
\begin{description}
\item [{attribute~Name}] is the name of the frame.
\item [{attribute~ID}] is the identifier of the frame.
\item [{attribute~ActivationType}] is the activation type of the frame.
\item [{attribute~ActivationClass}] is the activation class of the frame.
\item [{attribute~ActivationRate}] is the activation rate of the FRAME.
\item [{attribute~Length}] is the length of the frame.
\end{description}

\paragraph{Example}

No example available for the \const{FRAME} element.


\subsection{\const{SIGNAL}}

The \const{SIGNAL} element is used to represent the behavior of OSEK
alarms, counters and events.

The \const{SIGNAL} element is currently not used.

A \const{SIGNAL} element has the following attributes:
\begin{description}
\item [{attribute~Name}] is the name of the signal.
\item [{attribute~Type}] is the type of the signal.
\end{description}

\paragraph{Example}

No example available for the \const{SIGNAL} element.


\subsection{\const{MUTEX}}

The \const{MUTEX} element is used to represent a RTOS mutex. A mutex
can be thought as a binary semaphore used to ensure mutual exclusion
between tasks (and on OSEK OS, also between ISRs). \const{MUTEX}
elements are currently used during the schedulability analysis to
compute blocking times of tasks. A \const{MUTEX} element has the
following attributes:

\begin{description}
\item [{attribute~Name}] is the name of the mutex.
\item [{attribute~Policy}] is the policy that must be used for the
  mutex.  The attribute is optional. Policy identifiers have not been
  specified yet. Currently, the behavior of the Tool set is to ignore
  the Policy attribute, considering that all the mutexes will use the
  Immediate Priority Ceiling Protocol.
\end{description}

\paragraph{Example}

This example shows how to declare a mutex named \const{Mutex2}.

\begin{lstlisting}
[...]
<ARCHITECTURAL>
[...]
  <MUTEX Name="Mutex2"/>
[...]
</ARCHITECTURAL>
[...]
\end{lstlisting}


\section{Mapping section}
\label{sec:Mapping-section}

The mapping section contains a set of elements that describes the
\emph{mapping} of the system. The mapping of the system is the result
of the mapping phase, which links the functional aspects (that is,
\const{PROC} elements, \const{VAR} elements, and so on) to the
architectural aspects of a system (that is, \const{TASK},
\const{MUTEX}es, and so on). Each architectural entity is represented
by an XML element inside the main \const{MAPPING} element. The mapping
section may contain the following elements: \const{PROCMAP},
\const{TASKMAP}, \const{VARMAP}.


\subsection{\const{PROCMAP}}

The \const{PROCMAP} element is used to represent the mapping
relationship between \const{PROC} and \const{TASK} elements: that is,
the fact that a \const{PROC} is executed by a particular
\const{TASK}. In general, different \const{PROC} elements will be
mapped to a single Task. The ordering between different \const{PROC}
elements can be also specified passing an integer number as
attribute. The \const{PROCMAP} element has the following attributes:

\begin{description}
\item [{attribute~TaskRef}] is the name (path) of the \const{TASK}. If
  the \const{TASK} name contains special characters, they must be
  protected using {}``\textbackslash{}''.
\item [{attribute~ProcRef}] is the full name (path) of the
  \const{PROC}.  If the \const{PROC} name contains special characters,
  they must be protected using {}``\textbackslash{}''.
\item [{attribute~ModeRef}] is the name of the mode to which this
  mapping is related. If not specified, the default mode is used.
\item [{attribute~Order}] is the mapping order (an integer
  number). This attribute is currently not used.
\end{description}

Please note that the \const{PROCMAP} elements do not have {}``names''.


\paragraph{Example}

The \const{Guide} task is composed by three \const{PROC} contained
inside the \const{Guide} subsystem. These \const{PROC} elements have
the following names: \const{decode}, \const{feedBack} and
\const{pwmRef}.  This example shows how to specify that the
\const{Guide} task is composed by the sequence of the three
\const{PROC} elements in that particular order.

\begin{lstlisting}
[...]
<MAPPING>
[...]
  <PROCMAP ProcRef="Guide/decode"   TaskRef="Guide" Order="0"/>
  <PROCMAP ProcRef="Guide/feedBack" TaskRef="Guide" Order="1"/>
  <PROCMAP ProcRef="Guide/pwmRef"   TaskRef="Guide" Order="2"/>
[...]
</MAPPING>
[...]
\end{lstlisting}


\subsection{\const{TASKMAP}}

The \const{TASKMAP} element is used to represent the mapping
relationship between \const{TASK} and \const{RTOS} elements: that is,
the fact that a \const{TASK} is executed by a particular \const{RTOS}
on a given \const{CPU}. The \const{TASKMAP} element has the following
attributes:

\begin{description}
\item [{attribute~TaskRef}] is the name (path) of the \const{TASK}. If
  the \const{TASK} name contains special characters, they must be
  protected using {}``\textbackslash{}''.
\item [{attribute~RtosRef}] is the name of the \const{RTOS}. The name
  of the RTOS is NOT a pathname.
\item [{attribute~ModeRef}] is the name of the mode to which this
  mapping is related. If not specified, the default mode is used.
\end{description}

Please note that the \const{TASKMAP} elements does not have {}``names''.


\paragraph{Example}

This example shows how the \const{Guide} task is mapped to the
\const{RTOS} identified by the name \const{CPU0.erika}.

\begin{lstlisting}
[...]
<MAPPING>
[...]
  <TASKMAP rtosRef ="CPU0.erika" TaskRef="Guide"/>
[...]
</MAPPING>
[...]
\end{lstlisting}


\subsection{\const{VARMAP}}

The \const{VARMAP} element is used to represent the mapping
relationship between \const{VAR} and \const{MUTEX} elements, or
between a \const{VAR} element and \const{FRAME}/\const{BUS} elements.

The idea is that a data structure needs some kind of mutual exclusion
protection that is provided by a \const{RTOS} by means of some sort of
binary semaphore (in case of a shared memory paradigm) or by means of
some message that needs to be transmitted on a bus (in case of a
message passing paradigm).

The \const{TASKMAP} element has the following attributes:
\begin{description}
\item [{attribute~VarRef}] is the name (path) of the \const{VAR}
  element.  This attribute is mandatory. If the \const{VAR} name
  contains special characters, they must be protected using
  {}``\textbackslash{}''.
\item [{attribute~FrameRef}] is the name of the \const{FRAME} element
  to which the \const{VAR} is mapped. This attribute is optional. If
  specified, also the \const{BUS} element must be specified.
\item [{attribute~BusRef}] is the name of the \const{BUS} element to
  which the \const{VAR} is mapped. This attribute is optional. If
  specified, also the \const{FRAME} element must be specified.
\item [{attribute~MutexRef}] is the name (path) of the \const{MUTEX}
  element to which the \const{VAR} is mapped. This attribute is
  optional.  If the \const{MUTEX} name contains special characters,
  they must be protected using {}``\textbackslash{}''.
\item [{attribute~ModeRef}] is the name of the mode to which this
  mapping is related. If not specified, the default mode is used.
\end{description}

\paragraph{Example}

This example shows 2 global variables named \const{OmegaRef1} and
\const{OmegaRef2}, and two variables local to the \const{Guide}
subsystem, named \const{img1} and \const{img2}. These variables are
protected by two distinct mutexes: \const{InputMutex} that protects
\const{img1} and \const{img2}, and \const{SpeedMutex} that protects
\const{omegaRef1} and \const{omegaRef2}.

\begin{lstlisting}
[...]
<MAPPING>
[...]
  <VARMAP VarRef="Guide/img1" MutexRef="InputMutex"/>
  <VARMAP VarRef="Guide/img2" MutexRef="InputMutex"/>

  <VARMAP VarRef="omegaRef1" MutexRef="SpeedMutex"/>
  <VARMAP VarRef="omegaRef2" MutexRef="SpeedMutex"/> 
[...]
</MAPPING>
[...]
\end{lstlisting}



\section{Annotation section}
\label{sec:Annotation-section}

The annotation section contains a set of elements that describes
information about the actual performance of the system being modeled.

The idea is that the system is modeled, mapped, and then tested on the
final target. Once a test application have been run, some performance
results (such ad example the task execution times) are back-annotated
inside the model using the \const{ANNOTATION} section. Back-annotated
informations are then used for other purposes, such as schedulability
analysis or scheduling simulation.

The current version of the \rtd\ Tool set needs back-annotated
informations for the purpose of scheduling analysis.

The \const{ANNOTATION} section may contain only the element
\const{EXECTIME}.


\subsection{\const{EXECTIME}}

The exectime element is used to describe the execution time of an
entity in the system. Supported entities for \const{EXECTIME} are
\const{TASK} elements, \const{PROC} elements, \const{METHOD} elements
and \const{RESOURCE} \const{METHOD} elements.

The \const{EXECTIME} element has the following attributes:
\begin{description}
\item [{attribute~Ref}] is the identifier (path) of the element which
  the \const{EXECTIME} is referred to. Please note that when
  specifying the path you must not specify the section name (e.g.,
  ARCHITECTURAL), because it is implicit in the Type attribute.
\item [{attribute~Type}] is the type of element to which the
  \const{EXECTIME} element is referred to. Available types are:

  \begin{description}
  \item [{task}] The Ref attribute refers to a \const{TASK} name. The
    back annotation gives information about an entire \const{TASK}
    instance.
  \item [{proc}] The Ref attribute refers to a \const{PROC} name. The
    back annotation gives information about a \const{PROC} execution
    time.
    % FIXME: to be better specified which is the method to which the
    % annotation is referring
  \item [{method}] The Ref attribute refers to a specific
    \const{METHOD} of a \const{VAR} or of a \const{PROC}.
  \item [{resource\_method}] The Ref attribute refers to a
    \const{METHOD} of an architectural element \const{RESOURCE}.
  \end{description}
  
\item [{attribute~ModeRef}] is the name of the mode to which this
  mapping is related. If not specified, the default mode is used.
\end{description}

The current version of the schedulability analyzer use the Task exec
times as informations for the execution time of tasks. Then, it uses
the execution time of methods linked to \const{VAR} and
\const{RESOURCE} elements as the information from where the blocking
times are computed.

The \const{EXECTIME} element can contain the following elements:
\const{WORST}, \const{BEST}, \const{DISTRIBUTION}.

The current version of the \rtd\ Toolset uses only the \const{WORST}
element.


\subsubsection{\const{WORST}}

This element can be used to specify the worst case execution time of
the element being measured by the \const{EXECTIME}.

The \const{WORST} element has only one attribute:
\begin{description}
\item [{attribute~Value}] is the worst case execution time.
\end{description}

\subsubsection{\const{BEST}}

This element can be used to measure the best case execution time of
the element being measured by the \const{EXECTIME}.

The current version of the \rtd\ Tool set does not use this element.

The \const{BEST} element has only one attribute:
\begin{description}
\item [{attribute~Value}] is the minimum measured execution time.
\end{description}

\subsubsection{\const{DISTRIBUTION}}

This element can be used to represent the probabilistic distribution
of the execution time of the element being measured by the
\const{EXECTIME}.  The current version of the \rtd\ schedulability
analysis does not use this element. The \const{DISTRIBUTION} element
has only one attribute:

\begin{description}
\item [{attribute~Type}] is used to specify the kind of distribution
  (e.g., Gaussian, Poisson, ...).
\end{description}

The \const{DISTRIBUTION} element can contain the following elements:
\const{AVG}, \const{VARIANCE}, \const{SAMPLE}.


\paragraph{\const{AVG}}

The \const{AVG} element is used to contain an average execution time.
It only has an attribute \textbf{Value} that contains that
information.  Only zero or one element of this kind can appear inside
a \const{DISTRIBUTION} element.


\paragraph{\const{VARIANCE}}

The \const{VARIANCE} element is used to contain the variance of the
execution time. It only has an attribute \textbf{Value} that contains
that information. Only zero or one element of this kind can appear
inside a \const{DISTRIBUTION} element.


\paragraph{\const{SAMPLE}}

The \const{SAMPLE} element is used to represent a discrete probability
distribution of the execution times. Each \const{SAMPLE} element
represents a sample of the probability distribution. For that reason,
the element has two attributes, \textbf{Value} and
\textbf{Probability}, that are used to describe the value of the
sample and its probability.

Of course, zero or more \const{SAMPLE} elements can appear inside a
\const{DISTRIBUTION} element.


\paragraph{Example}

The following example shows how to annotate the execution time of the
\const{read} and \const{write} methods provided by variable
\const{img1} internal to the \const{Guide} subsystem. The annotation
also shows the amount of time the \const{Guide} task has executed on
the CPU.

\begin{lstlisting}
[...]
<ANNOTATION>
[...]
  <EXECTIME Ref="Guide/img1/read" Type="METHOD">
     <WORST Value="8us" />
  </EXECTIME>
  <EXECTIME Ref="Guide/img1/write" Type="METHOD">
     <WORST Value="10us" />
  </EXECTIME> 

  <EXECTIME Ref="Guide" Type="TASK">
     <WORST Value="0.5ms" />
  </EXECTIME> 
[...]
</ANNOTATION>
[...]
\end{lstlisting}


\section{Schedulability section}
\label{sec:Schedulability-section}

The schedulability section contains a set of elements that store the
result of the schedulability analysis. The idea is that the \rtd\ Tool
set (once the user has provided all the necessary informations about
tasks, methods, resources, ...) is able to compute some
schedulability analysis (see Chapter
\ref{cha:Schedulability-analysis}) that returns two kind of
informations:
% FIXME - ref not existing

\begin{itemize}
\item a textual report;
\item a detailed (for each \const{RTOS}, for each \const{TASK}) schedulability
information.
\end{itemize}

The schedulability section reports, for each application mode, a
scheduling \emph{scenario}, that is, the schedulability information
for the system.  The system retains the last detailed schedulability
informations and all the reports which have been generated. The
schedulability section is composed by a sequence of
\const{SCHEDULINGSCENARIO} elements.


\subsection{\const{SCHEDULINGSCENARIO}}

A scheduling scenario represents the system schedulability
information.  A scheduling scenario is explicitly referred to a single
application mode. Only a scheduling scenario may exist for a given
application mode.

The \const{SCHEDULINGSCENARIO} element has only one attribute:

\begin{description}
\item [{attribute~ModeRef}] is the application Mode to which the
  scenario refers to.
\end{description}

The \const{SCHEDULINGSCENARIO} element can contain one ore more of the
following elements: \const{REPORT}, \const{CPUSCHED},
\const{TASKSCHED}.


\subsubsection{\const{REPORT}}

The \const{REPORT} element is an unformatted text container that
contains the schedulability report produced by the schedulability
tool. More than one report can be present for the same application
mode.


\subsubsection{\const{CPUSCHED}}

The \const{CPUSCHED} element gives some overall scheduling
informations about an \const{RTOS} mapped to a \const{CPU}. The
\const{CPUSCHED} element contains the following attributes:

\begin{description}
\item [{attribute~CpuRef}] is the \const{CPU} name which this element
  refers to. \const{CpuRef} is not a pathname, because there is only a
  namespace for \const{CPU}s.
\item [{attribute~Utilization}] is the utilization factor of the
  taskset allocated to the \const{CPU}.
\item [{attribute~SpeedFactor}] is the speed factor of the taskset
  allocated to the \const{CPU}.
\item [{attribute~Boundary}] is the theoretical $U_{lub}$boundary for
  the taskset.
\item [{attribute~Schedulable}] is the answer of the schedulability
  test.  Possible values are {}``\emph{yes}'' and {}``\emph{no}''.
\end{description}

\subsubsection{\const{TASKSCHED}}

The \const{TASKSCHED} element gives some overall scheduling
informations about a task.

The \const{TASKSCHED} element contains the following attributes:

\begin{description}
\item [{attribute~TaskRef}] is the task (path) name which this element
  refers to. If the \const{TASKREF} name contains special characters,
  they must be protected using {}``\textbackslash{}''.
\item [{attribute~Utilization}] is the utilization of the task.
\item [{attribute~CDelta}] is the amount of computation time the task
  should decrease to let the task set being schedulable.
\item [{attribute~TDelta}] is the amount of time the period of a task
  should increase to let the task set being schedulable.
\item [{attribute~Schedulable}] is the answer of the schedulability
  test.  Possible values are {}``\emph{yes}'' and {}``\emph{no}''.
\item [{attribute~ResponseTime}] is the worst case response time of a
  task.
\end{description}

\paragraph{Example}

A detailed explanation of the report and on the results can be
obtained running the examples provided with the \rtd\ Toolset. The
meaning of each field is described in detail in Chapter
\ref{cha:Schedulability-analysis}.
% FIXME ref


\section{Conventions and references}

The XML file structure defined in this chapter is written using a set
of conventions that specifies how things should be written inside some
attribute or element descriptions. Moreover, some elements (e.g.,
\const{METHOD} elements) are common to different element
specifications (e.g., a \const{VAR} and a \const{PROC} can have a
\const{METHOD} element). Finally, some elements often need to refer
other elements (for example, to express that {}``a \const{PROC} is
using a \const{METHOD} provided by a \const{VAR}''). For that reason,
specific XML elements have to be used (e.g., a \const{METHODREF}), and
the meaning of these elements is simply to refer other elements.

This section lists all these reference elements, conventions, and
shared elements.


\subsection{Name convention}

The current implementation of the \rtd\ Toolset maps the different
entities in the system under analysis (e.g., Tasks, CPUs, execution
times, ...) to XML elements. Each XML element will typically
refer to a different system entity. Entities have a name specified in
the \const{name} attribute. Two XML elements of the same kind $\alpha$
and $\beta$ with the same \const{name} attribute value refer to the
same system entity. In case two (or more) declarations exist, the tool
will try to merge the parameter definitions in the declarations.  If
the same parameter is assigned different values in different
declarations (in case of conflicts), the last one that appears in the
input file will prevail, but \underbar{no error message will be
  provided}.

Future versions of the tool set will provide error messages to inform
the user about the overwriting of different system parameters.


\subsection{Time references}

Throughout this document, whenever a timing reference is needed, the
user can specify a timing unit chosen from the following list:

\begin{description}
\item [{h}] hours
\item [{m}] minutes
\item [{s}] seconds
\item [{ms}] milliseconds
\item [{us}] microseconds
\item [{ns}] nanoseconds
\end{description}

If not specified, milliseconds (\textbf{ms}) are used as default
timing unit.


\subsubsection{Example}

Example of time units are:
\begin{description}
\item [1ms] one millisecond;
\item [1m] one minute;
\item [60001ms] one minute plus one millisecond.
\end{description}

\subsection{Paths to elements}
\label{sub:Paths-to-elements}

The XML structure used by the \rtd\ Tool set describe a tree of XML
elements representing the various elements in a given system.
Throughout this document, whenever a reference to an existing element
is needed, the entire path to that element should be used.

The path should contain the values of the \const{Name} attribute
separated by the \emph{slash} \emph{character} {}``/''. As an example,
a valid path is {}``\const{mysubsystem/myproc/write}''. If the
\emph{slash character} {}``\const{/}'' is protected by \emph{backslash
  character} {}``\const{\\}'', then in this case it is
considered part of a name in the hierarchy. As an example, the path
{}``\const{mysubsystem\\/name/myproc/write}'' has three
elements: {}``\const{mysubsystem/name}'', {}``\const{myproc}'' and
{}``\const{write}''.

The \emph{backslash character} {}``\textbackslash{}'' can be protected
with another \emph{backslash character} {}``\textbackslash{}''.  As an
example, the path
{}``\const{mysub\\\\/name/myproc/write}''
has four elements: {}``\const{mysub\\}'',
{}``\const{name}'', {}``\const{myproc}'' and {}``\const{write}''.

All others characters are considered part of a path name. A local path
refers to a local object, in the form
{}``\const{objectname/methodname}''.  A global path refers to an
object in any place of the hierarchy, in the form
{}``\const{name/.../name}''.

% FIXME: see bug 149 in the old Evidence bugzilla


\subsection{\const{METHOD}}
\label{sub:METHOD}

A method in general can be thought as a function that implement some
part of the behavior of an object. For example, a \const{PROC}
implementing a control algorithm may have different methods each one
implementing a different version of the same control rule. A
\const{VAR} can export different methods that allow to access a shared
data structure in different ways.

A \const{METHOD} element has only one attribute:
\begin{description}
\item [{attribute~Name}] this attribute specifies the \const{METHOD}
  name. Method names are \emph{unique} names for each object (as
  \const{PROC} and \const{VAR}).
\end{description}

\subsection{\const{METHODREF}}
\label{sub:METHODREF}

A \const{METHODREF} is used to explicitly model the fact that a PROC
$\alpha$ has some relationship with another \const{PROC} $\beta$
because $\alpha$ is calling a \const{METHOD} exported by $\beta$.  A
\const{METHODREF} requires the following attributes:
\begin{description}
\item [{attribute~Name}] this attribute specifies a method inside the
  functional section. Please remind that method names are unique in
  all the functional section.
\item [{attribute~MethodName}] this attribute contains a path to a
  \const{METHOD}.  Paths inside a \const{METHODREF} are local
  pathnames in the form {}``objectname/methodname'' in the case of
  \const{PROC} and \const{REQUIRED\_INTERFACE}; are global pathnames
  in the case of \const{TRIGGER}, and \const{RESOURCEREF}.
% FIXME: is this 100% correct?
\end{description}
The \const{METHODREF} does not give any kind of information about how
many times the method pointed by the \const{METHODREF} is called.


\subsection{\const{EVENTREF}}
\label{sub:EVENTREF}

\const{EVENTREF} is an element used to reference an \const{EVENT}.  An
\const{EVENTREF} has only one attribute:
\begin{description}
\item [{attribute~Name}] the Name attribute is the name of the
  \const{EVENT} which the \const{EVENTREF} is referred to.
\end{description}

\subsection{\const{MUTEXREF}}
\label{sub:MUTEXREF}

\const{MUTEXREF} is an element used to reference a \const{MUTEX}.  It
is mainly used inside \const{RESOURCES} (see Section
\ref{sub:RESOURCE} and Figure \ref{fig:RESOURCE}). A \const{MUTEXREF}
has two attributes:
\begin{description}
\item [{attribute~MutexName}] the \const{MUTEX} name to which this
  element refers. MutexName is not a pathname, because all the
  references to mutexes can be done only in the same namespace of the
  entity that requires it.
\item [{attribute~ModeRef}] is used to specify to which Mode the
  \const{MUTEXREF} object is related.
\end{description}

\section{A complete example}

As a complete example, consider the example number 5 (under the
directory \file{examples/005} of the \rtd\ directory tree). The
example describes a model of a small toy composed by a little car with
two motors directly connected to the wheels. The toy can read a shaded
path using a light sensor, following the reference position in the
path using a differential control algorithm. The toy example is
composed by 2 subsystems:

\begin{description}
\item [Guida] reads the data from an optical sensor, computing the
  optimal reference speed of the two motors following a control rule.
\item [ControlloMotore] tries to change the power set to the motors to
  minimize the difference between the optimal values computed by the
  \const{Guida} subsystem and the values read by the rotational
  sensors on the wheels.
\end{description}

All the \const{PROC} elements of the two subsystems are then mapped on
three different \const{TASK}s:

\begin{description}
\item [irq\_sensori] this \const{TASK} element is an ISR
  that reads the light sensors with a period of 1 ms.
\item [Guida] this \const{TASK} element is a task that
  computes the optimal motor speed values with a period of 1 ms.
\item [ControlloMotore] this \const{TASK} element reads the
  rotational sensors on the wheels and controls the motor speed trying
  to reach the value of the optimal speed. The Period is set to 0.05
  ms.
\end{description}

Finally, two mutexes, \const{MutexIngressi} and \const{MutexVel} are
defined to allow data exchange between the various \const{TASK}
elements.
