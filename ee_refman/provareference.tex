\documentclass[12pt,a4paper,titlepage]{scrreprt}
%\documentclass[12pt,a4paper,titlepage]{refrep}

\usepackage{evman}

\pagestyle{fancy}

\title{Example of Reference manual}

\begin{document}

\maketitle

\thispagestyle{plain}

\tableofcontents

\chapter{API reference}
\label{sec:api_reference}

This is the first chapter!! 

\section{Types}
\label{sec:types}

\section{Functions}
\label{sec:functions}

%%%%%%%%%%%%%%%%%%%%%%%%%%%%%%%%%%%%%%%%%%%%%%%%%%%%%%%%%%%%%%%%%%%%%%

\begin{function}{DeclareTask}

\synopsis{DeclareTask(TaskType TaskID);}

  \begin{fundescription}
DeclareTask serves as an external declaration of a task. The function
and use of this service are similar to that of the external
declaration of variables.
  \end{fundescription}

  \begin{funparameters}
    \fpar{TaskID}{Task reference}
  \end{funparameters}

  \begin{funconformance}
    BCC1, BCC2, ECC1, ECC2
  \end{funconformance}
\end{function}

%%%%%%%%%%%%%%%%%%%%%%%%%%%%%%%%%%%%%%%%%%%%%%%%%%%%%%%%%%%%%%%%%%%%%%


%%%%%%%%%%%%%%%%%%%%%%%%%%%%%%%%%%%%%%%%%%%%%%%%%%%%%%%%%%%%%%%%%%%%%%

\begin{function}{ActivateTask}

\synopsis{StatusType ActivateTask(TaskType TaskID);}

  \begin{fundescription}
    The task \vr{TaskID} is transferred from the suspended state into the
    ready state. The operating system ensures that the task code is being
    executed from the first statement.
  \end{fundescription}

  \begin{funparameters}
    \fpar{TaskID}{Task reference}
  \end{funparameters}

  \begin{funreturn}
  \fret{E_OK}{No error}
  \fret{E_OS_LIMIT}{Too many task activations of \vr{TaskID}}
  \fret{E_OS_ID}{(Extended) Task \vr{TaskID} is invalid}.
  \end{funreturn}

  \begin{funconformance}
    BCC1, BCC2, ECC1, ECC2
  \end{funconformance}
\end{function}

%%%%%%%%%%%%%%%%%%%%%%%%%%%%%%%%%%%%%%%%%%%%%%%%%%%%%%%%%%%%%%%%%%%%%%
\begin{function}{TerminateTask}

\synopsis{StatusType Terminatetask(void)}

  \begin{fundescription}
    This service causes the termination of the calling task. The
    calling task is transferred from the running state into the
    suspended state.

    An internal resource assigned to the calling task is automatically
    released. Other resources occupied by the task shall have been
    released before the call to \fn{TerminateTask}. If a resource is
    still occupied in standard status the behaviour is undefined.

    If the call was successful, \fn{TerminateTask} does not return to
    the call level and the status can not be evaluated.  If the
    version with extended status is used, the service returns in case
    of error, and provides a status which can be evaluated in the
    application.

    If the service \fn{TerminateTask} is called successfully, it
    enforces a rescheduling.

    Ending a task function without call to \fn{TerminateTask} or
    \reffun{ChainTask} is strictly forbidden and may leave the system in
    an undefined state.
  \end{fundescription}

  \begin{funparameters}
  \item No parameters
  \end{funparameters}

  \begin{funreturn}
    \fret{no return}{(Standard)}
    \fret{E_OS_RESOURCE}{(Extended) Task still occupies resources}
    \fret{E_OS_CALLEVEL}{(Extended) Call at interrupt level}.
  \end{funreturn}

  \begin{funconformance}
    BCC1, BCC2, ECC1, ECC2
  \end{funconformance}
\end{function}

%%%%%%%%%%%%%%%%%%%%%%%%%%%%%%%%%%%%%%%%%%%%%%%%%%%%%%%%%%%%%%%%%%%%%%

%%%%%%%%%%%%%%%%%%%%%%%%%%%%%%%%%%%%%%%%%%%%%%%%%%%%%%%%%%%%%%%%%%%%%%

\begin{function}{ChainTask}

\synopsis{StatusType ChainTask(TaskType TaskID);}

  \begin{fundescription}
This service causes the termination of the calling task. After
termination of the calling task a succeeding task \vr{TaskID} is
activated. Using this service, it ensures that the succeeding task
starts to run at the earliest after the calling task has been
terminated.

If the succeeding task is identical with the current task, this does
not result in multiple requests. The task is not transferred to the
suspended state, but will immediately become ready again.  An internal
resource assigned to the calling task is automatically released, even
if the succeeding task is identical with the current task. Other
resources occupied by the calling shall have been released before
\fn{ChainTask} is called. If a resource is still occupied in standard
status the behaviour is undefined.

If called successfully, \fn{ChainTask} does not return to the call level
and the status can not be evaluated.

In case of error the service returns to the calling task and provides
a status which can then be evaluated in the application.

If the service \fn{ChainTask} is called successfully, this enforces a
rescheduling.  Ending a task function without call to
\reffun{TerminateTask} or \fn{ChainTask} is strictly forbidden and may
leave the system in an undefined state.

If \const{E_OS_LIMIT} is returned the activation is ignored.  When an extended
task is transferred from suspended state into ready state all its
events are cleared.
  \end{fundescription}

  \begin{funparameters}
    \fpar{TaskID}{Task reference}
  \end{funparameters}

  \begin{funreturn}
  \fret{E_OK}{No error}
  \fret{E_OS_LIMIT}{Too many task activations of \vr{TaskID}}
  \fret{E_OS_ID}{(Extended) Task \vr{TaskID} is invalid}.
  \end{funreturn}

  \begin{funconformance}
    BCC1, BCC2, ECC1, ECC2
  \end{funconformance}
\end{function}

%%%%%%%%%%%%%%%%%%%%%%%%%%%%%%%%%%%%%%%%%%%%%%%%%%%%%%%%%%%%%%%%%%%%%%

\end{document}
