\chapter{API reference}
\label{sec:api_reference}

\section{Introduction}
The \ee\ Operating System provides a interface according to OSEK/VDX
specification, version 2.2.3. In addition, extended that specification
allowing the development of multicore applications.

The interface proposed is suited for small 8 to 32 bit
architectures, and proposes an environment where tasks can execute
concurrently exchanging data with a shared memory paradigm. Support
for limited synchronization primitives is also provided. 

Tasks in \ee\ are scheduled according to fixed priorities, and share
resources using the Immediate Priority Ceiling protocol.

On top of task execution there are interrupts, that always preempt the
running task to execute urgent operations required by peripherals, or,
in case of a multicore system, by other CPUs.

\ee\ can be configured to allow a normal (named Standard Status) or
extended (named Extended Status) error recognition.

Extended status is mainly used to check as many errors as possible
during the Debug phase. Once the application has been debugged, some
error checking can be disabled, saving execution time and code
footprint. Standard and Extended status are enabled in the OIL
configuration file. When describing \ee\ primitive return values, an
``(Extended)'' token near the error description means that the error
is raised only when the system runs with extended status.





\subsection{Conformance Classes}
To further reduce the overall footprint of the kernel, \ee\ provides
subsets of the OS API. These subsets are called {\em Conformance
Classes}. There are four Conformance Classes, named BCC1, BCC2, ECC1,
ECC2.

Conformance classes starting with the letter {\bf B} (that is, BCC1
and BCC2) only support Basic Tasks. Conformance classes starting with
the letter {\bf E} (that is, ECC1 and ECC2) support both Basic Tasks
and Extended Tasks. 

BCC1 and BCC2 conformance classes are designed to be as small as
possible, and in particular these conformance classes are the most
suited to implement small concurrent systems with little RAM
footprint, thanks to the stack sharing that can be obtained between
basic tasks.

ECC1 and ECC2 conformance classes are designed to be higher end
conformance classes supporting synchronization primitives that implies
the usage of separate stacks, and of course higher OS overheads due to
the stack change mechanisms.

Conformance classes ending with the number 1 (that is, BCC1 and ECC1)
does not store pending activations. The ready queue implementation is
done using a linear queue with $O(n)$ access time where n is the
number of tasks in the ready queue (using a linear queue allow the
minimization of the overall OS RAM footprint). Priorities are stored
as bit fields, implying that the number of different priorities in the
system is limited by the register width (for example, a 32 bit CPU can
have up to 32 different priorities).

Conformance classes ending with the number 2 (that is, BCC2 and ECC2)
allow a task to store one or more pending activations (the maximum
number of pending activation is specified in the \vr{ACTIVATION}
attribute of a Task in the OIL Specification). The ready queue
implementation is done using a bit field that exposes an $O(1)$
complexity that is independent on the number of tasks in the
system. The BCC2 Conformance class have only up to 8 different task
priorities, whereas ECC2 has up to 16 different task priorities.

More than one task with the same priority can coexist at the same time
for all the four conformance classes.

Resources, Alarms, and Application modes are codified as integers, so
there can be up to $2^n-1$ different entities, where $n$ is the number
of bits of the CPU register width (e.g., 32 for a 32 bit CPU).

Figure \ref{tab:minimum-requirements} shows the current limits with
respect to the number of OS objects allowed in the system.

%%%%%%%%%%%%%%%%%%%%%%%%%%%%%%%%%%%%%%%%%%%%%%%%%%%%%%%%%%%%
% INIZIO PARTE IMPORTATA DA minimumrequirements.lyx - NON MODIFICARE
% MODIFICATE IL LYX, ESPORTATE e commentate all'inizio ed alla fine!!!
\begin{table}
\begin{centering}
\begin{tabular}{|>{\centering}p{4cm}|>{\centering}p{1.4cm}|>{\centering}p{1.4cm}|>{\centering}p{1.4cm}|>{\centering}p{1.4cm}|}
\hline 
Feature & BCC1 & BCC2 & ECC1 & ECC2\tabularnewline
\hline 
Multiple requesting of task activation & no & yes & no & BT: yes; ET: no\tabularnewline
\hline 
Number of tasks which are not in the \emph{suspended} state & \multicolumn{2}{c|}{at least 255} & \multicolumn{2}{>{\centering}p{2.8cm}|}{at least 255 (any combination of BT/ET)}\tabularnewline
\hline 
More than one task per priority & yes & yes & yes & yes\tabularnewline
\hline 
Number of events per task & \multicolumn{2}{c|}{-} & \multicolumn{2}{c|}{\emph{nbits}}\tabularnewline
\hline 
Number of task priorities & \emph{nbits} & 16 & \emph{nbits} & 16\tabularnewline
\hline 
Resources & \multicolumn{4}{>{\centering}p{5.6cm}|}{$2^{nbits}-1$ (including RES\_SCHEDULER)}\tabularnewline
\hline 
Internal Resources & \multicolumn{4}{>{\centering}p{5.6cm}|}{no limit (they are automatically computed by the OIL Compiler)}\tabularnewline
\hline 
Alarms & \multicolumn{4}{c|}{$2^{nbits}-1$}\tabularnewline
\hline 
Application modes & \multicolumn{4}{c|}{$2^{nbits}-1$}\tabularnewline
\hline 
\end{tabular}
\par\end{centering}

\caption{\label{tab:minimum-requirements}This table lists the current limits
with respect to the number of OS objects allowed in the system. \emph{nbits}
means the number of bits stored in a microcontroller register (16
on a 16 bit machine, 32 on a 32 bit machine, with a minimum value
of 16).}
\end{table}

% FINE PARTE IMPORTATA DA TABELLAOSEK.LYX - NON MODIFICARE
%%%%%%%%%%%%%%%%%%%%%%%%%%%%%%%%%%%%%%%%%%%%%%%%%%%%%%%%%%

\subsection{Available primitives}
\ee\ provides a set of primitives that can be called in different
situations. The complete list of primitives is listed in Table
\ref{tab:api-restrictions}, together with the locations where it is
legal to call these functions.

The Background Task is the context where the application \fn{main()}
function is executed, after the call to \reffun{StartOS}. Please note that by
default the current implementation of the \reffun{StartOS} primitive never
returns. To make it return, tou need to add the following option to the OIL
file:

\begin{lstlisting}
  EE_OPT = __OO_STARTOS_OLD__
\end{lstlisting}

%%%%%%%%%%%%%%%%%%%%%%%%%%%%%%%%%%%%%%%%%%%%%%%%%%%%%%%%%%%%
% INIZIO PARTE IMPORTATA DA TabellaOSEK.lyx - NON MODIFICARE
% MODIFICATE IL LYX, ESPORTATE e commentate all'inizio ed alla fine!!!
%
\begin{table}
\begin{centering}\begin{tabular}{|c|c|c|c|c|c|}
\hline 
Service&
\begin{sideways}
Background Task%
\end{sideways}&
\begin{sideways}
Task%
\end{sideways}&
\begin{sideways}
ISR1%
\end{sideways}&
\begin{sideways}
ISR2%
\end{sideways}&
\begin{sideways}
Alarm Callback%
\end{sideways}\tabularnewline
\hline
\hline 
\reffun{ActivateTask}&
$\surd$&
$\surd$&
&
$\surd$&
\tabularnewline
\hline 
\reffun{Schedule}&
&
$\surd$&
&
&
\tabularnewline
\hline 
\reffun{GetResource}&
&
$\surd$&
&
$\surd$&
\tabularnewline
\hline 
\reffun{ReleaseResource}&
&
$\surd$&
&
$\surd$&
\tabularnewline
\hline 
\reffun{CounterTick}&
&
$\surd$&
&
$\surd$&
\tabularnewline
\hline 
\reffun{GetAlarm}&
$\surd$&
$\surd$&
&
$\surd$&
\tabularnewline
\hline 
\reffun{SetRelAlarm}&
$\surd$&
$\surd$&
&
$\surd$&
\tabularnewline
\hline 
\reffun{SetAbsAlarm}&
$\surd$&
$\surd$&
&
$\surd$&
\tabularnewline
\hline 
\reffun{CancelAlarm}&
$\surd$&
$\surd$&
&
$\surd$&
\tabularnewline
\hline 
\reffun{InitSem}&
$\surd$&
$\surd$&
&
$\surd$&
\tabularnewline
\hline 
\reffun{WaitSem}&
&
$\surd$&
&
&
\tabularnewline
\hline 
\reffun{TryWaitSem}&
$\surd$&
$\surd$&
&
$\surd$&
\tabularnewline
\hline 
\reffun{PostSem}&
$\surd$&
$\surd$&
&
$\surd$&
\tabularnewline
\hline 
\reffun{GetValueSem}&
$\surd$&
$\surd$&
&
$\surd$&
\tabularnewline
\hline
\reffun{GetTime}&
$\surd$&
$\surd$&
&
$\surd$&
\tabularnewline
\hline
\end{tabular}\par\end{centering}


\caption{\label{tab:api-restrictions}This table lists the environments where
primitives can be called. }
\end{table}

% FINE PARTE IMPORTATA DA TABELLAOSEK.LYX - NON MODIFICARE
%%%%%%%%%%%%%%%%%%%%%%%%%%%%%%%%%%%%%%%%%%%%%%%%%%%%%%%%%%





%% 
%% Basic template 
%%
%% \begin{function}{FunctionName}
%% \synopsis{}
%%   \begin{fundescription}
%%   \end{fundescription}
%%   \begin{funparameters}
%%     \fpar{}{}
%%   \end{funparameters}
%%   \begin{funreturn}
%%   \fret{}{}
%%   \fret{}{(Extended)}.
%%   \end{funreturn}
%%   \begin{funconformance}
%%     BCC1, BCC2, ECC1, ECC2
%%   \end{funconformance}
%% \end{function}


\pagebreak

\section{Constants}
\label{sec:constants}

This is a list of the \ee\ constants that can be used by the developer
for writing applications.

\begin{constant}{Error List}
  \begin{constantdescription}
    This is the list of the error values returned by the kernel primitives:
    \begin {lstlisting}
#define E_OK          0
#define E_OS_ACCESS   1
#define E_OS_CALLEVEL 2
#define E_OS_ID       3
#define E_OS_LIMIT    4
#define E_OS_NOFUNC   5
#define E_OS_RESOURCE 6
#define E_OS_STATE    7
#define E_OS_VALUE    8
#define E_OS_SYS_INIT 9
    \end{lstlisting}
  \end{constantdescription}
\end{constant}

\begin{constant2}{INVALID\_TASK}{INVALID_TASK}
  \begin{constantdescription}
    This constant represent an invalid task ID, and is returned by
    \reffun{GetTaskID} when the function is called and no task is
    running.
  \end{constantdescription}
\end{constant2}

\begin{constant}{OSService IDs}
  \begin{constantdescription}
    This is the list of Service IDs values that can be returned by
    \reffun{OSErrorGetServiceId}:
    \begin {lstlisting}
#define OSServiceId_ActivateTask      1U
#define OSServiceId_TerminateTask     2U
#define OSServiceId_ChainTask         3U
#define OSServiceId_Schedule          4U
#define OSServiceId_GetTaskID         5U
#define OSServiceId_GetTaskState      6U
#define OSServiceId_GetResource       7U
#define OSServiceId_ReleaseResource   8U
#define OSServiceId_SetEvent          9U
#define OSServiceId_ClearEvent        10U
#define OSServiceId_GetEvent          11U
#define OSServiceId_WaitEvent         12U
#define OSServiceId_GetAlarmBase      13U
#define OSServiceId_GetAlarm          14U
#define OSServiceId_SetRelAlarm       15U
#define OSServiceId_SetAbsAlarm       16U
#define OSServiceId_CancelAlarm       17U
#define OSServiceId_IncrementCounter  18U
#define OSServiceId_GetCounterValue   19U
#define OSServiceId_GetElapsedValue   20U
#define OSServiceId_StartOS           21U
#define OSServiceId_ForceSchedule     22U
    \end{lstlisting}
    Please note that the primitives:
    \begin{itemize}                              \vspace{-2mm}
      \item \reffun{DisableAllInterrupts}        \vspace{-2mm}
      \item \reffun{EnableAllInterrupts}         \vspace{-2mm}
      \item \reffun{SuspendAllInterrupts}        \vspace{-2mm}
      \item \reffun{ResumeAllInterrupts}         \vspace{-2mm}
      \item \reffun{SuspendOSInterrupts}         \vspace{-2mm}
      \item \reffun{ResumeOSInterrupts}          \vspace{-2mm}
      \item \reffun{GetActiveApplicationMode}    \vspace{-2mm}
      \item \reffun{ShutdownOS}                  \vspace{-2mm}
    \end{itemize}
    never return an error, and for that reason they are not listed
    here.
  \end{constantdescription}
\end{constant}

\begin{constant2}{RES\_SCHEDULER}{RES_SCHEDULER}
  \begin{constantdescription}
    This is the ID of the \const{RES_SCHEDULER} resource.

    That resource exists only when \const{USERESSCHEDULER} is set to
    \const{TRUE} within the OIL configuration file. The
    \const{RES_SCHEDULER} ceiling depends on the tasks that exists in
    the system, and it is computed when \rtd\ generates the \ee\ 
    configuration code.
  \end{constantdescription}
\end{constant2}

\begin{constant}{Task States}
  \begin{constantdescription}
    This is the list of the task states a task can have during its life:
    \begin {lstlisting}
#define RUNNING   0
#define WAITING   1
#define READY     2
#define SUSPENDED 3
    \end{lstlisting}
    Please note that the \const{WAITING} state is only available in
    the conformance classes ECC1 and ECC2.
  \end{constantdescription}
\end{constant}

\begin{constant}{Counters Constants}
  \begin{constantdescription}
    For all configurated counters \rtd\ generate the return values of
    \reffun{GetAlarmbase} as constants:
    \begin{description}
      \item[OSMAXALLOWEDVALUE\_x] Maximum possible allowed value of counter x in ticks
      \item[OSTICKSPERBASE\_x]    Number of ticks required to reach a specific unit of counter x
      \item[OSMINCYCLE\_x]        Minimum allowed number of ticks for a cyclic alarm of counter x
    \end{description}

    Thus, if the counter name is known, it is not necessary to call \reffun{GetAlarmbase}.
    When system counter is configured, the constants of this counter are
    additionally accessible via the following constants:
    \begin{description}
      \item[OSMAXALLOWEDVALUE] Maximum possible allowed value of the system counter in ticks.
      \item[OSTICKSPERBASE]    Number of ticks required to reach a specific unit of the system counter.
      \item[OSMINCYCLE]        Minimum allowed number of ticks for a cyclic alarm of the system counter.
    \end{description}

    Additionally the following constant is supplied:
    \begin{description}
      \item [OSTICKDURATION]  Duration of a tick of the system counter in nanoseconds.
    \end{description}
  \end{constantdescription}
\end{constant}

\begin{constant}{OSDEFAULTAPPMODE}
  \begin{constantdescription}
    This is the default Application Mode. This value is always a valid
    Application Mode that can be passed to \reffun{StartOS}.
  \end{constantdescription}
\end{constant}


\pagebreak

%%%%%%%%%%%%%%%%%%%%%%%%%%%%%%%%%%%%%%%%%%%%%%%%%%%%%%%%%%%%%%%%%%%%%%%%%%%%




\section{Types}
\label{sec:types}

This Section contains a description of the data types used by the OS
interface of \ee. When the size of a type is specified to be of the
size of a machine register, it is intended that the type has the same
size of the CPU general purpose register.

\begin{type}{AlarmBaseType}
  \begin{typedescription}
    This structure is used to store the basic information about
    Counters. It has the following fields:
    \begin{description}
    \item[TickType maxallowedvalue] Is the maximum allowed count value
      in ticks for a counter.
    \item[TickType ticksperbase] It is the number of ticks required to
      reach a counter-specific significant unit.
    \item[TickType mincycle] It is the smallest allowed value for the
      \vr{cycle} parameter of the primitives 
      \reffun{SetRelAlarm}/\reffun{SetAbsAlarm}. This field is only
      present when Extended status is selected.
    \end{description}
  \end{typedescription}
\end{type}

\begin{type}{AlarmBaseRefType}
  \begin{typedescription}
    This is a pointer to \reftype{AlarmBaseType}.
  \end{typedescription}
\end{type}

\begin{type}{AlarmType}
  \begin{typedescription}
    This (signed) type is used to store Alarm IDs, and it has the size
    of a register.
  \end{typedescription}
\end{type}

\begin{type}{AppModeType}
  \begin{typedescription}
    This (unsigned) type is used to store Application Mode IDs, and it
    has the size of a register.
  \end{typedescription}
\end{type}

\begin{type}{CounterType}
  \begin{typedescription}
    This (signed) type is used to store Counter IDs, and it has the size
    of a register.
  \end{typedescription}
\end{type}

\begin{type}{EventMaskType}
  \begin{typedescription}
    This (unsigned) type is used to store Event masks as bit fields,
    and it has the size of a register.
  \end{typedescription}
\end{type}

\begin{type}{EventMaskRefType}
  \begin{typedescription}
    This is a pointer to \reftype{EventMaskType}.
  \end{typedescription}
\end{type}

\begin{type}{OSServiceIdType}
  \begin{typedescription}
    This unsigned 8-bit integer type is used to store Service IDs, and it
    is used within the \reffun{OSErrorGetServiceId}.
  \end{typedescription}
\end{type}

\begin{type}{ResourceType}
  \begin{typedescription}
    This (unsigned) type is used to store Resource ID values, and it
    has the size of a register.
  \end{typedescription}
\end{type}

\begin{type}{SemType}
  \begin{typedescription}
    This type is a structure storing the information related to a
    counting semaphore.
  \end{typedescription}
\end{type}

\begin{type}{SemRefType}
  \begin{typedescription}
    This is a pointer to \reftype{SemType}.
  \end{typedescription}
\end{type}

\begin{type}{StatusType}
  \begin{typedescription}
    This type is an \const{unsigned char} used to store function error
    return values.
  \end{typedescription}
\end{type}

\begin{type}{TaskType}
  \begin{typedescription}
    This (signed) type is used to store Task ID, and it has the size
    of a register.
  \end{typedescription}
\end{type}

\begin{type}{TaskRefType}
  \begin{typedescription}
    This is a pointer to \reftype{TaskType}.
  \end{typedescription}
\end{type}

\begin{type}{TaskStateType}
  \begin{typedescription}
    This (unsigned) type is used to store Task Status values, and it
    has the size of a register.
  \end{typedescription}
\end{type}

\begin{type}{TaskStateRefType}
  \begin{typedescription}
    This is a pointer to \reftype{TaskStateType}.
  \end{typedescription}
\end{type}

\begin{type}{TickType}
  \begin{typedescription}
    This (unsigned) type is used to store Counter Ticks, and it has
    the size of a register.
  \end{typedescription}
\end{type}

\begin{type}{TickRefType}
  \begin{typedescription}
    This is a pointer to \reftype{TickType}.
  \end{typedescription}
\end{type}


\pagebreak






%%%%%%%%%%%%%%%%%%%%%%%%%%%%%%%%%%%%%%%%%%%%%%%%%%%%%%%%%%%%%%%%%%%%%%



\section{Object Declarations}
The following declarations have to be used to declare Tasks, Resources,
Alarms, and Events within the application code.

\begin{function_nopb}{DeclareAlarm}
  \synopsis{DeclareAlarm (AlarmIdentifier)}
  \begin{fundescription}
    Declares an alarm.

    This declaration is currently not mandatory because alarm
    identifiers are all declared within the code generated by \rtd.
  \end{fundescription}
  \begin{funconformance}
    BCC1, BCC2, ECC1, ECC2
  \end{funconformance}
\end{function_nopb}

%%%%%%%%%%%%%%%%%%%%%%%%%%%%%%%%%%%%%%%%%%%%%%%%%%%%%%%%%%%%%%%%%%%%%%

\begin{function_nopb}{DeclareEvent}
  \synopsis{DeclareEvent(EventID)}
  \begin{fundescription}
    \fn{DeclareEvent} serves as an external declaration of an
    event. The function and use of this service are similar to that of
    the external declaration of variables.

    This declaration is currently not mandatory because event
    identifiers are all declared within the code generated by \rtd.
  \end{fundescription}
  \begin{funconformance}
    ECC1, ECC2
  \end{funconformance}
\end{function_nopb}

%%%%%%%%%%%%%%%%%%%%%%%%%%%%%%%%%%%%%%%%%%%%%%%%%%%%%%%%%%%%%%%%%%%%%%

\begin{function_nopb}{DeclareResource}
  \synopsis{DeclareResource(ResourceID)}
  \begin{fundescription}
    \fn{DeclareResource} serves as an external declaration of a
    resource. The function and use of this service are similar to that
    of the external declaration of variables.

    
    This declaration is currently not mandatory because Resource
    identifiers are all declared within the code generated by \rtd.
  \end{fundescription}
  \begin{funparameters}
    \fpar{ResourceID}{Resource Identifier}
  \end{funparameters}
  \begin{funconformance}
    BCC1, BCC2, ECC1, ECC2
  \end{funconformance}
\end{function_nopb}

%%%%%%%%%%%%%%%%%%%%%%%%%%%%%%%%%%%%%%%%%%%%%%%%%%%%%%%%%%%%%%%%%%%%%%

\begin{function_nopb}{DeclareTask}
  \synopsis{DeclareTask(TaskType TaskID);}
  
  \begin{fundescription}
    DeclareTask serves as an external declaration of a task. The
    function and use of this service are similar to that of the
    external declaration of variables.
    
    This declaration is currently not mandatory because task
    identifiers are all declared within the code generated by \rtd.
  \end{fundescription}
  
  \begin{funparameters}
    \fpar{TaskID}{Task reference.}
  \end{funparameters}
  
  \begin{funconformance}
    BCC1, BCC2, ECC1, ECC2
  \end{funconformance}
\end{function_nopb}

\pagebreak










%%%%%%%%%%%%%%%%%%%%%%%%%%%%%%%%%%%%%%%%%%%%%%%%%%%%%%%%%%%%%%%%%%%%%%

\section{Object Definitions}
The following macro have to be used when defining Tasks, ISRs and
Alarm Callbacks.

\begin{function_nopb}{ALARMCALLBACK}
  \synopsis{ALARMCALLBACK(t)}
  \begin{fundescription}
    This macro is used to declare and to define an alarm callback.
  \end{fundescription}
  \begin{funparameters}
    \fpar{t}{Name of the alarm callback.}
  \end{funparameters}
  %  \begin{funreturn}
  %    \fret{E_OK}{No error.}
  %  \end{funreturn}
  \begin{funconformance}
    BCC1, BCC2, ECC1, ECC2
  \end{funconformance}
\end{function_nopb}

%%%%%%%%%%%%%%%%%%%%%%%%%%%%%%%%%%%%%%%%%%%%%%%%%%%%%%%%%%%%%%%%%%%%%%

\begin{function_nopb}{ISR}
  %PJ: da ricontrollare come � scritta
  \synopsis{ISR(Funcname) \{...\}}
  
  \begin{fundescription}
    The ISR keyword must be used when declaring an ISR function, to
    distinguish it from other function types and from tasks.
  \end{fundescription}
  
  \begin{funconformance}
    BCC1, BCC2, ECC1, ECC2
  \end{funconformance}
\end{function_nopb}

%%%%%%%%%%%%%%%%%%%%%%%%%%%%%%%%%%%%%%%%%%%%%%%%%%%%%%%%%%%%%%%%%%%%%%

\begin{function_nopb}{TASK}
  %PJ: da ricontrollare come � scritta
  \synopsis{TASK(Funcname) \{...\}}
  
  \begin{fundescription}
    The TASK keyword must be used when declaring a TASK function.
  \end{fundescription}
  
  \begin{funconformance}
    BCC1, BCC2, ECC1, ECC2
  \end{funconformance}
\end{function_nopb}

%%%%%%%%%%%%%%%%%%%%%%%%%%%%%%%%%%%%%%%%%%%%%%%%%%%%%%%%%%%%%%%%%%%%%%









\pagebreak

%%%%%%%%%%%%%%%%%%%%%%%%%%%%%%%%%%%%%%%%%%%%%%%%%%%%%%%%%%%%%%%%%%%%%%

\section{Task Primitives}

\ee\ supports two flavors of tasks:
\begin{description}
\item[Basic Tasks] A basic task is the simplest task in \ee,
  providing concurrency together with a one-shot task model. Basic
  Tasks can share their stack to reduce the overall RAM usage.
\item[Extended Tasks] An extended task is a task that can block on
  the synchronization primitive \reffun{WaitEvent}.
\end{description}

Basic Tasks are typically implemented as normal C functions, that
executes their code and then ends. One of these executions is called
also {\em Task Instance}\index{Task Instance}. After the end of a
basic task, its stack is freed. Basic Tasks never block, and they are
the ideal kind of tasks for implementing stack sharing techniques.

Extended Tasks, on the converse, are typically implemented as a never
ending task in which each instance ends with a synchronization
implemented with a call to the \reffun{WaitEvent} primitive. Extended
tasks always have a private stack.

\rtd identifies a task as Extended when its OIL definition contains
the specification of \const{Event}s. A task without any \const{Event}
assigned is a Basic Task.

The scheduling policy of \ee\ is a Fixed Priority Scheduling with
Immediate Priority Ceiling. As a result, the following case of tasks
may be implemented:
\begin{description}
\item[Full Preemptive Task] A Full Preemptive task is a task that can
  be preempted in each instant by higher priority tasks.
\item[Non Preemptive Task] A Non Preemptive task is like a Full
  Preemptive task that executes all the time locking a resource with
  its ceiling equal to the maximum priority in the system. As a
  result, a non preemptive task cannot be preempted by other tasks:
  only interrupts can preempt it.
\item[Mixed Preemptive Task] A Mixed Preemptive task is like a task
  that executes all the time locking a pseudo-resource (also called
  {\em Internal Resource}\index{Internal Resource}. As a result, only
  tasks with higher priority than the ceiling of the Internal
  Resource, and interrupts, can preempt it.
\end{description}

Independently of the task type all kernel primitives that may cause
rescheduling may be called in any Tasks' sub-functions.

Tasks are activated using the primitives \reffun{ActivateTask} or
\reffun{ChainTask}. Activating a task means that the activated task,
may be selected for scheduling, and may execute one Task Instance. A
task activation while a task is already waiting its execution or while
being the running task has an effect that depends on how many pending
activations\index{Pending activations} the particular Conformance
Class can store. BCC1 and ECC1 does not store pending activations,
whereas BCC2 and ECC2 can store pending activations if the task has
been properly configured in the OIL Configuration file.

Tasks {\bf must end} with a call either to \reffun{TerminateTask} or
\reffun{ChainTask}. Terminating a task without one of these two
primitives leads to indefinite results.

On multiprocessor systems, Tasks are statically assigned to CPUs at
compile time. The CPU a task is assigned to is specified within the
OIL configuration file.

\pagebreak

%%%%%%%%%%%%%%%%%%%%%%%%%%%%%%%%%%%%%%%%%%%%%%%%%%%%%%%%%%%%%%%%%%%%%%

\begin{function}{ActivateTask}
  \synopsis{StatusType ActivateTask(TaskType TaskID);}
  
  \begin{fundescription}
    This primitive activates a task \vr{TaskID}, putting it in the
    \const{READY} state, or in the \const{RUNNING} state if the
    scheduler finds that the activated task should become the running
    task.

    Once activated, the task will run for an instance, starting from
    its first instruction. For the BCC2 and ECC2 Conformance classes,
    pending activations can be stored if the task has been configured
    with a number of activations greater than 1 within the OIL
    configuration file.

    The function can be called from the Background task (typically,
    the \fn{main()} function).

  \end{fundescription}
  
  \begin{funparameters}
    \fpar{TaskID}{Task reference.}
  \end{funparameters}
  
  \begin{funreturn}
    \fret{E_OK}{No error.}
    \fret{E_OS_LIMIT}{Too many pending activations of \vr{TaskID}.}
    \fret{E_OS_ID}{(Extended) \vr{TaskID} is invalid.}
  \end{funreturn}
  
  \begin{funconformance}
    BCC1, BCC2, ECC1, ECC2
  \end{funconformance}
\end{function}

%%%%%%%%%%%%%%%%%%%%%%%%%%%%%%%%%%%%%%%%%%%%%%%%%%%%%%%%%%%%%%%%%%%%%%

\begin{function}{TerminateTask}
  \synopsis{StatusType TerminateTask(void)}
  
  \begin{fundescription}
    This primitive terminates the calling task. The function can be
    called from any function nesting: the stack space used by the task
    is also freed. The calling task should not have any Resource
    locked when this primitive is called (apart an Internal Resource
    that is automatically released with this call).

    After the call, the calling task is set in the \const{SUSPENDED}
    state, and it can be reactivated again using
    \reffun{ActivateTask}, \reffun{ChainTask}, or using Alarm
    notifications.

    All the tasks must terminate with a call to \fn{TerminateTask} or
    \reffun{ChainTask}. Otherwise, the behavior after task end is
    undefined.

    With Standard Status, the primitive never returns. With Extended
    Status, the primitive may return in case of errors.
  \end{fundescription}

  %  \begin{funparameters}
  %  \item No parameters
  %  \end{funparameters}
  
  \begin{funreturn}
    \fret{no return}{(Standard) -- in this case, the function is
      declared as returning \const{void}.}
    \fret{E_OS_RESOURCE}{(Extended) The task still occupies resources.}
    \fret{E_OS_CALLEVEL}{(Extended) The function was called at interrupt level.}
  \end{funreturn}
  
  \begin{funconformance}
    BCC1, BCC2, ECC1, ECC2
  \end{funconformance}
\end{function}

%%%%%%%%%%%%%%%%%%%%%%%%%%%%%%%%%%%%%%%%%%%%%%%%%%%%%%%%%%%%%%%%%%%%%%

\begin{function}{ChainTask}
  \synopsis{StatusType ChainTask(TaskType TaskID);}
  
  \begin{fundescription}
    This primitive is similar to \reffun{TerminateTask}, with the
    differences listed below.

    After the calling task is terminated, \vr{TaskID} is activated
    again.

    If \vr{TaskID} is the calling task ID, then the calling task is
    terminated, the Internal Resource is unlocked, and then the
    calling task is put again in the ready queue to be scheduled. The
    Internal Resource will be locked again when the task will be again
    selected for scheduling.

    When called successfully, \fn{ChainTask} does not return to the
    caller. In case of error the primitive returns, and the returned
    error value can be evaluated by the application.
    
    When an extended task is transferred from suspended state into
    ready state all its events are cleared.
  \end{fundescription}
  
  \begin{funparameters}
    \fpar{TaskID}{Task reference}
  \end{funparameters}
  
  \begin{funreturn}
    \fret{No return}{If the call is successful.}
    \fret{E_OS_LIMIT}{Too many task activations of \vr{TaskID}. The
      Task activation in this case is ignored.}
    \fret{E_OS_ID}{(Extended) Task \vr{TaskID} is invalid.}
    \fret{E_OS_RESOURCE}{(Extended) Calling task still occupies
      resources.}  
    \fret{E_OS_CALLEVEL}{(Extended) Call at interrupt
      level.}
  \end{funreturn}
  
  \begin{funconformance}
    BCC1, BCC2, ECC1, ECC2
  \end{funconformance}
\end{function}

%%%%%%%%%%%%%%%%%%%%%%%%%%%%%%%%%%%%%%%%%%%%%%%%%%%%%%%%%%%%%%%%%%%%%%

\begin{function}{Schedule}
  \synopsis{StatusType Schedule(void)}
  \begin{fundescription}
    This primitive can be used as a rescheduling point for tasks that
    have Internal Resources and for non preemptive tasks\footnote{Non
    preemptive tasks are tasks with an Internal Resource with the
    highest priority ceiling available assigned.}.

    When this primitive is called, the task releases its Internal
    Resource, and checks if there are higher priority tasks that have
    to preempt (In that case, a preemption is implemented). When the
    primitive returns, the task will reacquire its internal resource.
    
    The primitive does nothing if the calling task has no internal
    resource assigned.
  \end{fundescription}
  \begin{funreturn}
    \fret{void}{The function is redefined as returning \const{void} when
      Standard Error is used.}  
    \fret{E_OK}{No error.}
    \fret{E_OS_CALLEVEL}{(Extended) The primitive was called at interrupt level.}
    \fret{E_OS_RESOURCE}{(Extended) The calling task occupies resources.}
  \end{funreturn}
  \begin{funconformance}
    BCC1, BCC2, ECC1, ECC2
  \end{funconformance}
\end{function}

%%%%%%%%%%%%%%%%%%%%%%%%%%%%%%%%%%%%%%%%%%%%%%%%%%%%%%%%%%%%%%%%%%%%%%

\begin{function}{ForceSchedule}
  \synopsis{StatusType ForceSchedule(void)}
  \begin{fundescription}
    This function implements a preemption check. If a higher-priority
    task is ready, the current task is put into the ready state, its
    context is saved and the higher-priority task is
    executed. Otherwise the calling task is continued.

    The difference of this primitive with respect to \reffun{Schedule}
    is that Internal Resources are not released.

  \end{fundescription}
  \begin{funreturn}
    \fret{void}{The function is redefined as returning \const{void} when
      Standard Error is used.}  
    \fret{E_OK}{No error.}
    \fret{E_OS_CALLEVEL}{(Extended) Call at interrupt level.}
  \end{funreturn}
  \begin{funconformance}
    BCC1, BCC2, ECC1, ECC2
  \end{funconformance}
\end{function}


%%%%%%%%%%%%%%%%%%%%%%%%%%%%%%%%%%%%%%%%%%%%%%%%%%%%%%%%%%%%%%%%%%%%%%

\begin{function}{GetTaskID}
  %PJ: OK nota: qui ritorna void, lo standard ritorna un errore
  \synopsis{void GetTaskID ( TaskRefType TaskID )}
  \begin{fundescription}
    \fn{GetTaskID} returns the TaskID of the running task.
    
    If no task is running, \const{INVALID_TASK} is returned.
  \end{fundescription}
  %%   \begin{funparameters}
  %%     \fpar{}{}
  %%   \end{funparameters}
  %%   \begin{funreturn}
  %%   \fret{}{}
  %%   \end{funreturn}
  \begin{funconformance}
    BCC1, BCC2, ECC1, ECC2
  \end{funconformance}
\end{function}

%%%%%%%%%%%%%%%%%%%%%%%%%%%%%%%%%%%%%%%%%%%%%%%%%%%%%%%%%%%%%%%%%%%%%%

\begin{function}{GetTaskState}
  \synopsis{StatusType GetTaskState(TaskType TaskID, TaskStateRefType State)}
  \begin{fundescription}
    This primitive returns the state of a given task. possible states
    are listed in Section \ref{con:Task States}. 

    If the task \const{TaskID} supports pending activation, and the
    task has been activated more than once, the results refer to the
    state of its oldest activation.
    % lo standard diceva:
    % When the service is called for a task,
    % which is activated more than once, the state is set to running if
    % any instance of the task is running.

  \end{fundescription}
  \begin{funparameters}
    \fpar{TaskID}{(in) Task reference.}
    \fpar{State}{(out) Reference to the state of task \vr{TaskID}.}
  \end{funparameters}
  \begin{funreturn}
    \fret{void}{(Standard) The function returns void if Standard Mode is used.}
    \fret{E_OK}{(Extended) No error.}
    \fret{E_OS_ID}{(Extended) Task \vr{TaskID} is invalid.}
  \end{funreturn}
  \begin{funconformance}
    BCC1, BCC2, ECC1, ECC2
  \end{funconformance}
\end{function}


\pagebreak








%%%%%%%%%%%%%%%%%%%%%%%%%%%%%%%%%%%%%%%%%%%%%%%%%%%%%%%%%%%%%%%%%%%%%%

\section{Interrupt primitives}

\ee\ gives support for interrupts. Interrupts are modeled considering
typical microcontroller designs featuring interrupt controllers with a
prioritized view of the interrupt sources.

To map the requirements of fast OS-independent requests, \ee\ supports
the definition of fast interrupts handlers, called {\em ISR Type 1},
that on one side can handle interrupts in the fastest way possible,
but on the other side lack the possibility to call OS services. 

On the other end, lower priority interrupts, called {\em ISR Type 2} and
used (for example) for hardware timers, can call selected OS primitives
but are slower than ISR Type 1 due to the OS bookkeeping needed to
implement preemption.

\ee\ also offers a set of primitives to directly control interrupt
disabling and enabling, with also a nested version of these
primitives.

\pagebreak

\begin{function}{DisableAllInterrupts}
  \synopsis{void DisableAllInterrupts(void)}
  \begin{fundescription}
    \reffun{DisableAllInterrupts} and \reffun{EnableAllInterrupts} are
    used to implement critical sections with interrupt disabled.

    This primitive disables all the interrupts sources in the system,
    and saves the interrupt state that will be restored by a call to
    \reffun{EnableAllInterrupts}.
    
    The primitive may be called from an ISR category 1 and category 2
    and from the task level, but not from hook routines. No primitives
    can be called within critical sections surrounded by
    \reffun{DisableAllInterrupts} and \reffun{EnableAllInterrupts}.

    Critical sections using \reffun{DisableAllInterrupts}
    \reffun{EnableAllInterrupts} cannot be nested.  If you need nested
    critical sections, please use \reffun{SuspendOSInterrupts} /
    \reffun{ResumeOSInterrupts} or \reffun{SuspendAllInterrupts} /
    \reffun{ResumeAllInterrupts}.
  \end{fundescription}
  \begin{funconformance}
    BCC1, BCC2, ECC1, ECC2
  \end{funconformance}
\end{function}

%%%%%%%%%%%%%%%%%%%%%%%%%%%%%%%%%%%%%%%%%%%%%%%%%%%%%%%%%%%%%%%%%%%%%%

\begin{function}{EnableAllInterrupts}
  \synopsis{void EnableAllInterrupts(void)}
  \begin{fundescription}
    \reffun{DisableAllInterrupts} and \reffun{EnableAllInterrupts} are
    used to implement critical sections with interrupt disabled.

    This primitive restores the state saved by
    \reffun{DisableAllInterrupts}, enabling the recognition of interrupts.
    
    The primitive may be called from an ISR category 1 and category 2
    and from the task level, but not from hook routines.  No primitives
    can be called inside critical sections surrounded by
    \reffun{DisableAllInterrupts} and \reffun{EnableAllInterrupts}.
  \end{fundescription}
  \begin{funconformance}
    BCC1, BCC2, ECC1, ECC2
  \end{funconformance}
\end{function}

%%%%%%%%%%%%%%%%%%%%%%%%%%%%%%%%%%%%%%%%%%%%%%%%%%%%%%%%%%%%%%%%%%%%%%

\begin{function}{SuspendAllInterrupts}
  \synopsis{void SuspendAllInterrupts(void)}
  \begin{fundescription}
    \reffun{SuspendAllInterrupts} and \reffun{ResumeAllInterrupts} are
    used to implement critical sections with interrupt disabled, with
    nesting support.

    This primitive disables all the interrupts sources in the system,
    and saves the interrupt state that will be restored by a call to
    \reffun{ResumeAllInterrupts}.
    
    The service may be called from an ISR category 1 and category 2,
    from alarm-callbacks and from the task level, but not from all
    hook routines.

    No primitive calls beside \fn{SuspendAllInterrupts} /
    \reffun{ResumeAllInterrupts} pairs and
    \reffun{SuspendOSInterrupts} / \reffun{ResumeOSInterrupts} pairs
    are allowed within this critical section.
  \end{fundescription}
  \begin{funconformance}
    BCC1, BCC2, ECC1, ECC2
  \end{funconformance}
\end{function}

%%%%%%%%%%%%%%%%%%%%%%%%%%%%%%%%%%%%%%%%%%%%%%%%%%%%%%%%%%%%%%%%%%%%%%

\begin{function}{ResumeAllInterrupts}
  \synopsis{void ResumeAllInterrupts(void)}
  \begin{fundescription}
    \reffun{SuspendAllInterrupts} and \reffun{ResumeAllInterrupts} are
    used to implement critical sections with interrupt disabled, with
    nesting support.

    This primitive restores the state saved by
    \reffun{SuspendAllInterrupts}, enabling the recognition of
    interrupts if it is the last call in a series of nested calls of
    \fn{SuspendAllInterrupts} / \reffun{ResumeAllInterrupts} and
    \reffun{SuspendOSInterrupts} / \reffun{ResumeOSInterrupts} pairs.

    This primitive may be called from an ISR category 1 and category
    2, from alarm-callbacks and from the task level, but not from all
    hook routines.

  \end{fundescription}
  \begin{funconformance}
    BCC1, BCC2, ECC1, ECC2
  \end{funconformance}
\end{function}

%%%%%%%%%%%%%%%%%%%%%%%%%%%%%%%%%%%%%%%%%%%%%%%%%%%%%%%%%%%%%%%%%%%%%%

\begin{function}{SuspendOSInterrupts}
  \synopsis{void SuspendOSInterrupts(void)}
  \begin{fundescription}
    \reffun{SuspendOSInterrupts} and \reffun{ResumeOSInterrupts} are
    used to implement critical sections with interrupt category 2
    disabled, with nesting support.

    This primitive disables all the interrupts sources of category 2
    in the system, and saves the interrupt state that will be restored
    by a call to \reffun{ResumeOSInterrupts}.
    
    The service may be called from an ISR category 1 and category 2,
    and from the task level.

    No primitive calls beside \fn{SuspendAllInterrupts} /
    \reffun{ResumeAllInterrupts} pairs and
    \reffun{SuspendOSInterrupts} / \reffun{ResumeOSInterrupts} pairs
    are allowed within this critical section.
  \end{fundescription}
  \begin{funconformance}
    BCC1, BCC2, ECC1, ECC2
  \end{funconformance}
\end{function}

%%%%%%%%%%%%%%%%%%%%%%%%%%%%%%%%%%%%%%%%%%%%%%%%%%%%%%%%%%%%%%%%%%%%%%

\begin{function}{ResumeOSInterrupts}
  \synopsis{void ResumeOSInterrupts(void)}
  \begin{fundescription}
    \reffun{SuspendOSInterrupts} and \reffun{ResumeOSInterrupts} are
    used to implement critical sections with interrupt category 2
    disabled, with nesting support.

    This primitive restores the state saved by
    \reffun{SuspendOSInterrupts}, enabling the recognition of
    interrupts if it is the last call in a series of nested calls of
    \fn{SuspendAllInterrupts} / \reffun{ResumeAllInterrupts} and
    \reffun{SuspendOSInterrupts} / \reffun{ResumeOSInterrupts} pairs.

    The primitive may be called from an ISR category 1 and category 2
    and from the task level, but not from hook routines.
  \end{fundescription}
  \begin{funconformance}
    BCC1, BCC2, ECC1, ECC2
  \end{funconformance}
\end{function}



\pagebreak









%%%%%%%%%%%%%%%%%%%%%%%%%%%%%%%%%%%%%%%%%%%%%%%%%%%%%%%%%%%%%%%%%%%%%%

\section{Resource primitives}
\label{sec:resource-primitives}

Resources are the term used by \ee\ to refer to binary semaphores used
to implement shared critical sections.

Resources are implemented using the Immediate Priority Ceiling
protocol. 
%
%
% \nb{mettere link alla sezione di Peppe}
%
%
A resource is locked using the primitive \reffun{GetResource}, and
unlocked using \reffun{ReleaseResource}.

A special resource named \refconst{RES_SCHEDULER} is also
supported. the \refconst{RES_SCHEDULER} resource has a ceiling equal
to the highest priority in the system. As a result, a task locking
\refconst{RES_SCHEDULER} becomes non-preemptive. If needed, the
\refconst{RES_SCHEDULER} resource have to be configured in the OIL
configuration file.

On multiprocessor systems, Resources are divided in:
\begin{description}
\item[Local resources] A Resource is local when all the tasks that
  uses it are assigned to the same processor.
\item[Global resources] A Resource is global when the tasks that uses
  it are assigned to different processors.
\end{description}

A special kind of resources, called {\em Internal Resources}, are also
supported by \ee. Internal Resources are locked when the tasks enter
the \const{RUNNING} state, and it is released when the task
ends. Internal resources are used by optimization algorithms to limit
the maximum stack space used by application tasks. Please note that
\reffun{Schedule} explicitly release any Internal Resource locked by
the running task, thus limiting the possibility to reduce the overall
stack in the system. Also \reffun{WaitEvent} always release the
internal resource of the task; however, this fact does not impact on
stack usage because tasks using \reffun{WaitEvent} must run on a
private stack since \reffun{WaitEvent} is a blocking primitive. Please
also note that on Multicore systems, Internal Resources can only
be {\em local}. Global Internal Resources are not supported.

The primitives \reffun{GetResource} and \reffun{ReleaseResource}
automatically internally uses a spin-lock mechanism when called on a
Global Resource.

\ee\ support resource management at ISR level for a selected number of
architectures. Please check on the \ee\ wiki whether the architecture
you are currently using supports this feature. The implementation of
this feature is implemented extending the OIL configuration to accept
ISR priority as extra field in ISR Object definition. The ISR Priority
field contains architecture independent values which are then mapped
by \rtd\ to real interrupt priority register values.

\pagebreak


\begin{function}{GetResource}
  \synopsis{StatusType GetResource (ResourceType ResID)}
  \begin{fundescription}
    This primitive can be used to implement a critical section guarded
    by Resource \vr{ResID}. The critical section will end with the
    call to \reffun{ReleaseResource}.

    Nesting between critical sections guarded by different resources
    is allowed.

    Calls to \reffun{TerminateTask}, \reffun{ChainTask},
    \reffun{Schedule}, and \reffun{WaitEvent} are not allowed inside
    the critical section. % ForceSchedule si

    % Resources on ISR level non implementate ancora
    The service may be called from task level only.
    %The service may be called from an ISR and from task level.
  \end{fundescription}
  \begin{funparameters}
    \fpar{ResID}{Reference to resource}
  \end{funparameters}
  \begin{funreturn}
    \fret{void} {The function is redefined as returning \const{void} when
      Standard Error is used.} 
    \fret{E_OK}{(Extended) No error.}
    \fret{E_OS_ID}{(Extended) Resource \vr{ResID} is invalid.}
    \fret{E_OS_ACCESS}{(Extended) Attempt to get a resource which is
      already occupied by any task or ISR, or the statically assigned
      priority of the calling task or interrupt routine is higher than
      the calculated ceiling priority.}
  \end{funreturn}
  \begin{funconformance}
    BCC1, BCC2, ECC1, ECC2
  \end{funconformance}
\end{function}

%%%%%%%%%%%%%%%%%%%%%%%%%%%%%%%%%%%%%%%%%%%%%%%%%%%%%%%%%%%%%%%%%%%%%%

\begin{function}{ReleaseResource}
  \synopsis{StatusType ReleaseResource (ResourceType ResID)}
  \begin{fundescription}
    \fn{ReleaseResource} is used to release a resource locked using
    \reffun{GetResource}, closing a critical section.
    
    For information on nested critical sections, see \reffun{GetResource}.

    % Resources on ISR level non implementate ancora
    The service may be called from task level only.
    %The service may be called from an ISR and from task level.
  \end{fundescription}
  \begin{funparameters}
    \fpar{ResID}{Resource identifier}
  \end{funparameters}
  \begin{funreturn}
    \fret{void} {The function is redefined as returning \const{void}
      when Standard Error is used.}

    \fret{E_OK}{(Extended) No error}

    \fret{E_OS_ID}{(Extended) \vr{ResID} is an invalid identifier}

    \fret{E_OS_NOFUNC}{(Extended) Attempt to release a resource which
      is not occupied by any task or ISR, or another resource shall be
      released before.}

    \fret{E_OS_ACCESS}{(Extended) Attempt to release a resource which
      has a lower ceiling priority than the statically assigned
      priority of the calling task or interrupt routine.}
  \end{funreturn}
  \begin{funconformance}
    BCC1, BCC2, ECC1, ECC2
  \end{funconformance}
\end{function}

\pagebreak








%%%%%%%%%%%%%%%%%%%%%%%%%%%%%%%%%%%%%%%%%%%%%%%%%%%%%%%%%%%%%%%%%%%%%%

\section{Event related primitives}
Events represents a technique used by \ee\ to implement synchronization
primitives. Events are assigned to tasks. Tasks with events assigned
to are called {\em Extended tasks}. Tasks without events assigned to
are called {\em Basic Tasks}.

Extended tasks are supported only in the ECC1 and ECC2 conformance
classes. To assign an event to a task, the event have to be listed
inside the task declaration in the OIL configuration file.

Events are implemented as bits in a bit mask. Each task in the system is
associated to a bit mask, which is typically as large as a CPU data
register. The bit mask is initialized to $0$ at system startup.

The status of an extended task event mask can be read by tasks and
ISRs using the \reffun{GetEvent} primitive. Events can also be set
from tasks or ISRs using the \reffun{SetEvent} primitive (more than
one event can be set with a single call to \reffun{SetEvent}).

An extended task can wait for one ore more events from an event mask
to be set using the \reffun{WaitEvent} primitive. An extended task
needs then to explicitly clear an event calling the
\reffun{ClearEvent} primitive.

Calls to WaitEvent may provoke the task to block. For that reason,
extended tasks must have a private stack assigned to them, implying
also the fact that the multistack kernel have to be used.

\pagebreak

\begin{function}{SetEvent}
  \synopsis{StatusType SetEvent (TaskType TaskID, EventMaskType Mask)}
  \begin{fundescription}
    The events of task \vr{TaskID} are set according to the event mask
    \vr{Mask}. The call to \fn{SetEvent} may cause \vr{TaskID} to
    wakeup from a \reffun{WaitEvent} primitive. Any events not set in
    the event mask remain unchanged.  The service may be called from
    an interrupt service routine and from the task level, but not from
    hook routines.

    The function can be called from the Background task. Typically,
    it is called within the \fn{main()} function.

  \end{fundescription}
  \begin{funparameters}
    \fpar{TaskID}{Task identifier}
    \fpar{Mask}{Mask of the events to be set}
  \end{funparameters}
  \begin{funreturn}
    \fret{void} {The function is redefined as returning \const{void}
      when Standard Error is used.}
    \fret{E_OK}{(Extended) No error}
    \fret{E_OS_ID}{(Extended) Reference \vr{TaskID} is invalid.}
    \fret{E_OS_ACCESS}{(Extended) Task \vr{TaskID} is not an extended task.}
    \fret{E_OS_STATE}{(Extended) Events can not be set as the
      referenced task is in the suspended state.}
  \end{funreturn}
  \begin{funconformance}
    ECC1, ECC2
  \end{funconformance}
\end{function}

%%%%%%%%%%%%%%%%%%%%%%%%%%%%%%%%%%%%%%%%%%%%%%%%%%%%%%%%%%%%%%%%%%%%%%

\begin{function}{ClearEvent}
  \synopsis{StatusType ClearEvent (EventMaskType Mask)}
  \begin{fundescription}
    \fn{ClearEvent} clears the events \vr{Mask} of the calling task.

    This system call is restricted to extended tasks which own the
    event.
  \end{fundescription}
  \begin{funparameters}
    \fpar{Mask}{Mask of the event to be cleared}
  \end{funparameters}
  \begin{funreturn}
    \fret{void} {The function is redefined as returning \const{void}
      when Standard Error is used.}
    \fret{E_OK}{(Extended) No error}
    \fret{E_OS_ACCESS}{(Extended) The service has been invoked by a
      non-extended task.}
    \fret{E_OS_CALLEVEL}{(Extended) The service has been invoked at
      the interrupt level.}
  \end{funreturn}
  \begin{funconformance}
    ECC1, ECC2
  \end{funconformance}
\end{function}

%%%%%%%%%%%%%%%%%%%%%%%%%%%%%%%%%%%%%%%%%%%%%%%%%%%%%%%%%%%%%%%%%%%%%%

\begin{function}{GetEvent}
  \synopsis{StatusType GetEvent(TaskType TaskID, EventMaskRefType Event)}
  \begin{fundescription}
    This primitive returns the current state of all event bits of the
    task \vr{TaskID}.  The service may be called from interrupt
    service routines, task level and some hook routines.  The current
    status of the event mask of task \vr{TaskID} is copied to
    \vr{Event}.  The referenced task shall be an extended task.
  \end{fundescription}
  \begin{funparameters}
    \fpar{TaskID}{task whose mask is to be returned.}
    \fpar{Event}{(out) returned mask.}   
  \end{funparameters}
  \begin{funreturn}
    \fret{void} {The function is redefined as returning \const{void}
      when Standard Error is used.}
    \fret{E_OK}{(Extended) No error}
    \fret{E_OS_ID}{(Extended) \vr{TaskID} is an invalid reference.}
    \fret{E_OS_ACCESS}{(Extended) \vr{TaskID} is not an extended task.}
    \fret{E_OS_STATE}{(Extended) \vr{TaskID} is in suspended state.}
  \end{funreturn}
  \begin{funconformance}
    ECC1, ECC2
  \end{funconformance}
\end{function}

%%%%%%%%%%%%%%%%%%%%%%%%%%%%%%%%%%%%%%%%%%%%%%%%%%%%%%%%%%%%%%%%%%%%%%

\begin{function}{WaitEvent}
  \synopsis{StatusType WaitEvent (EventMaskType Mask)}
  \begin{fundescription}
    The calling task blocks if none of the events specified in
    \vr{Mask} are set.

    If the calling task blocks, the system is reschedule, and the
    Internal resource of the task is released. This service shall only
    be called from the extended task owning the events.
  \end{fundescription}
  \begin{funparameters}
    \fpar{Mask}{mask of the events waited for}
  \end{funparameters}
  \begin{funreturn}
    \fret{void} {The function is redefined as returning \const{void}
      when Standard Error is used.}
    \fret{E_OK}{(Extended) No error}
    \fret{E_OS_ACCESS}{(Extended) The calling task is not extended}
    \fret{E_OS_RESOURCE}{(Extended) Calling task occupies resources.}
    \fret{E_OS_CALLEVEL}{(Extended) Call at interrupt level.}
  \end{funreturn}
  \begin{funconformance}
    ECC1, ECC2
  \end{funconformance}
\end{function}

\pagebreak






%%%%%%%%%%%%%%%%%%%%%%%%%%%%%%%%%%%%%%%%%%%%%%%%%%%%%%%%%%%%%%%%%%%%%%

\section{Counter and Alarms primitives}
\ee\ supports a notification mechanism based on {\em Counter}s and {\em
Alarm}s.

A Counter is basically an integer value that can be incremented by $1$
``Tick'' using the primitive \reffun{IncrementCounter}. 

An Alarm is a notification that is attached to a specific Counter (the
link between a Counter and an Alarm is specified at compile time in
the OIL Configuration file).

An Alarm can be set to fire at a specified tick value using the
primitives \reffun{SetRelAlarm} and \reffun{SetAbsAlarm}. Alarms can
be set to be cyclically reactivated. Alarms can be canceled using the
primitive \reffun{CancelAlarm}.

When an Alarm fires, a notification takes place. A notification is set
to be one of the following actions:
\begin{description}
\item[Task activation.] In this case, a task is activated when the
  Alarm fires.
\item[Event set.] In this case, an event mask is set on a task when
  the Alarm fires.
\item[Alarm callback.] In this case, an alarm callback (defined using
  \reffun{ALARMCALLBACK}) is called.
\end{description} 

The notifications are executed inside the \reffun{IncrementCounter}
function. It is up to the developer placing the counter in meaningful
places (e.g., a timer interrupt).

Counters, Alarms, and their notifications are specified inside the OIL
configuration file. 

On multiprocessor systems, Counters are statically assigned to CPUs at
compile time. Counters are local to (and only visible in) a
CPU. Alarms are local to the CPU that hosts the counter they are
linked to.

\begin{warning}
The OSEK/VDX standard provides support for \index{System Counter}s
(e.g., counters that are automatically linked to hardware
timers). Please check on the \ee\ wiki whether the architecture you
are using supports this feature. If the architecture does not support
the feature, then all the counters have to be defined inside the OIL
Configuration file, and the user have to call
\reffun{IncrementCounter} to increment them.
\end{warning}

\pagebreak

\begin{function}{IncrementCounter}
  %PJ: vedere il commento
  \synopsis{StatusType IncrementCounter(CounterType c)}
  \begin{fundescription}
    This function receives a counter identifier as parameter, and it
    increments it by 1. This function is typically called inside an
    ISR type 2 or inside a task to notify that the trigger the counter
    is counting has happened.
    
    The function also implements the notification of expired alarms,
    that is implemented, depending on the alarm configuration, as:
    \begin{itemize}
      \item an alarm callback function;
      \item a task activation;
      \item an event mask set on an extended task.
    \end{itemize}

    The function is atomic, and a reschedule will happen at the end if
    the primitive is called at task level. If called at ISR level, the
    reschedule will happen at the end of the outermost nested IRQ.
  \end{fundescription}
  \begin{funparameters}
    \fpar{c}{The counter that needs to be incremented.}
  \end{funparameters}
  \begin{funreturn}
    \fret{E_OK}{No error.}
    \fret{E_OS_ID}{(Extended) Reference \vr{CounterID} is invalid or counter
    is implemented in hardware and can not be incremented by software.}
  \end{funreturn}
  \begin{funconformance}
    BCC1, BCC2, ECC1, ECC2
  \end{funconformance}
\end{function}

\begin{function}{GetCounterValue}
  \synopsis{StatusType GetCounterValue(CounterType CounterID, TickRefType Value)}
  \begin{fundescription}
    This service reads the current count value of a counter returning or the software ticks.
  \end{fundescription}
  \begin{funparameters}
    \fpar{CounterID}{The Counter which tick value should be read.}
    \fpar{Value}{Will contains the current tick value of the counter.}
  \end{funparameters}
  \begin{funreturn}
    \fret{E_OK}{No error.}
    \fret{E_OS_ID}{(Extended) Reference \vr{CounterID} is invalid.}
  \end{funreturn}
  \begin{funconformance}
    BCC1, BCC2, ECC1, ECC2
  \end{funconformance}
\end{function}

\begin{function}{GetElapsedValue}
  \synopsis{StatusType GetElapsedValue(CounterType CounterID, TickRefType Value, TickRefType ElapsedValue)}
  \begin{fundescription}
    This service gets the number of ticks between the current tick value and a previously read tick value.
    If the timer already passed the <Value> value a second (or multiple) time, the result returned is wrong.
    The reason is that the service can not detect such a relative overflow.
  \end{fundescription}
  \begin{funparameters}
    \fpar{CounterID}{The Counter which tick value should be read.}
    \fpar{Value}{Contains the previously read tick value the counter.}
    \fpar{ElapsedValue}{Will contains the difference to the previous read value.}
  \end{funparameters}
  \begin{funreturn}
    \fret{E_OK}{No error.}
    \fret{E_OS_ID}{(Extended) Reference \vr{CounterID} is invalid.}
    \fret{E_OS_VALUE}{(Extended) The given Value is invalid.}
  \end{funreturn}
  \begin{funconformance}
    BCC1, BCC2, ECC1, ECC2
  \end{funconformance}
\end{function}

%%%%%%%%%%%%%%%%%%%%%%%%%%%%%%%%%%%%%%%%%%%%%%%%%%%%%%%%%%%%%%%%%%%%%%

\begin{function}{GetAlarmBase}
  \synopsis{StatusType GetAlarmBase(AlarmType AlarmID, AlarmBaseRefType Info)}
  \begin{fundescription}
    Returns the alarm base characteristics. The return value \vr{Info}
    is a structure in which the information of data type
    \tp{AlarmBaseType} is stored.
    
    Allowed on task level, ISR, and in several hook routines.
  \end{fundescription}
  \begin{funparameters}
    \fpar{AlarmID}{Alarm identifier.}

    \fpar{Info}{Reference to the structure containing the constants
      that define the alarm base.}
  \end{funparameters}
  \begin{funreturn}
    \fret{void} {The function is redefined as returning \const{void}
      when Standard Error is used.}
    \fret{E_OK}{(Extended) No error.}
    \fret{E_OS_ID}{(Extended) Reference \vr{AlarmID} is invalid.}
  \end{funreturn}
  \begin{funconformance}
    BCC1, BCC2, ECC1, ECC2
  \end{funconformance}
\end{function}

%%%%%%%%%%%%%%%%%%%%%%%%%%%%%%%%%%%%%%%%%%%%%%%%%%%%%%%%%%%%%%%%%%%%%%

\begin{function}{GetAlarm}
  \synopsis{StatusType GetAlarm (AlarmType AlarmID, TickRefType Tick)}
  \begin{fundescription}
    The system service GetAlarm returns the relative value in ticks
    before the alarm \vr{AlarmID} expires. If \vr{AlarmID} is not in
    use, \vr{Tick} has an undefined value. Allowed on task level, ISR,
    and in several hook routines.
  \end{fundescription}
  \begin{funparameters}
    \fpar{AlarmID}{Alarm identifier}
    \fpar{Tick}{(out) Relative value in ticks before the alarm expires}
  \end{funparameters}
  \begin{funreturn}
    \fret{E_OK}{No error}
    \fret{E_OS_NOFUNC}{ \vr{AlarmID} is not used}
    \fret{E_OS_ID}{(Extended) reference \vr{AlarmID} is invalid}
  \end{funreturn}
  \begin{funconformance}
    BCC1, BCC2, ECC1, ECC2
  \end{funconformance}
\end{function}

%%%%%%%%%%%%%%%%%%%%%%%%%%%%%%%%%%%%%%%%%%%%%%%%%%%%%%%%%%%%%%%%%%%%%%

\begin{function}{SetRelAlarm}
  \synopsis{StatusType SetRelAlarm (AlarmType AlarmID, TickType increment, TickType cycle)}
  \begin{fundescription}
    After \vr{increment} ticks have elapsed, the \vr{AlarmID}
    notification is executed.
    
    If the relative value \vr{increment} is very small, the alarm may
    expire, and the notification can be executed before the system
    service returns to the user.  If \vr{cycle} is unequal zero, the
    alarm element is logged on again immediately after expiry with the
    relative value \vr{cycle}.
    
    The alarm \vr{AlarmID} must not already be in use.  To change
    values of alarms already in use the alarm shall be canceled
    first.
    
    If the alarm is already in use, this call will be ignored and the
    error \const{E_OS_STATE} is returned.  Allowed on task level and
    in ISR, but not in hook routines.

  \end{fundescription}
  \begin{funparameters}
    \fpar{AlarmID}{Reference to alarm}

    \fpar{increment}{Relative value in ticks representing the offset
      with respect to the current time of the first alarm expiration.}

    \fpar{cycle}{Cycle value in case of cyclic alarm. In case of
      single alarms, this parameter must be set to 0.}
  \end{funparameters}
  \begin{funreturn}
    \fret{E_OK}{No error.}
    \fret{E_OS_STATE}{Alarm is already in use.}
    \fret{E_OS_ID}{(Extended) reference \vr{AlarmID} is invalid}
    \fret{E_OS_VALUE}{(Standard and Extended) Value of \vr{increment} equal to 0.}
    \fret{E_OS_VALUE}{(Extended) Value of \vr{increment} outside of
      the admissible limits, or value of \vr{cycle} unequal to 0 and
      outside of the admissible counter limits.}
  \end{funreturn}
  \begin{funconformance}
    BCC1, BCC2, ECC1, ECC2
  \end{funconformance}
\end{function}

%%%%%%%%%%%%%%%%%%%%%%%%%%%%%%%%%%%%%%%%%%%%%%%%%%%%%%%%%%%%%%%%%%%%%%

\begin{function}{SetAbsAlarm}
  \synopsis{StatusType SetAbsAlarm (AlarmType AlarmID, TickType start, TickType cycle)}
  \begin{fundescription}
    The primitive occupies the alarm \vr{AlarmID} element. When
    \vr{start} ticks are reached, the \vr{AlarmID} notification is
    executed.
    
    If the absolute value \vr{start} is equal to the actual counter
    value, the alarm not expire immediately but will expire the next
    time the counter will reach the \vr{start} value..
    
    If \vr{cycle} is unequal zero, the alarm element is logged on
    again immediately after expiry with the relative value \vr{cycle}.
    
    The alarm \vr{AlarmID} shall not already be in use.  To change
    values of alarms already in use the alarm shall be canceled
    first.  If the alarm is already in use, this call will be ignored
    and the error \const{E_OS_STATE} is returned.  Allowed on task
    level and in ISR, but not in hook routines.
  \end{fundescription}
  \begin{funparameters}
    \fpar{AlarmID}{reference to alarm}

    \fpar{start}{Absolute value in ticks representing the time of the
      first expiration of the alarm.}

    \fpar{cycle}{cycle value in case of cyclic alarm. In case of
      single alarms, this parameter must be set to 0.}
  \end{funparameters}
  \begin{funreturn}
    \fret{E_OK}{No error.}
    \fret{E_OS_STATE}{Alarm is already in use.}
    \fret{E_OS_ID}{(Extended) reference \vr{AlarmID} is invalid}
  
    \fret{E_OS_VALUE}{(Extended) Value of \vr{start} outside of the
      admissible limits, or value of \vr{cycle} unequal to 0 and
      outside of the admissible counter limits}
  \end{funreturn}
  \begin{funconformance}
    BCC1, BCC2, ECC1, ECC2
  \end{funconformance}
\end{function}

%%%%%%%%%%%%%%%%%%%%%%%%%%%%%%%%%%%%%%%%%%%%%%%%%%%%%%%%%%%%%%%%%%%%%%

\begin{function}{CancelAlarm}
  \synopsis{StatusType CancelAlarm (AlarmType AlarmID)}
  \begin{fundescription}
    The primitive cancels the alarm \vr{AlarmID}. Allowed on task
    level and in ISR, but not in hook routines.
  \end{fundescription}
  \begin{funparameters}
    \fpar{AlarmID}{reference to alarm}
  \end{funparameters}
  \begin{funreturn}
    \fret{E_OK}{No error.}
    \fret{E_OS_STATE}{Alarm is already in use.}
    \fret{E_OS_ID}{(Extended)  reference \vr{AlarmID} is invalid.}
  \end{funreturn}
  \begin{funconformance}
    BCC1, BCC2, ECC1, ECC2
  \end{funconformance}
\end{function}

\pagebreak





%%%%%%%%%%%%%%%%%%%%%%%%%%%%%%%%%%%%%%%%%%%%%%%%%%%%%%%%%%%%%%%%%%%%%%

\section{Internal Messages}

\ee\ supports internal messaging following the OSEK COM specifiations,
conformace classes CCCA and CCCB. Please refer to the \ee\ COM Manual
for more information.

%%%%%%%%%%%%%%%%%%%%%%%%%%%%%%%%%%%%%%%%%%%%%%%%%%%%%%%%%%%%%%%%%%%%%%

\section{Counting Semaphores}

This section describes in detail the primitives provided by \ee\ to
support counting semaphores as a way to implement mutual exclusion and
synchronization between tasks.

Counting semaphores are an RTOS abstractions of an integer counter
coupled with a blocking queue. Basically two main operations are
possible on a semaphore, which are {\em waiting} on a semaphore, which
results in decrementing the counter if the counter has a value greater
than 0, or blocking the running task if the counter is 0, and {\em
posting} on a semaphore, which results in a counter increment if there
are no task blocked, or in the unblock of a blocked task otherwise.

\ee\ counting semaphores exports a simple interface which covers the
basic functionalities of a semaphore, like:
\begin{itemize}
\item Initializing a semaphore (\reffun{InitSem});
\item Waiting on a semaphore in a blocking
  (\reffun{WaitSem}) or non-blocking way (\reffun{TryWaitSem};
\item Posting on a semaphore (\reffun{PostSem});
\item Getting the value of a semaphore (\reffun{GetValueSem}).
\end{itemize}

Since waiting on a semaphore may result in blocking the running task,
the \reffun{WaitSem} primitive should be called only if the calling
task has a separate stack allocated to it. For this reason, the
\reffun{WaitSem} primitive can only be called by extended tasks in
conformance classes ECC1 and ECC2. Semaphores are available as non
blocking in conformance classes BCC1 and BCC2.

Semaphores can be allocated statically as a global variable, which
allow to bypass the call to \reffun{InitSem}.

Semaphores definition are not listed in the OIL file; semaphore
primitives receive as a parameter a pointer to the semaphore
descriptor.

The current semaphore implementation does {\em not} support
multiprocessor systems. That is, semaphores must be defined and used
locally to the same CPU.

\begin{warning}
Counting semaphores {\em do not} avoid Priority Inversion
problems. Please use Resources instead (see Section
\ref{sec:resource-primitives}).
\end{warning}

\pagebreak

\begin{function}{STATICSEM}
  \synopsis{SemType s = STATICSEM(value);}
  \begin{fundescription}
    This macro can be used to statically initialize a semaphore. It
    must be used inside the definition of a global semaphore variable
    to initialize a semaphore to a given value.
  \end{fundescription}
  \begin{funparameters}
    \fpar{value}{The counter value for the semaphore being initialized.}
  \end{funparameters}
  \begin{funconformance}
    BCC1, BCC2, ECC1, ECC2
  \end{funconformance}
\end{function}

%%%%%%%%%%%%%%%%%%%%%%%%%%%%%%%%%%%%%%%%%%%%%%%%%%%%%%%%%%%%%%%%%%%%%%

\begin{function}{InitSem}
  \synopsis{void InitSem(SemType s, int value);}
  \begin{fundescription}
    This macro can be used to initialize a semaphore at runtime. It
    receives as a parameter the init value of the semaphore counter.
  \end{fundescription}
  \begin{funparameters}
    \fpar{s}{The semaphore being initialized.}
    \fpar{value}{The counter value for the semaphore being initialized.}
  \end{funparameters}
  \begin{funreturn}
    \fret{void}{The function is a macro and it does not return an error.}
  \end{funreturn}
  \begin{funconformance}
    BCC1, BCC2, ECC1, ECC2
  \end{funconformance}
\end{function}

%%%%%%%%%%%%%%%%%%%%%%%%%%%%%%%%%%%%%%%%%%%%%%%%%%%%%%%%%%%%%%%%%%%%%%

\begin{function}{WaitSem}
  \synopsis{StatusType WaitSem(SemRefType s);}
  \begin{fundescription}
    If the semaphore counter is greater than 0, then the counter is
    decremented by one. If the counter has a value of 0, then the
    calling (running) task blocks. 

    This function can only be called by extended tasks in conformance
    classes ECC1 and ECC2.
  \end{fundescription}
  \begin{funparameters}
    \fpar{s}{The semaphore used by the primitive.}
  \end{funparameters}
  \begin{funreturn}
    \fret{void}{The function is redefined as returning \const{void} when
      Standard Error is used.}  
    \fret{E_OK}{No error.}
    \fret{E_OS_CALLEVEL}{(Extended) The primitive was called at interrupt level.}
    \fret{E_OS_RESOURCE}{(Extended) The calling task occupies resources.}
    \fret{E_OS_ACCESS}{(Extended) The calling task is not an extended task.}
  \end{funreturn}
  \begin{funconformance}
    ECC1, ECC2
  \end{funconformance}
\end{function}

%%%%%%%%%%%%%%%%%%%%%%%%%%%%%%%%%%%%%%%%%%%%%%%%%%%%%%%%%%%%%%%%%%%%%%


\begin{function}{TryWaitSem}
  \synopsis{int TryWaitSem(SemRefType s);}
  \begin{fundescription}
    This is a non-blocking version of \reffun{WaitSem}. If the
    semaphore counter is greater than 0, then the counter is
    decremented by one, and the primitive returns 0. If the counter
    has a value of 0, then the counter is decremented, and the
    primitive returns 1.
  \end{fundescription}
  \begin{funparameters}
    \fpar{s}{The semaphore used by the primitive.}
  \end{funparameters}
  \begin{funreturn}
    \fret{int}{0 if the semaphore counter has been decremented, 1 otherwise.}
  \end{funreturn}
  \begin{funconformance}
    BCC1, BCC2, ECC1, ECC2
  \end{funconformance}
\end{function}

%%%%%%%%%%%%%%%%%%%%%%%%%%%%%%%%%%%%%%%%%%%%%%%%%%%%%%%%%%%%%%%%%%%%%%

\begin{function}{PostSem}
  \synopsis{StatusType PostSem(SemRefType s);}
  \begin{fundescription}
    This primitive unblocks a task eventually blocked on the
    semaphore. If there are no tasks blocked on the semaphore, then
    the semaphore counter is incremented by one.
  \end{fundescription}
  \begin{funparameters}
    \fpar{s}{The semaphore used by the primitive.}
  \end{funparameters}
  \begin{funreturn}
    \fret{E_OK}{No error.}

    \fret{E_OS_VALUE}{The semaphore has not been incremented because
      its counter was equal to the semaphore maximum value
      \const{EE_MAX_SEM_COUNTER}.}
  \end{funreturn}
  \begin{funconformance}
    BCC1, BCC2, ECC1, ECC2
  \end{funconformance}
\end{function}

%%%%%%%%%%%%%%%%%%%%%%%%%%%%%%%%%%%%%%%%%%%%%%%%%%%%%%%%%%%%%%%%%%%%%%

\begin{function}{GetValueSem}
  \synopsis{int GetValueSem(SemRefType s);}
  \begin{fundescription}
    This primitive returns the value of the semaphore counter.
  \end{fundescription}
  \begin{funparameters}
    \fpar{s}{The semaphore used by the primitive.}
  \end{funparameters}
  \begin{funreturn}
    \fret{int}{The semaphore counter value.}
  \end{funreturn}
  \begin{funconformance}
    BCC1, BCC2, ECC1, ECC2
  \end{funconformance}
\end{function}

%%%%%%%%%%%%%%%%%%%%%%%%%%%%%%%%%%%%%%%%%%%%%%%%%%%%%%%%%%%%%%%%%%%%%%



%%%%%%%%%%%%%%%%%%%%%%%%%%%%%%%%%%%%%%%%%%%%%%%%%%%%%%%%%%%%%%%%%%%%%%

\section{Application modes, Startup and Shutdown primitives}
\ee\ supports the specification of a set of {\em Application
Modes}. Application modes are startup configurations that are used to
configure the running application for a certain mode of
operation. Examples of Application Modes are for example ``Normal
execution'', ``Flash Programming'', ``Debug Mode'', and so on.

The main idea is that the CPU startup code somehow discovers the
current Application mode\footnote{Typical ways to discover Application
Modes are for example the usage of dip switches on the device
board.}. Once done that, the application mode is passed to
\reffun{StartOS}, that sets the application mode for this run. Once
the system is started, the application mode value can be read using a
call to \reffun{GetActiveApplicationMode}.

Application modes are listed inside the OIL configuration file. There
always exist at least one application mode called
\refconst{OSDEFAULTAPPMODE}. Once set at startup into
\reffun{StartOS}, the Application mode cannot be changed.

Application modes are also useful to autostart tasks and alarms
following a call to \reffun{StartOS}. Tasks and alarms autostart
features must be specified inside the OIL configuration file.

Application modes should not be used to map application states that
may vary during the application execution.

The primitive \reffun{ShutdownOS} is used to prepare the system for
system shutdown. Currently the function simply calls
\reffun{ShutdownHook} and then it starts a forever loop waiting for an
hardware reset.

\pagebreak

\begin{function}{GetActiveApplicationMode}
  \synopsis{AppModeType GetActiveApplicationMode(void)}
  \begin{fundescription}
    The function returns the current Application Mode, that has been
    set up by \reffun{StartOS} at system startup.
  \end{fundescription}
  \begin{funreturn}
    \fret{Application mode}{The function returns the current Application Mode.}
  \end{funreturn}
  \begin{funconformance}
    BCC1, BCC2, ECC1, ECC2
  \end{funconformance}
\end{function}

%%%%%%%%%%%%%%%%%%%%%%%%%%%%%%%%%%%%%%%%%%%%%%%%%%%%%%%%%%%%%%%%%%%%%%

\begin{function}{StartOS}
  \synopsis{StatusType StartOS (AppModeType Mode)}
  \begin{fundescription}
    The user can call this system service to start the operating
    system in a specific Application mode. Only allowed outside of the
    operating system, at startup. The function calls in order the
    \reffun{StartupHook}, then it activates the tasks and set the
    alarms \const{AUTOSTART} set as \const{TRUE} inside the OIL
    configuration file. After that, the system is rescheduled and the
    highest priority ready task is executed.
    
    The \reffun{StartOS} primitive by default never returns to the caller.

    Please note that old versions of \ee\ implemented the
    \reffun{StartOS} primitive in a way that it was returning to the
    user, to enable the possibility to do some meaningful background
    activities. To re-enable this old behavior, please add the
    following line to the OIL file:

\begin{lstlisting}
  EE_OPT = "__OO_STARTOS_OLD__"
\end{lstlisting}

    If the function is configured to return to the caller, then it
    will return after the first idle time is reached. It is up to the
    caller to implement a meaningful background activity.  If unsure
    of a meaningful background activity, just use the \fn{for(;;)}
    construct.

  \end{fundescription}
  \begin{funparameters}
    \fpar{Mode}{application mode}
  \end{funparameters}
  \begin{funreturn}
    \fret{E_OK}{(Extended) No error.}
    \fret{E_OS_SYS_INIT}{(Extended) Error during initialization.}
    \fret{void}{The function returns void when Extended Status is not
                selected.}
  \end{funreturn}
  \begin{funconformance}
    BCC1, BCC2, ECC1, ECC2
  \end{funconformance}
\end{function}


%%%%%%%%%%%%%%%%%%%%%%%%%%%%%%%%%%%%%%%%%%%%%%%%%%%%%%%%%%%%%%%%%%%%%%

\begin{function}{ShutdownOS}
  \synopsis{void ShutdownOS (StatusType Error)}
  \begin{fundescription}
    The user can call this system service to abort the overall system
    (e.g. emergency off). 

    % non succede mai
    %The operating system also calls this
    %function internally, if it has reached an undefined internal state
    %and is no longer ready to run.
    
    If a \reffun{ShutdownHook} is configured the hook routine
    \reffun{ShutdownHook} is always called (with \vr{Error} as
    argument) before shutting down the operating system.
    
    The Operating system shutdown is currently implemented as a forever
    loop.

    %If \reffun{ShutdownHook} returns, further behaviour of
    %\fn{ShutdownOS} is implementation specific.
    
  \end{fundescription}
  \begin{funparameters}
    \fpar{Error}{The identifier of the error that occurred.}
  \end{funparameters}
  \begin{funconformance}
    BCC1, BCC2, ECC1, ECC2
  \end{funconformance}
\end{function}

\pagebreak






%%%%%%%%%%%%%%%%%%%%%%%%%%%%%%%%%%%%%%%%%%%%%%%%%%%%%%%%%%%%%%%%%%%%%%

\section{Hooks and Error handling primitives}
\ee supports five application callbacks that are called when specific
situations happens during application execution.

On multiprocessor systems, hooks are local callbacks to each processor.

The \reffun{ErrorHook} callback is called every time an error is
detected inside an \ee primitive. The callback can be used to
implement centralized error handling. A set of macros are also
available to better understand the source of the error (see Section
\ref{sec:errorhook-macros}).

The \reffun{StartupHook} callback is called inside the
\reffun{StartOS} primitive to implement the application startup
procedure.

The \reffun{ShutdownHook} callback is called inside the
\reffun{ShutdownOS} primitive to implement the Application specific
shutdown procedures.

The \reffun{PreTaskHook} and \reffun{PostTaskHook} callbacks are
called respectively every time a task becomes the \const{RUNNING}
Task, and every time a task stops to be the \const{RUNNING} Task.

The application can only call a limited set of primitives inside Hook
functions.

\subsection{Placement of the application startup code}
In general, there are three ways where the application startup code
can be placed:

\begin{enumerate}
\item before the call to \reffun{StartOS}. This case is typically used
  for non-OS related initializations, because calling \ee\ primitives
  before \reffun{StartOS} may have in general unpredictable results.
\item inside \reffun{StartupHook}. This case is used for the
  initializations that require a call to \ee\ primitives. Please note
  that \reffun{StartupHook} is called before the rescheduling in
  \reffun{StartOS} takes place.
\item after the call to \reffun{StartOS}. This case can be used also
  for \ee\ related initializations: the startup code is called {\em
  after} the end of \reffun{StartOS}, and that means at the first idle
  time {\em after} the autostart tasks has been activated and executed
  Please note that if the autostart tasks never terminate, these
  initializations are never called.
\end{enumerate}

\pagebreak

\begin{function}{ErrorHook}
  %PJ: ho riscritto in breve il contenuto dello standard
  \synopsis{void ErrorHook (StatusType Error)}
  \begin{fundescription}
    When configured in the OIL File, the system calls this callback
    every time the return value of a function is different from
    \const{E_OK}. The application can then get additional informations
    using the \fn{OSErrorGetServiceId} macro, that returns the
    function that is generating the Error. Once the function that
    generated the error is known, the application can also access the
    parameters that generated the error using the \fn{OSError_XXX_YYY}
    macros, where \fn{XXX} is the name of the function, and \fn{YYY}
    is the name of the parameter passed to the function.
  \end{fundescription}
  \begin{funparameters}
    \fpar{Error}{the identifier of the error that occurred}
  \end{funparameters}
  \begin{funconformance}
    BCC1, BCC2, ECC1, ECC2
  \end{funconformance}
\end{function}

%%%%%%%%%%%%%%%%%%%%%%%%%%%%%%%%%%%%%%%%%%%%%%%%%%%%%%%%%%%%%%%%%%%%%%

\begin{function}{PreTaskHook}
  %PJ: ho riscritto in breve il contenuto dello standard
  \synopsis{void PreTaskHook (void)}
  \begin{fundescription}
    When configured in the OIL File, this hook function is called
    every time a task becomes the \const{RUNNING} task due to the call
    of other functions, or due to the preemption done by
    interrupts. The ID of the task which has just become the
    \const{RUNNING} task can be read using the \reffun{GetTaskID}
    function.
  \end{fundescription}
  % \begin{funparameters}
  %   \fpar{Error}{the identifier of the error that occurred}
  % \end{funparameters}
  \begin{funconformance}
    BCC1, BCC2, ECC1, ECC2
  \end{funconformance}
\end{function}

%%%%%%%%%%%%%%%%%%%%%%%%%%%%%%%%%%%%%%%%%%%%%%%%%%%%%%%%%%%%%%%%%%%%%%

\begin{function}{PostTaskHook}
  %PJ: ho riscritto in breve il contenuto dello standard
  \synopsis{void PreTaskHook (void)}
  \begin{fundescription}
    When configured in the OIL File, this hook function is called
    every time a task is no more the \const{RUNNING} task due to the
    call of other functions, or due to the preemption done by
    interrupts. The ID of the task which has just finished to be the
    \const{RUNNING} task can be read using the \reffun{GetTaskID}
    function.
  \end{fundescription}
  % \begin{funparameters}
  %   \fpar{Error}{the identifier of the error that occurred}
  % \end{funparameters}
  \begin{funconformance}
    BCC1, BCC2, ECC1, ECC2
  \end{funconformance}
\end{function}

%%%%%%%%%%%%%%%%%%%%%%%%%%%%%%%%%%%%%%%%%%%%%%%%%%%%%%%%%%%%%%%%%%%%%%

\begin{function}{StartupHook}
  %PJ: ho riscritto in breve il contenuto dello standard
  \synopsis{void StartupHook (void)}
  \begin{fundescription}
    When configured in the OIL File, this hook function is called
    inside \reffun{StartOS}. Inside this callback, the Application can
    call the Operating System-related functions \reffun{ActivateTask},
    \reffun{SetRelAlarm}, and \reffun{SetAbsAlarm}. Please note that the
    simplest way to activate a task or set an alarm at startup is to
    specify their initial activation/setting inside the OIL File.
  \end{fundescription}
  % \begin{funparameters}
  %   \fpar{Error}{the identifier of the error that occurred}
  % \end{funparameters}
  \begin{funconformance}
    BCC1, BCC2, ECC1, ECC2
  \end{funconformance}
\end{function}

%%%%%%%%%%%%%%%%%%%%%%%%%%%%%%%%%%%%%%%%%%%%%%%%%%%%%%%%%%%%%%%%%%%%%%

\begin{function}{ShutdownHook}
  %PJ: ho riscritto in breve il contenuto dello standard
  \synopsis{void ShutdownHook (StatusType Error)}
  \begin{fundescription}
    When configured in the OIL File, this hook function is called
    inside \reffun{ShutdownOS}. Inside this callback, the Application
    can implement application dependent shutdown functions.
  \end{fundescription}
  \begin{funparameters}
    \fpar{Error}{the identifier of the error that occurred, that is the same value that has been passed to the ShutdownOS primitive}.
  \end{funparameters}
  \begin{funconformance}
    BCC1, BCC2, ECC1, ECC2
  \end{funconformance}
\end{function}

\pagebreak





%%%%%%%%%%%%%%%%%%%%%%%%%%%%%%%%%%%%%%%%%%%%%%%%%%%%%%%%%%%%%%%%%%%%%%

\section{ErrorHook Macros}
\label{sec:errorhook-macros}
These macros are meaningful inside the \reffun{ErrorHook} Hook
function, and are used to better understand the source of the
error. In particular, \reffun{ErrorHook} receives as parameter the
error that is raised by the primitive. Then, a call to
\reffun{OSErrorGetServiceId} returns informations about which
primitive caused the error. Finally, calls to the macros
\fn{OSError_XXX_YYY} returns the values of the \const{YYY} parameter
of the primitive \const{XXX}.

%%%%%%%%%%%%%%%%%%%%%%%%%%%%%%%%%%%%%%%%%%%%%%%%%%%%%%%%%%%%%%%%%%%%%%

\begin{function_nopb2}{OSErrorGetServiceId}{OSErrorGetServiceId}
  \synopsis{OSServiceIdType OSErrorGetServiceId(void)}
  \begin{fundescription}
    The function may be used inside \reffun{ErrorHook} to return the
    Service ID that generated the error that caused the call to
    \reffun{ErrorHook}.
  \end{fundescription}
  \begin{funreturn}
    \fret{Service ID}{The service ID causing the error.}
  \end{funreturn}
  \begin{funconformance}
    BCC1, BCC2, ECC1, ECC2
  \end{funconformance}
\end{function_nopb2}

\begin{function_nopb2}{OSError\_ActivateTask\_TaskID}{OSError_ActivateTask_TaskID}
  \synopsis{TaskType OSError_ActivateTask_TaskID(void)}
  \begin{fundescription}
    The function returns the \const{TaskID} parameter passed to
    \reffun{ActivateTask}. The function must be used inside
    \reffun{ErrorHook} after having discovered using
    \reffun{OSErrorGetServiceId} that the error was caused by that
    function.
  \end{fundescription}
  \begin{funreturn}
    \fret{Task ID}{The value of the \const{TaskID} parameter.}
  \end{funreturn}
  \begin{funconformance}
    BCC1, BCC2, ECC1, ECC2
  \end{funconformance}
\end{function_nopb2}

\begin{function_nopb2}{OSError\_ChainTask\_TaskID}{OSError_ChainTask_TaskID}
  \synopsis{TaskType OSError_ChainTask_TaskID(void)}
  \begin{fundescription}
    The function returns the \const{TaskID} parameter passed to
    \reffun{ChainTask}. The function must be used inside
    \reffun{ErrorHook} after having discovered using
    \reffun{OSErrorGetServiceId} that the error was caused by that
    function.
  \end{fundescription}
  \begin{funreturn}
    \fret{Task ID}{The value of the \const{TaskID} parameter.}
  \end{funreturn}
  \begin{funconformance}
    BCC1, BCC2, ECC1, ECC2
  \end{funconformance}
\end{function_nopb2}

\begin{function_nopb2}{OSError\_GetTaskState\_TaskID}{OSError_GetTaskState_TaskID}
  \synopsis{TaskType OSError_GetTaskState_TaskID(void)}
  \begin{fundescription}
    The function returns the \const{TaskID} parameter passed to
    \reffun{GetTaskState}. The function must be used inside
    \reffun{ErrorHook} after having discovered using
    \reffun{OSErrorGetServiceId} that the error was caused by that
    function.
  \end{fundescription}
  \begin{funreturn}
    \fret{Task ID}{The value of the \const{TaskID} parameter.}
  \end{funreturn}
  \begin{funconformance}
    BCC1, BCC2, ECC1, ECC2
  \end{funconformance}
\end{function_nopb2}

\begin{function_nopb2}{OSError\_GetTaskState\_State}{OSError_GetTaskState_State}
  \synopsis{TaskStateRefType OSError_GetTaskState_State(void)}
  \begin{fundescription}
    The function returns the \const{State} parameter passed to
    \reffun{GetTaskState}. The function must be used inside
    \reffun{ErrorHook} after having discovered using
    \reffun{OSErrorGetServiceId} that the error was caused by that
    function.
  \end{fundescription}
  \begin{funreturn}
    \fret{StateRef}{The value of the \const{State} parameter.}
  \end{funreturn}
  \begin{funconformance}
    BCC1, BCC2, ECC1, ECC2
  \end{funconformance}
\end{function_nopb2}

\begin{function_nopb2}{OSError\_GetResource\_ResID}{OSError_GetResource_ResID}
  \synopsis{ResourceType OSError_GetResource_ResID(void)}
  \begin{fundescription}
    The function returns the \const{ResID} parameter passed to
    \reffun{GetResource}. The function must be used inside
    \reffun{ErrorHook} after having discovered using
    \reffun{OSErrorGetServiceId} that the error was caused by that
    function.
  \end{fundescription}
  \begin{funreturn}
    \fret{Resource ID}{The value of the \const{ResID} parameter.}
  \end{funreturn}
  \begin{funconformance}
    BCC1, BCC2, ECC1, ECC2
  \end{funconformance}
\end{function_nopb2}

\begin{function_nopb2}{OSError\_ReleaseResource\_ResID}{OSError_ReleaseResource_ResID}
  \synopsis{ResourceType OSError_ReleaseResource_ResID(void)}
  \begin{fundescription}
    The function returns the \const{ResID} parameter passed to
    \reffun{ReleaseResource}. The function must be used inside
    \reffun{ErrorHook} after having discovered using
    \reffun{OSErrorGetServiceId} that the error was caused by that
    function.
  \end{fundescription}
  \begin{funreturn}
    \fret{Resource ID}{The value of the \const{ResID} parameter.}
  \end{funreturn}
  \begin{funconformance}
    BCC1, BCC2, ECC1, ECC2
  \end{funconformance}
\end{function_nopb2}

\begin{function_nopb2}{OSError\_SetEvent\_TaskID}{OSError_SetEvent_TaskID}
  \synopsis{TaskType OSError_SetEvent_TaskID(void)}
  \begin{fundescription}
    The function returns the \const{TaskID} parameter passed to
    \reffun{SetEvent}. The function must be used inside
    \reffun{ErrorHook} after having discovered using
    \reffun{OSErrorGetServiceId} that the error was caused by that
    function.
  \end{fundescription}
  \begin{funreturn}
    \fret{Task ID}{The value of the \const{TaskID} parameter.}
  \end{funreturn}
  \begin{funconformance}
    ECC1, ECC2
  \end{funconformance}
\end{function_nopb2}

\begin{function_nopb2}{OSError\_SetEvent\_Mask}{OSError_SetEvent_Mask}
  \synopsis{EventMaskType OSError_SetEvent_Mask(void)}
  \begin{fundescription}
    The function returns the \const{Mask} parameter passed to
    \reffun{SetEvent}. The function must be used inside
    \reffun{ErrorHook} after having discovered using
    \reffun{OSErrorGetServiceId} that the error was caused by that
    function.
  \end{fundescription}
  \begin{funreturn}
    \fret{Mask}{The value of the \const{Mask} parameter.}
  \end{funreturn}
  \begin{funconformance}
    ECC1, ECC2
  \end{funconformance}
\end{function_nopb2}

\begin{function_nopb2}{OSError\_ClearEvent\_Mask}{OSError_ClearEvent_Mask}
  \synopsis{EventMaskType OSError_ClearEvent_Mask(void)}
  \begin{fundescription}
    The function returns the \const{Mask} parameter passed to
    \reffun{ClearEvent}. The function must be used inside
    \reffun{ErrorHook} after having discovered using
    \reffun{OSErrorGetServiceId} that the error was caused by that
    function.
  \end{fundescription}
  \begin{funreturn}
    \fret{Mask}{The value of the \const{Mask} parameter.}
  \end{funreturn}
  \begin{funconformance}
    ECC1, ECC2
  \end{funconformance}
\end{function_nopb2}

\begin{function_nopb2}{OSError\_GetEvent\_TaskID}{OSError_GetEvent_TaskID}
  \synopsis{TaskType OSError_GetEvent_TaskID(void)}
  \begin{fundescription}
    The function returns the \const{TaskID} parameter passed to
    \reffun{GetEvent}. The function must be used inside
    \reffun{ErrorHook} after having discovered using
    \reffun{OSErrorGetServiceId} that the error was caused by that
    function.
  \end{fundescription}
  \begin{funreturn}
    \fret{Task ID}{The value of the \const{TaskID} parameter.}
  \end{funreturn}
  \begin{funconformance}
    ECC1, ECC2
  \end{funconformance}
\end{function_nopb2}

\begin{function_nopb2}{OSError\_GetEvent\_Event}{OSError_GetEvent_Event}
  \synopsis{EventMaskRefType OSError_GetEvent_Event(void)}
  \begin{fundescription}
    The function returns the \const{TaskID} parameter passed to
    \reffun{GetEvent}. The function must be used inside
    \reffun{ErrorHook} after having discovered using
    \reffun{OSErrorGetServiceId} that the error was caused by that
    function.
  \end{fundescription}
  \begin{funreturn}
    \fret{Event}{The value of the \const{Event} parameter.}
  \end{funreturn}
  \begin{funconformance}
    ECC1, ECC2
  \end{funconformance}
\end{function_nopb2}

\begin{function_nopb2}{OSError\_WaitEvent\_Mask}{OSError_WaitEvent_Mask}
  \synopsis{EventMaskType OSError_WaitEvent_Mask(void)}
  \begin{fundescription}
    The function returns the \const{Mask} parameter passed to
    \reffun{WaitEvent}. The function must be used inside
    \reffun{ErrorHook} after having discovered using
    \reffun{OSErrorGetServiceId} that the error was caused by that
    function.
  \end{fundescription}
  \begin{funreturn}
    \fret{Mask}{The value of the \const{Mask} parameter.}
  \end{funreturn}
  \begin{funconformance}
    ECC1, ECC2
  \end{funconformance}
\end{function_nopb2}

\begin{function_nopb2}{OSError\_GetAlarmBase\_AlarmID}{OSError_GetAlarmBase_AlarmID}
  \synopsis{AlarmType OSError_GetAlarmBase_AlarmID(void)}
  \begin{fundescription}
    The function returns the \const{AlarmID} parameter passed to
    \reffun{GetAlarmBase}. The function must be used inside
    \reffun{ErrorHook} after having discovered using
    \reffun{OSErrorGetServiceId} that the error was caused by that
    function.
  \end{fundescription}
  \begin{funreturn}
    \fret{Alarm ID}{The value of the \const{AlarmID} parameter.}
  \end{funreturn}
  \begin{funconformance}
    BCC1, BCC2, ECC1, ECC2
  \end{funconformance}
\end{function_nopb2}

\begin{function_nopb2}{OSError\_GetAlarmBase\_Info}{OSError_GetAlarmBase_Info}
  \synopsis{AlarmBaseRefType OSError_GetAlarmBase_Info(void)}
  \begin{fundescription}
    The function returns the \const{Info} parameter passed to
    \reffun{GetAlarmBase}. The function must be used inside
    \reffun{ErrorHook} after having discovered using
    \reffun{OSErrorGetServiceId} that the error was caused by that
    function.
  \end{fundescription}
  \begin{funreturn}
    \fret{Info}{The value of the \const{Info} parameter.}
  \end{funreturn}
  \begin{funconformance}
    BCC1, BCC2, ECC1, ECC2
  \end{funconformance}
\end{function_nopb2}

\begin{function_nopb2}{OSError\_GetAlarm\_AlarmID}{OSError_GetAlarm_AlarmID}
  \synopsis{AlarmType OSError_GetAlarm_AlarmID(void)}
  \begin{fundescription}
    The function returns the \const{AlarmID} parameter passed to
    \reffun{GetAlarm}. The function must be used inside
    \reffun{ErrorHook} after having discovered using
    \reffun{OSErrorGetServiceId} that the error was caused by that
    function.
  \end{fundescription}
  \begin{funreturn}
    \fret{Alarm ID}{The value of the \const{AlarmID} parameter.}
  \end{funreturn}
  \begin{funconformance}
    BCC1, BCC2, ECC1, ECC2
  \end{funconformance}
\end{function_nopb2}

\begin{function_nopb2}{OSError\_GetAlarm\_Tick}{OSError_GetAlarm_Tick}
  \synopsis{TickRefType OSError_GetAlarm_Tick(void)}
  \begin{fundescription}
    The function returns the \const{Tick} parameter passed to
    \reffun{GetAlarm}. The function must be used inside
    \reffun{ErrorHook} after having discovered using
    \reffun{OSErrorGetServiceId} that the error was caused by that
    function.
  \end{fundescription}
  \begin{funreturn}
    \fret{Tick}{The value of the \const{Tick} parameter.}
  \end{funreturn}
  \begin{funconformance}
    BCC1, BCC2, ECC1, ECC2
  \end{funconformance}
\end{function_nopb2}

\begin{function_nopb2}{OSError\_SetRelAlarm\_AlarmID}{OSError_SetRelAlarm_AlarmID}
  \synopsis{AlarmType OSError_SetRelAlarm_AlarmID(void)}
  \begin{fundescription}
    The function returns the \const{AlarmID} parameter passed to
    \reffun{SetRelAlarm}. The function must be used inside
    \reffun{ErrorHook} after having discovered using
    \reffun{OSErrorGetServiceId} that the error was caused by that
    function.
  \end{fundescription}
  \begin{funreturn}
    \fret{Alarm ID}{The value of the \const{AlarmID} parameter.}
  \end{funreturn}
  \begin{funconformance}
    BCC1, BCC2, ECC1, ECC2
  \end{funconformance}
\end{function_nopb2}

\begin{function_nopb2}{OSError\_SetRelAlarm\_increment}{OSError_SetRelAlarm_increment}
  \synopsis{TickType OSError_SetRelAlarm_increment(void)}
  \begin{fundescription}
    The function returns the \const{increment} parameter passed to
    \reffun{SetRelAlarm}. The function must be used inside
    \reffun{ErrorHook} after having discovered using
    \reffun{OSErrorGetServiceId} that the error was caused by that
    function.
  \end{fundescription}
  \begin{funreturn}
    \fret{increment}{The value of the \const{increment} parameter.}
  \end{funreturn}
  \begin{funconformance}
    BCC1, BCC2, ECC1, ECC2
  \end{funconformance}
\end{function_nopb2}

\begin{function_nopb2}{OSError\_SetRelAlarm\_cycle}{OSError_SetRelAlarm_cycle}
  \synopsis{TickType OSError_SetRelAlarm_cycle(void)}
  \begin{fundescription}
    The function returns the \const{cycle} parameter passed to
    \reffun{SetRelAlarm}. The function must be used inside
    \reffun{ErrorHook} after having discovered using
    \reffun{OSErrorGetServiceId} that the error was caused by that
    function.
  \end{fundescription}
  \begin{funreturn}
    \fret{cycle}{The value of the \const{cycle} parameter.}
  \end{funreturn}
  \begin{funconformance}
    BCC1, BCC2, ECC1, ECC2
  \end{funconformance}
\end{function_nopb2}

\begin{function_nopb2}{OSError\_SetAbsAlarm\_AlarmID}{OSError_SetAbsAlarm_AlarmID}
  \synopsis{AlarmType OSError_SetAbsAlarm_AlarmID(void)}
  \begin{fundescription}
    The function returns the \const{AlarmID} parameter passed to
    \reffun{SetAbsAlarm}. The function must be used inside
    \reffun{ErrorHook} after having discovered using
    \reffun{OSErrorGetServiceId} that the error was caused by that
    function.
  \end{fundescription}
  \begin{funreturn}
    \fret{Alarm ID}{The value of the \const{AlarmID} parameter.}
  \end{funreturn}
  \begin{funconformance}
    BCC1, BCC2, ECC1, ECC2
  \end{funconformance}
\end{function_nopb2}

\begin{function_nopb2}{OSError\_SetAbsAlarm\_start}{OSError_SetAbsAlarm_start}
  \synopsis{TickType OSError_SetAbsAlarm_start(void)}
  \begin{fundescription}
    The function returns the \const{start} parameter passed to
    \reffun{SetAbsAlarm}. The function must be used inside
    \reffun{ErrorHook} after having discovered using
    \reffun{OSErrorGetServiceId} that the error was caused by that
    function.
  \end{fundescription}
  \begin{funreturn}
    \fret{start}{The value of the \const{start} parameter.}
  \end{funreturn}
  \begin{funconformance}
    BCC1, BCC2, ECC1, ECC2
  \end{funconformance}
\end{function_nopb2}

\begin{function_nopb2}{OSError\_SetAbsAlarm\_cycle}{OSError_SetAbsAlarm_cycle}
  \synopsis{TickType OSError_SetAbsAlarm_cycle(void)}
  \begin{fundescription}
    The function returns the \const{cycle} parameter passed to
    \reffun{SetAbsAlarm}. The function must be used inside
    \reffun{ErrorHook} after having discovered using
    \reffun{OSErrorGetServiceId} that the error was caused by that
    function.
  \end{fundescription}
  \begin{funreturn}
    \fret{cycle}{The value of the \const{cycle} parameter.}
  \end{funreturn}
  \begin{funconformance}
    BCC1, BCC2, ECC1, ECC2
  \end{funconformance}
\end{function_nopb2}


\begin{function_nopb2}{OSError\_CancelAlarm\_AlarmID}{OSError_CancelAlarm_AlarmID}
  \synopsis{AlarmType OSError_CancelAlarm_AlarmID(void)}
  \begin{fundescription}
    The function returns the \const{AlarmID} parameter passed to
    \reffun{CancelAlarm}. The function must be used inside
    \reffun{ErrorHook} after having discovered using
    \reffun{OSErrorGetServiceId} that the error was caused by that
    function.
  \end{fundescription}
  \begin{funreturn}
    \fret{Alarm ID}{The value of the \const{AlarmID} parameter.}
  \end{funreturn}
  \begin{funconformance}
    BCC1, BCC2, ECC1, ECC2
  \end{funconformance}
\end{function_nopb2}

\begin{function_nopb2}{OSError\_IncrementCounter\_AlarmID}{OSError_IncrementCounter_AlarmID}
  \synopsis{AlarmType OSError_IncrementCounter_AlarmID(void)}
  \begin{fundescription}
    The function returns the \const{AlarmID} of the Alarm notification
    triggered by \reffun{IncrementCounter} that raised the error. The function must
    be used inside \reffun{ErrorHook} after having discovered using
    \reffun{OSErrorGetServiceId} that the error was caused by that
    function.
  \end{fundescription}
  \begin{funreturn}
    \fret{Alarm ID}{The value of the \const{AlarmID} of the Alarm
      notification that triggered the error.}
  \end{funreturn}
  \begin{funconformance}
    BCC1, BCC2, ECC1, ECC2
  \end{funconformance}
\end{function_nopb2}

\begin{function_nopb2}{OSError\_IncrementCounter\_TaskID}{OSError_IncrementCounter_TaskID}
  \synopsis{TaskType OSError_IncrementCounter_TaskID(void)}
  \begin{fundescription}
    The function returns the \const{TaskID} related to the Alarm
    notification triggered by \reffun{IncrementCounter} that raised the
    error. The function must be used inside \reffun{ErrorHook} after
    having discovered using \reffun{OSErrorGetServiceId} that the
    error was caused by that function.
  \end{fundescription}
  \begin{funreturn}
    \fret{Task ID}{The value of the \const{TaskID} of the Alarm
      notification that triggered the error.}
  \end{funreturn}
  \begin{funconformance}
    BCC1, BCC2, ECC1, ECC2
  \end{funconformance}
\end{function_nopb2}

\begin{function_nopb2}{OSError\_IncrementCounter\_Mask}{OSError_IncrementCounter_Mask}
  \synopsis{EventMaskType OSError_IncrementCounter_Mask(void)}
  \begin{fundescription}
    The function returns the Event \const{Mask} related to the Alarm
    notification triggered by \reffun{IncrementCounter} that raised the
    error. The function must be used inside \reffun{ErrorHook} after
    having discovered using \reffun{OSErrorGetServiceId} that the
    error was caused by that function.
  \end{fundescription}
  \begin{funreturn}
    \fret{Mask}{The value of the Event Mask of the Alarm
      notification that triggered the error.}
  \end{funreturn}
  \begin{funconformance}
    ECC1, ECC2
  \end{funconformance}
\end{function_nopb2}

\begin{function_nopb2}{OSError\_IncrementCounter\_action}{OSError_IncrementCounter_action}
  \synopsis{EE_TYPENOTIFY OSError_IncrementCounter_action(void)}
  \begin{fundescription}
    The function returns the \const{action} related to the Alarm
    notification triggered by \reffun{IncrementCounter} that raised the
    error. The function must be used inside \reffun{ErrorHook} after
    having discovered using \reffun{OSErrorGetServiceId} that the
    error was caused by that function.
  \end{fundescription}
  \begin{funreturn}
    \fret{action}{The value of the action of the Alarm notification
      that triggered the error. Possible values of this parameter are:
      \const{EE_ALARM_ACTION_TASK} for task notifications,
      \const{EE_ALARM_ACTION_CALLBACK} if the notification is an alarm
      callback, and \const{EE_ALARM_ACTION_EVENT} (only available with
      ECC1 and ECC2), if the notification is an event set.}
  \end{funreturn}
  \begin{funconformance}
    BCC1, BCC2, ECC1, ECC2
  \end{funconformance}
\end{function_nopb2}

\begin{function_nopb2}{OSError\_StartOS\_Model}{OSError_StartOS_Mode}
  \synopsis{AppModeType OSError_StartOS_Mode(void)}
  \begin{fundescription}
    The function returns the \const{Model} related to the StartOS
    Call. The function must be used inside \reffun{ErrorHook} after
    having discovered using \reffun{OSErrorGetServiceId} that the
    error was caused by that function.
  \end{fundescription}
  \begin{funreturn}
    \fret{Mode}{The Mode parameter that was passed to the
                \reffun{StartOS} primitive.}
  \end{funreturn}
  \begin{funconformance}
    BCC1, BCC2, ECC1, ECC2
  \end{funconformance}
\end{function_nopb2}


%%%%%%%%%%%%%%%%%%%%%%%%%%%%%%%%%%%%%%%%%%%%%%%%%%%%%%%%%%%%%%%

% EG: General ISR section
\section{Interrupt service routines}

\ee\ Set that all interrupts of category 1 have to have a higher or equal
hardware priority compared with interrupts of category 2. This limitation has
been introduced to avoid that any activities corresponding to the calls of the
operating system performed in the context of an ISR2 interruting an ISR1 are
unbounded delayed, until the next rescheduling, beacuse ISR1 are outside of
\ee\ bounduaries, so any action can be done at the end of the one of them.
This is the only limitation common to all \ee\ porting, so you should check
specific architecture manuals and/or wiki pages if more limitation
regarding ISR priority levels has been introduced.
