\chapter{Introduction}

\section{\ee\ and \rtd}

\ee\ is a free of charge, open-source RTOS implementation of the
OSEK/VDX API, available for various microcontrollers on the web page
\url{http://erika.tuxfamily.org}.

\ee\ offers the availability of a real-time scheduler and resource 
managers allowing the full exploitation of the power of new generation 
micro-controllers and multicore platforms while guaranteeing predictable 
real-time performance and retaining the programming model of 
conventional single processor architectures.

The advanced features provided by \ee\ are:

\begin{itemize}
\item Support for four conformance classes to match different
application requirements;
\item Support for preemptive and non-preemptive multitasking;
\item Support for fixed priority scheduling;
\item Support for stack sharing techniques, and one-shot task
model to reduce the overall stack usage;
\item Support for shared resources;
\item Support for periodic activations using Alarms;
\item Support for centralized Error Handling;
\item Support for hook functions before and after each context
switch;
\end{itemize}

The \ee\ kernel is a complete OSEK/VDX environment, which can be used
to implement multithreading applications. The \ee\ API providing
support for thread activation, mutual exclusion, alarms, events and
counting semaphores.

The OSEK/VDX consortium provides the OIL language (OSEK Implementation
Language) as a standard configuration language, which is used for the
static definition of the RTOS objects which are instantiated and used
by the application. \ee\ fully supports the OIL language for the
configuration of real-time applications.

\ee\ is natively supported by \rtd, a tool suite for the automatic
configuration and deployment of embedded applications which enables to
easily exploit multi-processor architectures and achieve the desired
performance without modifying the application source code. More details
about \rtd\ are available in a dedicated reference manual.

The purpose of this document is to describe in detail the \ee\
API. Details about specific architecture can be found in dedicated manuals,
available for download on the Erika web site.

%% \section{OSEK/VDX Copyright notice}

%% We acknowledge that OSEK/VDX is a registered trademark by:

%% \begin{lstlisting}
%%         Continental Automotive GmbH
%%         Vahrenwalderstra{\ss}e 9
%%         30165 Hannover
%%         Germany
%% \end{lstlisting}




