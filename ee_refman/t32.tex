%
% Tool: this chapter should be decoupled from the general ee_refman
%       and should be put int oa separate manual to be made available 
%		into the general decomentation section.
%
\chapter{ORTI and Lauterbach Trace32 support}
\label{sec:orti_t32}

\ee\ provides direct support for application debugging using the
Lauterbach Trace32 debugger \cite{Lauterbach}.

The provided support includes:
\begin{itemize}
\item Automatic generation of Trace32 scripts to simplify the load and
  debugging of \ee\ applications.
\item Automatic generation of batch scripts for single and multicore
  Nios II designs. An instance of Trace32 will be started and
  configurated for each core in the design.
\item Automatic generation of the ORTI files to enable kernel
  awareness with Lauterbach Trace32. In this way, all the object
  declared in the system (Tasks, Resources, Alarms, Stacks, Context
  switches, ...) can be evaluated from the Trace32 debugger.
\end{itemize}

This chapter describes the impact of the ORTI usage in \ee\
applications with respect to the impact to the code footprint of the
ORTI code, and with respect to the stack usage statistics performed by
Lauterbach Trace32. More information on how to generate the Lauterbach
Trace32 scripts, and the ORTI files can be found in the \rtd\
Reference Manual.

%%%%%%%%%%%%%%%%%%%%%%%%%%%%%%%%%%%%%%%%%%%%%%%%%%%%%%%%%%%%%%%%%%%%%%

\section{ORTI and \ee\ footprint}
The ORTI support provided by Lauterbach Trace32 is a nice feature that
allow developers to track in a graphical way the kernel objects like
tasks, resources, alarms, and stacks. When using ORTI, \ee\ is
configured appropriately by \rtd\ by adding a set of parameters inside
the makefiles. As a result, the kernel records a set of additional
information which are then referred inside the ORTI file. For that
reason, for a maximum reduction of the \ee\ footprint, the ORTI options
should be disabled.

%%%%%%%%%%%%%%%%%%%%%%%%%%%%%%%%%%%%%%%%%%%%%%%%%%%%%%%%%%%%%%%%%%%%%%

\section{ORTI and stack usage statistics}
Another interesting feature of the ORTI support for Lauterbach Trace32
is the possibility of tracking the stack usage in the system. To use
the feature, the user should remind to fill up all the stack space
with the value \const{0xa5a5a5a5}. That can be usually done by putting
a call to \reffun{EE_trace32_stack_init} as the first instruction into
\fn{main} or into \fn{alt_main}.

%%%%%%%%%%%%%%%%%%%%%%%%%%%%%%%%%%%%%%%%%%%%%%%%%%%%%%%%%%%%%%%%%%%%%%

\begin{function2}{EE\_trace32\_stack\_init}{EE_trace32_stack_init}
  \synopsis{void EE_trace32_stack_init(void);}
  
  \begin{fundescription}
    This inline function can be used on the Altera Nios II platform to
    fill up the memory space from \const{__alt_stack_limit} to the
    current stack pointer with the pattern \const{0xa5a5a5a5}.

    The memory region from \const{__alt_stack_limit} to
    \const{0xa5a5a5a5} is the memory region that is typically
    allocated as stack space by the Altera System Library.

    This function should be called among the first functions in the
    system, before any call to \fn{malloc}, because the heap region is
    typically allocated at the end of the available stack
    space. Failing to do that may result in overwriting the data
    structures allocated with malloc, with potential unexpected
    results and application crashes.

    This function is only useful to get an estimation of the used
    stack space, and should be used in conjunction with the ORTI stack
    informations produced for Lauterbach Trace32. If this function is
    not called, it is likely that Lauterbach Trace32 will return a
    stack utilization of 100\% for all the stacks defined in the system.
  \end{fundescription}
  
  \begin{funconformance}
    BCC1, BCC2, ECC1, ECC2
  \end{funconformance}
\end{function2}

%%%%%%%%%%%%%%%%%%%%%%%%%%%%%%%%%%%%%%%%%%%%%%%%%%%%%%%%%%%%%%%%%%%%%%

