\chapter[Basic concepts]{Basic concepts of Real-time programming}
\label{cha:basics}

In this chapter, the basic concept behind real-time programming are
presented. We also anticipate the basic mechanisms underlying \ee\ and
we give an overview of the theory behind the \ee\ real-time services.

%-----------------------------------------------------------------%
\section{Real-time applications}

A real-time application is a program whose correctness depends not
only on the correctness of the produced results, but also on the {\em
time at which they are produced}. If results are produced too late,
the program is incorrect. 

In a {\em safety critical} real-time application, a late result can
cause catastrophic consequences. Examples of safety critical
applications include industrial plant control systems, automotive
control systems, aerospace systems, etc.

Many real-time applications are less critical. In thses systems, a
late result does not compromise the functionality. However, the
performance and the ``quality'' of the produced results depends on
their timeliness. These systems are often referred to as {\em soft
real-time}. Examples include telecommunication applications,
multimedia systems, some control system.

To ensure a correct behaviour, the designer of a real-time system must
adopt programming techniques and operating system services that
guarantee a {\em predictable behaviour}. Then, by performing an
appropriate {\em analysis}, it is possible to know a priori
(i.e. before delivering the system to the final user) if the timing
behaviour is correct under all operating conditions.

\ee\ is a real-time operating system that provides {\bf predictable
behaviour}. This means that it is possible to perform a {\em
scheduling analysis}\index{scheduling analysis} on programs developed
with \ee, by means of the {\bf \rtd\ Scheduling Analysis Plug-in}
(see the \rtd\ reference manual), to know if the application will
meet the timing constraints.

In most case, real-time systems interact with the external environment
(e.g. the system to be controlled) and must respond within certain
intervals of time to external events. As different events may happen
at different time instants, and must be processed with a different
priority, a natural choice for these systems is to use concurrent
programming. Therefore, in many cases a real-time application consists
of a set of concurrent tasks\index{task} that are managed by the RTOS
and its ``scheduler''. 

\subsection{Differences between real-time and non-real-time programming}

It is important to underline the main differences between classical
multitask programming for non-real-time applicatrions, and programming
with a real-time application with a RTOS in general, and with \ee\ in
particular. The following issues must be considered in a real-time
system.

\begin{itemize}
\item The number of tasks\index{task creation} and their behaviour
  must be known a-priori. Since the system must be predictable and to
  enable a {\em scheduling analysis}, it is necessary to estimate the
  worst-case situation for the system in terms of load. For this
  reason, the number of tasks and their code is usually fixed before
  run-time. In \ee, the tasks are created at system start-up and exist
  during the entire system lifetime. Tasks cannot be created
  dynamically at run-time.

\item The same rule applies to memory\index{dynamic memory} and shared
  objects, including alarms\index{alarm} and communication
  objects. Memory is statically allocated to tasks at compile time,
  and cannot be dynamically allocated. Mutex\index{mutex}, and in
  general any communication or synchronization objects is created at
  system start-up and cannot be dynamically created.

\item Not all synchronization and communication primitives are
  allowed. For example, generic semaphores are not used in real-time
  systems, as they could produce the well-known {\em priority
  inversion}\index{priority inversion} problem. We will present the
  priority inversion problem in Section \ref{??}. Also, tasks cannot
  self-suspend for a certain interval of time, as this complicates the
  analysis and introduces a potential source of unpredictability in
  the system.

\item Appropriate {\em scheduling algorithms}\index{scheduling
  algorithm} are used in RTOS, to ensure predictable behaviour. The
  most commonly used scheduling algorithm is the {\em fixed priority
  scheduling}, and it is the one used in \ee. Moreover, future version
  of \ee\ will allow different scheduling algorithms like {\em
  earliest deadline first} (EDF). Scheduling algorithms will be
  briefly described in Section \ref{sec:basics_scheduling}.

\item \nb{Altro??}

\end{itemize}

%-----------------------------------------------------------------%

\section{Tasks}
\label{sec:basics_tasks}

A tasks (or {\em thread})\index{task}\index{thread} is a piece of code
that execute sequentially. In a real-time applications, tasks are
usually activated by timers or by external or internal events, and
must complete their execution before a {\em deadline}\index{deadline}. 

\nb{basic tasks}

\nb{extended tasks}

A task goes though a set of {\em states}\index{task!states}. In Figure
\ref{fig:ready_diagram}, we report the state diagram of a
task. Initally, a task is {\em idle} and waits for an activation. When
a task is activated, it goes into the {\em ready} state. Since more
than one task can be ready at the same time, a {\em scheduler} selects
which tasks to execute among the ready tasks (see Section
\ref{sec:basics_scheduling}). When a task completes its execution, it
goes back into the idle state waiting for the next activation
event. The task's code executed between an activation and the time it
completes and becomes idle again is a task's {\em
instance}\index{task!instance} or {\em job}\index{task!job}. The
interval of time between the activation and the corresponding
completion is called {\em response time}\index{task!response time}. In
a real-time system, it is important to control the response time of
all tasks and guarantee that it is always within an interval of time
called {\em deadline}. More specifically, the {\em relative
deadline}\index{task!deadline} of a task is the maximum response time
that is allowed to a task for the system to be correct.

While a task is executing, if it tries to access a shared resource, it
may be blocked because the resource is busy. In this case, it goes to
the {\em blocked} state until the resource gets free.

\begin{figure}
  \begin{center}
    \nb{immagine mancante!!!}
   % \includegraphics[width=5cm]{images/ready_diagram.eps}
  \end{center}
  \caption{State diagram for a task.}
  \label{fig:ready_diagram}
\end{figure}

\nb{Put a typical example of thead code from erika, or remand to the
next chapter}.

In the real-time literature, tasks are classified according to their
activation type. A task can be {\em periodic}\index{task!periodic} if
it is periodically activated by a timer (or by an external events that
is triggered at constant intervals of time); it can be {\em
sporadic}\index{task!sporadic} if is is activated by an even that
occurs at variable intervals of time, but with a minimum interarrival
time between two consecutive events; or {\em
aperiodic}\index{task!aperiodic} if such minimum interarrival time
cannot be specified. Other models include {\em burst} activations,
recurring etc.

The first time a periodic task is activated is called {\em offset} or
{\em phase}. As we will briefly explain in the next Section, it is
often necessary to impose different offsets to periodic tasks, to
avoid the situation in which they are all activated at the same time.
In Section \ref{??}, we will present how to program a periodic task in
\ee.

\section{Scheduling}
\label{sec:basics_scheduling}

In a multi-task operating system, the {\em scheduler}\index{scheduler}
selects which task to execute among the set of active tasks. In a
RTOS, the scheduler is particularly important, as a {\em predictable}
schedule allows off-line analysis under all operating conditions. In
particular, it is necessary to compute the worst-case response time
(or an upper bound to it) of all tasks, to ensure that they will
complete before their deadline even in the worst possible situation.

The simplest and most common scheduler in RTOS is the {\em Fixed
Priority Scheduler}, and this scheduler is also the standard scheduler
for \ee. Another important scheduler is the {\em Earliest Deadline
First} and it will be available in a future version of \ee.

\subsection{Fixed priority}
\label{sec:basics_fp}

In fixed priority scheduling, each task is assigned a {\em
priority}\index{task!priority} between $[0,15]$ \nb{check this}, where
$15$ is the highest and $1$ the lowest priority: the ready task with
the highest priority is selected to execute. If two tasks have the
same priority, they are executed in FIFO order. The logical structure
used by the fixed priority scheduler is depicted in Figure
\ref{??}. There are $15$ FIFO queues, one for each priority. When a
task is activated or unblocked, is inserted at the end of the
corresponding queue. The first task of the highest priority non-empty
queue is selected to execute.  The actual implementation of this
structure in \ee\ has been highly optimized for minimizing the
overhead.

\begin{figure}
  \begin{center}
%%     \includegraphics[width=8cm]{images/ready_queues.eps}
  \end{center}
  \caption{Logical structure of the ready queue in fixed priority
    scheduling.}
  \label{fig:ready_queues}
\end{figure}

\nb{schedulability test?}


\subsection{Earliest Deadline First}
\label{sec:basics_edf}



%-----------------------------------------------------------------%

\section{Interrupt handlers}

Most microcontrollers and processor provide an interrupt mechanisms,
which is mostly used for asynchronous I/O.

When an external device wants to send some data (or when it has
completed some operation), it sends an ``interrupt signal'' to the
processor. Under normal conditions, when the interrupt signal is
raised, the processor suspends what it was doing and jumps into a
special section of code that processes the interrupt request. This
special section of code is usually an function called {\em interrupt
handler}. When the interrupt handler completes, the processor resumes
execution from where it was interrupted.

Usually, a processor provide a certain number of ``interrupt
signals'', one for each device. Each interrupt is identified by an
interrupt ID. For each interrupt it is possible to ``install'' a {\em
Interrupt handler}\index{interrupt!handler} routine, which is a
function that will be invoked each time the corresponding signal is
activated.

Some interrupt signals are used for RTOS internal purpouses. For
example, it is common to have a internal or external timer device that
periodically send interrupts to the processor. The RTOS uses this
signal and installs its own handler to keep track ot the time.

It is possible to disable interrupts with special processor
instructions. If interrupts are disabled, when an interrupt signal is
raised, it remains pending and the processor continues to process what
is was executing. When interrupts are enabled, the pending interrupts
are served.

The RTOS usually provides special system calls to
disable interrupts. \ee\ provides a set of function to enable/disable
interrupts (see \reffun{DisableAllInterrupts},
\reffun{EnableAllInterrupts}, \reffun{SuspendAllInterrupts},
\reffun{ResumeAllInterrupts}, \reffun{SuspendOSInterrupts},
\reffun{ResumeOSInterrupts}).

Each interrupt signal has a {\em priority}. If more than one interrupt
signal arrives (or is pending), it is processed in order of
priority. The priority order of each interrupt depends on the specific
processor/board. Please see the target manual of your reference
architecture for more details.

It is possible to {\em nest} interrupts. If the processor is executing
the interrupt handler of a lower priority interrupt signal, and
another higher priority interrupt signal is raised, and interrupts are
enabled, the processor suspend what it was doing to execute the
handler corresponding to the higher priority interrupt signal. When
this new handler is completed, the processor resume execution of the
lower priority handler.

The priority of interrupt handlers is higher than the priority of the
tasks. Therefore, an interrupt handler always preempts a task (unless
interrupts are disabled). The logical relationship between task
priorities and interrupt handlers priorities is shown in Figure
\ref{???}.

In \ee, interrupt handlers are of two types: 

\begin{itemize}
\item interrupt handlers of type 1; from these handlers, no RTOS
  service can be invoked;
\item interrupt handlers of type 2; from these handler, a certain
  number of RTOS service can be invoked.
\end{itemize}


%-----------------------------------------------------------------%

\section{Shared resources and blocking times}



\subsection{Priority Inversion}

\subsection{Stack Resource Policy}

%-----------------------------------------------------------------%

\section{Non preemptive scheduling}

\subsection{Preemption thresholds}

\subsection{Pseudo-resources}

\subsection{Equivalence with SRP}

%-----------------------------------------------------------------%

\section{Communication}

\subsection{CABs}

%-----------------------------------------------------------------%

\section{Application modes}

%-----------------------------------------------------------------%

\section{Multiprocessor scheduling}

\subsection{Allocation}

\subsection{Scheduling}

\subsection{Resource Sharing}

\subsection{Communication}

%-----------------------------------------------------------------%

\section{Bibliography on RT scheduling}

\nb{Citare: Libro di Giorgio e di J. Liu, libro di Alan Burns su FP e
dei vari su RM analysis. Citare papers pi� importanti.}
