\chapter[Introduction]{Introduction}
\label{cha:introduction}

Embedded microcontroller units are spreading in thousands of
applications, ranging from single to distributed systems, control
applications, multimedia, communication, medical applications and many
others. Modern microcontrollers, which are growing in computational
power, speed and interfacing capabilities, are more and more feeling
the need of tools to make the development of complex scalable
applications easier.

The \dspic\ family represents one of the latest products of Microchip
Technology Inc., the world leading company in the field of
microcontroller units. With a speed of up to 40 MHz, the \dspic\ family
seamlessly integrates a DSP core for high performance computation with
a full range of interfaces to several buses like CAN, I2C, SPI, serial
lines, and so on.  

\section{\ee\ and \rtd\ for \dspic}

Embedded applications often require tight control on the temporal
behavior of each single activity in the system. The research in the
field of real-time systems brought the team of Evidence Srl to design
a small, efficient, modular real-time kernel that can be used to
easily guarantee real-time constraints in every embedded applications.

\ee\ and \rtd\ represent the answer of Evidence Srl for the
development of scalable real-time applications for the \dspic\
family.

\ee\ provides \dspic\ developers the following features:

\begin{description}
\item[Traditional RTOS features] ~
  \begin{itemize}
  \item Support for four conformance classes to match different
    application requirements;
  \item Support for preemptive and non-preemptive multitasking;
  \item Support for fixed priority scheduling;
  \item Support for stack sharing techniques, and one-shot task model
    to reduce the overall stack usage;
  \item Support for shared resources;
  \item Support for periodic activations using Alarms;
  \item Support for centralized Error Handling;
  \item Support for hook functions before and after each context
    switch.
  \end{itemize}

\item[\rtd\ development environment] ~
  \begin{itemize}
  \item Development environment based on the Eclipse IDE;
  \item Support for the OIL language for the specification of the RTOS
    configuration;
  \item Graphical configuration plugin to easily generate the OIL
    configuration file and to easily configure the RTOS parameters;
  \item Full integration with the Cygwin development environment to
    provide a Unix-style scripting environment;
  \item Apache ANT scripting support for code generation;
  \end{itemize}

\item[\dspic\ integration features] ~
  \begin{itemize}
  \item Installation setup which integrates Microchip software
    together with fully configured Evidence \ee\ + \rtd;
  \item Full support for the Microchip devices libraries;
  \item Full support for the Microchip C30 compiler;
  \item Full support of the MPLAB IDE debugging environment;
  \item Full support for the Microchip ICD2 debugger;
  \item Full support for \dspic\ series 30 and 33, and for PIC24;
  \item Support for the 802.15.4 (ZigBee) wireless communication
    protocol (coming soon);
  \item Support for I/O to Multimedia Card (MMC) / Secure Digital with
    FAT filesystem (coming soon);
  \item Development of many specific hardware drivers for \dspic, like
    multiple servomotor driving, bus EIB support (domotic), and many
    other (coming soon);
  \item Support for the FLEX development board and for some other
    \dspic\ evaluation boards;
  \end{itemize}
\end{description}


\section{Integration with Microchip Inc. products}

\ee\ and \rtd\ aims to the best integration with the existing tools
for development available from Microchip Inc.

\rtd\ will be used to quickly configure the application, setting
temporal parameters of real-time tasks, memory requirements, stack
allocation and many other parameters. \rtd\ generates the application
template, and leaves the developer the task to implement the logic of
each single task.

While programming the application, the developer can exploit the power
and flexibility offered by the primitives of the \ee\ real-time
kernel.

The application can be imported into MPLAB IDE to be written into the
\dspic\ EPROM flash memory. Moreover, the application can be debugged
from within the MPLAB IDE.

\section{Content of this document}

The purpose of this document is to describe all the information needed
to create, develop and modify an \ee\ application under the Microchip
\dspic\ family of microcontrollers. In particular, the document describes:
\begin{itemize}
\item The design flow which should be used to generate an \ee\ application;
\item The configuration of the development environment;
\item The options which are available to configure the system.
\end{itemize}

\begin{note}
If you are looking for a step-by-step / quick guide tutorial on how to
use \ee\ and \rtd\ with \dspic, please read the ``\ee\
Tutorial for the microchip \dspic\ Platform'', available for download on
the Evidence Web site.
\end{note}
