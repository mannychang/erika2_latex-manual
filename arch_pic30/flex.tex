\chapter[Flex Board]{Flex Board}

\section{Introduction}

This chapter describes the support done in \ee\ for the
Evidence/Embedded solutions Flex Board.

Flex is an embedded board which can be used by all the developers who
want to fully exploit the potential of the latest Microchip
micro-controllers: the \dspic\ family.

Flex is born as a development board where to easily develop and test
real-time applications for the \dspic\ micro-controller. The
main features of Flex are:
\begin{itemize}
\item robust electronic design;
\item modular architecture;
\item availability of a growing number of application notes;
\item the full support of \ee.
\end{itemize}

%
% aggiungere screenshot della board
%

To configure the usage of the Flex Board, the user has to specify an
appropriate \const{BOARD_DATA}, as in the following example:

\begin{lstlisting}
  ...
  BOARD_DATA = EE_FLEX {
    ...
  }
  ...
\end{lstlisting}

The Flex board supports a set of devices which are directly mounted on
it, plus a set of additional devices mounted on specific add-on
boards. These devices can be configured by adding attributes inside
the \const{BOARD_DATA} section.

The supported devices and the API functions needed to use them are
described in the following sections.

\begin{warning}
The current version of the board support for Flex only supports the
\dspic\ model 33FJ256GP710.
\end{warning}

\section{System LED}

The Flex Board has a system LED attached to a GPIO pin of the
microcontroller. To use the system LED on the Flex Board, the
developer should specify the \const{USELEDS} attribute as TRUE, as in
the following example:

\begin{lstlisting}
  ...
  BOARD_DATA = EE_FLEX {
    USELEDS = TRUE;
    ...
  }
  ...
\end{lstlisting}

The following subsections will describe the functions available to
control the Flex System LED.

\begin{function_nopb2}{EE\_leds\_init}{EE_leds_init:flex}
  \synopsis{void EE_leds_init(void);}
  
  \begin{fundescription}
    The function configures the GPIO pin. The LED starts turned off.
  \end{fundescription}
  
%  \begin{funparameters}
%    \fpar{none}{None in this function.}
%  \end{funparameters}
  
%  \begin{funreturn}
%    \fret{void}{The function does not return a value.}
%  \end{funreturn}
  
%  \begin{funconformance}
%  \end{funconformance}
\end{function_nopb2}

\begin{function_nopb2}{EE\_led\_sys\_on}{EE_led_sys_on}
  \synopsis{void EE_led_sys_on(void);}
  
  \begin{fundescription}
    The function turns on the LED.
  \end{fundescription}
  
%  \begin{funparameters}
%    \fpar{none}{None in this function.}
%  \end{funparameters}
  
%  \begin{funreturn}
%    \fret{void}{The function does not return a value.}
%  \end{funreturn}
  
%  \begin{funconformance}
%  \end{funconformance}
\end{function_nopb2}

\begin{function_nopb2}{EE\_led\_sys\_off}{EE_led_sys_off}
  \synopsis{void EE_led_sys_off(void);}
  
  \begin{fundescription}
    The function turns off the LED.
  \end{fundescription}
  
%  \begin{funparameters}
%    \fpar{none}{None in this function.}
%  \end{funparameters}
  
%  \begin{funreturn}
%    \fret{void}{The function does not return a value.}
%  \end{funreturn}
  
%  \begin{funconformance}
%  \end{funconformance}
\end{function_nopb2}

\begin{function_nopb2}{EE\_led\_on}{EE_led_on:flex}
  \synopsis{void EE_led_on(void);}
  
  \begin{fundescription}
    The function turns on the LED.
  \end{fundescription}
  
%  \begin{funparameters}
%    \fpar{none}{None in this function.}
%  \end{funparameters}
  
%  \begin{funreturn}
%    \fret{void}{The function does not return a value.}
%  \end{funreturn}
  
%  \begin{funconformance}
%  \end{funconformance}
\end{function_nopb2}

\begin{function_nopb2}{EE\_led\_off}{EE_led_off:flex}
  \synopsis{void EE_led_off(void);}
  
  \begin{fundescription}
    The function turns off the LED.
  \end{fundescription}
  
%  \begin{funparameters}
%    \fpar{none}{None in this function.}
%  \end{funparameters}
  
%  \begin{funreturn}
%    \fret{void}{The function does not return a value.}
%  \end{funreturn}
  
%  \begin{funconformance}
%  \end{funconformance}
\end{function_nopb2}
















\chapter[Flex Demo Daughter Board]{Flex Demo Daughter Board}

\section{Introduction}

This chapter describes the support done in \ee\ for the
Evidence/Embedded solutions Flex Demo Daughter Board.

The Demo Daughter board is a small board that plugs on the Flex Light
/ Flex Full connectors and which provides a set of devices useful to
implement small demos and demonstrators of control algorithms.

%\nb{nino, controlla!!!}

The main features of the Flex Demo Daughter board are:
\begin{itemize}
\item 8 leds;
\item 4 buttons;
\item 1 accelerometer;
\item 8 Analog Inputs;
\item 1 buzzer;
\item 1 DAC;
\item direct availability of the MCU encoder pins
\item Infrared receiver;
\item 16x2 characters LCD;
\item PWM outputs;
\item Temperature sensor;
\item Light sensor;
\item Trimmer;
\item Connections to implement the USB communication on the FLEX Full
\item Zigbee connector for Microchip or Easybee modules.
\end{itemize}

%
% aggiungere screenshot della board
%

To configure the usage of the Flex Demo Daughter Board, the user has
to specify an appropriate \const{BOARD_DATA}, as in the following
example:

\begin{lstlisting}
  ...
  BOARD_DATA = EE_FLEX {
    TYPE = DEMO {
	...
    };
    ...
  }
  ...
\end{lstlisting}

The supported devices and the API functions needed to use them are
described in the following sections.

\section{LEDS}

The Flex Demo Daughter board hosts 8 LEDs which are attached to GPIO
pins. To use the LEDs on the Flex Demo Daughter Board, the developer should
include the following fragment in the application OIL file:
\begin{lstlisting}
  ...
  BOARD_DATA = EE_FLEX {
    TYPE = DEMO { OPTIONS = LEDS; };
    ...
  }
  ...
\end{lstlisting}

The following subsections will describe the functions available to
control the Flex Demo Daughter Board LEDs.

\begin{function_nopb2}{EE\_demoboard\_leds\_init}{EE_demoboard_leds_init:flex}
  \synopsis{void EE_demoboard_leds_init(void);}
  
  \begin{fundescription}
    The function configures the LEDs, which starts turned off.
  \end{fundescription}
  
%  \begin{funparameters}
%    \fpar{none}{None in this function.}
%  \end{funparameters}
  
%  \begin{funreturn}
%    \fret{void}{The function does not return a value.}
%  \end{funreturn}
  
%  \begin{funconformance}
%  \end{funconformance}
\end{function_nopb2}


\begin{function_nopb2}{EE\_leds}{EE_leds:flex}
  \synopsis{void EE_leds(EE_UINT8 data);}
  
  \begin{fundescription}
    The function sets the led values using the \const{data} parameter.
  \end{fundescription}
\end{function_nopb2}

\begin{function_nopb2}{EE\_leds\_on}{EE_leds_on:flex}
  \synopsis{void EE_leds_on(void);}
  
  \begin{fundescription}
    The function turns all the LEDs on.
  \end{fundescription}
\end{function_nopb2}

\begin{function_nopb2}{EE\_leds\_off}{EE_leds_off:flex}
  \synopsis{void EE_leds_off(void);}
  
  \begin{fundescription}
    The function turns all the LEDs off.
  \end{fundescription}
\end{function_nopb2}

\begin{function_nopb2}{EE\_led\_0\_on}{EE_led_0_on:flex}
  \synopsis{void EE_led_0_on(void);}
  
  \begin{fundescription}
    The function turns LED 0 on.
  \end{fundescription}
\end{function_nopb2}

\begin{function_nopb2}{EE\_led\_0\_off}{EE_led_0_off:flex}
  \synopsis{void EE_led_0_off(void);}
  
  \begin{fundescription}
    The function turns LED 0 off.
  \end{fundescription}
\end{function_nopb2}

\begin{function_nopb2}{EE\_led\_1\_on}{EE_led_1_on:flex}
  \synopsis{void EE_led_1_on(void);}
  
  \begin{fundescription}
    The function turns LED 1 on.
  \end{fundescription}
\end{function_nopb2}

\begin{function_nopb2}{EE\_led\_1\_off}{EE_led_1_off:flex}
  \synopsis{void EE_led_1_off(void);}
  
  \begin{fundescription}
    The function turns LED 1 off.
  \end{fundescription}
\end{function_nopb2}

\begin{function_nopb2}{EE\_led\_2\_on}{EE_led_2_on:flex}
  \synopsis{void EE_led_2_on(void);}
  
  \begin{fundescription}
    The function turns LED 2 on.
  \end{fundescription}
\end{function_nopb2}

\begin{function_nopb2}{EE\_led\_2\_off}{EE_led_2_off:flex}
  \synopsis{void EE_led_2_off(void);}
  
  \begin{fundescription}
    The function turns LED 2 off.
  \end{fundescription}
\end{function_nopb2}

\begin{function_nopb2}{EE\_led\_3\_on}{EE_led_3_on:flex}
  \synopsis{void EE_led_3_on(void);}
  
  \begin{fundescription}
    The function turns LED 3 on.
  \end{fundescription}
\end{function_nopb2}

\begin{function_nopb2}{EE\_led\_3\_off}{EE_led_3_off:flex}
  \synopsis{void EE_led_3_off(void);}
  
  \begin{fundescription}
    The function turns LED 3 off.
  \end{fundescription}
\end{function_nopb2}

\begin{function_nopb2}{EE\_led\_4\_on}{EE_led_4_on:flex}
  \synopsis{void EE_led_4_on(void);}
  
  \begin{fundescription}
    The function turns LED 4 on.
  \end{fundescription}
\end{function_nopb2}

\begin{function_nopb2}{EE\_led\_4\_off}{EE_led_4_off:flex}
  \synopsis{void EE_led_4_off(void);}
  
  \begin{fundescription}
    The function turns LED 4 off.
  \end{fundescription}
\end{function_nopb2}

\begin{function_nopb2}{EE\_led\_5\_on}{EE_led_5_on:flex}
  \synopsis{void EE_led_5_on(void);}
  
  \begin{fundescription}
    The function turns LED 5 on.
  \end{fundescription}
\end{function_nopb2}

\begin{function_nopb2}{EE\_led\_5\_off}{EE_led_5_off:flex}
  \synopsis{void EE_led_5_off(void);}
  
  \begin{fundescription}
    The function turns LED 5 off.
  \end{fundescription}
\end{function_nopb2}

\begin{function_nopb2}{EE\_led\_6\_on}{EE_led_6_on:flex}
  \synopsis{void EE_led_6_on(void);}
  
  \begin{fundescription}
    The function turns LED 6 on.
  \end{fundescription}
\end{function_nopb2}

\begin{function_nopb2}{EE\_led\_6\_off}{EE_led_6_off:flex}
  \synopsis{void EE_led_6_off(void);}
  
  \begin{fundescription}
    The function turns LED 6 off.
  \end{fundescription}
\end{function_nopb2}

\begin{function_nopb2}{EE\_led\_7\_on}{EE_led_7_on:flex}
  \synopsis{void EE_led_7_on(void);}
  
  \begin{fundescription}
    The function turns LED 7 on.
  \end{fundescription}
\end{function_nopb2}

\begin{function_nopb2}{EE\_led\_7\_off}{EE_led_7_off:flex}
  \synopsis{void EE_led_7_off(void);}
  
  \begin{fundescription}
    The function turns LED 7 off.
  \end{fundescription}
\end{function_nopb2}




% -------------------------------------------------------------------

\section{Buttons}

The Flex Demo Daughter Board has a set of four buttons attached to GPIO pins
of the microcontroller. To use the buttons, the developer should
include the following fragment in the application OIL file:

\begin{lstlisting}
  ...
  BOARD_DATA = EE_FLEX {
    TYPE = DEMO { OPTIONS = BUTTONS; };
    ...
  }
  ...
\end{lstlisting}

The following subsections will describe the functions available to
control the Flex Demo Daughter Board buttons.

\begin{function_nopb2}{EE\_buttons\_init}{EE_buttons_init}
  \synopsis{void EE_buttons_init( void (*isr_callback)(void), EE_UINT8 mask );}
  
  \begin{fundescription}
    The function configures the GPIO pins used by the buttons. Buttons
    can be configured to be controlled only by using polling functions
    (no \fn{isr_callback} is specified), or can be configured to raise
    an interrupt (if \fn{isr_callback} is specified).

    When the \fn{isr_callback} is specified, the \fn{mask} parameter
    is used to control for which buttons the interrupt will be
    generated.
  \end{fundescription}
  
  \begin{funparameters}
    \fpar{isr_callback}{The function is called inside an ISR2 upon a
      button press.}

    \fpar{mask}{If \fn{isr_callback} is specified, then this parameter
      controls which buttons will generate an interrupt request. In
      particular, bit \const{0x01} is used for button S1, bit
      \const{0x02} is used for button S2, bit \const{0x04} is used for
      button S3, bit \const{0x08} is used for button S4.}
  \end{funparameters}
  
  \begin{funreturn}
    \fret{void}{The function does not return a value.}
  \end{funreturn}
  
%  \begin{funconformance}
%  \end{funconformance}
\end{function_nopb2}

\begin{function_nopb2}{EE\_button\_get\_S1}{EE_button_get_S1}
  \synopsis{EE_UINT8 EE_button_get_S1(void);}
  
  \begin{fundescription}
    The function returns the status of the button number S1.
  \end{fundescription}
  
%  \begin{funparameters}
%    \fpar{none}{None in this function.}
%  \end{funparameters}
  
  \begin{funreturn}
    \fret{unsigned char}{1 of the button is pressed, 0 otherwise.}
  \end{funreturn}
  
%  \begin{funconformance}
%  \end{funconformance}
\end{function_nopb2}

\begin{function_nopb2}{EE\_button\_get\_S2}{EE_button_get_S2}
  \synopsis{EE_UINT8 EE_button_get_S2(void);}
  
  \begin{fundescription}
    The function returns the status of the button number S2.
  \end{fundescription}
  
%  \begin{funparameters}
%    \fpar{none}{None in this function.}
%  \end{funparameters}
  
  \begin{funreturn}
    \fret{unsigned char}{1 of the button is pressed, 0 otherwise.}
  \end{funreturn}
  
%  \begin{funconformance}
%  \end{funconformance}
\end{function_nopb2}

\begin{function_nopb2}{EE\_button\_get\_S3}{EE_button_get_S3}
  \synopsis{EE_UINT8 EE_button_get_S3(void);}
  
  \begin{fundescription}
    The function returns the status of the button number S3.
  \end{fundescription}
  
%  \begin{funparameters}
%    \fpar{none}{None in this function.}
%  \end{funparameters}
  
  \begin{funreturn}
    \fret{unsigned char}{1 of the button is pressed, 0 otherwise.}
  \end{funreturn}
  
%  \begin{funconformance}
%  \end{funconformance}
\end{function_nopb2}

\begin{function_nopb2}{EE\_button\_get\_S4}{EE_button_get_S4}
  \synopsis{EE_UINT8 EE_button_get_S4(void);}
  
  \begin{fundescription}
    The function returns the status of the button number S4.
  \end{fundescription}
  
%  \begin{funparameters}
%    \fpar{none}{None in this function.}
%  \end{funparameters}
  
  \begin{funreturn}
    \fret{unsigned char}{1 of the button is pressed, 0 otherwise.}
  \end{funreturn}
  
%  \begin{funconformance}
%  \end{funconformance}
\end{function_nopb2}

% -------------------------------------------------------------------

\section{LCD}

The Flex Demo Daughter Board has an alpha-numeric 16 x 2 LCD display mounted
on the board attached to the GPIO pins of the microcontroller. To use
the LCD on the board, the developer should
include the following fragment in the application OIL file:

\begin{lstlisting}
  ...
  BOARD_DATA = EE_FLEX {
    TYPE = DEMO { OPTIONS = LCD; };
    ...
  }
  ...
\end{lstlisting}

The functions available can be used to print character and strings to
the LCD Display, and to select the current cursor position (which is
the position where the next character will be printed).

To specify a character position in the LCD, the functions provided
uses an integer. Numbers from 0 to 15 represents the first line,
whereas numbers from 15 to 31 represents the second line.

The following subsections will describe the functions available to
control the Explorer 16 LCD.


\begin{function_nopb2}{EE\_lcd\_init}{EE_lcd_init}
  \synopsis{void EE_lcd_init(void);}
  
  \begin{fundescription}
    The function initializes the LCD display.
  \end{fundescription}
  
%  \begin{funparameters}
%    \fpar{none}{None in this function.}
%  \end{funparameters}
  
%  \begin{funreturn}
%    \fret{void}{The function does not return a value.}
%  \end{funreturn}
  
%  \begin{funconformance}
%  \end{funconformance}
\end{function_nopb2}

\begin{function_nopb2}{EE\_lcd\_command}{EE_lcd_command}
  \synopsis{void EE_lcd_command(EE_UINT8 cmd);}
  
  \begin{fundescription}
    The function sends a command to the LCD. Most of the LCD functions
    described in this chapters basically remap to this function. The
    developer can use this function to implement features which are
    currently not supported by the LCD API.
  \end{fundescription}
  
  \begin{funparameters}
    \fpar{cmd}{The LCD command.}
  \end{funparameters}
  
%  \begin{funreturn}
%    \fret{void}{The function does not return a value.}
%  \end{funreturn}
  
%  \begin{funconformance}
%  \end{funconformance}
\end{function_nopb2}

\begin{function_nopb2}{EE\_lcd\_putc}{EE_lcd_putc}
  \synopsis{void EE_lcd_putc(EE_INT8 data);}
  
  \begin{fundescription}
    The function puts a character on the LCD display, at the current
    cursor position.
  \end{fundescription}
  
%  \begin{funparameters}
%    \fpar{data}{The character to be printed on the LCD.}
%  \end{funparameters}
  
%  \begin{funreturn}
%    \fret{void}{The function does not return a value.}
%  \end{funreturn}
  
%  \begin{funconformance}
%  \end{funconformance}
\end{function_nopb2}

% this function seems to be missing
%\begin{function_nopb2}{EE\_lcd\_getc}{EE_lcd_getc}
%  \synopsis{EE_INT8 EE_lcd_getc(void);}
%  
%  \begin{fundescription}
%    The function returns the character which is present at the current
%    cursor position.
%  \end{fundescription}
%  
%%  \begin{funparameters}
%%    \fpar{none}{None in this function.}
%%  \end{funparameters}
%  
%  \begin{funreturn}
%    \fret{char}{The character which is displayed at the current cursor
%      position.}
%  \end{funreturn}
%  
%%  \begin{funconformance}
%%  \end{funconformance}
%\end{function_nopb2}



\begin{function_nopb2}{EE\_lcd\_puts}{EE_lcd_puts}
  \synopsis{void EE_lcd_puts(EE_INT8 *buf);}
  
  \begin{fundescription}
    The function prints a string to the display.
  \end{fundescription}
  
  \begin{funparameters}
    \fpar{buf}{The string to display. It must be a valid C-language
      string.}
  \end{funparameters}
  
%  \begin{funreturn}
%    \fret{void}{The function does not return a value.}
%  \end{funreturn}
  
%  \begin{funconformance}
%  \end{funconformance}
\end{function_nopb2}


\begin{function_nopb2}{EE\_lcd\_busy}{EE_lcd_busy}
  \synopsis{unsigned char EE_lcd_busy(void);}
  
  \begin{fundescription}
    The function returns 1 if the display is busy, 0 otherwise. This
    function can be used to check if the application can send a new
    command to the LCD, or if the command can not be sent because the
    LCD is still busy processing the previous command.
  \end{fundescription}
  
%  \begin{funparameters}
%    \fpar{none}{None in this function.}
%  \end{funparameters}
  
  \begin{funreturn}
    \fret{unsigned char}{1 if the display is busy, 0 otherwise.}
  \end{funreturn}
  
%  \begin{funconformance}
%  \end{funconformance}
\end{function_nopb2}

\begin{function_nopb2}{EE\_lcd\_clear}{EE_lcd_clear}
  \synopsis{void EE_lcd_clear(void);}
  
  \begin{fundescription}
    The function clears the LCD.
  \end{fundescription}
  
%  \begin{funparameters}
%    \fpar{none}{None in this function.}
%  \end{funparameters}
  
%  \begin{funreturn}
%    \fret{void}{The function does not return a value.}
%  \end{funreturn}
  
%  \begin{funconformance}
%  \end{funconformance}
\end{function_nopb2}

\begin{function_nopb2}{EE\_lcd\_home}{EE_lcd_home}
  \synopsis{void EE_lcd_home(void);}
  
  \begin{fundescription}
    The function sets the current cursor position to the top left
    display character.
  \end{fundescription}
  
%  \begin{funparameters}
%    \fpar{none}{None in this function.}
%  \end{funparameters}
  
%  \begin{funreturn}
%    \fret{void}{The function does not return a value.}
%  \end{funreturn}
  
%  \begin{funconformance}
%  \end{funconformance}
\end{function_nopb2}

\begin{function_nopb2}{EE\_lcd\_line2}{EE_lcd_line2}
  \synopsis{void EE_lcd_line2(void);}
  
  \begin{fundescription}
    The function sets the current cursor position to the bottom left
    display character.
  \end{fundescription}
  
%  \begin{funparameters}
%    \fpar{none}{None in this function.}
%  \end{funparameters}
  
%  \begin{funreturn}
%    \fret{void}{The function does not return a value.}
%  \end{funreturn}
  
%  \begin{funconformance}
%  \end{funconformance}
\end{function_nopb2}

\begin{function_nopb2}{EE\_lcd\_curs\_right}{EE_lcd_curs_right}
  \synopsis{void EE_lcd_curs_right(void);}
  
  \begin{fundescription}
    The function sets the current cursor position on the next
    character on the right.
  \end{fundescription}
  
%  \begin{funparameters}
%    \fpar{none}{None in this function.}
%  \end{funparameters}
  
%  \begin{funreturn}
%    \fret{void}{The function does not return a value.}
%  \end{funreturn}
  
%  \begin{funconformance}
%  \end{funconformance}
\end{function_nopb2}

\begin{function_nopb2}{EE\_lcd\_curs\_left}{EE_lcd_curs_left}
  \synopsis{void EE_lcd_curs_left(void);}
  
  \begin{fundescription}
    The function sets the current cursor position on the next
    character on the left.
  \end{fundescription}
  
%  \begin{funparameters}
%    \fpar{none}{None in this function.}
%  \end{funparameters}
  
%  \begin{funreturn}
%    \fret{void}{The function does not return a value.}
%  \end{funreturn}
  
%  \begin{funconformance}
%  \end{funconformance}
\end{function_nopb2}


\begin{function_nopb2}{EE\_lcd\_shift}{EE_lcd_shift}
  \synopsis{void EE_lcd_shift(void);}
  
  \begin{fundescription}
    The function can be used to enable the shift mode of the LCD. When
    in shift mode, each character sent provoke the shifting of all the
    characters of the LCD.
  \end{fundescription}
  
%  \begin{funparameters}
%    \fpar{none}{None in this function.}
%  \end{funparameters}
  
%  \begin{funreturn}
%    \fret{void}{The function does not return a value.}
%  \end{funreturn}
  
%  \begin{funconformance}
%  \end{funconformance}
\end{function_nopb2}

\begin{function_nopb2}{EE\_lcd\_goto}{EE_lcd_goto}
  \synopsis{void EE_lcd_goto(EE_UINT8 posx, EE_UINT8 posy);}
  
  \begin{fundescription}
    The function sets the current cursor position to $(posx, posy)$.
  \end{fundescription}
  
  \begin{funparameters}
    \fpar{posx}{The LCD column, from 0 to 15.}
    \fpar{posy}{The LCD row, 0 or 1.}
  \end{funparameters}
  
%  \begin{funreturn}
%    \fret{void}{The function does not return a value.}
%  \end{funreturn}
  
%  \begin{funconformance}
%  \end{funconformance}
\end{function_nopb2}


% -------------------------------------------------------------------

\section{Analog sensors}

The Flex Demo Daughter Board has a set of analog channels available,
in particular (in parenthesis the pin assignments): 

\begin{itemize}
\item Temperature sensor (AN12/RB12);
\item Light sensor (AN13/RB13);
\item Trimmer (AN15/RB15);
\item Accelerometer X axis (AN16/RC1);
\item Accelerometer Y axis (AN17/RC2);
\item Accelerometer Z axis (AN18/RC3);
\item ADC Aux (AN19/RC4).
\end{itemize}

To use these inputs, the developer should
include the following fragment in the application OIL file:

\begin{lstlisting}
  ...
  BOARD_DATA = EE_FLEX {
    TYPE = DEMO {
      OPTIONS = ADC_IN;        // for the analog inputs 
      OPTIONS = ACCELEROMETER; // for the accelerometer
      OPTIONS = SENSORS;       // for the sensors
      OPTIONS = TRIMMER;       // for the potentiometer
  };
  ...
\end{lstlisting}

The functions available can be used to start and stop the A/D
converter, and to read the values from the various sensors.


\begin{function_nopb2}{EE\_analog\_init}{EE_analog_init}
  \synopsis{void EE_analog_init(void);}
  
  \begin{fundescription}
    The function initializes the A/D converter. 
    The ADC is initialized in polling mode.
  \end{fundescription}
  
%  \begin{funparameters}
%    \fpar{none}{None in this function.}
%  \end{funparameters}
  
%  \begin{funreturn}
%    \fret{void}{The function does not return a value.}
%  \end{funreturn}
  
%  \begin{funconformance}
%  \end{funconformance}
\end{function_nopb2}

\begin{function_nopb2}{EE\_analog\_close}{EE_analog_close}
  \synopsis{void EE_analog_close(void);}
  
  \begin{fundescription}
    The function turns off the A/D converter. 
  \end{fundescription}
\end{function_nopb2}


\begin{function_nopb2}{EE\_adcin\_init}{EE_adcin_init}
  \synopsis{void EE_adcin_init(void);}
  
  \begin{fundescription}
    The function initializes the A/D converter. 
    It has the same functionality as \const{EE_analog_init}.
  \end{fundescription}
\end{function_nopb2}

\begin{function_nopb2}{EE\_adcin\_get\_volt}{EE_adcin_get_volt}
  \synopsis{EE_UINT16 EE_adcin_get_volt(void);}
  
  \begin{fundescription}
    The function reads the ADC Aux channel and returns its value. The
    A/D converter should have
    been already initialized using \reffun{EE_adcin_init}.
  \end{fundescription}
  
  \begin{funreturn}
    \fret{EE_UINT16}{The voltage read from the ADC Aux channel, in millivolt.}
  \end{funreturn}
\end{function_nopb2}



\begin{function_nopb2}{EE\_trimmer\_init}{EE_trimmer_init}
  \synopsis{void EE_trimmer_init(void);}
  
  \begin{fundescription}
    The function initializes the A/D converter. 
    It has the same functionality as \const{EE_analog_init}.
  \end{fundescription}
\end{function_nopb2}

\begin{function_nopb2}{EE\_trimmer\_get\_volt}{EE_trimmer_get_volt}
  \synopsis{EE_UINT16 EE_trimmer_get_volt(void);}
  
  \begin{fundescription}
    The function reads the Trimmer channel and returns its value. The
    A/D converter should have
    been already initialized using \reffun{EE_adcin_init}.
  \end{fundescription}
  
  \begin{funreturn}
    \fret{EE_UINT16}{The voltage read from the Trimmer channel, in millivolt.}
  \end{funreturn}
\end{function_nopb2}



\begin{function_nopb2}{EE\_analogsensors\_init}{EE_analogsensors_init}
  \synopsis{void EE_analogsensors_init(void);}
  
  \begin{fundescription}
    The function initializes the A/D converter. 
    It has the same functionality as \const{EE_analog_init}.
  \end{fundescription}
\end{function_nopb2}

\begin{function_nopb2}{EE\_analog\_get\_temperature}{EE_analog_get_temperature}
  \synopsis{EE_UINT16 EE_nalog_get_temperature(void);}
  
  \begin{fundescription}
    The function reads the temperature sensor and returns its value. The
    A/D converter should have
    been already initialized using \reffun{EE_adcin_init}.
  \end{fundescription}
  
  \begin{funreturn}
    \fret{EE_UINT16}{The voltage read from the temperature sensor, in millivolt.}
  \end{funreturn}
\end{function_nopb2}

\begin{function_nopb2}{EE\_analog\_get\_light}{EE_analog_get_light}
  \synopsis{EE_UINT16 EE_analog_get_light(void);}
  
  \begin{fundescription}
    The function reads the Light sensor and returns its value. The
    A/D converter should have
    been already initialized using \reffun{EE_adcin_init}.
  \end{fundescription}
  
  \begin{funreturn}
    \fret{EE_UINT16}{The voltage read from the Light sensor, in millivolt.}
  \end{funreturn}
\end{function_nopb2}





\begin{function_nopb2}{EE\_accelerometer\_init}{EE_accelerometer_init}
  \synopsis{void EE_accelerometer_init(void);}
  
  \begin{fundescription}
    The function initializes the A/D converter. 
    It has the same functionality as \const{EE_analog_init}, plus some
    initialization specific for the 3-axis accelerometer..
  \end{fundescription}
\end{function_nopb2}

\begin{function_nopb2}{EE\_accelerometer\_getglevel}{EE_accelerometer_getglevel}
  \synopsis{EE_UINT8 EE_accelerometer_getglevel(void);}
  
  \begin{fundescription}
    Description to be done. 
    %\nb{Nino, provide a description}
  \end{fundescription}
  
  \begin{funreturn}
    \fret{EE_UINT8}{} %\nb{Nino, provide a description}}
  \end{funreturn}
\end{function_nopb2}


\begin{function_nopb2}{EE\_accelerometer\_setglevel}{EE_accelerometer_setglevel}
  \synopsis{void EE_accelerometer_setglevel(EE_UINT8 level);}
  
  \begin{fundescription}
	Description to be done. 
	%\nb{Nino, provide a description}
  \end{fundescription}
  
  \begin{funreturn}
    \fret{EE_UINT8}{}
%\nb{Nino, provide a description}}
  \end{funreturn}
\end{function_nopb2}


\begin{function_nopb2}{EE\_accelerometer\_sleep}{EE_accelerometer_sleep}
  \synopsis{void EE_accelerometer_sleep(void);}
  
  \begin{fundescription}
    Description to be done. 
%    \nb{Nino, provide a description}
  \end{fundescription}
  
  \begin{funreturn}
    \fret{EE_UINT8}{}
%\nb{Nino, provide a description}}
  \end{funreturn}
\end{function_nopb2}

\begin{function_nopb2}{EE\_accelerometer\_wakeup}{EE_accelerometer_wakeup}
  \synopsis{void EE_accelerometer_wakeup(void);}
  
  \begin{fundescription}
    Description to be done. 
%    \nb{Nino, provide a description}
  \end{fundescription}
  
  \begin{funreturn}
    \fret{EE_UINT8}{}
%\nb{Nino, provide a description}}
  \end{funreturn}
\end{function_nopb2}


\begin{function_nopb2}{EE\_accelerometer\_gety}{EE_accelerometer_gety}
  \synopsis{float EE_accelerometer_gety(void);}
  
  \begin{fundescription}
    Description to be done. 
%    \nb{Nino, provide a description}
  \end{fundescription}
  
  \begin{funreturn}
    \fret{EE_UINT8}{}
%\nb{Nino, provide a description}}
  \end{funreturn}
\end{function_nopb2}

\begin{function_nopb2}{EE\_accelerometer\_getz}{EE_accelerometer_getz}
  \synopsis{float EE_accelerometer_getz(void);}
  
  \begin{fundescription}
    Description to be done. 
%    \nb{Nino, provide a description}
  \end{fundescription}
  
  \begin{funreturn}
    \fret{EE_UINT8}{}
%\nb{Nino, provide a description}}
  \end{funreturn}
\end{function_nopb2}








\section{Buzzer}

The Flex Demo Daughter Board has a buzzer which can be used to produce
simple sounds.

To use the buzzer, the developer should
include the following fragment in the application OIL file:

\begin{lstlisting}
  ...
  BOARD_DATA = EE_FLEX {
    TYPE = DEMO {
      OPTIONS = BUZZER;        // for the analog inputs 
  };
  ...
\end{lstlisting}

The functions available can be used to setup the buzzer, and to play notes.

\begin{warning}
The buzzer driver uses timer T3.
\end{warning}

\begin{function_nopb2}{EE\_buzzer\_init}{EE_buzzer_init}
  \synopsis{void EE_buzzer_init(void);}
  
  \begin{fundescription}
    The function initializes the buzzer.
  \end{fundescription}
\end{function_nopb2}

\begin{function_nopb2}{EE\_buzzer\_set\_freq}{EE_buzzer_set_freq}
  \synopsis{void EE_buzzer_set_freq(EE_UINT16 new_freq);}
  
  \begin{fundescription}
    The function sets an output frequency for the buzzer. Frequencies
    should be higher than 10 Hz. No action is taken if the new
    frequency differs from the previous one by less than 10 Hz.
  \end{fundescription}
  
\begin{funparameters}
  \fpar{new_freq}{The new buzzer frequency, in Hz.}
\end{funparameters}
\end{function_nopb2}

\begin{function_nopb2}{EE\_buzzer\_get\_freq}{EE_buzzer_get_freq}
  \synopsis{EE_UINT16 EE_buzzer_get_freq(void);}
  
  \begin{fundescription}
    The function returns the current buzer frequency.
  \end{fundescription}
  
\begin{funreturn}
  \fret{EE_UINT16}{The current buzzer frequency.}
\end{funreturn}
\end{function_nopb2}


\begin{function_nopb2}{EE\_buzzer\_mute}{EE_buzzer_mute}
  \synopsis{void EE_buzzer_mute(void);}
  
  \begin{fundescription}
    The function mutes the buzzer.
  \end{fundescription}
\end{function_nopb2}

\begin{function_nopb2}{EE\_buzzer\_unmute}{EE_buzzer_unmute}
  \synopsis{void EE_buzzer_unmute(void);}
  
  \begin{fundescription}
    The function unmutes the buzzer.
  \end{fundescription}
\end{function_nopb2}

\begin{function_nopb2}{EE\_buzzer\_close}{EE_buzzer_close}
  \synopsis{void EE_buzzer_close(void);}
  
  \begin{fundescription}
    The function resets the buzzer.
  \end{fundescription}
\end{function_nopb2}


% -------------------------------------------------------------------

\section{PWM Output}

The Flex Demo Daughter Board has a PWM output which is attached to the
Output Compare 3 of the Timer 2. %\nb{Nino, please check}

To use the PWM, the developer should
include the following fragment in the application OIL file:

\begin{lstlisting}
  ...
  BOARD_DATA = EE_FLEX {
    TYPE = DEMO {
      OPTIONS = PWM_OUT;
  };
  ...
\end{lstlisting}

The functions available can be used to start, stop and set the duty
cycle of the PWM.


\begin{function_nopb2}{EE\_pwm\_init}{EE_pwm_init}
  \synopsis{void EE_pwm_init(EE_UINT16 Period);}
  
  \begin{fundescription}
    The function initializes the PWM, setting also its period.
  \end{fundescription}
  
\begin{funparameters}
  \fpar{Period}{The PWM period.}
\end{funparameters}
  
%  \begin{funreturn}
%    \fret{void}{The function does not return a value.}
%  \end{funreturn}
  
%  \begin{funconformance}
%  \end{funconformance}
\end{function_nopb2}


\begin{function_nopb2}{EE\_pwm\_set\_duty}{EE_pwm_set_duty}
  \synopsis{void EE_pwm_set_duty(float duty);}
  
  \begin{fundescription}
    The function sets the duty cycle of the PWM.
  \end{fundescription}
  
\begin{funparameters}
  \fpar{duty}{The PWM duty cycle.}
\end{funparameters}
  
%  \begin{funreturn}
%    \fret{void}{The function does not return a value.}
%  \end{funreturn}
  
%  \begin{funconformance}
%  \end{funconformance}
\end{function_nopb2}

\begin{function_nopb2}{EE\_pwm\_close}{EE_pwm_close}
  \synopsis{void EE_pwm_close(void);}
  
  \begin{fundescription}
    The function shuts down the PWM.
  \end{fundescription}
\end{function_nopb2}


% -------------------------------------------------------------------

\section{DAC Output}

The Flex Demo Daughter Board has a DAC output connected to the
Microcontroller I2C port which can be used to convert digital signals
to analog values.

To use the DAC, the developer should
include the following fragment in the application OIL file:

\begin{lstlisting}
  ...
  BOARD_DATA = EE_FLEX {
    TYPE = DEMO {
      OPTIONS = DAC;
  };
  ...
\end{lstlisting}

The functions available can be used to start, stop and set the duty
cycle of the PWM.


\begin{function_nopb2}{EE\_dac\_general\_call}{EE_dac_general_call}
  \synopsis{EE_INT8 EE_dac_general_call(EE_UINT8 second);}
  
  \begin{fundescription}
    Description to be done. 
%    \nb{nino, una descrizione!}
  \end{fundescription}
  
\begin{funparameters}
  \fpar{second}{}
%\nb{nino, decsrizione!}}
\end{funparameters}
  
\begin{funreturn}
  \fret{EE_INT8}{}
%\nb{nino, decsrizione}.}
\end{funreturn}
  
%  \begin{funconformance}
%  \end{funconformance}
\end{function_nopb2}

\begin{function_nopb2}{EE\_dac\_fast\_write}{EE_dac_fast_write}
  \synopsis{EE_INT8 EE_dac_fast_write(EE_UINT16 data, EE_UINT8 port, EE_UINT8 power);}
  
  \begin{fundescription}
    Description to be done. 
%    \nb{nino, una descrizione!}
  \end{fundescription}
  
\begin{funparameters}
  \fpar{data}{}
%\nb{nino, decsrizione!}}
  \fpar{port}{}
%\nb{nino, decsrizione!}}
  \fpar{power}{}
%\nb{nino, decsrizione!}}
\end{funparameters}
  
\begin{funreturn}
  \fret{EE_INT8}{}
%\nb{nino, decsrizione}.}
\end{funreturn}
  
%  \begin{funconformance}
%  \end{funconformance}
\end{function_nopb2}


\begin{function_nopb2}{EE\_dac\_write}{EE_dac_write}
  \synopsis{EE_INT8 EE_dac_write(EE_UINT16 data, EE_UINT8 port, EE_UINT8 power, EE_UINT8 save);}
  
  \begin{fundescription}
    Description to be done. 
    %\nb{nino, una descrizione!}
  \end{fundescription}
  
\begin{funparameters}
  \fpar{data}{}
%\nb{nino, decsrizione!}}
  \fpar{port}{}
%\nb{nino, decsrizione!}}
  \fpar{power}{}
%\nb{nino, decsrizione!}}
  \fpar{save}{}
%\nb{nino, decsrizione!}}
\end{funparameters}
  
\begin{funreturn}
  \fret{EE_INT8}{}
%\nb{nino, decsrizione}.}
\end{funreturn}
  
%  \begin{funconformance}
%  \end{funconformance}
\end{function_nopb2}

\begin{function_nopb2}{EE\_dac\_init}{EE_dac_init}
  \synopsis{void EE_dac_init(void);}
  
  \begin{fundescription}
    Description to be done. 
    %\nb{nino, una descrizione!}
  \end{fundescription}
\end{function_nopb2}
