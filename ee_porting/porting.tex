\documentclass[12pt,a4paper,normalheadings,titlepage]{scrreprt}

\usepackage{../common/evman}
\usepackage{graphicx}

\pagestyle{headings}

\def\title{Porting applications over the various conformance classes of \ee}
\def\subtitle{Quick guide}
\include{dynamic_version}

\begin{document}

\maketitle

\pagebreak

\tableofcontents

\chapter{Introduction}

The ERIKA Enterprise kernel provides various operating system APIs:
\begin{itemize}
\item a minimal multithreading API which
  offers multithreading and resource usage support for tiny
  microcontrollers, and
\item a superset of the minimal API
  which follows more closely the OSEK/VDX OS specification.
\end{itemize}

Both APIs allow similar programming capabilities, being able to support
multithreading applications for small microcontrollers.

However, there are a few differences which have to be taken into
account when porting an application from the minimal API to the OSEK API and
viceversa. The purpose of this document is to compare in detail the two APIs,
for letting the user to choose the development platform more suitable to its 
needs.

% ------------------------------------------------------

\chapter{Comparison between the various APIs}

\section{System differences}

\subsection{Conformance classes}

\ee\ supports four conformance classes named BCC1, BCC2, ECC1, and
ECC2. The main idea is that these conformance classes contain a subset
of the OSEK/VDX API features, allowing a fine tuning of performance vs
code and memory footprint.

The minimal API supports a conformance class named FP, which is similar to the 
BCC2 conformance class of \ee\ (or ECC2 if multistack is selected). 

% ------------------------------------------------------

\subsection{Error Handling}

\ee\ primitives typically return error values to inform the correct
execution of the primitive. There are various error codes returned,
which may be tuned to reduce the code footprint. In particular, there
is support for an {\em extended status}, where the primitives return
all the kind of errors which can be detected, and a {\em standard
status}, where only part of the errors are raised. 

To reduce the code footprint, the minilal API primitives typically do not return 
any error code. The system always assumes the correctness of
parameter values, and acts in a default way upon particular conditions 
(e.g., task activations are dropped over a given number of pending 
activations). For this reason, moreover, there is no distinction between 
{\em extended status} and {\em standard status}. 

\ee\ supports the \fn{ErrorHook} hook function and its macro,
allowing the access to primitive parameters to cause the
error. When defined in the OIL file, \fn{ErrorHook} is called
everytime an error different from \const{E_OK} is returned by a
primitive. The minimal API does not support neither \fn{ErrorHook} nor its
macros.

% ------------------------------------------------------

\subsection{PreTaskHook and PostTaskHook}

\ee\ supports the \fn{PreTaskHook} and the \fn{PostTaskHook} hook
functions. These hooks are called by the kernel whenever a context
change occurs. The minimal API does not support \fn{PreTaskHook} and
\fn{PostTaskHook}.

% ------------------------------------------------------

\subsection{System startup}

\ee\ supports system startup using \fn{StartOS}. The application 
developer has to put a call to \fn{StartOS} within the \fn{main} 
function to start the system. In this way, the \fn{main} function 
becomes the Background Task. A call to \fn{StartOS} provoke the 
execution of the \fn{StartupHook} hook function, the autostart of tasks 
and alarms if specified inside the OIL file. The \fn{StartOS} primitive 
is also used to set the application mode. 

The minimal API does not support \fn{StartOS}, \fn{StartupHook}, application modes 
and the autostart of tasks and alarms. With the minimal API, the kernel is already 
started at the first instruction of \fn{main}. Even in this case, the 
\fn{main} function is the Background Task. Task activations and Alarm 
arming at syystem startup must be done explicitly by the developer. 

\begin{warning}
When porting an application from the minimal API to the OSEK/VDX API, a call to \fn{StartOS}
{\em must} be added in the system startup routine (typically \fn{main}).
\end{warning}

\begin{warning}
When using the minimal API, remember that activating a task within the \fn{main} 
always causes a preemption to the activated task. This must be taken 
into account when more than one task has to be activated at startup. A 
possible solution is to activate the tasks starting from the 
highest priority one. 
\end{warning}


% ------------------------------------------------------

\subsection{System Shutdown}

In \ee\ , the user should call \fn{ShutdownOS} when the system must
end. Calling \fn{ShutdownOS} also causes a call to the
\fn{ShutdownHook} hook function.

The minimal API does not support \fn{ShutdownOS} and \fn{ShutdownHook}.


% ------------------------------------------------------

\section{Tasks}

\subsection{Task termination}

In \ee, a task instance must always terminate with a call to
\fn{TerminateTask} or to \fn{ChainTask}. Failing to terminate a task
with one of these primitives brings to an undefined result; typically,
it provokes an application crash. \fn{TerminateTask} and
\fn{ChainTask} provide a simple way to clean and throw away the task
stack.

The minimal API does not support \fn{TerminateTask} and
\fn{ChainTask}. A task terminates at the last \const{\}} of the task
function. No explicit stack cleanup functions are supported.
 
\begin{warning}
When porting an application from the minimal API to the OSEK/VDX API,
the developer {\em must} add a call to \fn{TerminateTask} at the end
of every task body.
\end{warning}

\subsection{Informations on tasks}

\ee\ supports the \fn{GetTaskID} and \fn{GetTaskState} primitives to
get information about the running task ID and the task statuses.

The minimal API does not support neither \fn{GetTaskID} nor \fn{GetTaskState}. 

\subsection{Basic tasks and extended tasks}

\ee\ distinguishes between {\em Basic} Tasks and {\em Extended}
Tasks. Basic tasks typically run on a shared stack, whereas extended
tasks must run on a private stack. Extended tasks are tasks which use
events and counting semaphores.

The minimal API does not have an explicit dinstinction between basic and extended
tasks. The designer must take care to call counting semaphores and blocking
primitives only within tasks with a private stack.

\subsection{Number of pending activations}

Considering the conformance classes BCC1 and ECC1 of \ee, tasks can
have only one pending activation. In conformance classes BCC2 and
ECC2, tasks can have more than one pending activation. The maximum
number of pending activations is specified inside the OIL file and
can not be changed at runtime. Pending activations of tasks with the
same priority are processed in a FIFO order, meaning that the ready
queue enqueues {\em activations} and not {\em tasks}, consuming RAM
space for each pending activation which have to be stored.

When using the minimal API, tasks store the number of pending activations as an integer value. 
Therefore, the maximum value is implementation dependent. The developer 
can not rely on a particular order in the processing of pending 
activations of tasks with the same priority. 

% ------------------------------------------------------

\section{Interrupt handling}

There is always a distinction between ISR type 1 and type 2.

While \ee\ supports the primitives for disabling interrupts, the
minimal API does not (please refer to the architecture manual for
functions to disable interrupts).

% ------------------------------------------------------

\section{Event handling}

\ee\ supports events for the two conformance classes: ECC1 and ECC2. 
Events are not supported by the minimal API.

% ------------------------------------------------------

\section{Support for non-blocking semaphores}

\ee\ supports the non-blocking counting semaphores primitives in the
BCC1 and BCC2 conformance classes. BCC1 and BCC2 can also run on a
monostack configuration.

Semaphore primitives are supported by the minimal API only in the multistack
configuration.

% ------------------------------------------------------

\section{ORTI support}

If configured, \ee\ maintains information to support ORTI debugger
awareness. The RT-Druid code generator is able to generate appropriate
ORTI files which can be interpreted by debuggers such as Lauterbach
Trace32.

The minimal API does not support ORTI kernel awareness.

\chapter{History}

\begin{tabular}{|p{0.4\columnwidth}|p{0.6\columnwidth}|}
\hline 
Version&
Comment\tabularnewline
\hline
\hline 
1.0.0&
Initial version of this document.\tabularnewline
\hline 
1.0.1&
Added new versioning mechanism.\tabularnewline
\hline 
\end{tabular}


%\bibliographystyle{plain}
%\bibliography{../common/biblio}

\printindex

\end{document}
