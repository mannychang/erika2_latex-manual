\chapter{Introduction}

\section{\ee\ }
Evidence presents the \ee\ RTOS, a minimal RTOS for single
chip microcontrollers, which provides a simple and tiny multithreading
environment with support for advanced real-time scheduling algorithms
and which supports stack sharing.

The \ee\ kernel has been developed with the idea of providing the
minimal set of primitives which can be used to implement a
multithreading environment. The \ee\ API is available as a reduced
OSEK/VDX API, providing support for thread activation, mutual
exclusion, alarms, and counting semaphores.

Moreover, the \ee\ kernel offers support for Fixed Priority (FP),
Earliest Deadline First (EDF), and Contract-based scheduling (FRSH)
\cite{frescor} scheduling algorithms, to offer a choice between the
tradiction and innovative efficient ways of scheduling concurrent
threads.

The OSEK/VDX consortium provides the OIL language (OSEK Implementation 
Language) as a standard configuration language, which is used for the 
static definition of the RTOS objects which are instantiated and used 
by the application. \ee\ fully supports the OIL language for the 
configuration of real-time applications.

To face the complexity of dealing with the OIL language and 
configuration files, Evidence ships the \rtd\ configuration and 
profiling environment, which allows to configure all the application 
parameters through a easy-to-use visual interface that automatically 
generates the application configuration file using the OIL language. 

The typical application design flow include the definition of an OIL 
configuration file which defines the RTOS objects used by the 
application; after that, \rtd\ is run for generating appropriate makefiles 
and source code to configure the \ee. Finally, the application is 
compiled to produce an executable file which can be run on the target. 

The features provided by \ee\ to developers are the following:
\begin{itemize}
\item Traditional RTOS features:
  \begin{itemize}
  \item Support for preemptive and non-preemptive multitasking;
  \item Support for fixed priority scheduling;
  \item Support for shared resources;
  \item Support for periodic activations using Alarms;
  \end{itemize}
\item Innovative features
  \item Support for stack sharing techniques, and one-shot task
    model to reduce the overall stack usage;
  \item Support for EDF scheduling by using a circular timer approach
    \cite{Carlini03}.
  \item Support for contract-based resource reservations using the
    IRIS scheduling algorithm \cite{iris}
\end{itemize}

The purpose of this document is to describe in detail the minimal API
implemented by \ee. Please check the Evidence web site for other
documents describing the details of \ee\ portings for the different
supported embedded targets.
