%=================================================================
\chapter{Software for \flex}
\label{ch:software}
%=================================================================
%\chapter{Sofware development for the \flex\ boards}
%\label{ch:ee-flex}
The \flex\ boards comes with a rich software infrastructure which symplifies the appli-
cation development.

\section{\ee}
First of all, \flex\ comes with \ee\ as the default software
development environment. In particular, \ee\ for Microchip dsPIC (R)
DSC micro-controller family is a complete open-source\footnote{\ee\ is
distributed under the GPL+Linking exception license} RTOS implementing
the OS, OIL, and ORTI part of the OSEK/VDX standard
(\url{http://www.osek-vdx.org}). \ee\ includes the state of the art
real- time technology as well as the \rtd\ configuration tool, which
allows easy design and optimization of a real-time application.

\section{Libraries for \flex}
\ee\ fully supports the \flex\ boards and all the
Daughter Boards. A complete set of libraries allows the exploitation
of all the features provided. The development of complex applications
based on the \flex\ Base Board and available Daughter Boards is
simplified by a well documented and clear set of primitives.  The
needed libraries can be configured using the \rtd\ tool, letting the
developer to dedicate the efforts to the implementation of the program
logic.

\section{Template applications}
A set of template applications using the \flex\ boards are also
available. These applica- tions can be instantiated as \rtd\ projects
by selecting the appropriate template at project creation time.

\section{Scilab and Scicos code generator}
Finally, a code generator for Scilab and Scicos designs is also
available. The code generator has been developed in collaboration
with Simone Mannori from INRIA (FR), and Roberto Bucher from SUPSI
Lugano. 

Please check the Evidence web site
\url{http://www.evidence.eu.com} to get updated documentation and
manuals about the Scilab/Scicos code generator support.
